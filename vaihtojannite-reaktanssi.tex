
\frame{
\frametitle{Sinimuotoinen vaihtojännite}
\begin{itemize}
\item Pistorasiasta saatava jännite on {\em sinimuotoista}.
\item Kaikki jaksolliset vaihtojännitteet voidaan esittää siniaaltojen summana
(Fourier'n teoreema).
\item Lineaarisessa piirissä sinimuotoinen heräte tuottaa sinimuotoisen vasteen.
\item Sinimuotoisilla signaaleilla laskeminen on helppoa.
\end{itemize}
}



\frame{
\frametitle{Sinimuotoinen jännite (tai virta)}
Sinimuotoinen jännite (tai virta) määritellään
\[
u(t)=\hat{u}\sin(2\pi f t+ \phi)
\]
Usein merkitään $\omega=2\pi f$, jolloin kaava lyhenee:
\[
u(t)=\hat{u}\sin(\omega t+ \phi)
\]
Suuretta $\omega$ kutsutaan {\em kulmataajuudeksi} ja sen yksikkö on $\frac{\rm rad}{\rm s}$. Kulma $\phi$ on {\em vaihekulma}.
Kerroin $\hat{u}$ (luetaan u-hattu) on jännitteen {\em amplitudi} eli huippuarvo. Huippuarvosta saadaan jännitteen {\em tehollisarvo} jakamalla
se luvulla $\sqrt{2}$.
}

\frame{
\frametitle{Kondensaattori ja sinimuotoinen jännite}
Yhtälöstä
\[
i=C\Du
\]

\begin{center}
\begin{picture}(50,50)(0,0)
\vst{-25,0}{u(t)=\hat{u}\sin(\omega t+ \phi)}
\vc{25,0}{C}
\di{25,40}{i}
%\du{40,0}{u}
\hln{-25,0}{50}
\hln{-25,50}{50}

\end{picture}
\end{center}
Virta on siis
\[
i=C\Du=C\omega\hat{u}\cos(\omega t+ \phi)=C\omega\hat{u}\sin(\omega t+ \phi+\frac{\pi}{2})
\]
Eli jännitteen ja virran suhde on $\frac{1}{C\omega}$, ja niiden välillä on 90 asteen ($\frac{\pi}{2}$
radiaanin) vaihe-ero. Virta on 90 astetta jännitettä edellä.
}

\frame{
\frametitle{Reaktanssin käsite}
Kondensaattorille
\[
i=C\Du
\]

\begin{center}
\begin{picture}(50,50)(0,0)
\vst{-25,0}{u(t)=\hat{u}\sin(\omega t+ \phi)}
\vc{25,0}{C}
\di{25,40}{i}
%\du{40,0}{u}
\hln{-25,0}{50}
\hln{-25,50}{50}

\end{picture}
\end{center}
\[
i(t)=C\Du=C\omega\hat{u}\cos(\omega t+ \phi)=C\omega\hat{u}\sin(\omega t+ \phi+\frac{\pi}{2})
\]
Jännitteen ja virran {\bf amplitudien} suhde on:
\[
X=\frac{\hat{u}}{C\omega \hat{u}}=\frac{1}{\omega C}
\]
Tätä jännitteen ja virran suhdetta $X$ kutsutaan {\bf reaktanssiksi}, aivan kuten jännitteen ja virran suhdetta
vastuksessa kutsutaan resistanssiksi.
}

\frame{
\frametitle{Kela ja sinimuotoinen jännite}
Yhtälöstä
\[
u=L\Di
\]

\begin{center}
\begin{picture}(50,50)(0,0)
\vj{-25,0}{i(t)=\hat{i}\sin(\omega t+ \phi)}
\vl{25,0}{L}
\di{25,40}{i}
\du{40,0}{u}
\hln{-25,0}{50}
\hln{-25,50}{50}

\end{picture}
\end{center}
Jännite on 
\[
u=L\Di=L\omega\hat{i}\cos(\omega t+ \phi)=L\omega\hat{i}\sin(\omega t+ \phi+\frac{\pi}{2})
\]
Eli jännitteen ja virran suhde on ${L\omega}$, ja niiden välillä on 90 asteen ($\frac{\pi}{2}$
radiaanin) vaihe-ero. Virta on 90 astetta jännitettä jäljessä.
}

\frame{
\frametitle{Kelan reaktanssi}
Yhtälöstä
\[
u=L\Di
\]

\begin{center}
\begin{picture}(50,50)(0,0)
\vj{-25,0}{i(t)=\hat{i}\sin(\omega t+ \phi)}
\vl{25,0}{L}
\di{25,40}{i}
\du{40,0}{u}
\hln{-25,0}{50}
\hln{-25,50}{50}

\end{picture}
\end{center}
Jännite on 
\[
u=L\Di=L\omega\hat{i}\cos(\omega t+ \phi)=L\omega\hat{i}\sin(\omega t+ \phi+\frac{\pi}{2})
\]
Jännitteen ja virran suhde eli kelan reaktanssi on:
\[
X=\frac{L\omega\hat{i}}{\hat{i}}=\omega L
\]
}

\frame{
\begin{block}{Esimerkki} % TODO: Ratkaisu puuttuu
Kuinka suuri on virran $I$ a) tehollisarvo b) huippuarvo ($\hat{i}$). Onko virta jännitettä edellä vai jäljessä,
ja kuinka paljon? Jännitelähteen taajuus on 50 Hz ja tehollisarvo 230 volttia.
\end{block}
\begin{center}
\begin{center}
\begin{picture}(50,50)(0,0)
\vst{-25,0}{E}
\vc{25,0}{C}
\di{25,40}{I}
%\du{40,0}{u}
\hln{-25,0}{50}
\hln{-25,50}{50}

\end{picture}
\end{center}
\[
C=1\uF
\]
\end{center}

}


\frame{
\frametitle{Esimerkki} % Arska 1 kotitehtävä 1 2009
\[
J=1\A\ (f=50\Hz)\quad L=1\,{\rm H} \quad C=100\uF
\]

\begin{center}
\begin{picture}(150,110)(0,0)
\vj{0,0}{J}
\vc{50,0}{C}
\vl{50,50}{L}
\hln{0,0}{50}
\hln{0,100}{50}
\vln{0,50}{50}
\du{70,0}{u_{\rm C}}
\du{70,50}{u_{\rm L}}

\end{picture}
\end{center}
Laske jännitteet $u_{\rm C}$ ja $u_{\rm L}$ ja ilmoita ne ajan funktiona muodossa $u=\hat{u}\sin(\omega t+\phi)$. Miksi näiden jännitteiden tehollisarvojen (tai huippuarvojen) summa
ei ole suoraan virtalähteen jännite?

\tiny $u_{\rm L}\approx 444\V\cdot\sin(2\pi 50\Hz\cdot t+90^\circ)$, $u_{\rm C}\approx 45\V\cdot\sin(2\pi 50\Hz\cdot t-90^\circ)$
}

