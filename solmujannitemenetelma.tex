\frame{
\frametitle{Kirchhoffin lakien systemaattinen soveltaminen}
Virtapiiriyhtälöt kannattaa kirjoittaa systemaattisesti, ettei sekoa omaan näppäryyteensä.
Yksi tapa on {\bf solmujännitemenetelmä}:
\begin{enumerate}
\item Valitse joku solmuista maasolmuksi
\item Nimeä jännitteet maasolmua vasten eli piirrä jännitenuoli jokaisesta solmusta maasolmuun.
\item Lausu vastusten jännitteet nimettyjen jännitteiden avulla (piirrä jokaisen vastuksen yli jännitenuoli).
\item Kirjoita virtayhtälö jokaiselle solmulle, jossa on tuntematon jännite.
\item Ratkaise jännitteet virtayhtälöistä.
\item Ilmoita kysytty jännite/jännitteet ja/tai virta/virrat.
\end{enumerate}

}

\frame{
\frametitle{Esimerkki}
Ratkaise virta $I$.
\begin{center}
\begin{picture}(150,100)(0,0)
\vst{0,0}{E_1}
\vst{100,0}{E_2}
\vz{50,0}{R_3\hspace{-0.65cm}}
\hz{0,50}{R_1\vspace{0.8cm}}
\hz{50,50}{R_2\vspace{0.8cm}}
\di{50,1}{I}
\hln{0,0}{100}

\pause
\color{red}
\hgp{25,0}
%\ri{48,50}{I_1}
%\li{52,50}{}
%\txt{55,60}{I_2}

\pause
\color{blue}

\du{58,0}{U_3}


%\pause
%\color{violet}
%\txt{160,25}{I=I_1+I_2}
%\tx{160,10}


\pause
\color{cyan}
\ru{0,60}{E_1-U_3}
\lu{50,60}{E_2-U_3}


\end{picture}
\end{center}
\pause
\[
\color{magenta} \frac{U_3}{R_3}=\frac{E_1-U_3}{R_1}+\frac{E_2-U_3}{R_2}
\pause
\color{orange} \Longrightarrow
U_3=R_3\frac{R_2E_1+R_1E_2}{R_1R_2+R_2R_3+R_1R_3}
\]
\pause
\[
\color{brown} I=\frac{U_3}{R_3}=\frac{R_2E_1+R_1E_2}{R_1R_2+R_2R_3+R_1R_3}
\]
}

\frame{
\frametitle{Huomautuksia}
\begin{itemize}
\item Yhtälöt voi kirjoittaa monella eri logiikalla, ei ole yhtä oikeaa menetelmää.
\item Vaatimuksena ainoastaan a) Kirchhoffin lakien noudattaminen b) Ohmin lain\footnote{
Ohmin lakia voi käyttää vain vastuksille. Jos piirissä on muita komponentteja, tulee tietää
niiden virta-jänniteyhtälö eli tietää, miten komponentin virta riippuu jännitteestä.}
 noudattaminen
sekä se, että yhtälöitä on yhtä monta kuin tuntemattomia.
\item Jos piirissä on virtalähde, se säästää (yleensä) laskentatyötä, koska silloin tuntemattomia
virtoja on yksi vähemmän.
\item Käyttämällä konduktansseja yhtälöt näyttävät siistimmiltä.
\end{itemize}
}

\frame{
\frametitle{Toinen esimerkki}

\begin{center}
\begin{picture}(150,100)(0,0)
\vst{0,0}{E_1}
\vst{150,0}{E_2}
\vz{50,0}{R_3\hspace{-0.65cm}}
\hz{0,50}{R_1\vspace{0.8cm}}
\hz{50,50}{R_2\vspace{0.8cm}}
\hln{0,0}{150}
\vz{100,0}{R_4\hspace{-0.65cm}}
\hz{100,50}{R_5\vspace{0.8cm}}

%\color{green}
%\ri{8,50}{I_1}
%\ri{58,50}{I_2}
%\di{50,1}{I_3}
%\ri{108,50}{I_5}
%\di{100,1}{I_4}
%\txt{200,25}{I_1=I_2+I_3}
%\txt{200,10}{I_2=I_4+I_5}

\color{blue}
\du{58,0}{U_3}
\du{108,0}{U_4}


\end{picture}
\end{center}

\[\color{blue}
\frac{E_1-U_3}{R_1}=\frac{U_3-U_4}{R_2}+\frac{U_3}{R_3} \quad \mbox{ja}\quad
\frac{U_3-U_4}{R_2}=\frac{U_4}{R_4}+\frac{U_4-E_2}{R_5}
\]
\[\color{red}
G_1(E_1-U_3)=G_2(U_3-U_4)+G_3U_3\ \mbox{ja}\
G_2(U_3-U_4)=G_4U_4+G_5(U_4-E_2)
\]
Kaksi yhtälöä, kaksi tuntematonta, voidaan ratkaista. Lopputulos on sama, käytitpä konduktansseja tai resistansseja!
}


\frame{
\frametitle{Huomattavaa}
\begin{itemize}
\item Virtapiirin ratkaisemiseksi on useita muitakin menetelmiä kuin solmujännitemenetelmä: haaravirtamenetelmä,
silmukkamenetelmä, solmumenetelmä, modifioitu solmupistemenetelmä\ldots
\item Mikäli piirissä on ideaalisia jännitelähteitä (=jännitelähteitä, jotka liittyvät suoraan
solmuun ilman että välissä on vastus), yhtälöihin tulee yksi tuntematon arvo lisää (=jännitelähteen
virta) sekä yksi yhtälö lisää (jännitelähde määrää solmujen jännite-eron).
\end{itemize}
}

\frame{
\frametitle{Jännitelähde kahden solmun välillä}
Jos piirissä on kahden (muun kuin maa)solmun välillä jännitelähde, niin ongelmaan tulee yksi yhtälö ja yksi muuttuja lisää.
\begin{enumerate}
\item Jännitelähteen virtaa ei tunneta $\to$ merkitse sitä $I$:llä.
\item Jännitelähde pakottaa kahden solmun välille tietyn jännite-eron. Kirjoita tälle jännite-erolle yhtälö.
\end{enumerate}
}

\frame{
\frametitle{Jännitelähde kahden solmun välillä: esimerkki}
Entäpä jos edellisessä esimerkissä olisi ollut $R_2$:n tilalla jännitelähde $E_3$:
\begin{center}
\begin{picture}(150,70)(0,0)
\vst{0,0}{E_1}
\vst{150,0}{E_2}
\vz{50,0}{R_3\hspace{-0.65cm}}
\hz{0,50}{R_1\vspace{0.8cm}}
\hst{50,50}{E_3\vspace{0.2cm}}
\hln{0,0}{150}
\vz{100,0}{R_4\hspace{-0.65cm}}
\hz{100,50}{R_5\vspace{0.8cm}}

%\ri{8,50}{I_1}
%\ri{58,50}{I_2}
%\di{50,1}{I_3}
%\ri{108,50}{I_5}
%\di{100,1}{I_4}
%\txt{200,25}{I_1=I_2+I_3}
%\txt{200,10}{I_2=I_4+I_5}

\pause

\color{blue}
\du{58,0}{U_3}
\du{108,0}{U_4}

\pause
\color{red}
\ru{0,60}{E_1-U_3}
\ru{50,65}{U_3-U_4}

\ru{100,60}{U_4-E_2}


\end{picture}
\end{center}
Jännitelähteen virtaa ei voi laskea Ohmin lailla, koska Ohmin laki pätee vain vastukselle, ei jännitelähteelle:
\pause
\[%\color{blue}
\frac{E_1-U_3}{R_1}=\cancel{\color{orange}\frac{U_3-U_4}{R_2}}+\frac{U_3}{R_3} \quad \mbox{ja}\quad
\cancel{\color{orange} \frac{U_3-U_4}{R_2}}=\frac{U_4}{R_4}+\frac{U_4-E_2}{R_5}
\]
Ratkaisu: merkitään jännitelähteen virtaa $I$:llä.
\[%\color{blue}
\frac{E_1-U_3}{R_1}=I+\frac{U_3}{R_3} \quad \mbox{ja}\quad
I=\frac{U_4}{R_4}+\frac{U_4-E_2}{R_5}
\]

}

\frame{
\frametitle{Jännitelähde kahden solmun välillä: esimerkki}
\begin{center}
\begin{picture}(150,70)(0,0)
\vst{0,0}{E_1}
\vst{150,0}{E_2}
\vz{50,0}{R_3\hspace{-0.65cm}}
\hz{0,50}{R_1\vspace{0.8cm}}
\hst{50,50}{E_3\vspace{0.2cm}}
\hln{0,0}{150}
\vz{100,0}{R_4\hspace{-0.65cm}}
\hz{100,50}{R_5\vspace{0.8cm}}

\color{orange}
\ri{58,50}{}
\txt{56,58}{I}

\color{blue}
\du{58,0}{U_3}
\du{108,0}{U_4}

\color{red}
\ru{0,60}{E_1-U_3}
\ru{50,65}{U_3-U_4}

\ru{100,60}{U_4-E_2}


\end{picture}
\end{center}
Ratkaisu: merkitään jännitelähteen virtaa $I$:llä.
\[%\color{blue}
\frac{E_1-U_3}{R_1}={\color{orange} I}+\frac{U_3}{R_3} \quad \mbox{ja}\quad
{\color{orange} I}=\frac{U_4}{R_4}+\frac{U_4-E_2}{R_5}
\]
Koska meillä on nyt kolme tuntematonta, tarvitaan vielä yksi yhtälö:
\[
-E_3=U_3-U_4
\]
Kolme yhtälöä, kolme tuntematonta $\to$ voidaan ratkaista.
}

\frame{
\frametitle{Jännitelähde kahden solmun välillä: yhteenveto}
\begin{itemize}
\item Koska jännitelähteen virtaa ei voida laskea Ohmin laista, jännitelähteen virta tuo yhden tuntemattoman lisää piiriyhtälöihin:
\[%\color{blue}
\frac{E_1-U_3}{R_1}={\color{orange} I}+\frac{U_3}{R_3} \quad \mbox{ja}\quad
{\color{orange} I}=\frac{U_4}{R_4}+\frac{U_4-E_2}{R_5}
\]
\item Koska meille tuli yksi tuntematon lisää, tarvitaan myös yksi yhtälö lisää.
\item Tämä yhtälö saadaan Kirchhoffin jännitelaista:
\[
-E_3=U_3-U_4
\]
\end{itemize}

}


\frame{
\begin{block}{Esimerkki}
Ratkaise virta $I_4$. Tarkista tuloksesi siten, että merkitset kuvaan kaikki jännitteet ja virrat ja toteat, että tuloksesi ei ole ristiriidassa Kirchhoffin lakien kanssa.
\end{block}
\begin{center}
\begin{picture}(150,50)(0,0)
\vj{0,0}{J}
\hz{0,50}{R_1}
%\hz{50,50}{R_3}
\hst{50,50}{E}
\hz{100,50}{R_4}
\vz{50,0}{R_2}
\vz{100,0}{R_3}
\vz{150,0}{R_5}
\hln{0,0}{150}
\di{100,1}{I_4}
\end{picture}
$R_1=R_2=R_3=R_4=R_5=1\ohm\qquad E=9\V\qquad J=1\A$
\end{center}

}



\frame {
  \frametitle{Ratkaisu}
\begin{center}
\begin{picture}(150,50)(0,0)
\vj{0,0}{J}
\hz{0,50}{R_1}
%\hz{50,50}{R_3}
\hst{50,50}{E}
\hz{100,50}{R_4}
\vz{50,0}{R_2}
\vz{100,0}{R_3}
\vz{150,0}{R_5}
\hln{0,0}{150}
\di{100,1}{I_4}
\color{red}
\du{57,0}{U_2}
\du{107,0}{U_3}
\hgp{100,0}
\ri{96,50}{I}
\end{picture}
$R_1=R_2=R_3=R_4=R_5=1\ohm\qquad E=9\V\qquad J=1\A$
\end{center}

Kirjoitetaan kaksi virtayhtälöä ja yksi jänniteyhtälö. Merkitään vastusten $R_4$ ja $R_5$
sarjaankytkennän konduktanssia symbolilla $G_{45}$.
\begin{eqnarray*}
J&=&U_2G_2+I\\
I&=&U_3G_3+U_3G_{45} \\
U_2+E&=&U_3
\end{eqnarray*}

Sijoittamalla toisesta yhtälöstä $I$:n ensimmäiseen yhtälöön ja sijoittamalla tähän kolmannesta yhtälöstä
saatavan $U_2$:n, saadaan
\[
J=(U_3-E)G_2+U_3(G_3+G_{45})
\]
}
\frame{
Sijoitetaan yhtälöön lukuarvot ja ratkaistaan:

\[
U_3=4 \V
\]

\begin{itemize}
\item Joten kysytty virta on $4 \V \cdot 1 \Siemens=4\A$.
\item Jänniteyhtälöstä $U_2+E=U_3$ ratkeaa $U_2=-5\V$, siispä vastuksen $R_2$ virta on $5\A$ alhaalta ylöspäin.
\item Virraksi $I$ saadaan $1\A+5\A=6\A$, josta $4\A$ kulkee $R_3$:n läpi ja loput $2\A$ vastusten $R_4$ ja $R_5$ läpi.
\item Jännitteet ja virrat täsmäävät Kirchhoffin lakien kanssa, joten piiri on laskettu oikein.
\end{itemize}
}



\frame{
\frametitle{Esimerkki 1}
% Silvonen 153 http://users.tkk.fi/~ksilvone/Lisamateriaali/teht100.pdf
Ratkaise $I$ ja $U$.
\begin{center}
\begin{picture}(150,100)(0,0)
\vst{0,0}{E_1}
\vst{50,0}{E_2}
\hst{50,100}{E_3}
\hz{0,50}{R_1}
\vz{100,0}{R_2}
\vj{150,0}{J}
\di{0,42}{I}
\du{110,0}{U}

\hln{0,0}{150}
\hln{100,100}{50}
\hln{0,100}{50}

\vln{0,50}{50}
\vln{100,50}{50}
\vln{150,50}{50}

\uncover<2->{\color{red} \li{50,100}{I_3}}
\end{picture}
\end{center}
\pause
\begin{eqnarray*}
J&=&UG_2+I_3\\
I_3&=&I+(E_1-E_2)G_1\\
U&=&E_1+E_3
\end{eqnarray*}
}


\frame{
\frametitle{Esimerkki 2}
% Silvonen 162
Ratkaise $U_2$ ja $I_1$.
\begin{center}
\begin{picture}(150,100)(0,0)
\vst{0,0}{E_1}
\vst{50,0}{E_2}
\vst{100,0}{E_3}
\hj{0,50}{J_1}
\hlj{50,50}{J_2}
\hz{25,100}{R}
\hln{0,0}{100}
\vln{0,50}{50}
\vln{100,50}{50}
\hln{0,100}{25}
\hln{75,100}{25}
\lu{50,67}{U_2}
\li{25,0}{I_1}

\end{picture}
\end{center}
\pause
\begin{eqnarray*}
I_1&=&(E_1-E_3)G+J_1\\
E_2+U_2&=&E_3
\end{eqnarray*}
}

\frame{
\frametitle{Mistä lisäharjoitusta?}
\begin{itemize}
\item Silvosen kirjaan on lisämateriaalia osoitteessa \url{http://users.tkk.fi/~ksilvone/Lisamateriaali/lisamateriaali.htm}
\item Sieltä löytyy tasavirtapiiritehtäviä 175 kappaletta \url{http://users.tkk.fi/~ksilvone/Lisamateriaali/teht100.pdf}
\item Tehtäviin on pdf:n lopussa myös ratkaisut, joten saat välittömän palautteen osaamisestasi!
\item Jos intoa riittää, voi opetella käyttämään piirisimulaattoria. Sillä on helppo mm. tarkistaa kotitehtävät:
\url{http://www.linear.com/designtools/software/ltspice.jsp}
\end{itemize}

}

\frame{
\frametitle{Esimerkki 3}
% Silvonen 162
Ratkaise $U_4$.
\begin{center}
\begin{picture}(150,100)(0,0)
\vst{0,0}{E}
\vz{50,0}{R_2}
\vz{100,0}{R_4}
\hz{0,50}{R_1}
\hz{50,50}{R_3}
\hln{0,0}{100}
\du{110,0}{U_4}
\pause
\color{red}
\du{58,0}{U_2}
\hgp{25,0}
\end{picture}
\end{center}
\begin{eqnarray*}
(E-U_2)G_1&=&U_2G_2+(U_2-U_4)G_3\\
(U_2-U_4)G_3&=&G_4U_4
\end{eqnarray*}
}

\begin{comment}
\frame{
\frametitle{Esimerkki 4}
% Silvonen 162
Ratkaise $U_4$.
\begin{center}
\begin{picture}(150,100)(0,0)
\vst{0,0}{E_1}
\vz{50,0}{R_2}
\vz{100,0}{R_4}
\hz{0,50}{R_1}
\hz{50,50}{R_3}
\hln{0,0}{50}
\hst{50,0}{E_2}
\hj{50,100}{J}
\vln{50,50}{50}
\vln{100,50}{50}
\cn{50,50}

\du{110,0}{U_4}

\pause
%\color{red}
%\du{58,0}{U_2}
%\hgp{25,0}
\end{picture}
\end{center}
\begin{eqnarray*}
(E-U_2)G_1&=&U_2G_2+(U_2-U_4)G_3\\
(U_2-U_4)G_3&=&G_4U_4
\end{eqnarray*}
}
\end{comment}

\frame{
\begin{block}{Esimerkki}
Ratkaise jännite $U_1$. Kaikki vastukset ovat $10 \ohm$ vastuksia, $E=10\V$ ja $J=1\A$.
\end{block}

\begin{center}
\begin{picture}(150,50)(0,0)
\vj{0,0}{J}
\vz{50,0}{R_1}
\vz{100,0}{R_3}
\hz{50,50}{R_2}
\hz{50,0}{R_4}
\vst{150,0}{E}
\hln{0,0}{50}
\hln{100,0}{50}
\hln{0,50}{50}
\hln{100,50}{50}
\du{57,0}{U_1}
\end{picture}
\end{center}


}




\frame{
\begin{block}{Esimerkki}
Ratkaise jännite $U_1$. Kaikki vastukset ovat $10 \ohm$ vastuksia, $E=10\V$ ja $J=1\A$.
\end{block}

\begin{center}
\begin{picture}(150,50)(0,0)
\vj{0,0}{J}
\vz{50,0}{R_1}
\vz{100,0}{R_3}
\hz{50,50}{R_2}
\hz{50,0}{R_4}
\vst{150,0}{E}
\hln{0,0}{50}
\hln{100,0}{50}
\hln{0,50}{50}
\hln{100,50}{50}
\du{57,0}{U_1}

\color{red}
\put(97,47){\vector(-1,-1){45}}
\stx{89,31}{$U_2$}
\lu{50,-19}{}
\txt{75,-25}{U_3}
\hgp{25,0}
\ri{125,50}{I}

\color{blue}
\ru{50,60}{U_1-U_2}
\du{110,0}{}
\stx{125,15}{$U_2-U_3$}
\end{picture}
\end{center}

\begin{eqnarray*}
J&=&U_1G_1+(U_1-U_2)G_2\\
(U_1-U_2)G_2&=&(U_2-U_3)G_3+I\\
G_3(U_2-U_3)+I&=&U_3G_4\\
U_2-U_3&=&E
\end{eqnarray*}

}


\frame{
\frametitle{Ratkaisu jatkuu}
\begin{eqnarray*}
J&=&U_1G_1+(U_1-U_2)G_2\\
(U_1-U_2)G_2&=&EG_3+I\\
G_3E+I&=&U_3G_4\\
U_2-U_3&=&E
\end{eqnarray*}
Ratkaistaan kolmannesta yhtälöstä $I$ ja sijoitetaan se toiseen yhtälöön. Ratkaistaan viimeisestä yhtälöstä $U_3$
ja sijoitetaan se paikalleen.
\begin{eqnarray*}
J&=&U_1G_1+(U_1-U_2)G_2\\
(U_1-U_2)G_2&=&EG_3+(U_2-E)G_4-G_3E\\
\end{eqnarray*}
\vspace{-1cm}
\begin{eqnarray*}
1&=&0,2U_1-0,1U_2\\
0,1U_1-0,1U_2&=&0,1U_2-1
\end{eqnarray*}
}
\frame{
\frametitle{Ratkaisu jatkuu}
\begin{eqnarray*}
1&=&0,2U_1-0,1U_2\\
0,1U_1-0,1U_2&=&0,1U_2-1
\end{eqnarray*}

Jonka ratkaisu on
\begin{eqnarray*}
U_1&=&10\\
U_2&=&10
\end{eqnarray*}
Eli kysytty jännite $U_1$ on 10 volttia. Tämän voi vielä tarkistaa simulaattorilla.

}


\frame{
\frametitle{Esimerkki} % Piirianalyysi laskari 2 tehtävä 2
\[
R_1=100\ohm\quad R_2=500\ohm\quad R_3=1,5\kohm\quad R_4=1\kohm \quad E_1=5\V
\]
\[
\quad J_1=100\mA\quad J_2=150\mA
\]
\begin{center}
\begin{picture}(150,100)(0,0)
\vj{0,0}{J_1}
\vz{50,0}{R_1}
\vst{100,0}{E_1}
\vz{150,0}{R_4}
\hz{50,50}{R_2}
\hz{100,50}{R_3}
\hj{100,100}{J_2}
\hln{0,0}{150}
\hln{0,50}{50}
\vln{100,50}{50}
\cn{100,50}
\vln{150,50}{50}
\du{160,0}{U_4}
\end{picture}
\end{center}
Ratkaise $U_4$.\\
\tiny $U_4=92\V$
}

\frame{
\frametitle{Esimerkki} % http://users.tkk.fi/ksilvone/Lisamateriaali/teht100.pdf t. 114
\[
R_1=12\ohm\quad R_2=25\ohm\quad J=1 \A \quad E_1=1\V\quad
E_2=27\V
\]
\begin{center}
\begin{picture}(150,100)(0,0)
\vst{0,0}{E_1}
\vj{50,0}{J}
\hst{75,50}{E_2}
\hz{0,50}{R_1}
\vz{125,0}{R_2\hspace{-1.3cm}}
\hln{0,0}{125}
\hln{50,50}{25}
\du{70,0}{U}
\end{picture}
\end{center}
Laske jännite $U$.\\
\tiny
Vastaus: $\frac{1}{37}\V \approx 27 \mV$
}

\frame{
\begin{block}{Esimerkki}
Ratkaise jännite $U$.
\end{block}
\[
R_1=1\ohm \quad R_2=2\ohm\quad J_1=1\A \quad J_2=2\A\quad E=3\V
\]
\begin{center}
\begin{picture}(100,100)(0,0)
\vj{0,0}{J_1}
\vst{100,0}{E}
\vz{50,0}{R_1}
\hz{50,50}{R_2}
\hlj{50,100}{J_2}
\hln{0,0}{100}
\hln{0,50}{50}
\vln{50,50}{50}
\vln{100,50}{50}
\cn{50,50}
\cn{100,50}
\du{58,0}{U}
\end{picture}
\end{center}


}




\frame{
\frametitle{Esimerkki}
%\begin{block}{Kotitehtävä 4}
%Ratkaise jännite $U$.
%\end{block}
\[
R_1=1\ohm \quad R_2=2\ohm\quad J_1=1\A \quad J_2=2\A\quad E=3\V
\]
\begin{center}
\begin{picture}(100,100)(0,0)
\vj{0,0}{J_1}
\vst{100,0}{E}
\vz{50,0}{R_1}
\hz{50,50}{R_2}
\hlj{50,100}{J_2}
\hln{0,0}{100}
\hln{0,50}{50}
\vln{50,50}{50}
\vln{100,50}{50}
\cn{50,50}
\cn{100,50}
\du{58,0}{U}
\color{blue}
\lu{50,58}{E-U}
\end{picture}
\end{center}
\[
J_1+J_2+G_2(E-U)=G_1U
\]
\[
1+2+0,5(3-U)=1\cdot U
\]
\[
4,5=1,5U
\]
\[
U=3
\]


}

