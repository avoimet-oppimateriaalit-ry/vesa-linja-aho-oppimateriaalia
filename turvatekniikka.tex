% Tammen kirjasta Turvalaitteet 556 ja alustan sähköjärjestelmät luku 16

\frame{
\frametitle{Turvatekniikka}
\begin{itemize}
\item Turvallisuus on hyvä myyntiargumentti, olipa kyseessä sanomalehtiotsikko tai kulkuväline.
\item Turvatekniikka on kehittynyt reilusti eteenpäin viimeisen 20 vuoden aikana.
\item Moni asia on keksintönä vanha: esimerkiksi turvatyyny on 1950-luvun keksintö. Joihinkin autoihin se tuli 1970-luvulla. Vakiovarusteena se
yleistyi vasta 1990-luvulla.
\item Aktiiviset turvalaitteet pyrkivät estämään onnettomuuden. Passiiviset turvalaitteet pyrkivät minimoimaan onnettomuuden vaikutukset.
\end{itemize}
}

\frame{
\frametitle{Turvatyyny}
\begin{itemize}
\item Idea on yksinkertainen: kaasulla täyttyvä tyyny ottaa vastaan kuljettajan liike-energian.
\item Useita eri nimiä: Airbag, SRS (Supplement Restraint System), SIR (Supplemental Inflatable Restraint). 1970-luvulla GM:n autoissa
kauppanimellä Air Cushion Restraint System (ACRS).
\end{itemize}
}

\frame{
\frametitle{Turvatyyny}
\begin{itemize}
\item Laukaisu voidaan tehdä mekaanisesti tai elektronisesti tai molempien yhdistelmänä.
\item Täyttö tapahtuu räjähdyspanoksen avulla. Räjähdyksessä vapautuva kaasu (yleensä typpeä) täyttää tyynyn.
\item Joissain järjestelmissä räjähdyspanos ainoastaan rikkoo korkeapainesäiliön, josta vapautuva kaasu täyttää turvatyynyn.
\item Tyynyssä olevat reiät huolehtivat siitä, että tyyny tyhjenee välittömästi onnettomuuden jälkeen.
\end{itemize}
}

\frame{
\frametitle{Turvatyynyn turvallisuus}
\begin{itemize}
\item Turvatyynyn laukeaminen turhaan voi aiheuttaa vakavan vaaratilanteen.
\item Nykyjärjestelmissä usean kiihtyvyysanturin tietoja analysoidaan ja turvatyynyt laukaistaan vain, jos on pakko.
\item Vaiheittainen täyttö (2 eri räjähdyspanosta), kuljettajan istuimen sijainti ja tieto turvavyön käytöstä vaikuttavat laukaisuun.
\item Etumatkustajan turvatyynyn tulee olla poiskytkettävissä lastenistuimen asentamista varten.
\item Vanhoissa järjestelmissä turvatyynyn laukaisulaite sijaitsi itse tyynyssä. Nykyjärjestelmissä erillinen ohjain laukaisee tyynyn/tyynyt.
\end{itemize}
% http://en.wikipedia.org/w/index.php?title=Airbag&oldid=386324286
Vuosina 1990--2008, Yhdysvalloissa tapahtui 175 turvatyynyn {\em aiheuttamaa} kuolemaa. Suuri osa (104) oli lapsia. Kyseisenä ajanjaksona
turvatyynylaukeamisia tapahtui 3,3 miljoonaa ja turvatyynyt pelastivat arviolta 6400 ihmisen hengen ja estivät lukuisia loukkaantumisia.
}

\frame{
\frametitle{Anturit}
\begin{itemize}
\item Perinteinen laukaisuanturi perustuu massajousiviritelmään.
\item Nykyisin käytetään useisiin paikkoihin sijoitettuja mikromekaanisia kiihtyvyysantureita.
\end{itemize}
}

\frame{
\frametitle{Turvavyön esikiristimet}
\begin{itemize}
\item Vammautumisriskiä pienentää, jos turvavöistä otetaan "löysät pois"\ törmäystilanteessa.
\item Esikiristyslaitteet toimivat räjähdyspanoksen avulla. Toteutustavat vaihtelevat valmistajittain.
\item Pienissä törmäyksissä laukaistaan pelkät esikiristimet, ei turvatyynyjä.
\end{itemize}
}

\frame{
\frametitle{Rengaspaineen valvonta}
\begin{itemize}
\item Renkaan tyhjeneminen kesken ajon voi aiheuttaa vakavan vaaratilanteen.
\item Kokematon kuljettaja ei välttämättä huomaa renkaan tyhjenemistä. Osa renkaista on suunniteltu sellaisiksi, että niillä pystyy ajamaan myös
niiden tyhjennyttyä. Tällaisesta renkaasta on paineen alenemista vaikea havaita.
\item Elektroninen rengaspaineen valvonta perustuu rengasventtiiliin integroituun anturiin, joka lähettää radioimpulssina painetiedon muutaman sekunnin välein.
\item Kuljettajaa varoitetaan merkkivalolla tai teksti-ilmoituksella tyhjentyneestä renkaasta.
\end{itemize}
}

\frame{
\frametitle{Ajonvakautusjärjestelmä (ESP)}
\begin{itemize}
\item ESP = Electronic Stability Program
\item Perusajatus: tietokone seuraa anturitietoja ja vertaa niitä kuljettajan syötteeseen.
\item Jos auton käyttäytyminen aikoo poiketa kuljettajan ohjaustoimenpiteistä, aloitetaan korjaustoimenpiteet:
\begin{itemize}
\item Jarrutetaan yksittäisiä pyöriä.
\item Kiihdytetään vetäviä pyöriä.
\end{itemize}
ESP tunnistaa ongelmatilanteen yleensä ennen kuljettajaa.
\end{itemize}
}

\frame{
\frametitle{Ajonvakautusjärjestelmä (ESP)}
ESP koostuu useista alijärjestelmistä (joista moni on ja on ollut käytössä yksinäänkin)
\begin{itemize}
\item ASR = Vetoluiston esto (Anti-Slip Regulation, myös: TCS eli Traction Control System)
\item \begin{itemize}
\item ABD = luiston tasaus jarrulla (Automatic Brake Differential) % Lisää wikipediaan
\item MSR = moottorijarrutusmomentin säätö (Motor Slip Regulation) % Lisää wikipediaan
\end{itemize}
\item ABS = lukkiutumattomat jarrut (Anti-lock Braking System)
\begin{itemize}
\item EBV = Elektroninen jarruvoiman jako (Elektronischer Bremskraftverteiler) % Lisää wikipediaan
\end{itemize}
\item BAS = Jarruassistentti (Brake Assist System)
\end{itemize}
}

\frame{
\frametitle{ESP}
Tulodata
\begin{itemize}
\item Ohjauspyörän kääntökulma
\item Momenttipyyntö (kaasupolkimen asento)
\item Esipaine (jarrutustoivomus)
\item Ajoneuvon nopeus
\item Kiihtyvyysanturitiedot
\item Kitkakerroin (lasketaan kiihtyvyys-, nopeus- ja jarrupainetiedoista)
\end{itemize}
Toimenpiteet:
\begin{itemize}
\item Jarrutetaan yksittäisiä pyöriä.
\item Kiihdytetään vetäviä pyöriä.
\end{itemize}
Auton käyttäytymistä mallinnetaan differentiaaliyhtälöillä ja auton kulkutila pyritään säätämään kuljettajan ohjaustoiveen mukaiseksi.
}

\frame{
\frametitle{Lukkiutumattomat jarrut (ABS)}
\begin{itemize}
\item Pitämällä pyörien luisto erittäin pienenä, saadaan minimoitua jarrutusmatka sekä ennen kaikkea säilytettyä ohjattavuus.
\item Tietokone laskee 1000 kertaa sekunnissa pyörien pyörimisnopeuden ja säätää jarruvoimaa magneettiventtiilin ja pumpun avulla.
\end{itemize}
}


\frame{
\frametitle{Elektroninen jarruvoiman jako (EBV)}
\begin{itemize}
\item Elektroninen järjestelmä korvaa 2000-luvun autoissa mekaanisen taka-akselin jarrutusvoiman säätimen.
\item Jarruvoima jaetaan pyörimisnopeustunnistimien tietojen perusteella. Kaarrejarrutuksessa säätö optimoidaan mahdollisimman suuren
sivuttaispidon aikaansaamiseksi.
\end{itemize}
}

\frame{
\frametitle{Luistonesto (ASR)}
\begin{itemize}
\item ASR-luistonesto pyrkii estämään pyörien sutimisen kiihdytettäessä.
\item Auton nopeus tunnistetaan ei-vetävien pyörien nopeudesta.
\item Moottorin momenttia alennetaan, kunnes pito palaa.
\item Momentin nopea säätö on mahdollista nykyaikaisissa suorasuihkutusjärjestelmissä.
\item ASR on yleensä kytkettävissä pois päältä.
\end{itemize}
}

\frame{
\frametitle{Jarruassistentti (BAS)}
\begin{itemize}
\item Usein hätäjarrutuksessa poljinta painetaan riittävän nopeasti, mutta ei riittävän kovaa.
\item BAS-järjestelmässä on erillinen painevaraaja. Jarrupolkimen nopea liike tunnistetaan ja magneettiventtiili aukeaa tuoden
painevaraajasta välittömästi riittävän paineen jarrupiiriin.
\end{itemize}
}

\frame{
\frametitle{Luistontasaus jarrulla (ABD)}
\begin{itemize}
\item Jarruttamalla luistamaan pyrkivää pyörää, siirtyy tasauspyörästön ansiosta suurempi momentti pitävälle pyörälle.
\item Säätöä ei voida tehdä jatkuvasti, etteivät jarrut kuumene.
\end{itemize}
}

\frame{
\frametitle{Moottorijarrutusmomentin säätö (MSR)}
\begin{itemize}
\item Liukkaalla kelillä äkillinen moottorijarrutus voi johtaa auton epästabiiliin käyttäytymiseen.
\item MSR-järjestelmä viestittää luisutiedon moottorinohjausjärjestelmälle, joka lisää hieman moottorin vääntömomenttia.
\end{itemize}
}


\frame{
\frametitle{Lainsäädäntö patistaa turvatekniikan käyttöön}
\begin{itemize}
\item Marraskuun 2011 jälkeen kaikki uudet tyyppihyväksytyt henkilöautot ja pikkupakettiautot varustettava ESP:lla.
\item Marraskuun 2014 jälkeen kaikki uudet ajoneuvot varustettava ESP:lla.
\end{itemize}
}
