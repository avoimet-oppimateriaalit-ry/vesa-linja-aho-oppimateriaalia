% KAIPAA JOHDANTOKALVOA
\frame{
\frametitle{Solmumenetelmä}
Tarkastellaan edellisen esimerkin yhtälöitä
\[
G_1(E_1-U_3)=G_2(U_3-U_4)+G_3U_3\ \mbox{ja}\
G_2(U_3-U_4)=G_4U_4+G_5(U_4-E_2)
\]
Kerrotaan sulut auki
\begin{eqnarray*}
G_1E_1-G_1U_3 &=& G_2U_3-G_2U_4+G_3U_3\\
G_2U_3-G_2U_4 &=& G_4U_4+G_5U_4-G_5E_2
\end{eqnarray*}
Ja järjestellään termejä ja siirretään vakiotermit toiselle puolelle
\begin{eqnarray*}
G_1U_3+G_2U_3-G_2U_4+G_3U_3&=& G_1E_1\\
-G_2U_3+G_2U_4+G_4U_4+G_5U_4&=& G_5E_2
\end{eqnarray*}
Ja otetaan jännitteet yhteisiksi tekijöiksi
\begin{eqnarray*}
(G_1+G_2+G_3)U_3-G_2U_4&=& G_1E_1\\
-G_2U_3+(G_2+G_4+G_5)U_4&=& G_5E_2
\end{eqnarray*}
}
\frame{
\frametitle{Solmumenetelmä jatkuu}
\begin{eqnarray*}
(G_1+G_2+G_3)U_3-G_2U_4&=& G_1E_1\\
-G_2U_3+(G_2+G_4+G_5)U_4&=& G_5E_2
\end{eqnarray*}
Yhtälöiden logiikka on seuraava:
\begin{itemize}
\item Jokaiselle solmulle (=tuntemattomalle jännitteelle) on yksi yhtälö.
\item Kyseiseen solmuun liittyvien konduktanssien summa on kunkin
solmujännitteen kertoimena.
\item Yhtälön vasemmalla puolella lähtevät virrat ovat positiivisia, oikealla puolella
saapuvat virrat ovat positiivisia.
\end{itemize}
}


\frame{
\frametitle{Huomattavaa}
\begin{itemize}
\item Solmumenetelmä ja solmujännitemenetelmä ovat kaksi eri menetelmää (vaikkakin
hyvin samankaltaisia).
\item Jos yhtälöiden muodostamislogiikassa on vähänkin epäselvää,
ratkaise piiri suoraan Kirchhoffin laeilla (älä yritä oikaista).
\item Virtapiirin ratkaisemiseksi on useita muitakin menetelmiä kuin solmujännitemenetelmä: haaravirtamenetelmä,
silmukkamenetelmä, solmumenetelmä, modifioitu solmupistemenetelmä\ldots
\item Mikäli piirissä on ideaalisia jännitelähteitä (=jännitelähteitä, jotka liittyvät suoraan
solmuun ilman että välissä on vastus), yhtälöihin tulee yksi tuntematon arvo lisää (=jännitelähteen
virta) sekä yksi yhtälö lisää (jännitelähde määrää solmujen jännite-eron).
\end{itemize}
}