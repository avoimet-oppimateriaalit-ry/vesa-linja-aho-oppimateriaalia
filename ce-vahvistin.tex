 \frame{
 \frametitle{Transistorivahvistin (yhteisemitterikytkentä)}
\begin{center}
\begin{picture}(180,160)(-30,-50)

\vz{0,0}{R_2=33\kohm\vspace{-1cm}}
\vz{0,50}{R_1=82\kohm}

\vst{200,0}{12\V}
\vln{200,50}{75}
\vz{100,75}{R_{\rm C}=1\kohm}
\hln{100,125}{100}

\hln{0,125}{100}
\vln{0,100}{25}
\hln{0,50}{50}
\npn{50,50}{}
\hln{0,-50}{200}
\vln{0,-50}{50}
\vln{200,-50}{50}
\vln{100,0}{25}
\vz{100,-50}{R_{\rm E}=470\ohm}

\hc{-50,50}{C_{\rm IN}}
\hc{100,75}{C_{\rm OUT}}
\cn{0,50}

\hln{-100,50}{50}
\du{-100,00}{\Uin}
\hg{-100,0}

\du{150,25}{\Uout}
\hg{150,25}
\hgp{0,-50}

\end{picture}
\end{center}
 }


 \frame{
 \frametitle{Transistorivahvistin}
 \begin{itemize}
 \item Edellisen kalvon piiri vahvistaa vaihtojännitesignaalia $\Uin$.
\item Miksi signaalia ei voi vain syöttää suoraan transistorin kannalle?
\item Mihin piirissä tarvitaan kondensaattoreita?
\item Kuinka suuri on piirin jännitevahvistus $\frac{\Uout}{\Uin}$?
 \end{itemize}
 }

 \frame{
 \frametitle{Transistorivahvistin}
 \begin{itemize}
\item Jotta transistorivahvistin vahvistaisi myös heikkoja signaaleja, se pitää {\em biasoida} eli {\em esijännittää}. Myös termiä {\em tasajänniteasettelu} käytetään.
\item Kondensaattorit estävät esijännitykseen käytettävää tasasähköä vuotamasta vahvistimen tuloon ja lähtöön.
\item Kuvan vahvistinkytkentää kutsutaan yhteisemitteri- eli CE-vahvistimeksi (engl. common emitter).
\item Analysoidaan piirin toimintaa kerrostamismenetelmällä. Mallinnetaan transistoria virtaohjatulla virtalähteellä.
\item Kaikki muut jännitelähteet (kanta-emitteridiodin jännite ja piirin käyttöjännite) asetetaan nollaksi.
\item Oletetaan kondensaattorit oikosuluksi (= signaalin taajuus on niin suuri, että kondensaattorien impedanssi on pieni).
 \end{itemize}
 }


 \frame{
 \frametitle{Transistorivahvistimen analysointi}
\begin{center}
\begin{picture}(180,160)(-30,-50)

\vz{0,0}{R_2=33\kohm\vspace{-1cm}}
\vz{0,50}{R_1=82\kohm}

\vln{200,0}{50}
\vln{200,50}{75}
\vz{100,75}{R_{\rm C}=1\kohm}
\hln{100,125}{100}

\hln{0,125}{100}
\vln{0,100}{25}
\hln{0,50}{50}
\vdj{100,25}{\beta I_{\rm B}}
\hln{50,25}{50}
\vln{50,25}{25}
\ri{75,25}{I_{\rm B}}

\hln{0,-50}{200}
\vln{0,-50}{50}
\vln{200,-50}{50}
\vln{100,0}{25}
\vz{100,-50}{R_{\rm E}=470\ohm}
\hln{-50,50}{50}
\hln{100,75}{50}

\ri{-25,50}{I_{\rm in}}

\cn{0,50}

\hln{-100,50}{50}
\du{-100,00}{\Uin}
\hg{-100,0}

\du{150,25}{\Uout}
\hg{150,25}
\hgp{0,-50}

\end{picture}
\end{center}
 }


 \frame{
 \frametitle{Vahvistuskerroin}
Emitterivirralle voidaan kirjoittaa
\[
\frac{\Uin}{R_{\rm E}}=I_{\rm B}+\beta I_{\rm B} \quad \Rightarrow \quad I_{\rm B}=\frac{\Uin}{R_{\rm E}(1+\beta)}
\]
josta
\[
\Uout = -\beta I_{\rm B}\cdot R_{\rm C}= -\beta \frac{\Uin}{R_{\rm E}(1+\beta)} R_{\rm C}=-\Uin \frac{\beta}{1+\beta} \frac{R_{\rm C}}{R_{\rm E}}
\]
ja
\[
\frac{\Uout}{\Uin}=-\frac{\beta}{1+\beta} \frac{R_{\rm C}}{R_{\rm E}}\approx -\frac{R_{\rm C}}{R_{\rm E}},
\]
koska $\frac{\beta}{1+\beta}\approx 1$.
 }

 \frame{
 \frametitle{Tuloresistanssi}
\[
R_{\rm in}=\frac{\Uin}{I_{\rm in}}
\]
\[
I_{\rm in}=\frac{\Uin}{R_1}+\frac{\Uin}{R_2}+I_{\rm B}=\frac{\Uin}{R_1}+\frac{\Uin}{R_2} +\frac{\Uin}{R_{\rm E}(1+\beta)}
\]
\[
R_{\rm in}=\frac{\Uin}{I_{\rm in}}=\frac{\Uin}{\frac{\Uin}{R_1}+\frac{\Uin}{R_2} +\frac{\Uin}{R_{\rm E}(1+\beta)}}=\frac{1}{\frac{1}{R_1}+\frac{1}{R_2} +\frac{1}{R_{\rm E}(1+\beta)}}
\]

Eli tuloresistanssi on resistanssien $R_1$, $R_2$ ja $R_{\rm E}(1+\beta)$ rinnankytkentä. Voidaan ajatella, että emitterillä oleva resistanssi näkyy kannalla $(1+\beta)$-kertaisena.

 }

 \frame{
 \frametitle{Lähtöresistanssi}
Lähtöresistanssi on helppo johtaa tekemällä kollektorivastukselle ja virtalähteelle lähdemuunnos:
\begin{center}
\begin{picture}(180,160)(100,0)

\vz{100,75}{R_{\rm C}}

\vdj{100,25}{\beta I_{\rm B}}
\hgp{50,125}

\hln{50,125}{50}

\uu{115,75}{\Uout\hspace{-1cm}}
\hgp{100,25}


\vdst{225,50}{\beta I_{\rm B}R_{\rm C}}
\hg{225,50}
\hz{225,100}{R_{\rm C}}
\hg{275,50}
\du{275,50}{\Uout}

\end{picture}
\end{center}
Lähdemuunnoksessa resistanssin arvo säilyy samana, eli vahvistimen lähtöresistanssi on suoraan kollektorivastus $R_{\rm C}$.

 }


 \frame{
 \frametitle{Esimerkki}
\begin{center}
\begin{picture}(180,160)(-30,-50)

\vz{0,0}{R_2=33\kohm\vspace{-1cm}}
\vz{0,50}{R_1=82\kohm}

\vst{200,0}{12\V}
\vln{200,50}{75}
\vz{100,75}{R_{\rm C}=1\kohm}
\hln{100,125}{100}

\hln{0,125}{100}
\vln{0,100}{25}
\hln{0,50}{50}
\npn{50,50}{}
\hln{0,-50}{200}
\vln{0,-50}{50}
\vln{200,-50}{50}
\vln{100,0}{25}
\vz{100,-50}{R_{\rm E}=500\ohm}
\txt{70,25}{\beta=100}

\hc{-50,50}{C_{\rm IN}}
\hc{100,75}{C_{\rm OUT}}
\cn{0,50}

\hln{-100,50}{50}
\du{-100,00}{\Uin}
\hg{-100,0}

\du{150,25}{\Uout}
\hg{150,25}
\hgp{0,-50}

\end{picture}
\end{center}
Kuinka suureksi $C_{\rm IN}$ tulee vähintään valita, jotta yli 20 $\Hz$ taajuiset signaalit eivät vaimene enempää kuin 3 desibeliä piirin ideaalivahvistukseen ($\approx 2$) verrattuna?
 }

\frame{
 \frametitle{Esimerkki}
$C_{\rm IN}$ ja vahvistimen tuloresistanssi $R_{\rm IN}$ muodostavat yhdessä ensimmäisen asteen ylipäästösuodattimen, joka määrää vahvistimen alarajataajuuden. Ensimmäisen asteen suodattimilla suodattimen ominaistaajuus on sama kuin -3 desibelin piste, eli ongelma ratkeaa suoraan laskemalla $C_{\rm IN}$ muodostuneen ylipäästösuodattimen ominaistaajuuden kaavasta ($f_0=20\Hz$):
\[
f_0=\frac{\omega_0}{2\pi}=\frac{\frac{1}{RC}}{2\pi}=\frac{1}{2\pi R C}=\frac{1}{2\pi R_{\rm IN} C_{\rm IN}}\Rightarrow  C_{\rm IN}=\frac{1}{2\pi R_{\rm IN} f_0}
\]
\[
C_{\rm IN}=\frac{1}{2\pi \frac{1}{\frac{1}{R_1}+\frac{1}{R_2} +\frac{1}{R_{\rm E}(1+\beta)}} f_0}\approx 500 \nF
\]

Huomaa, että myös $C_{\rm OUT}$ vaikuttaa alarajataajuuteen, mutta tätä ei käsitelty esimerkissä.


}


