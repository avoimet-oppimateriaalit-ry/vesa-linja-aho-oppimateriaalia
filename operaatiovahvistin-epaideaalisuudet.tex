\frame{
\frametitle{Operaatiovahvistimen epäideaalisuudet}
Ideaalinen operaatiovahvistin
\begin{itemize}
\item on äärettömän nopea.
\item $A \to \infty$
\item Tulonapoihin ei mene virtaa
\item  Lähtövirta ja -jännite voi olla kuinka suuri tahansa
\end{itemize}
Negatiivisesti takaisinkytkettynä ideaalisen operaatiovahvistimen tulonavoissa on täsmälleen sama jännite.
}

\frame{
\frametitle{Käytännön operaatiovahvistin}
\begin{itemize}
\item On hidas
\item Vahvistuskerroin $A$ on suuri muttei ääretön.
\item Tulonapoihin menee pieni virta.
\item Lähtövirta on rajallinen, kuten myös lähtöjännite.
\item Tulonavat eivät ole samassa potentiaalissa, vaan niiden välillä on pieni jännite-ero.
\end{itemize}
}

\frame{
\frametitle{Tulonsiirrosjännite}
\begin{itemize}
\item Negatiivisesti takaisinkytketyn operaatiovahvistimen tulonapojen välillä on pieni jännite, jota kutsutaan
offset-jännitteeksi eli tulonsiirrosjännitteeksi.
\item Suuruusluokka on muutamasta millivoltista satoihin mikrovoltteihin.
\item Joissain operaatiovahvistimissa on liitännät potentiometrille, jolla voidaan nollata tulonsiirrosjännite.
\item Joissain malleissa on automaattinen offset-jännitteen nollaus.
\item $U_{\rm offset}=U_+-U_-$.
\end{itemize}
}

\frame{
\frametitle{Tuloesivirta ja tulonsiirrosvirta}
\begin{itemize}
\item Molempiin tulonapoihin menee pieni virta. Virrat eivät ole samansuuruisia ($I_+ \not = I_-$).
\item Tuloesivirta = $\frac{I_+ -  I_-}{2}$
\item Tulonsiirrosvirta = $I_+ -  I_-$
\item Virrat ovat yleensä nanoampeeriluokkaa.
\end{itemize}
}

\frame{
\frametitle{Toteutustavat}
\begin{itemize}
\item Bipolaaritransistorioperaatiovahvistimilla offset-jännite on yleensä pieni ja tulovirrat suurempia kuin kanavatransistorioperaatiovahvistimilla, joilla puolestaan offset-jännite voi olla suurehko.
\end{itemize}
}

\frame{
\frametitle{Äärellinen nopeus}
\begin{itemize}
\item Operaatiovahvistimen lähtöjännite ei voi muuttua äärettömän nopeasti.
\item Slew rate eli seurantanopeus kertoo, kuinka nopeasti operaatiovahvistimen lähtöjännite voi muuttua.
\item Suuruusluokka on voltteja tai kymmeniä voltteja per mikrosekunti.
\end{itemize}
}


\frame{
\frametitle{Äärellinen lähtövirta ja lähtöjännite}
\begin{itemize}
\item Lähtöjännite ei voi mennä ylitse (eikä alitse) käyttöjännitteiden.
\item Kaikilla operaatiovahvistimilla lähtöjännite ei yllä edes käyttöjännitteisiin.
\item Jos lähtöjännite yltää molempiin käyttöjännitteisiin, puhutaan rail-to-rail -operaatiovahvistimesta.
\item Lähtövirta on myös äärellinen.
\end{itemize}
}

\frame{
\frametitle{CMRR}
\begin{itemize}
\item Common Mode Rejection Ratio eli yhteismuodon vaimennussuhde.
\item Ideaalinen operaatiovahvistin vahvistaa vain jännite-eroa, ei yhteismuotoista tulojännitettä.
\item CMRR = $\frac{A_{\rm D}}{A_{\rm C}}$.
\item Ilmoitetaan yleensä desibeleinä, kuten myös käyttöjännitteen vaimennussuhde (PSRR), joka kertoo, kuinka
paljon käyttöjännitteen heilahtelu näkyy lähtöjännitteessä.
\end{itemize}
}
