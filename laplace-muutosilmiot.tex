

\frame{
\frametitle{Muutosilmiöt}
\begin{itemize}
\item Muutosilmiöiden käsittely on helppoa, jos piirissä on vain yksi kela tai kondensaattori.
\item Jos niitä on useampi, muodostuva differentiaaliyhtälö on erittäin hankala ratkaista (ilman syvällisempää
differentiaaliyhtälöiden osaamista).
\item Monimutkaisempi differentiaaliyhtälö on helppo ratkaista Laplace-muunnoksen avulla.
\item Laplace-muunnoksen avulla voidaan differentiaaliyhtälö muuntaa tavalliseksi yhtälöksi, josta selviää
tavallisella kaavanpyörittelyllä. Avuksi tarvitaan {\bf Laplace-muunnostaulukko} (löytyy Tuubissa olevasta kaavakokoelmasta).
\end{itemize}
}

\frame{
\frametitle{Laplace-muunnoksen idea}
Laplace-muunnos on niin kutsuttu integraalimuunnos. Se määritellään seuraavasti:
\[
\lap (f(t))=\int_0^\infty f(t)e^{-st}{\rm d}t
\]
Laplace-muunnosten laskeminen on työlästä. Siksi käytännön sovelluksissa käytetään valmista Laplace-muunnostaulukkoa.

Muuttuja $s$ on nimeltään Laplace-muuttuja. Laplace-muunnettua ajan funktiota $f(t)$ merkitään usein isolla kirjaimella $F(s)$.

}

\frame{
\frametitle{Mitä hyötyä Laplace-muunnoksesta on?}
\begin{itemize}
\item Laplace-muunnoksen avulla voidaan differentiaaliyhtälö muuntaa tavalliseksi (=algebralliseksi) yhtälöksi.
\item Sitten algebrallinen yhtälö ratkaistaan.
\item Ja lopuksi se Laplace-käänteismuunnetaan takaisin ajan funktioksi.
\item VERTAUS: Aivan kuten osoitinlaskennassa ensin sinimuotoinen jännitelähde muutetaan kompleksiluvuksi,
sitten lasketaan kompleksiluvuilla ja lopuksi tulos muunnetaan sinimuotoiseksi.
\end{itemize}
}

\frame{
\frametitle{Askelfunktio ja impulssifunktio}
Määritellään pari uutta funktiota:
\begin{itemize}
\item Askelfunktiolla tarkoitetaan sellaista (ajan $t$) funktiota $\epsilon(t)$, joka saa arvon 1, jos $t\ge 0$
ja 0, jos $t<0$. Tällä funktiolla voidaan esimerkiksi mallintaa kytkintä, joka suljetaan ajanhetkellä $t=0$.
\item Impulssifunktiolla tarkoitetaan funktiota\footnote{Pilkunviilaus: impulssifunktio ei tarkalleen ottaen ole funktio, koska
siltä puuttuu joitain funktiolle tyypillisiä ominaisuuksia.} $\delta(t)$, jonka arvo on $\infty$, kun $t=0$, ja 0, kun $t\neq 0$. Impulssifunktiolla
voidaan mallintaa äkillistä virta- tai jännitepiikkiä. Esimerkiksi jos kelan virta katkaistaan yhtäkkiä, niin kelan jännite hyppää (teoriassa)
äärettömäksi.
\end{itemize}
}

\frame{
\frametitle{Muutama tavallinen Laplace-muunnos}
Laplace-muunnos on lineaarinen, joten vakiokerroin voidaan siirtää Laplace-muunnoksen "läpi" (aivan kuten derivoinnissakin),
ja summalausekkeessa termit voidaan Laplace-muuntaa yksi kerrallaan (aivan kuten summalauseke voidaan derivoida termi kerrallaan).
Alla pari tärkeää muunnosta:
\begin{itemize}
\item $\lap \{A\epsilon(t)\}=\frac{A}{s}$
\item $\lap \{\delta(t)\}=1$
\item $\lap \{t\}=\frac{1}{s^2}$
\item $\lap \{{\rm e}^{-at}\}=\frac{1}{s+a}$

\end{itemize}
Laplace-muunnos on määritelty vain ajanhetkillä $t\ge 0$. Tämä tarkoittaa, että jokainen muunnettava funktio voidaan ajatella kerrotuksi
askelfunktiolla. Siis: $\lap \{\epsilon(t)\}=\lap \{1\}=\frac{1}{s}$ ja vakiolle $\lap \{A\}=\lap \{A\epsilon(t)\}=\frac{A}{s}$.
}

\frame{
\frametitle{Komponenttien Laplace-muunnokset}
Virtapiirejä laskettaessa ei tarvitse ensin kirjoittaa differentiaaliyhtälöä ja sitten muuntaa sitä, vaan komponentit
voidaan muuntaa suoraan. Kela ja kondensaattori Laplace-muunnettuna ovat:
\begin{center}
\begin{picture}(100,70)(0,0)
\hl{0,0}{sL}
\hc{0,50}{\frac{1}{sC}}

\end{picture}
\end{center}
Kaavat on helppo muistaa, ne muistuttavat impedanssin kaavoja, nyt vain $\jj \omega$:n tilalla on Laplace-muuttuja $s$.
Tämä ei ole sattumaa --- aiheesta lisää ensi tunnilla! 

}

\frame{
\frametitle{Komponenttien Laplace-muunnokset}
Mikäli kelassa kulkee alkuvirta (virta hetkellä $t=0$) tai kondensaattorissa on alkujännite (jännite hetkellä $t=0$), muunnos
tapahtuu seuraavasti:
\begin{center}
\begin{picture}(100,70)(0,0)
\hl{0,0}{L}
\ri{50,0}{I_{\rm L0}}

\hc{0,50}{C}
\ru{0,65}{U_{\rm C0}}

\hl{100,0}{sL}
\hst{150,0}{LI_{\rm L 0}}

\hc{100,50}{\frac{1}{sC}}
\hlst{150,50}{\frac{U_{\rm C 0}}{s}}


\end{picture}
\end{center}

}

\frame{
\frametitle{Lähteiden Laplace-muunnokset}
Tasajännitelähde ja tasavirtalähde muunnetaan yksinkertaisesti jakamalla tasavirta tai tasajännite Laplace-muuttujalla $s$.
\begin{center}
\begin{picture}(100,70)(0,-10)
\hst{0,0}{E}

\hj{0,50}{J}

\hst{100,0}{\frac{E}{s}}

\hj{100,50}{\frac{J}{s}}

\end{picture}
\end{center}
Mikäli lähde on ajasta riippuva, esimerkiksi sinimuotoinen, niin lähteen ajan funktio vain muunnetaan
Laplace-muunnostaulukon avulla.
  \begin{alertblock}{Tärkeää (ja kätevää)}
Laplace-muunnetulle piirille pätevät kaikki tutut laskusäännöt (Kirchhoff, Ohm, jännitteenjako \ldots)!
  \end{alertblock}


}

\frame{
\frametitle{Yksinkertainen esimerkki}
Ratkaistaan kelan jännite $u$ Laplace-muunnoksen avulla.
\[
E=12\V  \qquad L=2\,{\rm H}\qquad R=6\ohm %\qquad U_0=5\V
\]
\begin{center}
\begin{picture}(100,50)(0,0)
\vl{100,0}{L}
\vst{0,0}{E}
\hso{0,50}{t=0}
\hln{0,0}{100}
%\hln{0,50}{50}
\hz{50,50}{R}
\du{115,0}{u}
\ri{50,50}{i}

\end{picture}
\end{center}
Laplace-muunnetaan piiri (ks. seuraava kalvo).

}

\frame{
\frametitle{Yksinkertainen esimerkki - jatkuu}
Ratkaistaan kelan jännite $u$ Laplace-muunnoksen avulla.
\[
E=12\V  \qquad L=2\,{\rm H}\qquad R=6\ohm %\qquad U_0=5\V
\]
\begin{center}
\begin{picture}(100,50)(0,0)
\vl{100,0}{sL}
\vst{0,0}{\frac{E}{s}}
%\hso{0,50}{t=0}
\hln{0,0}{100}
\hln{0,50}{50}
\hz{50,50}{R}
\du{115,0}{U}
\ri{50,50}{I}

\end{picture}
\end{center}
Nyt kelan jännite ratkeaa jännitteenjakosäännöllä:
\[
U=\frac{E}{s}\frac{sL}{R+sL}={E}\frac{1}{\frac{R}{L}+s}
\]
Muunnetaan vastaus takaisin ajan funktioksi:
\[
\lap^{-1}\left\{{E}\frac{1}{\frac{R}{L}+s}\right\}=E{\rm e}^{-\frac{R}{L}t}
\]

}
\frame{
\frametitle{Osamurtokehitelmä}
\begin{itemize}
\item Edellinen esimerkki oli helppo, koska Laplace-muodossa oleva ratkaisu oli helppo pyöräyttää muotoon, johon löytyi suora
muunnoskaava Laplace-taulukosta.
\item Jos lauseke on hankalampi, se tulee pilkkoa murtolausekkeiden summaksi (=osamurtokehitelmä).
\item Osamurtokehitelmän laskeminen on helppoa, mutta siinä on myös helppo tehdä virheitä (monta välivaihetta).
\item Osamurtokehitelmän idea on, että siinä ikään kuin tehdään takaperin kahden tai useamman murtolausekkeen yhteenlasku.
\end{itemize}
}

\frame{
\frametitle{Osamurtokehitelmä}
Esimerkiksi lauseke $\frac{1}{s^2+sa}$ eli $\frac{1}{s(s+a)}$ voidaan lausua muodossa
\[
\frac{A}{s}+\frac{B}{s+a},
\]
missä $A$ ja $B$ ovat vakioita, joita emme vielä tiedä. Ne selvitetään seuraavasti:
\begin{eqnarray*}
\frac{A}{s}+\frac{B}{s+a}&=&\frac{1}{s(s+a)}\\
\frac{A(s+a)}{s(s+a)}+\frac{Bs}{s(s+a)}&=&\frac{1}{s(s+a)}\\
\frac{As+Aa+Bs}{s(s+a)}&=&\frac{1}{s(s+a)}\\
\end{eqnarray*}
Jatkuu seuraavalla kalvolla\ldots
}

\frame{
\frametitle{Osamurtokehitelmä}

Jotta osoittajat olisivat samat, tulee olla $A=-B$ (jotta $s$-termi häviää) ja  $A=\frac{1}{a}$ (jotta osoittajaan jää ykkönen).
Eli:
\[
\frac{\frac{1}{a}}{s}+\frac{-\frac{1}{a}}{s+a}=\frac{1}{s(s+a)}
\]
Nyt vasemmanpuoleiset termit voidaan muuntaa taulukkoa käyttämällä.

Lisää esimerkkejä löytyy googlaamalla "Partial Fraction Decomposition"\ tai vaikkapa Wikipediasta
\url{http://en.wikipedia.org/wiki/Partial_fraction}.

Vinkki: jotkut laskimet (esim. TI-89) osaavat laskea osamurtokehitelmän.

}

\frame{
\frametitle{Esimerkki: muutosilmiökalvojen esimerkki Laplace-muunnoksella}
Ratkaise kondensaattorin jännite $u$ ajan funktiona. Kytkin suljetaan ajanhetkellä $t=0$.
\[
R_1=R_2=1\ohm\quad C=1\,{\rm F}\quad E=10\V\quad U_0=0\V
\]
\begin{center}
\begin{picture}(100,50)(0,0)
\vc{100,0}{C}
\vst{-50,0}{E}
\hso{-50,50}{t=0}
%\hln{-50,50}{50}
\vz{50,0}{R_2}
\hln{-50,0}{150}
\hln{50,50}{50}
\hz{0,50}{R_1}
\du{115,0}{u}
\ri{75,50}{i}

\end{picture}
\end{center}


}

\frame{
\frametitle{Esimerkki}
Laplace-muunnetaan piiri
\[
R_1=R_2=1\ohm\quad C=1\,{\rm F}\quad E=10\V\quad U_0=0\V
\]
\begin{center}
\begin{picture}(100,50)(0,0)
\vc{100,0}{\frac{1}{sC}}
\vst{-50,0}{\frac{E}{s}}
%\hso{-50,50}{t=0}
\hln{-50,50}{50}
\vz{50,0}{R_2}
\hln{-50,0}{150}
\hln{50,50}{50}
\hz{0,50}{R_1}
\du{115,0}{U}
%\ri{75,50}{}

\end{picture}
\end{center}
$R_2$ ja $C$ ovat rinnakkain, ja niiden rinnankytkennän impedanssi on $Z_{R_2C}=\frac{\frac{R_2}{sC}}{R_2+\frac{1}{sC}}=\frac{R_2}{R_2sC+1}$.
Jännitteenjakosäännön mukaan jännite U on
\[
U=\frac{E}{s}\frac{Z_{R_2C}}{R_1+Z_{R_2C}}=\frac{E}{s}\frac{1}{\frac{R_1}{Z_{R_2C}}+1}= \frac{E}{s}\frac{1}{R_1\frac{R_2sC+1}{R_2}+1}
\] 


}

\frame{
\frametitle{Esimerkki}
\[
U= \frac{E}{s}\frac{1}{R_1\frac{R_2sC+1}{R_2}+1}=\frac{E}{R_1C}\frac{1}{s(s+\frac{R_1+R_2}{R_1R_2C})}
\] 
Tehdään lyhennysmerkintä (tilan säästämiseksi): $a=\frac{R_1+R_2}{R_1R_2C}$. Nyt:
\[
U=\frac{E}{R_1C}\frac{1}{s(s+a)}
\]
Nyt pitää tehdä osamurtokehitelmä, mutta onneksi se on tehty tällaiselle kaavalle jo kolme kalvoa sitten:
\[
U=\frac{E}{R_1C}\frac{1}{s(s+a)}=\frac{E}{R_1C}\left (\frac{\frac{1}{a}}{s}+\frac{-\frac{1}{a}}{s+a}\right )
\]
}

\frame{
\frametitle{Esimerkki}
Suoritetaan Laplace-käänteismuunnos:
\[
u(t)=\lap^{-1}\{U(s)\}=\lap^{-1}\left\{\frac{E}{R_1C}\left(\frac{\frac{1}{a}}{s}+\frac{-\frac{1}{a}}{s+a}\right)\right\}
\]
\[
=\frac{E}{R_1C}\frac{1}{a}\lap^{-1}\left\{\left(\frac{1}{s}-\frac{1}{s+a}\right)\right\}=\frac{E}{R_1C}\frac{1}{a}(1-{\rm e}^{-at})
\]
\[
=\frac{E}{R_1C}\frac{1}{\frac{R_1+R_2}{R_1R_2C}}(1-{\rm e}^{-\frac{R_1+R_2}{R_1R_2C}t})={E}\frac{R_2}{R_1+R_2}(1-{\rm e}^{-\frac{R_1+R_2}{R_1R_2C}t})
\]
Huomautus! Tässä kyseisessä tapauksessa Laplace-muunnos oli työläämpi kuin yritteen käyttö. Jos piirissä on useampi kuin yksi kela tai kondensaattori,
Laplace-muunnoksen käyttö on käytännössä välttämätöntä.
}

\frame{
\frametitle{Esimerkki: kela ja kondensaattori}
Kondensaattorin alkujännite $U_{\rm C0}$ on 5 volttia. Millainen virta piirissä lähtee kulkemaan, kun kytkin suljetaan
ajanhetkellä $t=0$. 
\begin{center}
\begin{picture}(100,50)(0,0)
\vc{0,0}{C=1\,{\rm F}}
\vl{100,0}{L=1\,{\rm H}\hspace{-1.7cm}}
\ri{95,50}{i}
\hln{0,0}{100}
\hln{0,50}{50}
\hso{50,50}{t=0}
%\di{50,0}{I_{\rm L 0}}
\dcru{5,0}{U_{\rm C0}=5\V}

\end{picture}
\end{center}
Laplace-muunnetaan piiri. Koska kondensaattorilla on alkujännite, muunnokseen tulee mukaan
jännitelähde (ks. seuraava kalvo).
}

\frame{
\frametitle{Esimerkki: kela ja kondensaattori}
Kondensaattorin alkujännite $U_{\rm C0}$ on 5 volttia. Millainen virta piirissä lähtee kulkemaan, kun kytkin suljetaan
ajanhetkellä $t=0$. 
\begin{center}
\begin{picture}(100,60)(0,-10)
\vc{0,0}{\frac{1}{sC}}
\vl{100,0}{sL}
\ri{95,50}{I}
\hln{0,0}{100}
\hln{0,50}{100}
\hlst{0,0}{\frac{U_{\rm C0}}{s}}
%\hso{50,50}{t=0}
%\di{50,0}{I_{\rm L 0}}
%\dcru{5,0}{U_{\rm C0}=5\V}

\end{picture}
\end{center}
Seuraavaksi ratkaistaan Ohmin lain avulla munnetusta piiristä virta $I$:
\[
I=\frac{\frac{U_{\rm C0}}{s}}{\frac{1}{sC}+sL}=\frac{U_{\rm C0}}{\frac{1}{C}+s^2L}=\frac{U_{\rm C0}}{\frac{1}{C}+s^2L}
\]
Jatkuu\ldots

}

\frame{
\frametitle{Esimerkki: kela ja kondensaattori}
\scriptsize
Nyt lauseke pitää vääntää sellaiseen muotoon, että siihen voidaan soveltaa jotain kaavakokoelman
muunnoskaavaa. Virran lausekkeen nimittäjässä on toinen $s$:n potenssi plus vakio. Sopivalta
kaavalta näyttää
\[
\lap\{\sin \omega t\} = \displaystyle \frac{\omega}{s^2 + \omega^2}
\]
Järjestetään virran lauseke niin, että se näyttää samalta
\[
I=\frac{U_{\rm C0}}{\frac{1}{C}+s^2L}=U_{\rm C0}\frac{1}{\frac{1}{C}+s^2L}=\frac{U_{\rm C0}}{L}\frac{1}{\frac{1}{LC}+s^2}=
\frac{U_{\rm C0}}{L}\sqrt{LC}\frac{\frac{1}{\sqrt{LC}}}{\frac{1}{LC}+s^2}
\]
Ja käänteismuunnetaan ja sijoitetaan lukuarvot
\[
i(t)=\lap^{-1}\left\{ \frac{U_{\rm C0}}{L}\sqrt{LC}\frac{\frac{1}{\sqrt{LC}}}{\frac{1}{LC}+s^2} \right\}=\frac{U_{\rm C0}}{L}\sqrt{LC}\lap^{-1}\left\{\frac{\frac{1}{\sqrt{LC}}}{\frac{1}{LC}+s^2} \right\}
\]
\[
=\frac{U_{\rm C0}}{L}\sqrt{LC}\sin\frac{1}{\sqrt{LC}}t=5\sin(t)
\]
Eli piirissä jää kiertämään 5 ampeerin sinimuotoinen virta, kulmataajuudella $\omega=\frac{1}{\sqrt{LC}}=1$.

}



\frame{
\frametitle{Esimerkki}
Ajanhetkellä $t=0$ kondensaattorin jännite on 0 V ja kelan virta on 1 A. Miten
piirin virta $i$ käyttäytyy ajanhetken $t=0$ jälkeen? Selvitä se Laplace-muunnoksen avulla.
\[
C=1\,{\rm F}\quad L=1\,{\rm H} \quad U_{\rm C0}=0\V \quad I_{\rm L 0}=1\A
\]
\begin{center}
\begin{picture}(100,50)(0,0)
\vc{0,0}{C}
\vl{50,0}{L}
\ri{25,50}{i}
\hln{0,0}{50}
\hln{0,50}{50}
\di{50,0}{I_{\rm L 0}}
\dcru{5,0}{U_{\rm C0}}

\end{picture}
\end{center}

%{\tiny Oikea lopputulos: $i(t)=\sin(t+\frac{\pi}{2})$}

Ratkaisu: muunnetaan piiri (seuraava kalvo).

}

\frame{
\frametitle{Esimerkki}

\[
C=1\,{\rm F}\quad L=1\,{\rm H} \quad U_{\rm C0}=0\V \quad I_{\rm L 0}=1\A
\]
\begin{center}
\begin{picture}(100,50)(0,0)
\vc{0,0}{\frac{1}{sC}}
\vl{50,0}{sL}
\ri{25,50}{I}
\hln{0,0}{50}
\hln{0,50}{50}
%\di{50,0}{I_{\rm L 0}}
%\dcru{5,0}{U_{\rm C0}}
\hlst{0,0}{LI_{\rm L0}}
\end{picture}
\end{center}
Virta Laplace-tasossa on
\[
I=\frac{LI_{\rm L0}}{\frac{1}{sC}+sL}=\frac{1}{\frac{1}{s}+s}=\frac{s}{s^2+1}
\]
Käänteismuunnetaan:
\[
i(t)=\lap^{-1}\left\{ \frac{s}{s^2+1} \right\}=\cos(1\cdot t)=\cos(t)=\sin(t+\frac{\pi}{2})
\]
}


