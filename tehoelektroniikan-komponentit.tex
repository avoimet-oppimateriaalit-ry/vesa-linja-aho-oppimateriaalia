\frame{
\frametitle{Tehoelektroniikan komponentit}
Tehoelektroniikka = käytetään sähköä suurtehosovelluksissa elektronisesti ohjaten. Tärkeitä komponentteja
\begin{itemize}
\item IGBT eli eristehilabipolaaritransistori
\item SCR-tyristori
\item GTO-tyristori
\item Triac ja diac
\end{itemize}
}

\frame{
\frametitle{IGBT eli eristehilabipolaaritransistori}
\begin{itemize}
\item Kytkinkäyttöön suunniteltu jänniteohjattu komponentti.
\item Uudehkoa tekniikkaa (kehitetty 1980-luvulla).
\item Suurilla taajuuksilla MOSFET on edelleen yleisesti käytetty.
\item Suurilla tehoilla käytetään perinteisiä tyristorikomponentteja.
\end{itemize}
}

\frame{
\frametitle{Tyristorikomponentit}
\begin{itemize}
\item SCR-tyristori
\item GTO-tyristori
\item Triac
\item Diac
\end{itemize}
}

\frame{
\frametitle{SCR-tyristori}
\begin{itemize}
\item Neljästä puolijohdekerroksesta valmistettu "perinteinen"\ tyristori.
\item Toiminta lyhyesti: hilalle annettu jännitepulssi saa tyristorin johtamaan.
\item Tyristori johtaa niin kauan, kun virta anodilta katodille on riittävän suuri.
\item SCR-tyristori lakkaa johtamasta eli {\em sammuu} vasta, kun pitovirta alitetaan.
\end{itemize}
}

\frame{
\frametitle{GTO-tyristori}
\begin{itemize}
\item Kuten SCR-tyristori, mutta se voidaan sammuttaa antamalla hilalle negatiivinen jännitepulssi.
\end{itemize}
}

\frame{
\frametitle{Triac}
\begin{itemize}
\item Toimii kuten kaksi rinnakkain mutta vastakkaisiin suuntiin kytkettyä tyristoria.
\item Käytetään vaihtosähkötehon säätöön.
\end{itemize}
}

\frame{
\frametitle{Diac}
\begin{itemize}
\item Diacissa
\item Diacia käytetään Triacin liipaisuun.
\end{itemize}
}

\frame{
\frametitle{Tehonsäätö}
\begin{itemize}
\item Tyristorikomponenteilla tehonsäätö perustuu vaihesäätöön.
\item Tyristori/Triac kytketään johtamaan, kun siniaallon puolijakso on saavuttanut tietyn vaihekulman.
\end{itemize}
}

\frame{
\frametitle{Komponenttiesimerkkejä}
\begin{itemize}
\item Teho-MOSFET BUZ11
\item Tyristori TIC106
\item Triac TIC226D
\item Diac DB3
\item IGBT IRG4BC30U
\end{itemize}
}
