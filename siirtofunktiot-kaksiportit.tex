\frame{
\frametitle{Taajuusvaste ja siirtofunktiot}
\begin{itemize}
\item Siirtofunktiot ja erityisesti taajuusvaste ovat tärkeitä käsitteitä, kun tutkitaan, kuinka
jokin elementti (kaapeli, suodatin, vahvistin, \ldots) vaikuttaa signaalin kulkuun.
\end{itemize}
}

\frame{
\frametitle{Napa ja portti}
\begin{itemize}
\item Piirissä olevaa johdon liitäntäkohtaa nimitetään {\bf navaksi} tai nastaksi.
\item Kaksi napaa muodostavat {\bf portin} eli napaparin.
\item Helpoin esimerkki on auton akku, jolla sisäistä resistanssia.
\end{itemize}
\begin{center}
\begin{picture}(100,50)(0,0)
\vst{0,0}{E}
\hz{0,50}{R_{\rm S}}
\hln{0,0}{50}
\out{50,0}
\out{50,50}
\end{picture}
\end{center}

}


\frame{
\frametitle{Nelinapa eli kaksiportti}
\begin{itemize}
\item Todella moni elektroniikkapiiri toimii niin, että sinne syötetään jokin signaali, ja sieltä saadaan ulos jokin signaali.
\item Syötettyä signaalia kutsutaan herätteeksi ja ulos saatavaa signaalia vasteeksi.
\item Vasteen ja herätteen suhdetta kutsutaan syöttöpistefunktioksi (vaste ja heräte samassa portissa) tai siirtofunktioksi
(vaste ja heräte eri portissa).
\end{itemize}
\begin{center}
\begin{picture}(100,50)(0,0)
%\vst{0,0}{E}
\hstp{0,0}{}
%\hz{0,50}{R_{\rm S}}
%\hln{0,0}{50}
\out{50,0}
\out{50,50}
\out{0,0}
\out{0,50}
\du{0,0}{\Uin}
\du{50,0}{\Uout}
\ri{0,0}{{I_{\rm in}}}
\end{picture}
\end{center}
}

\frame{
\frametitle{Syöttöpiste- ja siirtofunktiot}
\begin{center}
\begin{picture}(100,50)(0,0)
%\vst{0,0}{E}
\hstp{0,0}{}
%\hz{0,50}{R_{\rm S}}
%\hln{0,0}{50}
\out{50,0}
\out{50,50}
\out{0,0}
\out{0,50}
\du{0,0}{\Uin}
\du{50,0}{\Uout}
\ri{10,50}{{I_{\rm in}}}
\li{40,50}{{I_{\rm out}}}
\end{picture}
\end{center}
\begin{itemize}
\item Syöttöpistefunktioita ovat muun muassa tuloimpedanssi $Z_{\rm in}=\frac{\Uin}{I_{\rm in}}$ ja lähtöimpedanssi $Z_{\rm out}=\frac{\Uout}{I_{\rm out}}$.
\item Siirtofunktioita ovat jännitevahvistus $A=\frac{\Uout}{\Uin}$ sekä virtavahvistus $B=\frac{I_{\rm out}}{I_{\rm in}}$
\end{itemize}
}

\frame{
\frametitle{Siirtofunktio, taajuusvaste, amplitudivaste ja vaihevaste}
\begin{itemize}
\item Amplitudivasteella tarkoitetaan sitä siirtofunktiota, josta käy ilmi, miten piiri vaikuttaa signaalin amplitudiin eli voimakkuuteen eri taajuuksilla.
\item Vaihevasteella tarkoitetaan siirtofunktiota, joka kertoo, miten piiri vaikutttaa signaalin vaiheeseen, eli käytännössä, kuinka paljon signaali viivästyy piirissä.
\item Amplitudi- ja vaihevasteesta käytetään yhteisnimitystä taajuusvaste. 
\item Yleinen tapa on käyttää siirtofunktioiden esityksessä ja matemaattisessa käsittelyssä kompleksilukuja.
\item Tällöin sekä amplitudi- että vaihevaste kulkevat mukana samassa kaavassa.
\item Kompleksiluvun itseisarvo kertoo amplitudin ja kulma eli argumentti kertoo vaihekulman.
\end{itemize}
}
