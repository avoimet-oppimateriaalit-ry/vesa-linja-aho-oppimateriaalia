

\documentclass[]{beamer}


\usepackage[utf8]{inputenc}
% Latex circuit diagram macros - version 1.2 (18th August 2012)
% Kimmo Silvonen, Vesa Linja-aho (License: LGPL)
% Homepage: http://code.google.com/p/latex-circuit-diagram/

\usepackage{upgreek} % For correct micro (\mu vs. \upmu) symbol

% The following are the macros:

% v standard voltage source
\newcommand{\vst}[2]
 {\put(#1){\begin{picture}(32,50)
 \put(0,25){\circle{20}}
 \put(0,0){\line(0,1){50}}
 \put(-8,38){\makebox(0,0){$+$}}
 \put(-8,12){\makebox(0,0){$-$}}
 \put(-13,25){\makebox(0,0)[r]{$#2$}}
\end{picture}}}

% v standard voltage source down
\newcommand{\vdst}[2]
 {\put(#1){\begin{picture}(32,50)
 \put(0,25){\circle{20}}
 \put(0,0){\line(0,1){50}}
 \put(-8,38){\makebox(0,0){$-$}}
 \put(-8,12){\makebox(0,0){$+$}}
 \put(-13,25){\makebox(0,0)[r]{$#2$}}
\end{picture}}}

% v dc voltage source
\newcommand{\vba}[2]
 {\put(#1){\begin{picture}(32,50)
 \put(-5,23){\line(1,0){10}}
 \put(-10,27){\line(1,0){20}}
 \put(0,0){\line(0,1){23}}
 \put(0,27){\line(0,1){23}}
 \put(-13,25){\makebox(0,0)[r]{$#2$}}
\end{picture}}}

% v dc voltage source down
\newcommand{\vdba}[2]
 {\put(#1){\begin{picture}(32,50)
 \put(-5,27){\line(1,0){10}}
 \put(-10,23){\line(1,0){20}}
 \put(0,0){\line(0,1){23}}
 \put(0,27){\line(0,1){23}}
 \put(-13,25){\makebox(0,0)[r]{$#2$}}
\end{picture}}}

% v current source
\newcommand{\vj}[2]
 {\put(#1){\begin{picture}(32,50)
 \put(0,25){\circle{20}}
 \put(-10,25){\line(1,0){20}}
 \put(0,35){\line(0,1){15}}
 \put(0,0){\line(0,1){15}}
 \put(-13,25){\makebox(0,0)[r]{$#2$}}
 \thicklines
 \put(0,42.5){\vector(0,1){0}}
\end{picture}}}

% v current source down
\newcommand{\vdj}[2]
 {\put(#1){\begin{picture}(32,50)
 \put(0,25){\circle{20}}
 \put(-10,25){\line(1,0){20}}
 \put(0,0){\line(0,1){15}}
 \put(0,35){\line(0,1){15}}
 \put(-13,25){\makebox(0,0)[r]{$#2$}}
 \thicklines
 \put(0,7.5){\vector(0,-1){0}}
\end{picture}}}

% v line
\newcommand{\vln}[2]
 {\put(#1){\begin{picture}(10,50)
 \put(0,0){\line(0,1){#2}}
\end{picture}}}

% h standard voltage source
\newcommand{\hst}[2]
 {\put(#1){\begin{picture}(50,32)
 \put(25,0){\circle{20}}
 \put(0,0){\line(1,0){50}}
 \put(12,-8){\makebox(0,0){$-$}}
 \put(38,-8){\makebox(0,0){$+$}}
 \put(25,-19){\makebox(0,0){$#2$}}
\end{picture}}}

% h standard voltage source left
\newcommand{\hlst}[2]
 {\put(#1){\begin{picture}(50,32)
 \put(25,0){\circle{20}}
 \put(0,0){\line(1,0){50}}
 \put(12,-8){\makebox(0,0){$+$}}
 \put(38,-8){\makebox(0,0){$-$}}
 \put(25,-19){\makebox(0,0){$#2$}}
\end{picture}}}

% h dc voltage source
\newcommand{\hba}[2]
 {\put(#1){\begin{picture}(50,32)
 \put(23,-5){\line(0,1){10}}
 \put(27,-10){\line(0,1){20}}
 \put(0,0){\line(1,0){23}}
 \put(27,0){\line(1,0){23}}
 \put(25,-19){\makebox(0,0){$#2$}}
\end{picture}}}

% h dc voltage source left
\newcommand{\hlba}[2]
 {\put(#1){\begin{picture}(50,32)
 \put(27,-5){\line(0,1){10}}
 \put(23,-10){\line(0,1){20}}
 \put(0,0){\line(1,0){23}}
 \put(27,0){\line(1,0){23}}
 \put(25,-19){\makebox(0,0){$#2$}}
\end{picture}}}

% h current source
\newcommand{\hj}[2]
 {\put(#1){\begin{picture}(50,32)
 \put(25,0){\circle{20}}
 \put(25,-10){\line(0,1){20}}
 \put(35,0){\line(1,0){15}}
 \put(0,0){\line(1,0){15}}
 \put(25,-19){\makebox(0,0){$#2$}}
 \thicklines
 \put(42.5,0){\vector(1,0){0}}
\end{picture}}}

% h current source left
\newcommand{\hlj}[2]
 {\put(#1){\begin{picture}(50,32)
 \put(25,0){\circle{20}}
 \put(25,-10){\line(0,1){20}}
 \put(0,0){\line(1,0){15}}
 \put(35,0){\line(1,0){15}}
 \put(25,-19){\makebox(0,0){$#2$}}
 \thicklines
 \put(7.5,0){\vector(-1,0){0}}
\end{picture}}}

% h line
\newcommand{\hln}[2]
 {\put(#1){\begin{picture}(400,10)
 \put(0,0){\line(1,0){#2}}
\end{picture}}}

% v voltage source (old)
\newcommand{\ve}[2]
 {\put(#1){\begin{picture}(32,50)
 \put(0,25){\circle{20}}
 \put(0,0){\line(0,1){15}}
 \put(0,35){\line(0,1){15}}
 \put(-13,25){\makebox(0,0)[r]{$#2$}}
 \thicklines
 \put(0,15){\vector(0,1){20}}
\end{picture}}}

% v voltage source down (old)
\newcommand{\vde}[2]
 {\put(#1){\begin{picture}(32,50)
 \put(0,25){\circle{20}}
 \put(0,0){\line(0,1){15}}
 \put(0,35){\line(0,1){15}}
 \put(-13,25){\makebox(0,0)[r]{$#2$}}
 \thicklines
 \put(0,35){\vector(0,-1){20}}
\end{picture}}}

% h voltage source
\newcommand{\he}[2]
 {\put(#1){\begin{picture}(50,32)
 \put(25,0){\circle{20}}
 \put(0,0){\line(1,0){15}}
 \put(35,0){\line(1,0){15}}
 \put(25,-19){\makebox(0,0){$#2$}}
 \thicklines
 \put(15,0){\vector(1,0){20}}
\end{picture}}}

% h voltage source left
\newcommand{\hle}[2]
 {\put(#1){\begin{picture}(50,32)
 \put(25,0){\circle{20}}
 \put(0,0){\line(1,0){15}}
 \put(35,0){\line(1,0){15}}
 \put(25,-19){\makebox(0,0){$#2$}}
 \thicklines
 \put(35,0){\vector(-1,0){20}}
\end{picture}}}

% v controlled voltage source
\newcommand{\vcvs}[2]
 {\put(#1){\begin{picture}(32,50)
 \put(0,0){\line(0,1){15}}
 \put(0,35){\line(0,1){15}}
 \put(0,15){\line(1,1){10}}
 \put(0,15){\line(-1,1){10}}
 \put(0,35){\line(-1,-1){10}}
 \put(0,35){\line(1,-1){10}}
 \put(-13,25){\makebox(0,0)[r]{$#2$}}
 \put(-20,40){\makebox(0,0){$+$}}
 \put(-20,10){\makebox(0,0){$-$}}
\end{picture}}}

% v controlled current source
\newcommand{\vccs}[2]
 {\put(#1){\begin{picture}(32,50)
 \put(0,0){\line(0,1){15}}
 \put(0,35){\line(0,1){15}}
 \put(0,18){\vector(0,1){14}}
 \put(0,15){\line(1,1){10}}
 \put(0,15){\line(-1,1){10}}
 \put(0,35){\line(-1,-1){10}}
 \put(0,35){\line(1,-1){10}}
 \put(-13,25){\makebox(0,0)[r]{$#2$}}
\end{picture}}}

% v nonstandard voltage source
\newcommand{\vnst}[2]
 {\put(#1){\begin{picture}(32,50)
 \put(0,25){\circle{20}}
 \put(0,0){\line(0,1){15}}
 \put(0,35){\line(0,1){15}}
 \put(-20,25){\makebox(0,0)[l]{$#2$}}
\end{picture}}}

% v impedance
\newcommand{\vz}[2]
 {\put(#1){\begin{picture}(28,50)
 \put(-5,10){\framebox(10,30)}
 \put(0,0){\line(0,1){10}}
 \put(0,40){\line(0,1){10}}
 \put(-12,25){\makebox(0,0)[r]{$#2$}}
\end{picture}}}

% v capasitance
\newcommand{\vc}[2]
 {\put(#1){\begin{picture}(32,50)
 \put(0,0){\line(0,1){23}}
 \put(0,27){\line(0,1){23}}
 \put(-10,23){\line(1,0){20}}
 \put(-10,27){\line(1,0){20}}
 \put(-12,25){\makebox(0,0)[r]{${#2}$}}
\end{picture}}}

% v coil
\newcommand{\vl}[2]
 {\put(#1){\begin{picture}(22,50)
 \put(0,0){\line(0,1){10}}
 \put(0,40){\line(0,1){10}}
 \multiput(0,15)(0,10){3}{\oval(10,10)[r]}
 \put(-3,25){\makebox(0,0)[r]{${#2}$}}
\end{picture}}}

% v coil reversed
\newcommand{\vlr}[2]
 {\put(#1){\begin{picture}(22,50)
 \put(0,0){\line(0,1){10}}
 \put(0,40){\line(0,1){10}}
 \multiput(0,15)(0,10){3}{\oval(10,10)[l]}
 \put(3,25){\makebox(0,0)[l]{${#2}$}}
\end{picture}}}

% v reactance
\newcommand{\vx}[2]
 {\put(#1){\begin{picture}(28,50)
 \put(-5,10){\framebox(10,30)}
 \thicklines
 \multiput(-5,10)(1,0){11}{\line(0,1){30}}
 \multiput(-5,10)(0,1){31}{\line(1,0){10}}
 \thinlines
 \put(0,0){\line(0,1){10}}
 \put(0,40){\line(0,1){10}}
 \put(-7,25){\makebox(0,0)[r]{$#2$}}
\end{picture}}}

% v resistance
\newcommand{\vr}[2]
 {\put(#1){\begin{picture}(22,50)
 \multiput(0,10)(0,10){3}{\line(2,1){10}}
 \multiput(10,15)(0,10){3}{\line(-2,1){10}}
 \put(0,40){\line(0,1){10}}
 \put(0,0){\line(0,1){10}}
 \put(-3,25){\makebox(0,0)[r]{${#2}$}}
\end{picture}}}

% v engine
\newcommand{\veng}[2]
 {\put(#1){\begin{picture}(32,50)
 \put(0,25){\circle{20}}
 \put(-5,33){\line(0,1){5}}
 \put(5,33){\line(0,1){5}}
 \put(-5,12){\line(0,1){5}}
 \put(5,12){\line(0,1){5}}
 \put(-5,38){\line(1,0){10}}
 \put(-5,12){\line(1,0){10}}
 \put(0,0){\line(0,1){12}}
 \put(0,38){\line(0,1){12}}
 \put(-10,25){\makebox(0,0)[r]{$#2$}}
\end{picture}}}

% h impedance
\newcommand{\hz}[2]
 {\put(#1){\begin{picture}(50,22)
 \put(10,-5){\framebox(30,10)}
 \put(0,0){\line(1,0){10}}
 \put(40,0){\line(1,0){10}}
 \put(25,-12){\makebox(0,0){$#2$}}
\end{picture}}}

% h capasitance
\newcommand{\hc}[2]
 {\put(#1){\begin{picture}(50,32)
 \put(0,0){\line(1,0){23}}
 \put(27,0){\line(1,0){23}}
 \put(23,-10){\line(0,1){20}}
 \put(27,-10){\line(0,1){20}}
 \put(25,-17){\makebox(0,0){${#2}$}}
\end{picture}}}

% h coil
\newcommand{\hl}[2]
 {\put(#1){\begin{picture}(50,22)
 \put(0,0){\line(1,0){10}}
 \put(40,0){\line(1,0){10}}
 \multiput(15,0)(10,0){3}{\oval(10,10)[t]}
 \put(25,-7){\makebox(0,0){${#2}$}}
\end{picture}}}

% h coil upside down
\newcommand{\hld}[2]
 {\put(#1){\begin{picture}(50,22)
 \put(0,0){\line(1,0){10}}
 \put(40,0){\line(1,0){10}}
 \multiput(15,0)(10,0){3}{\oval(10,10)[b]}
 \put(25,7){\makebox(0,0){${#2}$}}
\end{picture}}}

% h reactance
\newcommand{\hx}[2]
 {\put(#1){\begin{picture}(50,22)
 \put(10,-5){\framebox(30,10)}
 \thicklines
 \multiput(10,-5)(0,1){11}{\line(1,0){30}}
 \multiput(10,-5)(1,0){31}{\line(0,1){10}}
 \thinlines
 \put(0,0){\line(1,0){10}}
 \put(40,0){\line(1,0){10}}
 \put(25,-12){\makebox(0,0){${#2}$}}
\end{picture}}}

% h resistance
\newcommand{\hr}[2]
 {\put(#1){\begin{picture}(50,22)
 \multiput(10,0)(10,0){3}{\line(1,2){5}}
 \multiput(15,10)(10,0){3}{\line(1,-2){5}}
 \put(0,0){\line(1,0){10}}
 \put(40,0){\line(1,0){10}}
 \put(25,-7){\makebox(0,0){${#2}$}}
\end{picture}}}

% h resistance (old)
\newcommand{\hrold}[2]
 {\put(#1){\begin{picture}(50,22)
 \multiput(10,7)(12,0){3}{\line(1,0){6}}
 \multiput(16,0)(12,0){2}{\line(1,0){6}}
 \multiput(10,0)(6,0){6}{\line(0,1){7}}
 \put(0,0){\line(1,0){10}}
 \put(40,0){\line(1,0){10}}
 \put(25,-7){\makebox(0,0){${#2}$}}
\end{picture}}}

% v transformer 1 core
\newcommand{\vmlc}[1]
 {\put(#1){\begin{picture}(60,50)
 \put(0,0){\line(0,1){10}}
 \put(0,40){\line(0,1){10}}
 \multiput(0,15)(0,10){3}{\oval(10,10)[r]}
 \put(40,0){\line(0,1){10}}
 \put(40,40){\line(0,1){10}}
 \multiput(40,15)(0,10){3}{\oval(10,10)[l]}
 \put(14,5){\line(0,1){40}}
 \put(20,5){\line(0,1){40}}
 \put(26,5){\line(0,1){40}}
\end{picture}}}

% v transformer 2
\newcommand{\vml}[4]
 {\put(#1){\begin{picture}(60,50)
 \put(0,0){\line(0,1){10}}
 \put(0,40){\line(0,1){10}}
 \multiput(0,15)(0,10){3}{\oval(10,10)[r]}
 \put(50,0){\line(0,1){10}}
 \put(50,40){\line(0,1){10}}
 \multiput(50,15)(0,10){3}{\oval(10,10)[l]}
 \put(10,40){\circle*{3}}
 \put(40,40){\circle*{3}}
 \put(25,40){\vector(-1,0){10}}
 \put(25,40){\vector(1,0){10}}
 \put(25,47){\makebox(0,0){$#4$}}
 \put(25,25){\makebox(0,0){${#2}\hspace{2mm}{#3}$}}
\end{picture}}}

% v transformer 3
\newcommand{\vt}[4]
 {\put(#1){\begin{picture}(60,50)
 \put(5,10){\framebox(10,30)}
 \put(35,10){\framebox(10,30)}
 \thicklines
 \multiput(5,10)(1,0){11}{\line(0,1){30}}
 \multiput(5,10)(0,1){31}{\line(1,0){10}}
 \multiput(35,10)(1,0){11}{\line(0,1){30}}
 \multiput(35,10)(0,1){31}{\line(1,0){10}}
 \thinlines
 \put(10,0){\line(0,1){10}}
 \put(10,40){\line(0,1){10}}
 \put(0,0){\line(1,0){10}}
 \put(0,50){\line(1,0){10}}
 \put(40,0){\line(1,0){10}}
 \put(40,50){\line(1,0){10}}
 \put(40,0){\line(0,1){10}}
 \put(40,40){\line(0,1){10}}
 \put(20,40){\circle*{3}}
 \put(30,40){\circle*{3}}
 \put(22,44){\vector(1,0){8}}
 \put(28,44){\vector(-1,0){8}}
 \put(25,30){\makebox(0,0){\footnotesize $#4$}}
 \put(5,25){\makebox(0,0)[r]{\footnotesize ${#2}$}}
 \put(47,25){\makebox(0,0)[l]{\footnotesize ${#3}$}}
\end{picture}}}

% v transformer 4
\newcommand{\vm}[4]
 {\put(#1){\begin{picture}(60,50)
 \put(-5,10){\framebox(10,30)}
 \put(45,10){\framebox(10,30)}
 \thicklines
 \multiput(-5,10)(1,0){11}{\line(0,1){30}}
 \multiput(-5,10)(0,1){31}{\line(1,0){10}}
 \multiput(45,10)(1,0){11}{\line(0,1){30}}
 \multiput(45,10)(0,1){31}{\line(1,0){10}}
 \thinlines
 \put(0,0){\line(0,1){10}}
 \put(0,40){\line(0,1){10}}
 \put(50,0){\line(0,1){10}}
 \put(50,40){\line(0,1){10}}
 \put(10,40){\circle*{3}}
 \put(40,40){\circle*{3}}
 \put(25,40){\vector(-1,0){10}}
 \put(25,40){\vector(1,0){10}}
 \put(25,47){\makebox(0,0){$#4$}}
 \put(25,25){\makebox(0,0){${#2}\hspace{2mm}{#3}$}}
\end{picture}}}

% v transformer 5
\newcommand{\vlm}[4]
 {\put(#1){\begin{picture}(60,50)
 \put(0,0){\line(0,1){10}}
 \put(0,40){\line(0,1){10}}
 \multiput(0,15)(0,10){3}{\oval(10,10)[r]}
 \put(50,0){\line(0,1){10}}
 \put(50,40){\line(0,1){10}}
 \multiput(50,15)(0,10){3}{\oval(10,10)[l]}
 \put(10,40){\circle*{3}}
 \put(40,10){\circle*{3}}
 \put(14,40){\vector(3,-4){22}}
 \put(14,40){\vector(-3,4){0}}
 \put(25,44){\makebox(0,0){$#4$}}
 \put(25,25){\makebox(0,0){${#2}\hspace{4mm}{#3}$}}
\end{picture}}}

% v transformer 6
\newcommand{\vlmi}[4]
 {\put(#1){\begin{picture}(60,50)
 \put(0,0){\line(0,1){10}}
 \put(0,40){\line(0,1){10}}
 \multiput(0,15)(0,10){3}{\oval(10,10)[r]}
 \put(50,0){\line(0,1){10}}
 \put(50,40){\line(0,1){10}}
 \multiput(50,15)(0,10){3}{\oval(10,10)[l]}
 \put(10,10){\circle*{3}}
 \put(40,40){\circle*{3}}
 \put(14,10){\vector(3,4){22}}
 \put(14,10){\vector(-3,-4){0}}
 \put(25,44){\makebox(0,0){$#4$}}
 \put(25,25){\makebox(0,0){${#2} \hspace{4mm} {#3}$}}
\end{picture}}}

% v lamp
\newcommand{\vlamp}[2]
 {\put(#1){\begin{picture}(50,50)
 \put(0,25){\circle{20}}
 \put(0,0){\line(0,1){15}}
 \put(0,35){\line(0,1){15}}
 \put(-7,32){\line(1,-1){14}}
 \put(-7,18){\line(1,1){14}}
 \put(-12,17){\makebox(0,0)[r]{$#2$}}
\end{picture}}}

% horizontal switch open
\newcommand{\hso}[2]
 {\put(#1){\begin{picture}(50,20)
 \put(0,0){\line(1,0){15}}
 \put(35,0){\line(1,0){15}}
 \put(15,0){\line(2,1){20}}
 \put(15,0){\circle{3}}
 \put(35,0){\circle{3}}
 \put(25,-10){\makebox(0,0){$#2$}}
\end{picture}}}

% horizontal switch nearly closed
\newcommand{\hsc}[2]
 {\put(#1){\begin{picture}(50,20)
 \put(0,0){\line(1,0){15}}
 \put(35,0){\line(1,0){15}}
 \put(15,0){\line(6,1){20}}
 \put(15,0){\circle{3}}
 \put(35,0){\circle{3}}
 \put(25,-10){\makebox(0,0){$#2$}}
\end{picture}}}

% vertical switch open
\newcommand{\vso}[2]
 {\put(#1){\begin{picture}(20,50)
 \put(0,0){\line(0,1){15}}
 \put(0,35){\line(0,1){15}}
 \put(0,15){\line(1,2){10}}
 \put(0,15){\circle*{3}}
 \put(0,35){\circle*{3}}
 \put(-2,25){\makebox(0,0)[r]{$#2$}}
\end{picture}}}

% v avo meter
\newcommand{\vavo}[3]
 {\put(#1){\begin{picture}(50,50)
 \put(0,25){\circle{20}}
 \put(0,25){\makebox(0,0){$#2$}}
 \put(0,0){\line(0,1){15}}
 \put(0,35){\line(0,1){15}}
 \put(-9,15){\makebox(0,0)[r]{$#3$}}
\end{picture}}}

% h avo meter
\newcommand{\havo}[3]
 {\put(#1){\begin{picture}(50,50)
 \put(25,0){\circle{20}}
 \put(25,0){\makebox(0,0){$#2$}}
 \put(0,0){\line(1,0){15}}
 \put(35,0){\line(1,0){15}}
 \put(25,-17){\makebox(0,0){$#3$}}
\end{picture}}}

% v power meter
\newcommand{\vpm}[3]
 {\put(#1){\begin{picture}(50,65)
 \put(0,0){\circle{20}}
 \put(-3,12){\circle*{3}}
 \put(-13,3){\circle*{3}}
 \put(0,0){\makebox(0,0){$#2$}}
 \put(-10,-15){\makebox(0,0)[r]{$#3$}}
\end{picture}}}

% up voltage arrow
\newcommand{\uu}[2]
 {\put(#1){\begin{picture}(12,50)
 \put(-3,25){\makebox(0,0)[r]{${#2}$}}
% \thicklines
 \put(0,5){\vector(0,1){40}}
\end{picture}}}

% down voltage arrow
\newcommand{\du}[2]
 {\put(#1){\begin{picture}(12,50)
 \put(3,25){\makebox(0,0)[l]{$#2$}}
% \thicklines
 \put(0,45){\vector(0,-1){40}}
\end{picture}}}

% right voltage arrow
\newcommand{\ru}[2]
 {\put(#1){\begin{picture}(50,12)
 \put(25,8){\makebox(0,0){${#2}$}}
% \thicklines
 \put(5,0){\vector(1,0){40}}
\end{picture}}}

% left voltage arrow
\newcommand{\lu}[2]
 {\put(#1){\begin{picture}(50,12)
 \put(25,8){\makebox(0,0){${#2}$}}
% \thicklines
 \put(45,0){\vector(-1,0){40}}
\end{picture}}}

% down right-curved arrow
\newcommand{\dcru}[2]
 {\put(#1){\begin{picture}(32,50)
 \put(0,25){\oval(20,40)[r]}
 \put(15,25){\makebox(0,0)[l]{$#2$}}
% \thicklines
 \put(0,5){\vector(-4,-1){0}}
\end{picture}}}

% up right-curved arrow
\newcommand{\ucru}[2]
 {\put(#1){\begin{picture}(32,50)
 \put(0,25){\oval(20,40)[r]}
 \put(15,25){\makebox(0,0)[l]{${#2}$}}
% \thicklines
 \put(0,45){\vector(-4,1){0}}
\end{picture}}}

% right up-curved arrow
\newcommand{\rcuu}[2]
 {\put(#1){\begin{picture}(50,32)
 \put(25,0){\oval(40,20)[t]}
 \put(25,20){\makebox(0,0){${#2}$}}
% \thicklines
 \put(45,0){\vector(1,-4){0}}
\end{picture}}}

% left up-curved arrow
\newcommand{\lcuu}[2]
 {\put(#1){\begin{picture}(50,32)
 \put(25,0){\oval(40,20)[t]}
 \put(25,20){\makebox(0,0){${#2}$}}
% \thicklines
 \put(5,0){\vector(-1,-4){0}}
\end{picture}}}

% right current arrow
\newcommand{\ri}[2]
 {\put(#1){\begin{picture}(10,50)
 \put(0,-7){\makebox(0,0){${#2}$}}
 \thicklines
 \put(0,0){\vector(1,0){0}}
\end{picture}}}

% left current arrow
\newcommand{\li}[2]
 {\put(#1){\begin{picture}(10,50)
 \put(3,-7){\makebox(0,0)[r]{${#2}$}}
 \thicklines
 \put(0,0){\vector(-1,0){0}}
\end{picture}}}

% up current arrow
\newcommand{\ui}[2]
 {\put(#1){\begin{picture}(50,20)
 \put(5,0){\makebox(0,0)[l]{$#2$}}
 \thicklines
 \put(0,0){\vector(0,1){0}}
\end{picture}}}

% down current arrow
\newcommand{\di}[2]
 {\put(#1){\begin{picture}(50,20)
 \put(5,0){\makebox(0,0)[l]{$#2$}}
 \thicklines
 \put(0,0){\vector(0,-1){0}}
\end{picture}}}

% right current arrow (text above)
\newcommand{\rui}[2]
 {\put(#1){\begin{picture}(10,50)
 \put(0,7){\makebox(0,0){${#2}$}}
 \thicklines
 \put(0,0){\vector(1,0){0}}
\end{picture}}}

% h crossing jumper
\newcommand{\hcj}[1]
 {\put(#1){\begin{picture}(10,50)
 \put(0,0){\oval(10,10)[t]}
\end{picture}}}

% v crossing jumper
\newcommand{\vcj}[1]
 {\put(#1){\begin{picture}(10,50)
 \put(0,0){\oval(10,10)[r]}
\end{picture}}}

% h ground
\newcommand{\hg}[1]
 {\put(#1){\begin{picture}(50,10)
 \put(-10,0){\line(1,0){20}}
\end{picture}}}

% v ground
\newcommand{\vg}[1]
 {\put(#1){\begin{picture}(50,10)
 \put(0,-10){\line(0,1){20}}
\end{picture}}}

% h ground pin
\newcommand{\hgp}[1]
 {\put(#1){\begin{picture}(20,10)
 \put(-10,-10){\line(1,0){20}}
 \put(0,0){\line(0,-1){10}}
\end{picture}}}

% h ground pin upwards
\newcommand{\hgpu}[1]
 {\put(#1){\begin{picture}(20,10)
 \put(-10,10){\line(1,0){20}}
 \put(0,0){\line(0,1){10}}
\end{picture}}}

% h port
\newcommand{\hp}[3]
 {\put(#1){\begin{picture}(10,50)
 \put(0,0){\circle{3}}
 \put(0,50){\circle{3}}
 \put(0,10){\makebox(0,0){$#3$}}
 \put(0,40){\makebox(0,0){$#2$}}
\end{picture}}}

% v port
\newcommand{\vp}[3]
 {\put(#1){\begin{picture}(50,10)
 \put(0,0){\circle{3}}
 \put(50,0){\circle{3}}
 \put(40,0){\makebox(0,0){$#3$}}
 \put(10,0){\makebox(0,0){$#2$}}
\end{picture}}}

% mathematical text
\newcommand{\txt}[2]
 {\put(#1){\begin{picture}(50,50)
 \put(0,0){\makebox(0,0){$#2$}}
\end{picture}}}

% text non math
\newcommand{\tx}[2]
 {\put(#1){\begin{picture}(50,50)
 \put(0,0){\makebox(0,0){#2}}
\end{picture}}}

% small text
\newcommand{\stx}[2]
 {\put(#1){\begin{picture}(50,50)
 \put(0,0){\makebox(0,0){\scriptsize #2}}
\end{picture}}}

% text non math left-aligned
\newcommand{\txl}[2]
 {\put(#1){\begin{picture}(50,50)
 \put(0,0){\makebox(0,0)[l]{#2}}
\end{picture}}}

% output/input pin
\newcommand{\out}[1]
 {\put(#1){\begin{picture}(10,10)
 \put(0,0){\circle{3}}
\end{picture}}}

% connection node
\newcommand{\cn}[1]
 {\put(#1){\begin{picture}(10,50)
 \put(0,0){\circle*{3}}
\end{picture}}}

% big connection node
\newcommand{\cnb}[1]
 {\put(#1){\begin{picture}(10,50)
 \put(0,0){\circle*{5}}
\end{picture}}}

% down wave arrow
\newcommand{\dw}[2]
 {\put(#1){\begin{picture}(12,50)
 \put(0,50){\vector(0,-1){29}}
 \put(0,0){\line(0,1){25}}
 \put(-12,25){\makebox(0,0){${#2}$}}
\end{picture}}}

% up wave arrow
\newcommand{\uw}[2]
 {\put(#1){\begin{picture}(12,50)
 \put(0,0){\vector(0,1){29}}
 \put(0,25){\line(0,1){25}}
 \put(12,25){\makebox(0,0){${#2}$}}
\end{picture}}}

% left wave arrow
\newcommand{\lw}[2]
 {\put(#1){\begin{picture}(50,12)
 \put(50,0){\vector(-1,0){29}}
 \put(0,0){\line(1,0){25}}
 \put(25,-12){\makebox(0,0){${#2}$}}
\end{picture}}}

% right wave arrow
\newcommand{\rw}[2]
 {\put(#1){\begin{picture}(50,12)
 \put(0,0){\vector(1,0){29}}
 \put(25,0){\line(1,0){25}}
 \put(25,12){\makebox(0,0){${#2}$}}
\end{picture}}}

% h z arrow
\newcommand{\hza}[2]
 {\put(#1){\begin{picture}(50,20)
 \put(-20,25){\vector(1,0){20}}
 \put(-20,25){\line(0,-1){10}}
 \put(-20,8){\makebox(0,0){$#2$}}
\end{picture}}}

% h y arrow
\newcommand{\hya}[2]
 {\put(#1){\begin{picture}(50,20)
 \put(20,25){\vector(-1,0){20}}
 \put(20,25){\line(0,-1){10}}
 \put(20,8){\makebox(0,0){$#2$}}
\end{picture}}}

% v diode up
\newcommand{\ud}[2]
 {\put(#1){\begin{picture}(32,50)
 \put(-10,20){\line(1,0){20}}
 \put(-10,30){\line(1,0){20}}
 \put(-10,20){\line(1,1){10}}
 \put(10,20){\line(-1,1){10}}
 \put(0,0){\line(0,1){50}}
 \put(-20,25){\makebox(0,0)[r]{$#2$}}
\end{picture}}}

% v diode down
\newcommand{\dd}[2]
 {\put(#1){\begin{picture}(32,50)
 \put(-10,20){\line(1,0){20}}
 \put(-10,30){\line(1,0){20}}
 \put(-10,30){\line(1,-1){10}}
 \put(10,30){\line(-1,-1){10}}
 \put(0,0){\line(0,1){50}}
 \put(-20,25){\makebox(0,0)[r]{$#2$}}
\end{picture}}}

% h diode right
\newcommand{\rd}[2]
 {\put(#1){\begin{picture}(50,50)
 \put(0,0){\line(1,0){50}}
 \put(20,-10){\line(1,1){10}}
 \put(20,10){\line(1,-1){10}}
 \put(20,-10){\line(0,1){20}}
 \put(30,-10){\line(0,1){20}}
 \put(25,-20){\makebox(0,0)[c]{$#2$}}
\end{picture}}}

% h diode left
\newcommand{\ld}[2]
 {\put(#1){\begin{picture}(50,50)
 \put(0,0){\line(1,0){50}}
 \put(20,0){\line(1,1){10}}
 \put(20,0){\line(1,-1){10}}
 \put(20,-10){\line(0,1){20}}
 \put(30,-10){\line(0,1){20}}
 \put(25,-20){\makebox(0,0)[c]{$#2$}}
\end{picture}}}

% h diode right (black)
\newcommand{\rdb}[2]
 {\put(#1){\begin{picture}(50,50)
 \put(0,0){\line(1,0){50}}
 \put(20,-10){\line(1,1){10}}
 \put(20,10){\line(1,-1){10}}
 \put(20,-10){\line(0,1){20}}
 \put(30,-10){\line(0,1){20}}
 \thicklines
 \put(20,9){\line(1,0){1}}
 \put(20,8){\line(1,0){2}}
 \put(20,7){\line(1,0){3}}
 \put(20,6){\line(1,0){4}}
 \put(20,5){\line(1,0){5}}
 \put(20,4){\line(1,0){6}}
 \put(20,3){\line(1,0){7}}
 \put(20,2){\line(1,0){8}}
 \put(20,1){\line(1,0){9}}
 \put(20,0){\line(1,0){9.5}}
 \put(20,-1){\line(1,0){9}}
 \put(20,-2){\line(1,0){8}}
 \put(20,-3){\line(1,0){7}}
 \put(20,-4){\line(1,0){6}}
 \put(20,-5){\line(1,0){5}}
 \put(20,-6){\line(1,0){4}}
 \put(20,-7){\line(1,0){3}}
 \put(20,-8){\line(1,0){2}}
 \put(20,-9){\line(1,0){1}}
 \thinlines
 \put(25,-20){\makebox(0,0)[c]{$#2$}}
\end{picture}}}

% h diode right (empty)
\newcommand{\rde}[2]
 {\put(#1){\begin{picture}(50,50)
 \put(0,0){\line(1,0){20}}
 \put(30,0){\line(1,0){20}}
 \put(20,-10){\line(1,1){10}}
 \put(20,10){\line(1,-1){10}}
 \put(20,-10){\line(0,1){20}}
 \put(30,-10){\line(0,1){20}}
 \put(25,-20){\makebox(0,0)[c]{$#2$}}
\end{picture}}}

% h varactor diode right
\newcommand{\vrd}[2]
 {\put(#1){\begin{picture}(50,50)
 \put(0,0){\line(1,0){30}}
 \put(20,-10){\line(1,1){10}}
 \put(20,10){\line(1,-1){10}}
 \put(20,-10){\line(0,1){20}}
 \put(30,-10){\line(0,1){20}}
 \put(33,-10){\line(0,1){20}}
 \put(33,0){\line(1,0){17}}
 \put(25,-20){\makebox(0,0)[c]{$#2$}}
\end{picture}}}

% v zener diode up
\newcommand{\zud}[2]
 {\put(#1){\begin{picture}(32,50)
 \put(-10,20){\line(1,0){20}}
 \put(-10,30){\line(1,0){20}}
 \put(-10,20){\line(1,1){10}}
 \put(10,20){\line(-1,1){10}}
 \put(10,30){\line(0,-1){4}}
 \put(0,0){\line(0,1){50}}
 \put(-20,25){\makebox(0,0)[r]{$#2$}}
\end{picture}}}

% v zener diode down
\newcommand{\zdd}[2]
 {\put(#1){\begin{picture}(32,50)
 \put(-10,20){\line(1,0){20}}
 \put(-10,30){\line(1,0){20}}
 \put(0,20){\line(1,1){10}}
 \put(0,20){\line(-1,1){10}}
 \put(10,20){\line(0,1){4}}
 \put(0,0){\line(0,1){50}}
 \put(-20,25){\makebox(0,0)[r]{$#2$}}
\end{picture}}}

% h Schottky diode right
\newcommand{\srd}[2]
 {\put(#1){\begin{picture}(50,50)
 \put(0,0){\line(1,0){50}}
 \put(20,-10){\line(1,1){10}}
 \put(20,10){\line(1,-1){10}}
 \put(20,-10){\line(0,1){20}}
 \put(30,-10){\line(0,1){20}}
 \put(27,-10){\line(1,0){3}}
 \put(27,-10){\line(0,1){3}}
 \put(30,10){\line(1,0){3}}
 \put(33,7){\line(0,1){3}}
 \put(25,-20){\makebox(0,0)[c]{$#2$}}
\end{picture}}}

% h Schottky diode left
\newcommand{\sld}[2]
 {\put(#1){\begin{picture}(50,50)
 \put(0,0){\line(1,0){50}}
 \put(30,-10){\line(-1,1){10}}
 \put(30,10){\line(-1,-1){10}}
 \put(30,-10){\line(0,1){20}}
 \put(20,-10){\line(0,1){20}}
 \put(17,-10){\line(1,0){3}}
 \put(17,-10){\line(0,1){3}}
 \put(20,10){\line(1,0){3}}
 \put(23,7){\line(0,1){3}}
 \put(25,-20){\makebox(0,0)[c]{$#2$}}
\end{picture}}}

% h Schottky npn transistor
\newcommand{\snpn}[2]
 {\put(#1){\begin{picture}(50,50)
 \put(30,-13){\line(0,1){26}}
 \put(30,0){\line(-1,0){30}}
 \put(30,5){\line(1,1){20}}
 \put(30,-5){\vector(1,-1){8}}
 \put(30,-5){\line(1,-1){20}}
 \put(35,-22){\makebox(0,0){$#2$}}
 \put(27,-15){\line(1,0){3}}
 \put(27,-15){\line(0,1){3}}
 \put(30,15){\line(1,0){3}}
 \put(33,12){\line(0,1){3}}
 \put(30,-15){\line(0,1){30}}
\end{picture}}}

% h npn transistor CB
\newcommand{\npnb}[2]
 {\put(#1){\begin{picture}(50,50)
 \put(0,30){\circle{30}}
 \put(-10,30){\line(1,0){20}}
 \put(0,30){\line(0,-1){30}}
 \put(-25,50){\line(1,-1){20}}
 \put(-5,30){\vector(-1,1){8}}
 \put(5,30){\line(1,1){20}}
 \put(0,52){\makebox(0,0){$#2$}}
\end{picture}}}

% h pnp transistor CB
\newcommand{\pnpb}[2]
 {\put(#1){\begin{picture}(50,50)
 \put(0,30){\circle{30}}
 \put(-10,30){\line(1,0){20}}
 \put(0,30){\line(0,-1){30}}
 \put(-25,50){\line(1,-1){20}}
 \put(-14,39){\vector(1,-1){9}}
 \put(5,30){\line(1,1){20}}
 \put(0,52){\makebox(0,0){$#2$}}
\end{picture}}}

% h npn transistor
\newcommand{\npn}[2]
 {\put(#1){\begin{picture}(50,50)
 \put(30,0){\circle{30}}
 \put(30,-10){\line(0,1){20}}
 \put(30,0){\line(-1,0){30}}
 \put(30,5){\line(1,1){20}}
 \put(30,-5){\vector(1,-1){8}}
 \put(30,-5){\line(1,-1){20}}
 \put(30,-22){\makebox(0,0){$#2$}}
\end{picture}}}

% h pnp transistor
\newcommand{\pnp}[2]
 {\put(#1){\begin{picture}(50,50)
 \put(30,0){\circle{30}}
 \put(30,-10){\line(0,1){20}}
 \put(30,0){\line(-1,0){30}}
 \put(30,5){\line(1,1){20}}
 \put(50,-25){\vector(-1,1){20}}
 \put(30,-22){\makebox(0,0){$#2$}}
\end{picture}}}

% h npn transistor CC
\newcommand{\npnc}[2]
 {\put(#1){\begin{picture}(50,50)
 \put(30,0){\circle{30}}
 \put(30,-10){\line(0,1){20}}
 \put(30,0){\line(-1,0){30}}
 \put(30,-5){\line(1,-1){20}}
 \put(30,5){\vector(1,1){8}}
 \put(30,5){\line(1,1){20}}
 \put(30,-22){\makebox(0,0){$#2$}}
\end{picture}}}

% h pnp transistor CC
\newcommand{\pnpc}[2]
 {\put(#1){\begin{picture}(50,50)
 \put(30,0){\circle{30}}
 \put(30,-10){\line(0,1){20}}
 \put(30,0){\line(-1,0){30}}
 \put(30,-5){\line(1,-1){20}}
 \put(35,10){\vector(-1,-1){5}}
 \put(30,5){\line(1,1){20}}
 \put(30,-22){\makebox(0,0){$#2$}}
\end{picture}}}

% h npn transistor CC
\newcommand{\pnpcr}[2]
 {\put(#1){\begin{picture}(50,50)
 \put(-30,0){\circle{30}}
 \put(-30,-10){\line(0,1){20}}
 \put(-30,0){\line(1,0){30}}
 \put(-30,-5){\line(-1,-1){20}}
 \put(-35,10){\vector(1,-1){5}}
 \put(-30,5){\line(-1,1){20}}
 \put(-30,-22){\makebox(0,0){$#2$}}
\end{picture}}}

% h npn transistor reversed
\newcommand{\npr}[2]
 {\put(#1){\begin{picture}(50,50)
 \put(-30,0){\circle{30}}
 \put(-30,-10){\line(0,1){20}}
 \put(-30,0){\line(1,0){30}}
 \put(-30,5){\line(-1,1){20}}
 \put(-30,-5){\vector(-1,-1){8}}
 \put(-30,-5){\line(-1,-1){20}}
 \put(-30,-22){\makebox(0,0){$#2$}}
\end{picture}}}

% h pnp transistor reversed
\newcommand{\pnr}[2]
 {\put(#1){\begin{picture}(50,50)
 \put(-30,0){\circle{30}}
 \put(-30,-10){\line(0,1){20}}
 \put(-30,0){\line(1,0){30}}
 \put(-30,5){\line(-1,1){20}}
 \put(-35,-10){\vector(1,1){5}}
 \put(-30,-5){\line(-1,-1){20}}
 \put(-30,-22){\makebox(0,0){$#2$}}
\end{picture}}}

% h npn transistor (no circle)
\newcommand{\npnnc}[2]
 {\put(#1){\begin{picture}(50,50)
 \put(30,-13){\line(0,1){26}}
 \put(30,0){\line(-1,0){30}}
 \put(30,5){\line(1,1){20}}
 \put(30,-5){\vector(1,-1){8}}
 \put(30,-5){\line(1,-1){20}}
 \put(30,-22){\makebox(0,0){$#2$}}
\end{picture}}}

% h npn transistor w/o circle
\newcommand{\npnwoc}[2]
 {\put(#1){\begin{picture}(50,50)
 \put(30,-10){\line(0,1){20}}
 \put(30,0){\line(-1,0){30}}
 \put(30,5){\line(2,1){20}}
 \put(30,-5){\vector(2,-1){8}}
 \put(30,-5){\line(2,-1){20}}
 \put(50,15){\line(0,1){10}}
 \put(50,-25){\line(0,1){10}}
 \put(30,-22){\makebox(0,0){$#2$}}
\end{picture}}}

% h pnp transistor w/o circle
\newcommand{\pnpwoc}[2]
 {\put(#1){\begin{picture}(50,50)
 \put(30,-10){\line(0,1){20}}
 \put(30,0){\line(-1,0){30}}
 \put(30,5){\line(2,1){20}}
 \put(40,-10){\vector(-2,1){10}}
 \put(30,-5){\line(2,-1){20}}
 \put(50,15){\line(0,1){10}}
 \put(50,-25){\line(0,1){10}}
 \put(30,-22){\makebox(0,0){$#2$}}
\end{picture}}}

% h npn transistor w/o circle upside down
\newcommand{\npnwocu}[2]
 {\put(#1){\begin{picture}(50,50)
 \put(30,-10){\line(0,1){20}}
 \put(30,0){\line(-1,0){30}}
 \put(30,5){\line(2,1){20}}
 \put(40,10){\vector(-2,-1){8}}
 \put(30,-5){\line(2,-1){20}}
 \put(50,15){\line(0,1){10}}
 \put(50,-25){\line(0,1){10}}
 \put(30,-22){\makebox(0,0){$#2$}}
\end{picture}}}

% h npn transistor w/o circle reversed
\newcommand{\nprwoc}[2]
 {\put(#1){\begin{picture}(50,50)
 \put(-30,-10){\line(0,1){20}}
 \put(-30,0){\line(1,0){30}}
 \put(-30,5){\line(-2,1){20}}
 \put(-30,-5){\vector(-2,-1){8}}
 \put(-30,-5){\line(-2,-1){20}}
 \put(-50,15){\line(0,1){10}}
 \put(-50,-25){\line(0,1){10}}
 \put(-30,-22){\makebox(0,0){$#2$}}
\end{picture}}}

% h npn transistor w/o circle reversed upside down
\newcommand{\nprwocu}[2]
 {\put(#1){\begin{picture}(50,50)
 \put(-30,-10){\line(0,1){20}}
 \put(-30,0){\line(1,0){30}}
 \put(-30,5){\line(-2,1){20}}
 \put(-40,10){\vector(2,-1){8}}
 \put(-30,-5){\line(-2,-1){20}}
 \put(-50,15){\line(0,1){10}}
 \put(-50,-25){\line(0,1){10}}
 \put(-30,-22){\makebox(0,0){$#2$}}
\end{picture}}}

% h enmos, no arrow
\newcommand{\enmosna}[2]
 {\put(#1){\begin{picture}(50,50)
 \put(30,-10){\line(0,1){20}}
 \put(27,-5){\line(0,1){10}}
 \put(50,7.5){\line(0,1){17.5}}
 \put(50,-7.5){\line(0,-1){17.5}}
 \put(0,0){\line(1,0){27}}
 \put(30,7.5){\line(1,0){20}}
 \put(30,-7.5){\line(1,0){20}}
 \put(25,-22){\makebox(0,0){${#2}$}}
\end{picture}}}

% h enmos
\newcommand{\enmos}[2]
 {\put(#1){\begin{picture}(50,50)
 \put(30,-10){\line(0,1){20}}
 \put(27,-5){\line(0,1){10}}
 \put(50,7.5){\line(0,1){17.5}}
 \put(50,-7.5){\line(0,-1){17.5}}
 \put(0,0){\line(1,0){27}}
 \put(30,7.5){\line(1,0){20}}
 \put(30,-7.5){\vector(1,0){20}}
 \put(25,-27){\makebox(0,10){${#2}$}}
\end{picture}}}

% h epmos
\newcommand{\epmos}[2]
 {\put(#1){\begin{picture}(50,50)
 \put(30,-10){\line(0,1){20}}
 \put(27,-5){\line(0,1){10}}
 \put(50,7.5){\line(0,1){17.5}}
 \put(50,-7.5){\line(0,-1){17.5}}
 \put(0,0){\line(1,0){27}}
 \put(30,7.5){\line(1,0){20}}
 \put(50,-7.5){\vector(-1,0){20}}
 \put(25,-22){\makebox(0,0){${#2}$}}
\end{picture}}}

% h dnmos
\newcommand{\dnmos}[2]
 {\put(#1){\begin{picture}(50,50)
 \put(30,-10){\line(0,1){20}}
 \put(27,-5){\line(0,1){10}}
 \put(50,7.5){\line(0,1){17.5}}
 \put(50,-7.5){\line(0,-1){17.5}}
 \put(0,0){\line(1,0){27}}
 \put(30,7.5){\line(1,0){20}}
 \put(30,-7.5){\vector(1,0){20}}
 \put(25,-22){\makebox(0,0){${#2}$}}
\thicklines
 \put(31,-7.5){\line(0,1){15}}
 \put(31.5,-7.5){\line(0,1){15}}
\thinlines
\end{picture}}}

% h dpmos
\newcommand{\dpmos}[2]
 {\put(#1){\begin{picture}(50,50)
 \put(30,-10){\line(0,1){20}}
 \put(27,-5){\line(0,1){10}}
 \put(50,7.5){\line(0,1){17.5}}
 \put(50,-7.5){\line(0,-1){17.5}}
 \put(0,0){\line(1,0){27}}
 \put(30,7.5){\line(1,0){20}}
 \put(50,-7.5){\vector(-1,0){18}}
 \put(25,-22){\makebox(0,0){${#2}$}}
\thicklines
 \put(31,-7.5){\line(0,1){15}}
 \put(32,-7.5){\line(0,1){15}}
\thinlines
\end{picture}}}

% h njfet
\newcommand{\njfet}[2]
 {\put(#1){\begin{picture}(50,50)
 \put(30,-5){\line(0,1){20}}
 \put(50,12.5){\line(0,1){12.5}}
 \put(50,-2.5){\line(0,-1){22.5}}
 \put(0,0){\vector(1,0){30}}
 \put(30,12.5){\line(1,0){20}}
 \put(30,-2.5){\line(1,0){20}}
 \put(25,-22){\makebox(0,0){${#2}$}}
\end{picture}}}

% h pjfet
\newcommand{\pjfet}[2]
 {\put(#1){\begin{picture}(50,50)
 \put(30,-5){\line(0,1){20}}
 \put(50,12.5){\line(0,1){12.5}}
 \put(50,-2.5){\line(0,-1){22.5}}
 \put(30,0){\vector(-1,0){15}}
 \put(0,0){\line(1,0){15}}
 \put(30,12.5){\line(1,0){20}}
 \put(30,-2.5){\line(1,0){20}}
 \put(25,-22){\makebox(0,0){${#2}$}}
\end{picture}}}

% h enmos reversed, no arrow
\newcommand{\enmosrna}[2]
 {\put(#1){\begin{picture}(50,50)
 \put(-30,-10){\line(0,1){20}}
 \put(-27,-5){\line(0,1){10}}
 \put(-50,7.5){\line(0,1){17.5}}
 \put(-50,-7.5){\line(0,-1){17.5}}
 \put(0,0){\line(-1,0){27}}
 \put(-30,7.5){\line(-1,0){20}}
 \put(-30,-7.5){\line(-1,0){20}}
 \put(-25,-22){\makebox(0,0){${#2}$}}
\end{picture}}}

% h enmos reversed
\newcommand{\enmosr}[2]
 {\put(#1){\begin{picture}(50,50)
 \put(-30,-10){\line(0,1){20}}
 \put(-27,-5){\line(0,1){10}}
 \put(-50,7.5){\line(0,1){17.5}}
 \put(-50,-7.5){\line(0,-1){17.5}}
 \put(0,0){\line(-1,0){27}}
 \put(-30,7.5){\line(-1,0){20}}
 \put(-30,-7.5){\vector(-1,0){20}}
 \put(-25,-22){\makebox(0,0){${#2}$}}
\end{picture}}}

% h enh. pmos reversed
\newcommand{\epmosr}[2]
 {\put(#1){\begin{picture}(50,50)
 \put(0,0){\line(-1,0){27}}
 \put(-27,-5){\line(0,1){10}}
 \put(-30,-10){\line(0,1){20}}
 \put(-50,7.5){\vector(1,0){20}}
 \put(-30,-7.5){\line(-1,0){20}}
 \put(-50,-25){\line(0,1){17.5}}
 \put(-50,7.5){\line(0,1){17.5}}
 \put(-25,-20){\makebox(0,0){$#2$}}
\end{picture}}}

% h enh. pmosfet d down
\newcommand{\epmosdd}[2]
 {\put(#1){\begin{picture}(50,50)
 \put(0,0){\line(1,0){23}}
 \put(23,-5){\line(0,1){10}}
 \put(27,-10){\line(0,1){20}}
 \put(50,8){\vector(-1,0){23}}
 \put(27,-8){\line(1,0){23}}
% \put(50,-25){\line(1,0){0}}
% \put(50,25){\line(1,0){0}}
 \put(50,-25){\line(0,1){17}}
 \put(50,8){\line(0,1){17}}
 \put(25,-20){\makebox(0,0){$#2$}}
\end{picture}}}

% h fet (generic)
\newcommand{\fet}[2]
 {\put(#1){\begin{picture}(50,50)
 \put(30,0){\circle{30}}
 \put(30,-10){\line(0,1){20}}
 \put(50,7.5){\line(0,1){17.5}}
% \put(0,0){\line(0,1){25}}
 \put(50,-7.5){\line(0,-1){17.5}}
 \put(30,0){\line(-1,0){30}}
 \put(30,7.5){\line(1,0){20}}
 \put(30,-7.5){\line(1,0){20}}
 \put(35,-22){\makebox(0,0){${#2}$}}
% \put(0,25){\line(-1,0){10}}
% \put(50,25){\line(1,0){10}}
\end{picture}}}

% h benmos
\newcommand{\benmos}[2]
 {\put(#1){\begin{picture}(50,50)
 \put(30,-5){\line(0,1){5}}
 \put(30,2.5){\line(0,1){5}}
 \put(30,10){\line(0,1){5}}
 \put(27,0){\line(0,1){10}}
 \put(50,12.5){\line(0,1){12.5}}
 \put(50,-2.5){\line(0,-1){22.5}}
 \put(0,0){\line(1,0){27}}
 \put(30,12.5){\line(1,0){20}}
 \put(30,-2.5){\line(1,0){20}}
 \put(50,5){\vector(-1,0){20}}
 \put(25,-22){\makebox(0,0){${#2}$}}
\end{picture}}}

% h bdnmos
\newcommand{\bdnmos}[2]
 {\put(#1){\begin{picture}(50,50)
 \put(30,-5){\line(0,1){20}}
 \put(27,0){\line(0,1){10}}
 \put(50,12.5){\line(0,1){12.5}}
 \put(50,-2.5){\line(0,-1){22.5}}
 \put(0,0){\line(1,0){27}}
 \put(30,12.5){\line(1,0){20}}
 \put(30,-2.5){\line(1,0){20}}
 \put(50,5){\vector(-1,0){20}}
 \put(25,-22){\makebox(0,0){${#2}$}}
\end{picture}}}

% h bepmos
\newcommand{\bepmos}[2]
 {\put(#1){\begin{picture}(50,50)
 \put(30,-5){\line(0,1){5}}
 \put(30,2.5){\line(0,1){5}}
 \put(30,10){\line(0,1){5}}
 \put(27,0){\line(0,1){10}}
 \put(50,12.5){\line(0,1){12.5}}
 \put(50,-2.5){\line(0,-1){22.5}}
 \put(0,0){\line(1,0){27}}
 \put(30,12.5){\line(1,0){20}}
 \put(30,-2.5){\line(1,0){20}}
 \put(30,5){\vector(1,0){20}}
 \put(25,-22){\makebox(0,0){${#2}$}}
\end{picture}}}

% h bdpmos
\newcommand{\bdpmos}[2]
 {\put(#1){\begin{picture}(50,50)
 \put(30,-5){\line(0,1){20}}
 \put(27,0){\line(0,1){10}}
 \put(50,12.5){\line(0,1){12.5}}
 \put(50,-2.5){\line(0,-1){22.5}}
 \put(0,0){\line(1,0){27}}
 \put(30,12.5){\line(1,0){20}}
 \put(30,-2.5){\line(1,0){20}}
 \put(30,5){\vector(1,0){20}}
 \put(25,-22){\makebox(0,0){${#2}$}}
\end{picture}}}

% h njfet Millman
\newcommand{\njfetm}[2]
 {\put(#1){\begin{picture}(50,50)
 \put(30,-10){\line(0,1){20}}
 \put(50,7.5){\line(0,1){17.5}}
 \put(50,-7.5){\line(0,-1){17.5}}
 \put(0,0){\vector(1,0){30}}
 \put(30,7.5){\line(1,0){20}}
 \put(30,-7.5){\line(1,0){20}}
 \put(25,-22){\makebox(0,0){${#2}$}}
\end{picture}}}

% h njfet Rizzoni
\newcommand{\njfetr}[2]
 {\put(#1){\begin{picture}(50,50)
 \put(30,-10){\line(0,1){20}}
 \put(27,-5){\line(0,1){10}}
 \put(50,7.5){\line(0,1){17.5}}
 \put(50,-7.5){\line(0,-1){17.5}}
 \put(0,0){\vector(1,0){27}}
 \put(30,7.5){\line(1,0){20}}
 \put(30,-7.5){\line(1,0){20}}
 \put(25,-22){\makebox(0,0){${#2}$}}
\end{picture}}}

% h opa + up
\newcommand{\hoa}[2]
 {\put(#1){\begin{picture}(50,50)
 \put(0,0){\line(0,1){15}}
 \put(0,35){\line(0,1){15}}
 \put(0,15){\line(1,0){10}}
 \put(0,35){\line(1,0){10}}
 \put(40,25){\line(1,0){10}}
 \put(10,7){\line(0,1){36}}
 \put(10,7){\line(5,3){30}}
 \put(10,43){\line(5,-3){30}}
 \put(15,33){\makebox(0,0){$+$}}
 \put(15,17){\makebox(0,0){$-$}}
 \put(25,25){\makebox(0,0){${#2}$}}
\end{picture}}}

% h opa - up
\newcommand{\ho}[2]
 {\put(#1){\begin{picture}(50,50)
 \put(0,0){\line(0,1){15}}
 \put(0,35){\line(0,1){15}}
 \put(0,15){\line(1,0){10}}
 \put(0,35){\line(1,0){10}}
 \put(40,25){\line(1,0){10}}
 \put(10,7){\line(0,1){36}}
 \put(10,7){\line(5,3){30}}
 \put(10,43){\line(5,-3){30}}
 \put(15,33){\makebox(0,0){$-$}}
 \put(15,17){\makebox(0,0){$+$}}
 \put(25,25){\makebox(0,0){$#2$}}
\end{picture}}}

% h opa high (+ up)
\newcommand{\hop}[2]
 {\put(#1){\begin{picture}(50,100)
 \put(0,0){\line(0,1){40}}
 \put(0,60){\line(0,1){40}}
 \put(0,40){\line(1,0){10}}
 \put(0,60){\line(1,0){10}}
 \put(40,50){\line(1,0){10}}
 \put(10,32){\line(0,1){36}}
 \put(10,32){\line(5,3){30}}
 \put(10,68){\line(5,-3){30}}
 \put(15,58){\makebox(0,0){$+$}}
 \put(15,42){\makebox(0,0){$-$}}
 \put(25,50){\makebox(0,0){${#2}$}}
\end{picture}}}

% h opa high (- up)
\newcommand{\hopi}[2]
 {\put(#1){\begin{picture}(50,100)
 \put(0,0){\line(0,1){40}}
 \put(0,60){\line(0,1){40}}
 \put(0,40){\line(1,0){10}}
 \put(0,60){\line(1,0){10}}
 \put(40,50){\line(1,0){10}}
 \put(10,32){\line(0,1){36}}
 \put(10,32){\line(5,3){30}}
 \put(10,68){\line(5,-3){30}}
 \put(15,58){\makebox(0,0){$-$}}
 \put(15,42){\makebox(0,0){$+$}}
 \put(25,50){\makebox(0,0){${#2}$}}
\end{picture}}}

% h opa reversed - up
\newcommand{\hoar}[2]
 {\put(#1){\begin{picture}(50,50)
 \put(0,0){\line(0,1){15}}
 \put(0,35){\line(0,1){15}}
 \put(0,15){\line(-1,0){10}}
 \put(0,35){\line(-1,0){10}}
 \put(-40,25){\line(-1,0){10}}
 \put(-10,7){\line(0,1){36}}
 \put(-10,7){\line(-5,3){30}}
 \put(-10,43){\line(-5,-3){30}}
 \put(-15,33){\makebox(0,0){$-$}}
 \put(-15,17){\makebox(0,0){$+$}}
 \put(-25,25){\makebox(0,0){${#2}$}}
\end{picture}}}

% h opa output down
\newcommand{\hod}[2]
 {\put(#1){\begin{picture}(50,50)
 \put(0,0){\line(1,0){15}}
 \put(35,0){\line(1,0){15}}
 \put(35,-10){\line(0,1){10}}
 \put(15,-10){\line(0,1){10}}
 \put(25,-50){\line(0,1){10}}
 \put(7,-10){\line(1,0){36}}
 \put(7,-10){\line(3,-5){18}}
 \put(43,-10){\line(-3,-5){18}}
 \put(15,-15){\makebox(0,0){$+$}}
 \put(35,-15){\makebox(0,0){$-$}}
 \put(25,-22){\makebox(0,0){$#2$}}
\end{picture}}}

% h opa output up
\newcommand{\hou}[2]
 {\put(#1){\begin{picture}(50,50)
 \put(0,0){\line(1,0){15}}
 \put(35,0){\line(1,0){15}}
 \put(35,0){\line(0,1){10}}
 \put(15,0){\line(0,1){10}}
 \put(25,40){\line(0,1){10}}
 \put(7,10){\line(1,0){36}}
 \put(7,10){\line(3,5){18}}
 \put(43,10){\line(-3,5){18}}
 \put(15,15){\makebox(0,0){$-$}}
 \put(35,15){\makebox(0,0){$+$}}
 \put(25,23){\makebox(0,0){$#2$}}
\end{picture}}}

% h AND-port
\newcommand{\hand}[1]
 {\put(#1){\begin{picture}(50,50)
 \put(0,-5){\framebox(20,35){$\&$}}
 \put(20,12.5){\line(1,0){10}}
\end{picture}}}

% h NAND-port
\newcommand{\hnand}[1]
 {\put(#1){\begin{picture}(50,50)
 \put(22.5,12.5){\circle{5}}
 \put(0,-5){\framebox(20,35){$\&$}}
 \put(25,12.5){\line(1,0){5}}
\end{picture}}}

% h OR-port
\newcommand{\hor}[1]
 {\put(#1){\begin{picture}(50,50)
 \put(20,12.5){\line(1,0){10}}
 \put(0,-5){\framebox(20,35){$\geq 1$}}
 \put(20,12.5){\line(1,0){10}}
\end{picture}}}

% h NOR-port
\newcommand{\hnor}[1]
 {\put(#1){\begin{picture}(50,50)
 \put(22.5,12.5){\circle{5}}
 \put(25,12.5){\line(1,0){5}}
 \put(0,-5){\framebox(20,35){$\geq 1$}}
\end{picture}}}

% h XOR-port
\newcommand{\hxor}[1]
 {\put(#1){\begin{picture}(50,50)
 \put(20,12.5){\line(1,0){10}}
 \put(0,-5){\framebox(20,35){$=1$}}
\end{picture}}}

% h XNOR-port
\newcommand{\hxnor}[1]
 {\put(#1){\begin{picture}(50,50)
 \put(22.5,12.5){\circle{5}}
 \put(25,12.5){\line(1,0){5}}
 \put(0,-5){\framebox(20,35){$=1$}}
\end{picture}}}

% h NOT-port
\newcommand{\hnot}[1]
 {\put(#1){\begin{picture}(50,50)
 \put(22.5,12.5){\circle{5}}
 \put(25,12.5){\line(1,0){5}}
 \put(0,-5){\framebox(20,35){}}
\end{picture}}}

% h SR flip-flop
\newcommand{\sr}[1]
{\put(#1){\begin{picture}(50,60)
\put(10,-10){\framebox(30,50){}}
\txt{16,0}{R}
\txt{16,30}{S}
%\txt{11,15}{\overline{CK}}
\hln{0,15}{10}
\put(13,15){\makebox(0,0){$>$}}
\put(34,0){\makebox(0,0){$\overline{Q}$}}
\put(34,30){\makebox(0,0){$Q$}}
\hln{0,30}{10}
\hln{0,0}{10}
\hln{40,30}{10}
\hln{40,0}{10}
\end{picture}}}

% h JK flip-flop with S and R inverted
\newcommand{\jk}[1]
{\put(#1){\begin{picture}(50,60)
\put(10,-10){\framebox(30,50){}}
\put(7.5,15){\circle{5}}
\put(25,42.5){\circle{5}}
\put(25,-12.5){\circle{5}}
\hln{0,15}{5}
\hln{0,30}{10}
\hln{0,0}{10}
\hln{40,30}{10}
\hln{40,0}{10}
\put(13,15){\makebox(0,0){$>$}}
\end{picture}}}

% h JK flip-flop
\newcommand{\jkff}[1]
{\put(#1){\begin{picture}(50,60)
\put(10,-10){\framebox(30,50){}}
\put(7.5,15){\circle{5}}
\hln{0,15}{5}
\hln{0,30}{10}
\hln{0,0}{10}
\hln{40,30}{10}
\hln{40,0}{10}
\put(13,15){\makebox(0,0){$>$}}
\end{picture}}}

% h D flip-flop
\newcommand{\dff}[1]
{\put(#1){\begin{picture}(50,60)
\put(10,-10){\framebox(30,50){}}
\hln{0,30}{10}
\hln{0,0}{10}
\hln{40,30}{10}
\hln{40,0}{10}
\put(13,0){\makebox(0,0){$>$}}
\end{picture}}}

% h NOT-port with a triangle
\newcommand{\hnott}[1]
 {\put(#1){\begin{picture}(50,50)
 \put(20,4){\line(3,-2){6}}
 \put(20,0){\line(1,0){10}}
 \put(0,-17.5){\framebox(20,35){}}
\end{picture}}}

% h NOT-port small
\newcommand{\hnots}[1]
 {\put(#1){\begin{picture}(50,50)
 \put(22.5,0){\circle{5}}
 \put(25,0){\line(1,0){5}}
 \put(0,-7.5){\framebox(20,15){}}
\end{picture}}}

% h twoport
\newcommand{\htp}[2]
 {\put(#1){\begin{picture}(100,74)
 \put(15,-5){\framebox(70,60)}
 \put(0,0){\line(1,0){15}}
 \put(0,50){\line(1,0){15}}
 \put(85,0){\line(1,0){15}}
 \put(85,50){\line(1,0){15}}
 \put(50,25){\makebox(0,0){$#2$}}
\end{picture}}}

% h short twoport
\newcommand{\hstp}[2]
 {\put(#1){\begin{picture}(50,74)
 \put(10,-5){\framebox(30,60)}
 \put(0,0){\line(1,0){10}}
 \put(0,50){\line(1,0){10}}
 \put(40,0){\line(1,0){10}}
 \put(40,50){\line(1,0){10}}
 \put(25,25){\makebox(0,0){$#2$}}
\end{picture}}}

% h transmission line
\newcommand{\htl}[2]
 {\put(#1){\begin{picture}(100,60)
 \put(15,-5){\framebox(70,10)}
 \put(15,45){\framebox(70,10)}
 \put(0,0){\line(1,0){15}}
 \put(0,50){\line(1,0){15}}
 \put(85,0){\line(1,0){15}}
 \put(85,50){\line(1,0){15}}
 \put(50,25){\makebox(0,0){$#2$}}
\end{picture}}}

% h transmission line stub left
\newcommand{\hs}[2]
 {\put(#1){\begin{picture}(50,60)
 \put(-40,7){\framebox(30,10)}
 \put(-40,33){\framebox(30,10)}
 \put(0,12){\line(-1,0){10}}
 \put(0,38){\line(-1,0){10}}
 \put(0,0){\line(0,1){12}}
 \put(0,38){\line(0,1){12}}
 \put(-40,12){\line(-1,0){10}}
 \put(-40,38){\line(-1,0){10}}
 \put(-25,25){\makebox(0,0){${#2}$}}
\end{picture}}}

% h transmission line stub right
\newcommand{\hsr}[2]
 {\put(#1){\begin{picture}(50,60)
 \put(10,7){\framebox(30,10)}
 \put(10,33){\framebox(30,10)}
 \put(0,12){\line(1,0){10}}
 \put(0,38){\line(1,0){10}}
 \put(0,0){\line(0,1){12}}
 \put(0,38){\line(0,1){12}}
 \put(40,12){\line(1,0){10}}
 \put(40,38){\line(1,0){10}}
 \put(25,25){\makebox(0,0){$#2$}}
\end{picture}}}

% v transmission line
\newcommand{\vtl}[2]
 {\put(#1){\begin{picture}(60,50)
 \put(-5,10){\framebox(10,30)}
 \put(45,10){\framebox(10,30)}
 \put(0,0){\line(0,1){10}}
 \put(0,40){\line(0,1){10}}
 \put(50,0){\line(0,1){10}}
 \put(50,40){\line(0,1){10}}
 \put(25,25){\makebox(0,0){$#2$}}
\end{picture}}}

% h = horizontal, v = vertical, r = reversed

% General amplifier
\newcommand{\amp}[2]
 {\put(#1){\begin{picture}(50,50)
% \put(0,0){\line(0,1){15}}
\put(0,25){\line(1,0){10}}
% \put(0,35){\line(0,1){15}}
% \put(0,15){\line(1,0){10}}
% \put(0,35){\line(1,0){10}}
 \put(40,25){\line(1,0){10}}
 \put(10,7){\line(0,1){36}}
 \put(10,7){\line(5,3){30}}
 \put(10,43){\line(5,-3){30}}
% \put(15,33){\makebox(0,0){$-$}}
% \put(15,17){\makebox(0,0){$+$}}
 \put(25,25){\makebox(0,0){$#2$}}
\end{picture}}}

% An operational amplifier with supply voltages
\newcommand{\vo}[3]
 {\put(#1){\begin{picture}(50,50)(0,15)
 %\put(0,0){\line(0,1){15}}
 %\put(0,35){\line(0,1){15}}
 \put(0,15){\line(1,0){10}}
 \put(0,35){\line(1,0){10}}
 \put(40,25){\line(1,0){10}}
 \put(10,7){\line(0,1){36}}
 \put(10,7){\line(5,3){30}}
 \put(10,43){\line(5,-3){30}}
 \put(15,33){\makebox(0,0){$-$}}
 \put(15,17){\makebox(0,0){$+$}}
 \put(25,25){\makebox(0,0){$#2$}}
 % K‰yttikset
 \put(25,8){\line(0,1){8}}
 \put(25,34){\line(0,1){8}}
 \put(29,2){\makebox(0,0){\tiny$-#3\V$}}
 \put(29,47){\makebox(0,0){\tiny$+#3\V$}}
\end{picture}}}

% An operational amplifier with supply voltages, inverted
\newcommand{\voi}[3]
 {\put(#1){\begin{picture}(50,50)(0,15)
 %\put(0,0){\line(0,1){15}}
 %\put(0,35){\line(0,1){15}}
 \put(0,15){\line(1,0){10}}
 \put(0,35){\line(1,0){10}}
 \put(40,25){\line(1,0){10}}
 \put(10,7){\line(0,1){36}}
 \put(10,7){\line(5,3){30}}
 \put(10,43){\line(5,-3){30}}
 \put(15,33){\makebox(0,0){$+$}}
 \put(15,17){\makebox(0,0){$-$}}
 \put(25,25){\makebox(0,0){$#2$}}
 % K‰yttikset
 \put(25,8){\line(0,1){8}}
 \put(25,34){\line(0,1){8}}
 \put(29,2){\makebox(0,0){\tiny$-#3\V$}}
 \put(29,47){\makebox(0,0){\tiny$+#3\V$}}
\end{picture}}}

% An operational amplifier with supply voltages, reversed
\newcommand{\vor}[3]
 {\put(#1){\begin{picture}(50,50)(0,15)
 %\put(0,0){\line(0,1){15}}
 %\put(0,35){\line(0,1){15}}
 \put(0,15){\line(-1,0){10}}
 \put(0,35){\line(-1,0){10}}
 \put(-40,25){\line(-1,0){10}}
 \put(-10,7){\line(0,1){36}}
 \put(-10,7){\line(-5,3){30}}
 \put(-10,43){\line(-5,-3){30}}
 \put(-15,33){\makebox(0,0){$+$}}
 \put(-15,17){\makebox(0,0){$-$}}
 \put(-25,25){\makebox(0,0){${#2}$}}

 % K‰yttikset
 \put(-25,8){\line(0,1){8}}
 \put(-25,34){\line(0,1){8}}
 \put(-29,2){\makebox(0,0){\tiny$-#3\V$}}
 \put(-29,47){\makebox(0,0){\tiny$+#3\V$}}

\end{picture}}}

% Node name
\newcommand{\node}[2]
 {\put(#1){\begin{picture}(10,50)
 \thinlines \put(0,0){\circle{7}}
 \put(0,0){\makebox(1,0){\tiny #2}}
\end{picture}}}

% A rectifier bridge, REQUIRES \usepackage{rotating}
\newcommand{\bridge}[1]{

\put(#1){\rotatebox{-45}{\rd{0,0}{}}}
\put(#1){\rotatebox{45}{\rd{-50,50}{}}}
\put(#1){\rotatebox{135}{\ld{-50,-50}{}}}
\put(#1){\rotatebox{-135}{\ld{-50,50}{}}}
}

% v transformer (in power engineering diagrams)
\newcommand{\htf}[2]
 {\put(#1){\begin{picture}(50,32)
 \put(20,0){\circle{20}}
 \put(30,0){\circle{20}}
 \put(0,0){\line(1,0){10}}
 \put(40,0){\line(1,0){10}}
 \put(25,-19){\makebox(0,0){$#2$}}
\end{picture}}}

\newcommand{\vtf}[2]
 {\put(#1){\begin{picture}(32,50)
 \put(0,20){\circle{20}}
 \put(0,30){\circle{20}}
 \put(0,0){\line(0,1){10}}
 \put(0,40){\line(0,1){10}}
 \put(-13,25){\makebox(0,0)[r]{$#2$}}
\end{picture}}}

% Vesan pikamakroja
\newcommand{\kohm}{\,\mathrm{k}\Omega}
\newcommand{\mohm}{\,\mathrm{M}\Omega}
\newcommand{\ohm}{\,\Omega}
\newcommand{\V}{\,\mathrm{V}}
\newcommand{\A}{\,\mathrm{A}}
\newcommand{\Siemens}{\,\mathrm{S}}
\newcommand{\mA}{\,\mathrm{mA}}
\newcommand{\Hz}{\,\mathrm{Hz}}
\newcommand{\kHz}{\,\mathrm{kHz}}
\newcommand{\Uin}{U_\mathrm{in}}
\newcommand{\Uout}{U_\mathrm{out}}
\newcommand{\al}[2]{\parbox{20pt}{$#1$\\$#2$}}
\newcommand{\ad}[2]{\vspace{-10pt}\parbox{20pt}{$#1$\\$#2$}}
\newcommand{\Umin}{U_\mathrm{min}}
\newcommand{\Umax}{U_\mathrm{max}}
\newcommand{\mV}{\,\mathrm{mV}}
\newcommand{\jj}{\mathrm{j}}
\newcommand{\nF}{\,\mathrm{nF}}
\newcommand{\ms}{\,\mathrm{ms}}
\newcommand{\us}{\,\mathrm{\upmu s}}
\newcommand{\uF}{\,\mathrm{\upmu F}}
\newcommand{\uA}{\,\mathrm{\upmu A}}
\newcommand{\Ur}[1]{U_\mathrm{#1}}
\newcommand{\Ua}{U_{\rm a}}
\newcommand{\Du}{\frac{{\rm d}u}{{\rm d}t}}
\newcommand{\Di}{\frac{{\rm d}i}{{\rm d}t}}
\newcommand{\lap}{{\mathcal L}}
\newcommand{\dt}{{\,\rm d} t}
\newcommand{\degree}{^\circ}
\newcommand{\HH}{\,\mathrm{H}}
\newcommand{\mH}{\,\mathrm{mH}}

%\newcommand{\Ub}{U_\mathrm{b}}
%\newcommand{\Ua1}{U_\mathrm{a1}}
%\newcommand{\Ub1}{U_\mathrm{b1}}
%\newcommand{\Ua2}{U_\mathrm{a2}}
%\newcommand{\Ub2}{U_\mathrm{b2}}




\usepackage{comment}
\usepackage{icomma}
\usepackage{hyperref}
\usepackage{import}


\usetheme[secheader]{Boadilla}
\setbeamertemplate{navigation symbols}{}

\begin{document}

\title{Circuit analysis: DC Circuits (3 cr)}
\subtitle{Fall 2009 / Class AS09}
\author{Vesa Linja-aho}
\institute{Metropolia}
\frame{
\titlepage
	\vfill
	\begin{center}


The slides are licensed with CC By 1.0. \url{http://creativecommons.org/licenses/by/1.0/} \\
Slideset version: 1.1

	\end{center}
}
\date{\today}
\frame{
\frametitle{Table of Contents}
Click the lecture name to jump onto the first slide of the lecture.
\vfill
\begin{columns}
\begin{column}{0.4\textwidth}
\tableofcontents[sections={1-6}]
\end{column}
\begin{column}{0.4\textwidth}
\tableofcontents[sections={7-12}]
\end{column}
\end{columns}
}

\section{1. lecture}

\frame {
\frametitle{About the Course}
\begin{itemize}
\item Lecturer: M.Sc. Vesa Linja-aho
\item Lectures on Mon 11:00-14:00 and Thu 14:00-16:30, room P113
\item To pass the course: Home assignments and final exam. The exam is on Monday 12th October 2009 at 11:00-14:00.
\item All changes to the schedule are announced in the Tuubi-portal.
\end{itemize}
 }
 
 \frame{
 \frametitle{The Home Assignments}
 \begin{itemize}
 \item There are 12 home assignments.
 \item Each assignment is graded with 0, 0,5 or 1 points.
 \item To pass the course, the student must have at least 4 points from the assignments.
 \item Each point exceeding the minimum of 4 points will give you 0,5 extra points in the exam. 
 \item In the exam, there are 5 assignments, with maximum of 6 points each.
 \item To pass the exam, you need to get 15 points from the exam.
 \item All other grade limits (for grades 2-5) are flexible.
 \end{itemize}
 \begin{exampleblock}{Example}
 The student has 8 points from the home assignments. He gets 13 points from the exam.
He will pass the exam, because he gets extra points from the home assignments and
his total score is  $(8-4)\cdot 0,5+13=15$ points.
 \end{exampleblock}
However, is one gets 8 of 12 points from the home assignments, he usually gets more
than 13 points from the exam :-).
 }

\frame{
\frametitle{The Course Objectives}
From the curriculum:
\begin{block}{Learning outcomes of the course unit}
\small Basic concepts and basic laws of electrical engineering. Analysis of direct current (DC) circuits.
\end{block}
\begin{block}{Course contents}
\small Basic concepts and basic laws of electrical engineering, analysis methods, controlled sources. Examples and exercises.
\end{block}
}



\frame {
  \frametitle{The Course Schedule}
\begin{enumerate}
\item The basic quantities and units. Voltage source and resistance. Kirchhoff's laws and Ohm's law.
\item Conductance. Electric power. Series and parallel circuits. Node. Ground.
\item Current source. Applying the Kirchhoff's laws to solve the circuit. Node-voltage analysis.
\item Exercises on node-voltage analysis.
\item Source transformation.
\item Thévenin equivalent and Norton equivalent.
\item Superposition principle.
\item Voltage divider and current divider.
\item Inductance and capacitance in DC circuits.
\item Controlled sources.
\item Recap.
\item Recap.
\end{enumerate}
}


\frame {
  \frametitle{The Course is Solid Ground for Further Studies in Electronics}
  The basic knowledge on DC circuits is needed on the courses {\em Circuit Analysis: Basic AC-Theory}, {\em Measuring Technology},
  {\em Automotive Electronics 1}, {\em Automotive Electrical Engineering Labs}, $\ldots$
  
  \begin{alertblock}{Important!}
   By studying this course well, {\bf studying the upcoming courses will be easier}!
  \end{alertblock}
  The basics of DC circuits are vital for automotive electronics engineer, just like the
  basics of accounting are vital for an auditor, and basics of strength of materials are vital for a bridge-building engineer etc.
}

\frame{
\frametitle{What is Not Covered on This Course}
The basic physical characteristics of electricity is not covered on this course. Questions
like "What is electricity?" are covered on the course {\em Rotational motion and electromagnetism}.
}

\frame{
\frametitle{Studying in Our School}
\begin{itemize}
\item You have an opportunity to learn on the lectures. I can not force you to learn.
\item You have more responsibility on your learning than you had in vocational school or senior high school.
\item 1 cr $\approx$ 26,7 hours of work. 3 cr = 80 hours of work. You will spend 39 hours on the lectures.
\item Which means that you should use about 40 hours of your own time for studying!
\item If I proceed too fast or too slow, please interject me (or tell me by email).
\item Do not hesitate to ask. Ask also the "stupid questions".
\end{itemize}

}

\frame{
\frametitle{What Is Easy and What Is Hard?}
Different things are hard for different people. But my own experience shows that
\begin{itemize}
\item DC analysis is easy, because the math involved is very basic.
\item DC analysis is hard, because the circuits are not as intuitive as, for example, mechanical
systems are.
\end{itemize}
Studying your math courses well is important for the upcoming courses on circuit analysis.
For example, in AC circuits analysis you have to use {\em complex arithmetics}.
}



\frame {
\frametitle{Now, Let's Get into Business}
Any questions on the practical arrangements of the course?
}

\frame {
\frametitle{Electric Current}
\begin{itemize}
\item Electric current is a flow of electric charge.
\item The unit for electric current is the ampere (A).
\item The abbreviation for the quantity is $I$.
\item One may compare the electric current with water flowing in a pipe (so called {\em hydraulic analogy}).
\item The current always circulates in a loop: current does not compress nor vanish.
\item The current in a wire is denoted like this:
\end{itemize}

\begin{center}
\begin{picture}(100,25)(0,0)

\hln{0,0}{100}
\ri{50,0}{I=2\mA}

\end{picture}
\end{center}
}

\frame {
\frametitle{Kirchhoff's Current Law}
\begin{itemize}
\item As mentioned on the previous slide, the current can not vanish anywhere.
\end{itemize}
\begin{block}{Kirchhoff's Current Law (or: Kirchhoff's First Law)}
At any area in an electrical circuit, the sum of currents flowing into that area is equal to the sum of currents flowing out of that area. 
\end{block}

\begin{center}
\begin{picture}(100,50)(0,0)

\hln{0,0}{100}
\hln{0,50}{100}
\vln{0,0}{50}
\vln{50,0}{50}
\vln{100,0}{50}

\ri{25,0}{I_1=3\mA}
\ri{75,0}{I_2=2\mA}
\ui{50,25}{I_3=1\mA}
\end{picture}
\end{center}
If you draw a circle in any place in the circuit, you can observe that there is as the same amount of current flowing into
the circle and out from the circle!
}


\frame {
\frametitle{Be Careful with Signs}
\begin{itemize}
\item One can say: "The balance of my account -50 euros" or equally "I owe 50 euros to my bank".
\item One can say: "The profit of the company was -500000 euros" or equally "The loss of the company was 500000 euros".
\item If you measure a current with an ammeter and it reads $-15 \mA$, by reversing the wires of the ammeter it
will show  $15 \mA$.
\item The sign of the current shows the direction of the current. The two circuits below are exactly identical.
\end{itemize}

\begin{picture}(100,50)(0,0)
\hln{0,0}{100}
\hln{0,50}{100}
\vln{0,0}{50}
\vln{50,0}{50}
\vln{100,0}{50}
\ri{25,0}{I_1=3\mA}
\ri{75,0}{I_2=2\mA}
\ui{50,25}{I_3=1\mA}
\end{picture}
\begin{picture}(100,50)(-50,0)
\hln{0,0}{100}
\hln{0,50}{100}
\vln{0,0}{50}
\vln{50,0}{50}
\vln{100,0}{50}
\li{25,0}{I_{\rm a}=-3\mA}
\li{75,0}{I_{\rm b}=-2\mA\hspace{-1cm}}
\ui{50,25}{I_3=1\mA}
\end{picture}


}

\frame {
\frametitle{Voltage}
\begin{itemize}
\item The potential difference between two points is called voltage.
\item The abbreviation for the quantity is $U$.
\item In circuit theory, it is insignificant how the potential difference is generated (chemically, by induction etc.).

\item The unit of voltage is the volt (V).
\item  One may compare the voltage with a pressure difference in hydraulic system, or to a difference in altitude.
\item Voltage is denoted with an arrow between two points.
\end{itemize}

\begin{center}
\begin{picture}(50,50)(0,0)
\vst{0,0}{12 \V}
\du{20,0}{U=12 \V}
\hln{0,0}{20}
\hln{0,50}{20}
\end{picture}
\end{center}
}

\frame {
\frametitle{Kirchhoff's voltage law}
\begin{itemize}
\item The voltage between two points is the same, regardless of the path chosen.
\item This is easy to understand by using the analogy of differences in altitude. If you
leave your home, go somewhere and return to your home, you have traveled uphill as much 
as you have traveled downhill.
\end{itemize}
\begin{block}{Kirchhoff's Voltage Law (or: Kirchhoff's Second Law)}
The directed {\bf sum of the voltages around any closed circuit is zero}. 
\end{block}


\begin{center}
\begin{picture}(100,50)(0,0)
\hst{0,0}{1,5 \V}
\hst{50,0}{1,5 \V}
\hst{100,0}{1,5 \V}
\vln{0,0}{50}
\vln{150,0}{50}
\hln{0,50}{50}
\hln{100,50}{50}
\lu{50,50}{4,5 \V}
\cn{50,50}
\cn{100,50}
\end{picture}
\end{center}
}

\frame{
\frametitle{Ohm's law}
\begin{itemize}
\item {\em Resistance } is a measure of the degree to which an object opposes an electric current through it.
\item The larger the current, the larger the voltage -- and vice versa.
\item The abbreviation of the quantity is $R$ and the unit is ($\ohm$) (ohm).
\item The definition of resistance is the ratio of the voltage over the element divided with
the current through the element. $R=U/I$
\end{itemize}

\begin{center}
$U=RI$

\begin{picture}(50,50)(0,0)

\hz{0,0}{R}
\ru{0,10}{U}
\ri{5,0}{I}
\hln{50,0}{15}
\hln{-15,0}{15}

\end{picture}
\end{center}
}

\frame {
  \frametitle{Definitions}
\begin{description}
\item[Electric circuit] A system consisting of compontents, in where electric current flows.
\item[Direct current (DC)] The electrical quantities (voltage and current) are constant (or nearly constant) over time.
\item[Direct current circuit] An electric circuit, where voltages and currents are constant over time.
\end{description}
\begin{exampleblock}{Example}
In a flashlight, there is a direct current circuit consisting of a battery/batteries, a switch and a bulb.
In a bicycle there is an alternating current circuit (dynamo and bulb).
\end{exampleblock}

}

\frame {
  \frametitle{An alternate definition for direct current}
One may define also that direct current means a current, which does not change its direction (sign),
but the magnitude of the current can vary over time. For example, a simple lead acid battery charger
outputs a pulsating voltage, which varies between 0 V ... $\approx$ 18 V. This can be also called DC voltage.

\begin{alertblock}{Agreement}
On this course, we define DC to mean constant voltage and current. The magnitude and sign
are constant over time.
\end{alertblock}
}


\frame{
\frametitle{A simple DC circuit}
\begin{itemize}
\item A light bulb is wired to a battery. The resistance of the filament is $10 \ohm$. 
\end{itemize}

\begin{center}
\begin{picture}(50,50)(0,0)
\vst{0,0}{12 \V}
\uncover<-1>{\vlamp{50,0}{}}
\uncover<2->{\vz{50,0}{10\ohm}}
\uncover<-3>{\ri{25,50}{I=?}}
\uncover<4->{\ri{25,50}{I=1,2\A}}
\hln{0,0}{50}
\hln{0,50}{50}
\uncover<3->{\du{65,0}{12\V}}
\end{picture}
\end{center}


\uncover<4->{$U=RI$\\
$I=\frac{U}{R}=\frac{12 \V}{10 \ohm}=1,2 \A$}
}

\begin{comment}
\frame{
\frametitle{Oppikirja}
Tällä luennolla käsiteltiin kirjan Kimmo Silvonen: {\em Sähkötekniikka ja elektroniikka} kappaleet:
\begin{description}
\item[1.1.1] Sähkövirta ja Kirchhoffin virtalaki
\item[1.1.3] Potentiaaliero ja Kirchhoffin jännitelaki
\item[1.2.1] Ohmin laki
\end{description}
Koska sivunumerointi saattaa vaihdella painoksittain, viittaan kappaleen numeroihin.

}
\end{comment}


\frame{
\frametitle{Homework 1 (released 31st Aug, to be returned 3rd Sep)}
\begin{itemize}
\item The homework are to be returned at the beginning of the next lecture.
\item Remember to include your name and student number.
\end{itemize}
\begin{block}{Homework 1}
Find the current $I$.
\end{block}
\begin{center}
\begin{picture}(50,100)(0,0)
\vst{0,0}{1,5 \V}
\vst{0,50}{1,5 \V}
\vz{50,25}{R=20\ohm\hspace{-2.5cm}}
\vln{50,0}{25}
\vln{50,75}{25}
\hln{0,100}{50}
\hln{0,0}{50}
\di{50,25}{I}
\end{picture}
\end{center}

}





\section{2. lecture}

\frame{
\frametitle{Homework 1 - Model solution}
\begin{block}{Homework 1}
Find the current $I$.
\end{block}
\begin{center}
\begin{picture}(50,100)(0,0)
\vst{0,0}{1,5 \V}
\vst{0,50}{1,5 \V}
\vz{50,25}{R=20\ohm\hspace{-2.5cm}}
\vln{50,0}{25}
\vln{50,75}{25}
\hln{0,100}{50}
\hln{0,0}{50}
\di{50,25}{I}
\uncover<2->{\du{12,0}{U_1} } 
\uncover<2->{\du{12,50}{U_2} }
\uncover<2->{\du{41,25}{\hspace{-0.65cm}U_{\rm R}} }
\end{picture}
\end{center}%\uncover<3->{\[ U_1+U_2-U_{\rm R}=0 \Leftrightarrow U_{\rm R}=U_1+U_2 \]}\uncover<4->{\[ U=RI \Rightarrow U_{\rm R}=RI \Rightarrow I=\frac{U_{\rm R}}{R}=\frac{U_1+U_2}{R}=\frac{1,5 \V+1,5 \V}{20\ohm}=150 \mA \]}
\uncover<2->{\[ U_1+U_2-U_{\rm R}=0 \Leftrightarrow U_{\rm R}=U_1+U_2 \]}
\uncover<3->{\vspace{-\baselineskip}
\[ U=RI \Rightarrow U_{\rm R}=RI \Rightarrow I=\frac{U_{\rm R}}{R}=\frac{U_1+U_2}{R}=\frac{1,5 \V+1,5 \V}{20\ohm}=150 \mA \]}
}


\frame{
\frametitle{Conductance}
\begin{itemize}
\item  Resistance is a measure of the degree to which an object opposes an electric current through it.
\item The inverse of resistance is {\bf conductance}. The symbol for conductance is 
 $G$ and the unit is Siemens (S).
\item Conductance measures how easily electricity flows along certain element.
\item For example, if resistance $R=10 \ohm$ then conductance $G=0,1\Siemens$.
\end{itemize}

\begin{center}
$G=\frac{1}{R} \qquad U=RI \Leftrightarrow GU=I$ 

\begin{picture}(50,50)(0,0)
\hz{0,0}{G=\frac{1}{R}}
\ru{0,10}{U}
\ri{5,0}{I}
\hln{50,0}{15}
\hln{-15,0}{15}
\end{picture}
\end{center}
}


\frame {
\frametitle{Electric Power}
\begin{itemize}
\item In physics, power is the rate at which work is performed.
\item The symbol for power is $P$ and the unit is the Watt (W).
\item The DC power consumed by an electric element is $P=UI$\\[1cm] \begin{picture}(50,10)(-30,-5) \hz{0,0}{} \ri{8,0}{I} \ru{0,10}{U} \end{picture}
\item If the formula outputs a positive power, the element is consuming power from the circuit. If the formula outputs a negative power, the element is delivering power to the circuit.
\end{itemize}
}
\frame{
\frametitle{Electric Power}
\begin{block}{Energy can not be created nor destroyed}
The power consumed by the elements in the circuit = the power delivered by the elements in the circuit.
\end{block}

\begin{center}
\begin{picture}(200,50)(0,0)
\vst{0,0}{E}
\vz{100,0}{R}
\hln{0,0}{100}
\hln{0,50}{100}
\ui{0,45}{I}
\di{100,42}{I}
\txt{200,50}{I=\frac{U}{R}}
\txt{200,25}{P_R=UI=U\frac{U}{R}=\frac{U^2}{R}}
\txt{200,0}{P_E=U\cdot(-I)=U\frac{-U}{R}=-\frac{U^2}{R}}
\end{picture}
\end{center}
The power delivered by the voltage source is consumed by the resistor.
}

\frame {
\frametitle{Series and Parallel Circuits}
\begin{alertblock}{Definition: series circuit}
The elements are in series, if they are connected so that the same current flows through the elements.
\end{alertblock}
\begin{alertblock}{Definition: parallel circuit}
The elements are in parallel, if they are connected so that there is the same voltage across them.
\end{alertblock}

}

\frame {
\frametitle{Series and Parallel Circuits}
\begin{exampleblock}{Series circuit}
\begin{center}
\begin{picture}(100,20)(0,-10)
\hz{0,0}{}
\hz{50,0}{}
\ri{-10,0}{I}
\ri{110,0}{I}
\hln{-50,0}{50}
\hln{100,0}{50}
\end{picture}
\end{center}
\end{exampleblock}

\begin{exampleblock}{Parallel Circuit}
\begin{center}
\begin{picture}(50,75)(0,-5)
\hz{0,0}{}
\hz{0,50}{}
\vln{0,0}{50}
\vln{50,0}{50}
\ru{0,10}{U}
\ru{0,60}{U}
\end{picture}
\end{center}
\end{exampleblock}
}

\frame {
\frametitle{Resistors in series and in parallel}
\begin{exampleblock}{In series}
\begin{center}
\begin{picture}(100,20)(50,-15)
\hz{0,0}{R_1}
\hz{50,0}{R_2}
\txt{125,0}{\Longleftrightarrow}
\hz{150,0}{R=R_1+R_2}
\end{picture}
\end{center}
\end{exampleblock}
\begin{exampleblock}{In parallel}
\begin{center}
\begin{picture}(100,75)(0,-20)
\hz{0,0}{R_1}
\hz{0,50}{R_2}
\hln{-25,25}{25}
\hln{50,25}{25}
\vln{0,0}{50}
\vln{50,0}{50}
\txt{110,25}{\Longleftrightarrow}
\hz{130,25}{\vspace{-1cm}R=\frac{1}{\frac{1}{R_1}+\frac{1}{R_2}}}
\end{picture}
\end{center}
\end{exampleblock}
Or, by using conductances: $G=G_1+G_2$.
}



\frame{
\frametitle{Resistors in series and in parallel}
\begin{itemize}
\item The formulae on the previous slide can be applied to an arbitrary number of resistors. For instance,
the total resistance of five resistors in series is $R=R_1+R_2+R_3+R_4+R_5$.
\end{itemize}
}

\frame{
\frametitle{Voltage Sources in Series}
\begin{itemize}
\item The voltages can be summed like resistances, but be careful with correct signs.
\item Voltage sources in parallel are inadmissible in circuit theory. There can not be two different
voltages between two nodes at the same time.
\end{itemize}
\begin{center}
\begin{picture}(100,25)(0,-30)
\hst{0,0}{E_1}
\hlst{50,0}{E_2}
\hst{100,0}{E_3}
\cn{0,0}
\cn{150,0}
\end{picture}

\begin{picture}(100,25)(0,0)
\txt{75,23}{\Longleftrightarrow}
\hst{50,0}{E=E_1-E_2+E_3}
\cn{50,0}
\cn{100,0}
\end{picture}
\end{center}
}

\frame{
\frametitle{What Series and Parallel Circuits are NOT}
\begin{itemize}
\item Just the fact that two components seem to be one after the other, does not mean that they are in series.
\item Just the fact that two components seem to be side by side, does not mean that they are in parallel.
\item In the figure below, which of the resistors are in parallel and which are in series with each other?
\end{itemize}
\begin{center}
\begin{picture}(100,50)(0,0)
\vst{0,0}{E_1}
\vst{100,0}{E_2}
\vz{50,0}{R_3}
\hz{0,50}{R_1}
\hz{50,50}{R_2}
\hln{0,0}{100}
\end{picture}
\end{center}
\pause
\vspace{-0.2cm}
\begin{alertblock}{Solution}
\scriptsize
None! $E_1$ ja $R_1$ are in series and $E_2$ ja $R_2$ are in series. Both of these serial circuits are in parallel with $R_3$. But no two resistors are in parallel nor in series.
\end{alertblock}
}


\frame{
\frametitle{Terminal and Gate}
\begin{itemize}
\item A point which provides a point of connection to external circuits is called a terminal (or pole).
\item Two terminals form a gate.
\item An easy example: a car battery with internal resistance. 
\end{itemize}
\begin{center}
\begin{picture}(100,50)(0,0)
\vst{0,0}{E}
\hz{0,50}{R_{\rm S}}
\hln{0,0}{50}
\out{50,0}
\out{50,50}
\end{picture}
\end{center}

}

\frame{
\frametitle{Node}
\begin{itemize}
\item {\bf A node} means an area in the circuit where there are no potential differences, or alternatively a place where two or more circuit elements meet.
\item A "for dummies" --way to find nodes in the circuit: put your pen on a wire in the circuit. Start coloring
the wire, and backtrack when your pen meets a circuit element. The area you colored is one node.
\item How many nodes are there in the circuit below?
\end{itemize}
\begin{center}
\begin{picture}(150,50)(0,0)
\vst{0,0}{E}
\hz{0,50}{R_1}
\hz{50,50}{R_3}
\hz{100,50}{R_5}
\vz{50,0}{R_2}
\vz{100,0}{R_4}
\vz{150,0}{R_6}
\hln{0,0}{150}
\ri{8,50}{I}
\end{picture}
%$R_1=R_2=R_3=R_4=R_5=R_6=1\ohm\qquad E=9\V$
\end{center}
}

\frame{
\frametitle{Ground}
\begin{itemize}
\item One of the nodes in the circuit can be appointed the ground node.
\item By selecting one of the nodes to be the ground node, the circuit diagram usually appear cleaner. 
\item The car battery is connected to the chassis of the car. Therefore it is convenient to handle the chassis
as the ground node.
\item When we say "the voltage of this node is 12 volts" it means that the voltage between that node
and the ground node is 12 volts.
\end{itemize}
\begin{center}
\begin{picture}(150,50)(0,0)
\vst{0,0}{E}
\hz{0,50}{R_1}
\hz{50,50}{R_3}
\hz{100,50}{R_5}
\vz{50,0}{R_2}
\vz{100,0}{R_4}
\vz{150,0}{R_6}
\hln{0,0}{150}
\ri{8,50}{I}
\hgp{0,0}
\cn{0,0}
\end{picture}
\end{center}
}


\frame{
\frametitle{Ground}
\begin{itemize}
\item The ground node can be connected to the chassis of the device or it can be leave not connected to the chassis.
\item Therefore, the existence of the ground node does not mean that the device is "grounded".
\item The circuit on the previous slide can be presented also like this:
\end{itemize}
\begin{center}
\begin{picture}(150,50)(0,0)
\vst{0,0}{E}
\hz{0,50}{R_1}
\hz{50,50}{R_3}
\hz{100,50}{R_5}
\vz{50,0}{R_2}
\vz{100,0}{R_4}
\vz{150,0}{R_6}
%\hln{0,0}{150}
\ri{8,50}{I}
\hgp{0,0}
\hgp{50,0}
\hgp{100,0}
\hgp{150,0}

%\cn{0,0}
\end{picture}
\end{center}
}


%\subsection{Viitteet oppikirjaan ja kotitehtävä}
\begin{comment}
\frame{
\frametitle{Oppikirja}
Tällä luennolla käsiteltiin kirjan Kimmo Silvonen: {\em Sähkötekniikka ja elektroniikka} kappaleet:
\begin{description}
\item[1.2.2] [Siemensin laki ja] konduktanssi eli johtokyky
\item[1.3.1] Tehon ja energiankulutuksen laskeminen
\item[1.4.1] Sarjaankytkentä
\item[1.4.2] Rinnankytkentä
\item[1.5.1] Vastusten sarjaankytkentä
\item[1.5.2] Vastusten rinnankytkentä
\item[1.4.5] Napa, portti, maa
\end{description}
Koska sivunumerointi saattaa vaihdella painoksittain, viittaan kappaleen numeroihin.

}
\end{comment}

\frame{
\frametitle{Homework 2 (released 3rd Sep, to be returned 7th Sep)}
\begin{block}{Homework 2}
Find the current $I$.
\end{block}
\begin{center}
\begin{picture}(150,50)(0,0)
\vst{0,0}{E}
\hz{0,50}{R_1}
\hz{50,50}{R_3}
\hz{100,50}{R_5}
\vz{50,0}{R_2}
\vz{100,0}{R_4}
\vz{150,0}{R_6}
\hln{0,0}{150}
\ri{8,50}{I}
\end{picture}
$R_1=R_2=R_3=R_4=R_5=R_6=1\ohm\qquad E=9\V$
\end{center}

}




\section{3. lecture}

\frame{
\frametitle{Homework 2 - Model solution}
\begin{block}{Homework 2}
Find the current $I$.
\end{block}
\begin{center}
\begin{picture}(150,50)(0,0)
\vst{0,0}{E}
\hz{0,50}{R_1}
\hz{50,50}{R_3}
\hz{100,50}{R_5}
\vz{50,0}{R_2}
\vz{100,0}{R_4}
\vz{150,0}{R_6}
\hln{0,0}{150}
\ri{8,50}{I}
\end{picture}
$R_1=R_2=R_3=R_4=R_5=R_6=1\ohm\qquad E=9\V$
\end{center}
\begin{itemize}
\item $R_5$ ja $R_6$ are in series. The total resistance of the serial connection is $R_5+R_6=2\ohm$.
\item Furthermore, the serial connection is in parallel with $R_4$. The resistance of this parallel circuit is
$\frac{1}{\frac{1}{1}+\frac{1}{2}}\ohm=\frac{2}{3}\ohm$.
\end{itemize}
}
\frame{
\frametitle{Solution continues}
\begin{itemize}
\item $R_3$ is in series with the parallel circuit calculated on the previous slide. The resistance for this circuit is $R_3+\frac{2}{3}\ohm
=\frac{5}{3}\ohm$.
\item And the serial connection is in parallel with $R_2$. The resistance for the parallel circuit is
$\frac{1}{(\frac{5}{3})^{-1}+\frac{1}{1}}=\frac{5}{8}\ohm$.
\item Lastly, $R_1$ is in series with the resistance computed in the previous step. Therefore, the total resistance seen by voltage source $E$ is
 $\frac{5}{8}\ohm + R_1=\frac{13}{8}\ohm$.
\item The current $I$ is computed from Ohm's law $I=\frac{E}{\frac{13}{8}\ohm}=\frac{72}{13}\A\approx5,5\A$.
\end{itemize}
}







\frame{
\frametitle{The Current Source}
\begin{itemize}
\item The current source is a circuit element which delivers a certain current throught it, just like the voltage source
keeps a certain voltage between its nodes.
\item The current can be constant or it can vary by some rule.
\end{itemize}
\begin{center}
\begin{picture}(100,50)(0,0)
\vj{0,0}{J}
\vz{50,0}{R}
\hln{0,0}{50}
\hln{0,50}{50}
\end{picture}
\end{center}
}

\frame{
\frametitle{The Current Source}
\begin{itemize}
\item If there is a current source in a wire, you know the current of that wire.
\end{itemize}
\begin{center}
\begin{picture}(100,50)(0,0)
\vj{0,0}{J=1\A}
\vz{50,0}{R_1}
\hln{0,0}{100}
\hln{0,50}{100}
\ri{25,50}{I=1\A}
\vz{100,0}{R_2}
\end{picture}
\end{center}
}

\frame{
\frametitle{Applying Kirchhoff's Laws Systematically to the Circuit}
When solving a circuit, it is highly recommended to use a systematic mehtod to find the voltages and/or currents.
Otherwise it is easy to end up with writing a bunch of equations which can not be solved.
One systematic method is called the {\bf nodal analysis}:
\begin{enumerate}
\item Name each current in the circuit.
\item Select one node as the ground node. Assign a variable for each voltage between each node and ground node.
\item Write an equation based on Kirchhoff's current law for each node (except the ground node).
\item State the voltage of each resistor by using the node voltage variables in step 2. Draw the voltage arrows
at the same direction you used for the current arrows (this makes it easier to avoid sign mistakes).
\item State every current by using the voltages and substitute them into the current equations in step 2.
\item Solve the set of equations to find the voltage(s) asked.
\item If desired, solve the currents by using the voltages you solved.
\end{enumerate}

}

\frame{
\frametitle{Example}
Find the current $I$.
\begin{center}
\begin{picture}(150,100)(0,0)
\vst{0,0}{E_1}
\vst{100,0}{E_2}
\vz{50,0}{R_3\hspace{-0.65cm}}
\hz{0,50}{R_1\vspace{0.8cm}}
\hz{50,50}{R_2\vspace{0.8cm}}
\di{50,1}{I}
\hln{0,0}{100}

\pause
\color{red}
\ri{48,50}{I_1}
\li{52,50}{}
\txt{55,60}{I_2}

\pause
\color{blue}
\hgp{25,0} 
\du{58,0}{U_3}


\pause
\color{violet}
\txt{160,25}{I=I_1+I_2}


\pause
\color{cyan}
\ru{0,60}{E_1-U_3}
\lu{50,60}{E_2-U_3}


\end{picture}
\end{center}
\pause
\[
\color{magenta} \frac{U_3}{R_3}=\frac{E_1-U_3}{R_1}+\frac{E_2-U_3}{R_2}
\pause
\color{orange} \Longrightarrow
U_3=R_3\frac{R_2E_1+R_1E_2}{R_1R_2+R_2R_3+R_1R_3}
\]
\pause
\[
\color{brown} I=\frac{U_3}{R_3}=\frac{R_2E_1+R_1E_2}{R_1R_2+R_2R_3+R_1R_3}
\]
}

\frame{
\frametitle{Some Remarks}
\begin{itemize}
\item There are many methods for writing the circuit equations, and there is no such thing as "right" method.
\item The only requirement is that you follow Kirchhoff's laws and Ohm's law\footnote{Ohm's law can only be utilized for
resistors. If you have other elements, you must know their current-voltage equation.}
 and you have an equal
number of equations and unknowns.
\item If there is a current source in the circuit, it will (usually) make the circuit easier to solve, as
you then have one unknown less to solve.
\item By using conductances instead of resistances, the equations look a little cleaner.
\end{itemize}
}

\frame{
\frametitle{Another Example}

\begin{center}
\begin{picture}(150,100)(0,0)
\vst{0,0}{E_1}
\vst{150,0}{E_2}
\vz{50,0}{R_3\hspace{-0.65cm}}
\hz{0,50}{R_1\vspace{0.8cm}}
\hz{50,50}{R_2\vspace{0.8cm}}
\hln{0,0}{150}
\vz{100,0}{R_4\hspace{-0.65cm}}
\hz{100,50}{R_5\vspace{0.8cm}}

\color{green}
\ri{8,50}{I_1}
\ri{58,50}{I_2}
\di{50,1}{I_3}
\ri{108,50}{I_5}
\di{100,1}{I_4}
\txt{200,25}{I_1=I_2+I_3}
\txt{200,10}{I_2=I_4+I_5}

\color{blue}
\du{58,0}{U_3}
\du{108,0}{U_4}


\end{picture}
\end{center}

\[\color{blue}
\frac{E_1-U_3}{R_1}=\frac{U_3-U_4}{R_2}+\frac{U_3}{R_3} \quad \mbox{ja}\quad
\frac{U_3-U_4}{R_2}=\frac{U_4}{R_4}+\frac{U_4-E_2}{R_5}
\]
\[\color{red}
G_1(E_1-U_3)=G_2(U_3-U_4)+G_3U_3\ \mbox{ja}\
G_2(U_3-U_4)=G_4U_4+G_5(U_4-E_2)
\]
Two equations, two unknowns $\to$ can be solved. Use conductances!
}


\frame{
\frametitle{Some Remarks}
\begin{itemize}
\item There are many other methods available too: mesh analysis, modified nodal analysis, branch current method \ldots
\item If there are ideal voltage sources in the circuit (=voltage sources which are connected to a node without a series resistance),
you need one more unknown (the current of the voltage source) and one more equation (the voltage source will determine the voltage between
the nodes it is connected to).
\end{itemize}
}


\frame{
\frametitle{Homework 3 (released 7th Sep, to be returned 10th Sep)}
\begin{block}{Homework 3a)}
Find the current $I_4$.
\end{block}
\begin{block}{Homework 3b)}
Verify your solution by writing down all the voltages and currents to the circuit diagram and checking that
the solution does not contradict Ohm's and Kirchhoff's laws.
\end{block}
\begin{center}
\begin{picture}(150,50)(0,0)
\vj{0,0}{J}
\hz{0,50}{R_1}
\hst{50,50}{E}
\hz{100,50}{R_4}
\vz{50,0}{R_2}
\vz{100,0}{R_3}
\vz{150,0}{R_5}
\hln{0,0}{150}
\di{100,1}{I_4}
\end{picture}
$R_1=R_2=R_3=R_4=R_5=1\ohm\qquad E=9\V\qquad J=1\A$
\end{center}

}


\section{4. lecture}
\frame{
\frametitle{Homework 3 - Model Solution}
\begin{block}{Homework 3a)}
Find current $I_4$.
\end{block}
\begin{block}{Homework 3b)}
Verify your solution by writing down all the voltages and currents to the circuit diagram and checking that
the solution does not contradict Ohm's and Kirchhoff's laws.
\end{block}
\begin{center}
\begin{picture}(150,50)(0,0)
\vj{0,0}{J}
\hz{0,50}{R_1}
%\hz{50,50}{R_3}
\hst{50,50}{E}
\hz{100,50}{R_4}
\vz{50,0}{R_2}
\vz{100,0}{R_3}
\vz{150,0}{R_5}
\hln{0,0}{150}
\di{100,1}{I_4}
\end{picture}
$R_1=R_2=R_3=R_4=R_5=1\ohm\qquad E=9\V\qquad J=1\A$
\end{center}
}

\frame {
  \frametitle{Solution}
\begin{center}
\begin{picture}(150,50)(0,0)
\vj{0,0}{J}
\hz{0,50}{R_1}
%\hz{50,50}{R_3}
\hst{50,50}{E}
\hz{100,50}{R_4}
\vz{50,0}{R_2}
\vz{100,0}{R_3}
\vz{150,0}{R_5}
\hln{0,0}{150}
\di{100,1}{I_4}
\color{red}
\du{57,0}{U_2}
\du{107,0}{U_3}
\hgp{100,0}
\ri{96,50}{I}
\end{picture}
$R_1=R_2=R_3=R_4=R_5=1\ohm\qquad E=9\V\qquad J=1\A$
\end{center}
First we write two current equations and one voltage equation. The conductance of the series
circuit formed by $R_4$ ja $R_5$ is denoted with $G_{45}$.
\begin{eqnarray*}
J&=&U_2G_2+I\\
I&=&U_3G_3+U_3G_{45} \\
U_2+E&=&U_3
\end{eqnarray*}
By substituting $I$ from the second equation to the first equation and then substituting $U_2$ from the third
equation, we get
\[
J=(U_3-E)G_2+U_3(G_3+G_{45})
\]
}
By substituting the component values, we get
\[
U_3=4 \V
\]

\begin{itemize}
\item Therefore the current $I$ is $4 \V \cdot 1 \Siemens=4\A$.
\item From the voltage equation $U_2+E=U_3$ we can solve $U_2=-5\V$, therefore the current through $R_2$ is $5\A$ upwards.
\item The current $I$ is therefore $1\A+5\A=6\A$, of which $4\A$ goes through $R_3$:n and the remaining $2\A$ goes through $R_4$ ja $R_5$.
\item There is no contradiction with Kirchhoff's laws and therefore we can be certain that our solution is correct.
\end{itemize}





\frame{
\frametitle{Example 1}
Find $I$ and $U$.
\begin{center}
\begin{picture}(150,100)(0,0)
\vst{0,0}{E_1}
\vst{50,0}{E_2}
\hst{50,100}{E_3}
\hz{0,50}{R_1}
\vz{100,0}{R_2}
\vj{150,0}{J}
\di{0,42}{I}
\du{110,0}{U}

\hln{0,0}{150}
\hln{100,100}{50}
\hln{0,100}{50}

\vln{0,50}{50}
\vln{100,50}{50}
\vln{150,50}{50}

\uncover<2->{\color{red} \li{50,100}{I_3}}
\end{picture}
\end{center}
\pause
\begin{eqnarray*}
J&=&UG_2+I_3\\
I_3&=&I+(E_1-E_2)G_1\\
U&=&E_1+E_3
\end{eqnarray*}
}


\frame{
\frametitle{Example 2}
Find $U_2$ and $I_1$.
\begin{center}
\begin{picture}(150,100)(0,0)
\vst{0,0}{E_1}
\vst{50,0}{E_2}
\vst{100,0}{E_3}
\hj{0,50}{J_1}
\hlj{50,50}{J_2}
\hz{25,100}{R}
\hln{0,0}{100}
\vln{0,50}{50}
\vln{100,50}{50}
\hln{0,100}{25}
\hln{75,100}{25}
\lu{50,67}{U_2}
\li{25,0}{I_1}

\end{picture}
\end{center}
\pause
\begin{eqnarray*}
I_1&=&(E_1-E_3)G+J_1\\
E_2+U_2&=&E_3
\end{eqnarray*}
}

\frame{
\frametitle{How To Get Extra Exercise?}
\begin{itemize}
\item There are plenty of problems with solutions available at \url{http://users.tkk.fi/~ksilvone/Lisamateriaali/lisamateriaali.htm}
\item For example, you can find 175 DC circuit problems at \url{http://users.tkk.fi/~ksilvone/Lisamateriaali/teht100.pdf}
\item At the end of the pdf file you can find the model solutions, so you can check your solution.
\item If you are enthusiastic, you can install and learn to use a circuit simulator:
\url{http://www.linear.com/designtools/software/ltspice.jsp}
\end{itemize}

}

\frame{
\frametitle{Example 3}
Find $U_4$.
\begin{center}
\begin{picture}(150,100)(0,0)
\vst{0,0}{E}
\vz{50,0}{R_2}
\vz{100,0}{R_4}
\hz{0,50}{R_1}
\hz{50,50}{R_3}
\hln{0,0}{100}
\du{110,0}{U_4}
\pause
\color{red}
\du{58,0}{U_2}
\hgp{25,0}
\end{picture}
\end{center}
\begin{eqnarray*}
(E-U_2)G_1&=&U_2G_2+(U_2-U_4)G_3\\
(U_2-U_4)G_3&=&G_4U_4
\end{eqnarray*}
}

\begin{comment}
\frame{
\frametitle{Example 4}
Find $U_4$.
\begin{center}
\begin{picture}(150,100)(0,0)
\vst{0,0}{E_1}
\vz{50,0}{R_2}
\vz{100,0}{R_4}
\hz{0,50}{R_1}
\hz{50,50}{R_3}
\hln{0,0}{50}
\hst{50,0}{E_2}
\hj{50,100}{J}
\vln{50,50}{50}
\vln{100,50}{50}
\cn{50,50}

\du{110,0}{U_4}

\pause
\end{picture}
\end{center}
\begin{eqnarray*}
(E-U_2)G_1&=&U_2G_2+(U_2-U_4)G_3\\
(U_2-U_4)G_3&=&G_4U_4
\end{eqnarray*}
}
\end{comment}

\frame{
\frametitle{Homework 4 (released 10th Sep, to be returned 14th Sep)}
\begin{block}{Homework 4}
Find the voltage $U_1$. All resistors have the value $10 \ohm$, $E=10\V$ ja $J=1\A$.
\end{block}

\begin{center}
\begin{picture}(150,50)(0,0)
\vj{0,0}{J}
\vz{50,0}{R_1}
\vz{100,0}{R_3}
\hz{50,50}{R_2}
\hz{50,0}{R_4}
\vst{150,0}{E}
\hln{0,0}{50}
\hln{100,0}{50}
\hln{0,50}{50}
\hln{100,50}{50}
\du{57,0}{U_1}
\end{picture}
\end{center}


}


\section{5. lecture}

\frame{
\frametitle{Homework 4 - Model Solution}
\begin{block}{Homework 4}
Find the voltage $U_1$. All resistors have the value $10 \ohm$, $E=10\V$ ja $J=1\A$.
\end{block}

\begin{center}
\begin{picture}(150,50)(0,0)
\vj{0,0}{J}
\vz{50,0}{R_1}
\vz{100,0}{R_3}
\hz{50,50}{R_2}
\hz{50,0}{R_4}
\vst{150,0}{E}
\hln{0,0}{50}
\hln{100,0}{50}
\hln{0,50}{50}
\hln{100,50}{50}
\du{57,0}{U_1}

\color{red}
\put(97,47){\vector(-1,-1){45}}
\stx{89,31}{$U_2$}
\lu{50,-19}{}
\txt{75,-25}{U_3}
\hgp{25,0}
\ri{125,50}{I}

\color{blue}
\ru{50,60}{U_1-U_2}
\du{110,0}{}
\stx{125,15}{$U_2-U_3$}
\end{picture}
\end{center}

\begin{eqnarray*}
J&=&U_1G_1+(U_1-U_2)G_2\\
(U_1-U_2)G_2&=&(U_2-U_3)G_3+I\\
G_3(U_2-U_3)+I&=&U_3G_4\\
U_2-U_3&=&E
\end{eqnarray*}

}


\frame{
\frametitle{Solution}
\begin{eqnarray*}
J&=&U_1G_1+(U_1-U_2)G_2\\
(U_1-U_2)G_2&=&EG_3+I\\
G_3E+I&=&U_3G_4\\
U_2-U_3&=&E
\end{eqnarray*}
$I$ is solved from the third equation and substituted into the second equation, then $U_3$ is solved from the equation and substituted.
\begin{eqnarray*}
J&=&U_1G_1+(U_1-U_2)G_2\\
(U_1-U_2)G_2&=&EG_3+(U_2-E)G_4-G_3E\\
\end{eqnarray*}
\vspace{-1cm}
\begin{eqnarray*}
1&=&0,2U_1-0,1U_2\\
0,1U_1-0,1U_2&=&0,1U_2-1
\end{eqnarray*}
}
\frame{
\frametitle{Solution}
\begin{eqnarray*}
1&=&0,2U_1-0,1U_2\\
0,1U_1-0,1U_2&=&0,1U_2-1
\end{eqnarray*}

Which is solved
\begin{eqnarray*}
U_1&=&10\\
U_2&=&10
\end{eqnarray*}
Therefore the voltage $U_1$ is 10 Volts. This is easy to verify with a circuit simulator.

}





\frame {
  \frametitle{Circuit Transformation}
\begin{enumerate}
\item An operation which transforms a part of the circuit into an internally different, but externally equally acting
circuit, is called a {\em circuit transformation}.
\item For example, combining series resistors or parallel resistors into one resistor, is a circuit transformation. Combining series voltage sources into one voltage source is a circuit transformation too.  
\item On this lecture, we learn dealing with parallel current sources and the {\em source transformation}, with which
we can transform a voltage source with series resistance into a current source with parallel resistance. 
\end{enumerate}
}


\frame {
\frametitle{An Example of a Circuit Transformation}
Two (or more) resistors are combined to a single resistor, which acts just like the original circuit of resistors.
\begin{exampleblock}{Resistors in series}
\begin{center}
\begin{picture}(100,20)(50,-15)
\hz{0,0}{R_1}
\hz{50,0}{R_2}
\txt{125,0}{\Longleftrightarrow}
\hz{150,0}{R=R_1+R_2}
\end{picture}
\end{center}
\end{exampleblock}
\begin{exampleblock}{Resistors in parallel}
\begin{center}
\begin{picture}(100,75)(0,-20)
\hz{0,0}{R_1}
\hz{0,50}{R_2}
\hln{-25,25}{25}
\hln{50,25}{25}
\vln{0,0}{50}
\vln{50,0}{50}
\txt{110,25}{\Longleftrightarrow}
\hz{130,25}{\vspace{-1cm}R=\frac{1}{\frac{1}{R_1}+\frac{1}{R_2}}}
\end{picture}
\end{center}
\end{exampleblock}
Or, by using conductance: $G=G_1+G_2$.
}

\frame {
\frametitle{Current Sources in Parallel}
One or more current sources are transformed into single current source, which acts just like the original parallel circuit of current sources.
\begin{exampleblock}{Current sources in parallel}
\begin{center}
\begin{picture}(270,50)(0,0)
\vj{0,0}{J_1}
\vj{50,0}{J_2}
\vdj{100,0}{J_3}
\hln{0,0}{150}
\hln{0,50}{150}
\out{150,0}
\out{150,50}

\txt{170,25}{\Longleftrightarrow}
\vj{200,0}{}
\txt{250,25}{J=J_1+J_2-J_3}
\hln{200,0}{50}
\hln{200,50}{50}
\out{250,0}
\out{250,50}
\end{picture}
\end{center}
\end{exampleblock}
Just like connecting two or more voltage sources in parallel, connecting current sources in series is an undefined (read: forbidden) operation in circuit theory, just like divide by zero is undefined in mathematics. There can not be two
currents in one wire!
}

\frame {
\frametitle{The Source Transformation}
{\bf A voltage source with series resistance} acts just like {\bf current source with parallel resistance}.
\begin{exampleblock}{The source transformation}
\begin{center}
\begin{picture}(150,55)(0,0)
\vst{0,0}{E}
\hz{0,50}{R}
\hln{0,0}{50}
\out{50,0}
\out{50,50}

\txt{66,25}{\Longleftrightarrow}

\vj{100,0}{J}
\vz{150,0}{R}
\hln{100,0}{75}
\hln{100,50}{75}
\out{175,0}
\out{175,50}
\color{red}
\txt{200,25}{E=RJ}
\end{picture}
\end{center}
\end{exampleblock}
}

\frame{
\frametitle{Important}
\begin{itemize}
\item Note that an ideal voltage or current source can not be transformed like in previous slide. The voltage source to
be transformed must have series resistance and the current source must have parallel resistance. 
\item The resistance remains the same, and the value for the source is found from formula $E=RJ$, which is 
based on Ohm's law.
\item The source transform is not just a curiosity. It can save from many lines of manual calculations, for example when analyzing a transistor amplifier.
\end{itemize}
}

\frame {
\frametitle{Rationale for the Source Transformation}

\begin{exampleblock}{The source transformation}
\begin{center}
\begin{picture}(250,55)(0,0)
\vst{0,0}{E}
\hz{0,50}{R}
\hln{0,0}{50}
\hstp{50,0}{}
\ri{55,50}{I}
\du{50,0}{U}


\vj{150,0}{\frac{E}{R}}
\vz{183,0}{R}
\hln{150,0}{50}
\hln{150,50}{50}
\hstp{200,0}{}
\ri{205,50}{I}
\du{200,0}{U}
\color{red}
\end{picture}
\end{center}
\end{exampleblock}

In the figure left:
\[
I=\frac{E-U}{R}\qquad U=E-RI
\]
In the figure right:
\[
I=\frac{E}{R}-\frac{U}{R}=\frac{E-U}{R}\qquad U=(\frac{E}{R}-I)R=E-RI
\]
Both the circuits function equally.
}

\frame{
\frametitle{Example}
Solve $U$.
\begin{center}
\begin{picture}(150,50)(0,0)
\vst{0,0}{E_1}
\hz{0,50}{R_1}
\vz{50,0}{R_3}
\hz{50,50}{R_2}
\vst{100,0}{E}
\hln{0,0}{100}
\du{57,0}{U}
\end{picture}
\end{center}

The circuit is transformed
\begin{center}
\begin{picture}(150,50)(0,0)
\vj{0,0}{J_1}
\vz{50,0}{R_1}
\vz{100,0}{R_3\hspace{-0.2cm}}
\vz{75,0}{R_2\hspace{-0.2cm}}
%\hz{50,0}{R_4}
\vj{150,0}{J_2}
\hln{0,0}{150}
%\hln{100,0}{50}
\hln{0,50}{150}
\hln{100,50}{50}
%\du{57,0}{U_1}
%\ri{57,50}{I}
\end{picture}
\end{center}
And we get the result:
\[
U=\frac{J_1+J_2}{G_1+G_2+G_3}
\]
}
\frame{
\frametitle{A Very Important Notice!}
\begin{itemize}
\item The value of the resistance remains the same, but the resistor is not the same resistor! For instance, in the
previous example the current through the original resistor is not same as current through the transformed resistor!
\end{itemize}

}

\begin{comment}
\frame{
\frametitle{Oppikirja}
Tällä luennolla käsiteltiin kirjan Kimmo Silvonen: {\em Sähkötekniikka ja elektroniikka} kappaleet:
\begin{description}
\item[1.7.1] Jännitelähde-virtalähde-muunnos
\end{description}
Koska sivunumerointi saattaa vaihdella painoksittain, viittaan kappaleen numeroihin.
}
\end{comment}


\frame{
\frametitle{Homework 5 (released 14th Sep, to be returned 17th Sep)}
\begin{block}{Homework 5}
Find current $I$ by using source transformation. $J_1=10\A$, $J_2=1\A$, 
$R_1=100\ohm$, $R_2=200\ohm$ ja $R_3=300\ohm$.
\end{block}

\begin{center}
\begin{picture}(150,50)(0,0)
\vj{0,0}{J_1}
\vz{50,0}{R_1}
\vz{100,0}{R_3}
\hz{50,50}{R_2}
%\hz{50,0}{R_4}
\vj{150,0}{J_2}
\hln{0,0}{150}
\hln{100,0}{50}
\hln{0,50}{50}
\hln{100,50}{50}
%\du{57,0}{U_1}
\ri{57,50}{I}
\end{picture}
\end{center}
This is easy and fast assignment. If you find yourself writing many lines of equations, you have done something wrong.
}

%LUENTO6
\section{6. lecture}

\frame{
\frametitle{Homework 5 - Model Solution}
\begin{block}{Homework 5}
Find current $I$ by using source transformation. $J_1=10\A$, $J_2=1\A$, 
$R_1=100\ohm$, $R_2=200\ohm$ ja $R_3=300\ohm$.
\end{block}

\begin{center}
\begin{picture}(150,50)(0,0)
\vj{0,0}{J_1}
\vz{50,0}{R_1}
\vz{100,0}{R_3}
\hz{50,50}{R_2}
%\hz{50,0}{R_4}
\vj{150,0}{J_2}
\hln{0,0}{150}
\hln{100,0}{50}
\hln{0,50}{50}
\hln{100,50}{50}
%\du{57,0}{U_1}
\ri{57,50}{I}
\end{picture}
\end{center}


\begin{center}
\begin{picture}(150,50)(0,0)
\vst{0,0}{R_1J_1}
\hz{0,50}{R_1}
\hz{100,50}{R_3}
\hz{50,50}{R_2}
%\hz{50,0}{R_4}
\vst{150,0}{R_3J_2}
\hln{0,0}{150}
%\hln{100,0}{50}
%\hln{0,50}{50}
%\hln{100,50}{50}
%\du{57,0}{U_1}
\ri{57,50}{I}
\end{picture}
\end{center}
\[
I=\frac{R_1J_1-R_3J_2}{R_1+R_2+R_3}=\frac{1000\V-300\V}{600\ohm}=\frac{7}{6}\A
\approx 1,17 \A.
\]
}




\frame {
  \frametitle{Thévenin's Theorem and Norton's Theorem}
\begin{itemize}
 \item So far we have learned the following circuit transformations: voltage sources in series, current sources in parallel, resistances in parallel and in series and
the source transformation.
 \item Thévenin's theorem and Norton's theorem relate to circuit transformations too.
 \item By Thévenin's and Norton's theorems an arbitrary circuit constisting of voltage sources, current sources and resistances
 can be transformed into a single voltage source with series resistance (Thévenin's equivalent) or a single current source with parallel resistance (Norton's equivalent).
\end{itemize}
}


\frame {
\frametitle{Thévenin's Theorem and Norton's Theorem}

\begin{exampleblock}{Thévenin's Theorem}
An arbitrary linear circuit with two terminals is electrically equivalent to a single voltage source and a single series resistor, called
Thévenin's equivalent.

\end{exampleblock}


\begin{exampleblock}{Norton's Theorem}
An arbitrary linear circuit with two terminals is electrically equivalent to a single current source and a single parallel resistor, called
Norton's equivalent.
\end{exampleblock}

}

\frame {
\frametitle{Calculating the Thévenin Equivalent}
\begin{center}
\begin{picture}(200,55)(0,0)
\vst{0,0}{E}
\hz{0,50}{R_1}
\vz{50,0}{R_2}
\hln{0,0}{60}
\hln{50,50}{10}
\out{60,0}
\out{60,50}

\txt{100,25}{\Longleftrightarrow}

\vst{150,0}{E_{\rm T}}
\hz{150,50}{R_{\rm T}}
\hln{150,0}{50}
\out{200,0}
\out{200,50}

\end{picture}
\end{center}
The voltage $E_{\rm T}$ in the Thévenin equivalent is solved simply by calculating the voltage between the terminals.
For solving $R_{\rm T}$, there are two ways:
\begin{itemize}
\item By turning off all independent (= non-controlled) sources in the circuit, and calculating the resistance between the terminals.
\item By calculating the short circuit current of the port and applying Ohm's law.
\end{itemize}
Independent source is a source, whose value does not depend on any other voltage or current in the circuit.
All sources we have dealt with for now, have been independent.
Controlled sources covered later in this course.
}


\frame {
\frametitle{Calculating the Thévenin Equivalent}
\begin{center}
\begin{picture}(200,55)(0,0)
\vst{0,0}{E}
\hz{0,50}{R_1}
\vz{50,0}{R_2}
\hln{0,0}{60}
\hln{50,50}{10}
\out{60,0}
\out{60,50}

\txt{100,25}{\Longleftrightarrow}

\vst{150,0}{E_{\rm T}}
\hz{150,50}{R_{\rm T}}
\hln{150,0}{50}
\out{200,0}
\out{200,50}

\end{picture}
\end{center}
The voltage at the port is found by calculating the current through the resistors and multiplying it with
$R_2$. The voltage at the port, called also the {\em idle voltage} of the port, is equal to  $E_{\rm T}$.
\[
E_{\rm T}=\frac{E}{R_1+R_2}R_2
\]

}
\frame {
\frametitle{Calculating the Thévenin Equivalent}
There are two ways to solve $R_{\rm T}$. Way 1: turn off all the (independent) sources, and calculate the
voltage at the port. A turned-off voltage source is a voltage source, whose voltage is zero, which is same
as just a wire:

\begin{center}
\begin{picture}(200,55)(0,0)
%\vst{0,0}{E}
\vln{0,0}{50}
\hz{0,50}{R_1}
\vz{50,0}{R_2}
\hln{0,0}{60}
\hln{50,50}{10}
\out{60,0}
\out{60,50}

\txt{100,25}{\Longleftrightarrow}

%\vst{150,0}{E_{\rm T}}
\vln{150,0}{50}
\hz{150,50}{R_{\rm T}}
\hln{150,0}{50}
\out{200,0}
\out{200,50}

\end{picture}
\end{center}
Now it is easy to solve the resistance between the nodes of the port:
$R_1$ ja $R_2$ are in parallel, and therefore the resistance is
\[
R_{\rm T}=\frac{1}{G_1+G_2}=\frac{R_1R_2}{R_1+R_2}.
\]
This method is usually simpler than the other way with short circuit current!
}
\frame {
\frametitle{Solving $R_{\rm T}$ by Using the Short-circuit Current}
There are two ways to solve $R_{\rm T}$. Way 2: short-circuit the port, and calculate the current
through the short-circuit wire. This current is called the {\em short-circuit current}:
\begin{center}
\begin{picture}(200,55)(0,0)
\vst{0,0}{E}
\vln{60,0}{50}
\hz{0,50}{R_1}
\vz{50,0}{R_2}
\hln{0,0}{60}
\hln{50,50}{10}
\out{60,0}
\out{60,50}

\txt{100,25}{\Longleftrightarrow}

\vst{150,0}{E_{\rm T}}
\vln{200,0}{50}
\hz{150,50}{R_{\rm T}}
\hln{150,0}{50}
\out{200,0}
\out{200,50}

\di{60,25}{I_{\rm K}}
\di{200,25}{I_{\rm K}}


\end{picture}
\end{center}
The value for the short-circuit current is
\[
I_{\rm K}=\frac{E}{R_1}
\]
and the resistance $R_{\rm T}$ is (by applying Ohm's law to the figure on the right):
\[
R_{\rm T}=\frac{E_{\rm T}}{I_{\rm K}}=\frac{E_{\rm T}}{\frac{E}{R_1}}=\frac{\frac{E}{R_1+R_2}R_2}{\frac{E}{R_1}}=
\frac{R_1R_2}{R_1+R_2}
\]

}


\frame {
\frametitle{Norton Equivalent}
The Norton equivalent is simply a Thévenin equivalent, which has been source transformed into a current source and
parallel resistance (or vice versa). The resistance has the same value in both equivalents. The value of the
current source is the same as the short-circuit current of the port.

\begin{center}
\begin{picture}(100,50)(0,0)
\vj{0,0}{J_{\rm N}}
\vz{50,0}{R_{\rm N}}
\hln{0,0}{60}
\hln{0,50}{60}
\out{60,0}
\out{60,50}
\end{picture}
\end{center}
}

\frame {
\frametitle{Example 1}
Calculate the Thévenin equivalent. All component values = 1.
\begin{center}
\begin{picture}(150,50)(0,0)
\vj{0,0}{J_1}
\vz{50,0}{R_1}
\vz{100,0}{R_2}
%\hz{50,50}{R_2}
\hst{50,50}{E}
%\hz{50,0}{R_4}
\out{150,0}
\out{150,50}

\hln{0,0}{150}
\hln{100,0}{50}
\hln{0,50}{50}
\hln{100,50}{50}
%\du{57,0}{U_1}
%\ri{57,50}{I}
\end{picture}
\end{center}
}

\frame {
\frametitle{Example 2}
Calculate the Thévenin equivalent. All component values = 1.
\begin{center}
\begin{picture}(150,50)(0,0)
\vst{0,0}{E}
\vz{50,0}{R_2}
\vj{100,0}{J}
%\hz{50,50}{R_2}
\hz{50,50}{R_3}
%\hz{50,0}{R_4}
\out{150,0}
\out{150,50}

\hln{0,0}{150}
\hln{100,0}{50}
%\hln{0,50}{50}
\hz{0,50}{R_1}
\hln{100,50}{50}
%\du{57,0}{U_1}
%\ri{57,50}{I}
\end{picture}
\end{center}
}

\begin{comment}
\frame{
\frametitle{Oppikirja}
Tällä luennolla käsiteltiin kirjan Kimmo Silvonen: {\em Sähkötekniikka ja elektroniikka} kappaleet:
\begin{description}
\item[1.9.3] Théveninin ja Nortonin teoreemat
\end{description}
Koska sivunumerointi saattaa vaihdella painoksittain, viittaan kappaleen numeroihin.
}
\end{comment}

\frame{
\frametitle{Homework 6  (released 17th Sep, to be returned 21th Sep)}
\begin{block}{Homework 6}
Calculate the Thévenin equivalent. All the component values are 1.
(Every resistance is $1 \ohm$ and the current source is $J_1=1\A$.)
\end{block}

\begin{center}
\begin{picture}(150,50)(0,0)
\vj{0,0}{J_1}
\vz{50,0}{R_1}
\vz{100,0}{R_3}
\hz{50,50}{R_2}
%\hz{50,0}{R_4}
\out{150,0}
\out{150,50}

\hln{0,0}{150}
\hln{100,0}{50}
\hln{0,50}{50}
\hln{100,50}{50}
%\du{57,0}{U_1}
%\ri{57,50}{I}
\end{picture}
\end{center}

}

%LUENTO7

\section{7. lecture}

\frame{
\frametitle{Homework 6 - Model Solution}
\begin{block}{Homework 6}
Calculate the Thévenin equivalent. All the component values are 1.
(Every resistance is $1 \ohm$ and the current source is $J_1=1\A$.)
\end{block}

\begin{center}
\begin{picture}(150,50)(0,0)
\vj{0,0}{J_1}
\vz{50,0}{R_1}
\vz{100,0}{R_3}
\hz{50,50}{R_2}
%\hz{50,0}{R_4}
\out{150,0}
\out{150,50}

\hln{0,0}{150}
\hln{100,0}{50}
\hln{0,50}{50}
\hln{100,50}{50}
%\du{57,0}{U_1}
%\ri{57,50}{I}
\end{picture}
\end{center}
First we solve the voltage $E_{\rm T}$. This can be done by using applying source transformation:
\begin{center}
\begin{picture}(150,50)(0,0)
\vst{0,0}{J_1R_1}
\hz{0,50}{R_1}
\vz{100,0}{R_3}
\hz{50,50}{R_2}
%\hz{50,0}{R_4}
\out{150,0}
\out{150,50}

\hln{0,0}{150}
\hln{100,0}{50}
%\hln{0,50}{50}
\hln{100,50}{50}
%\du{57,0}{U_1}
%\ri{57,50}{I}
\du{110,0}{E_{\rm T}=\frac{J_1R_1}{R_1+R_2+R_3}R_3=\frac{1}{3}\V}
\end{picture}
\end{center}

}

\frame{
\frametitle{Solution}
Next we solve the resistance $R_{\rm T}$ of the Thévenin equivalent. The easiest way to do it is to 
turn off all the sources and calculate the resistance between the output port. (The other way is to find out the short-circuit current.)
The resistance can be solver either from the original or the transformed circuit. Let's use the transformed circuit and
turn off the voltage source:
\begin{center}
\begin{picture}(150,50)(0,0)
%\vst{0,0}{J_1R_1}
\vln{0,0}{50}
\hz{0,50}{R_1}
\vz{100,0}{R_3}
\hz{50,50}{R_2}
%\hz{50,0}{R_4}
\out{150,0}
\out{150,50}

\hln{0,0}{150}
\hln{100,0}{50}
%\hln{0,50}{50}
\hln{100,50}{50}
%\du{57,0}{U_1}
%\ri{57,50}{I}
\txt{170,25}{R_{\rm T}=\frac{1}{\frac{1}{R_1+R_2}+\frac{1}{R_3}}=\frac{2}{3}\ohm}
\end{picture}
\end{center}
Now the resistors  $R_1$ ja $R_2$  are in series, and the series circuit is in parallel with $R_3$.
Now we know both $E_{\rm T}$ ja $R_{\rm T}$ and we can draw the Thévenin equivalent (on the next slide).
}

\frame{
\frametitle{The Final Circuit}

\begin{center}
\begin{picture}(150,50)(0,0)
\vst{0,0}{E_{\rm T}=\frac{1}{3}\V}
\hln{0,0}{50}
\hz{0,50}{R_{\rm T}=\frac{2}{3}\ohm}
\out{50,50}
\out{50,0}
\end{picture}
\end{center}

}





\frame {
  \frametitle{Superposition Principle}
\begin{itemize}
\item A circuit consisting of resistances and constant-valued current and voltage sources is {\em linear}.
\item If a circuit is linear, all the voltages and currents can be solved by calculating the effect of each source one at the time.
\item This principle is called the {\bf method of superposition}.
\end{itemize}
}

\frame {
  \frametitle{Superposition Principle}
The method of superposition is applied as follows
\begin{itemize}
\item The current(s) and/or voltage(s) caused by each source is calculated one at a time so that all other sources are turned off.
\item A turned-off voltage source = short circuit (a wire), a turned-off current source = open circuit (no wire).
\item Finally, all results are summed together.
\end{itemize}
}


\frame {
\frametitle{Superposition Principle: an Example}
Find current $I_3$ by using the superposition principle.
\begin{center}
\begin{picture}(200,50)(0,0)
\vst{0,0}{E_1}
\vst{100,0}{E_2}
\vz{50,0}{R_3}
\di{50,1}{I_3}
\hz{0,50}{R_1}
\hz{50,50}{R_2}
\hln{0,0}{100}
\end{picture}
\end{center}
First, we turn off the rightmost voltage source:
\begin{center}
\begin{picture}(200,50)(0,0)
\vst{0,0}{E_1}
%\vst{100,0}{E_2}
\vln{100,0}{50}
\vz{50,0}{R_3}
\di{50,1}{I_{31}}
\hz{0,50}{R_1}
\hz{50,50}{R_2}
\hln{0,0}{100}
\txl{100,25}{$I_{31}=\frac{E_1}{R_1+\frac{1}{G_2+G_3}}\frac{1}{G_2+G_3}G_3$}
\end{picture}
\end{center}
Next, we turn off the leftmost voltage source:
\begin{center}
\begin{picture}(200,50)(-100,0)
%\vst{0,0}{E_1}
\vst{100,0}{E_2}
\vln{0,0}{50}
\vz{50,0}{R_3}
\di{50,1}{I_{31}}
\hz{0,50}{R_1}
\hz{50,50}{R_2}
\hln{0,0}{100}
\txl{-125,25}{$I_{32}=\frac{E_2}{R_2+\frac{1}{G_1+G_3}}\frac{1}{G_1+G_3}G_3$}
\end{picture}
\end{center}
}

\frame {
\frametitle{Superposition Principle: an Example}
The current $I_3$ is obtained by summing the partial currents $I_{31}$ and $I_{32}$.
\[
I_3=I_{31}+I_{32}=\frac{E_1}{R_1+\frac{1}{G_2+G_3}}\frac{1}{G_2+G_3}G_3+\frac{E_2}{R_2+\frac{1}{G_1+G_3}}\frac{1}{G_1+G_3}G_3
\]
}

\frame {
\frametitle{When Is It Handy to Use the Superposition Principle?}
\begin{itemize}
\item If one doesn't like solving equations but likes fiddling with the circuit.
\item If there are many sources and few resistors, the method of superposition is usually fast.
\item If there are sources with different frequencies (as we learn on the AC Circuits course), the analysis
of such a circuit is based on the superposition principle.
\end{itemize}
}


\frame {
\frametitle{Linearity and the Justification for the Superposition Principle}
\begin{itemize}
\item The method of superposition is based on the linearity of the circuit, which means that
every source affects every voltage and current with a constant factor.
\item This means that if there are sources $E_1$, $E_2$, $E_3$, $J_1$,
$J_2$ in the circuit, then every voltage and current is of form $k_1E_1+k_2E_2+k_3E_3+k_4J_1+k_5J_2$, where constants $k_n$ are real numbers.
\item If all the sources are turned off (= 0), then all currents and voltages in the circuit are zero. Therefore, by nullifying all sources except one,
we can find out the multiplier for the source in question.
\end{itemize}
}

\begin{comment}
\frame{
\frametitle{Oppikirja}
Tällä luennolla käsiteltiin kirjan Kimmo Silvonen: {\em Sähkötekniikka ja elektroniikka} kappaleet:
\begin{description}
\item[1.9.5] Superpositioperiaate eli kerrostamismenetelmä
\end{description}
}
\end{comment}


\frame{
\frametitle{Homework 7  (released 21st Sep, to be returned 24th Sep)}
\begin{block}{Homework 7}
Find current $I_2$ by using the superposition principle.
\end{block}
\[
J_1=1\A \quad R_1=10 \ohm \quad R_2= 20 \ohm \quad R_3=30 \ohm
\quad E_1=5\V
\]

\begin{center}
\begin{picture}(150,50)(0,0)
\vj{0,0}{J_1}
\vz{50,0}{R_1}
\vz{100,0}{R_3}
\hz{50,50}{R_2}
%\hz{50,0}{R_4}
%\out{150,0}
%\out{150,50}
\ri{58,50}{I_2}
\vst{150,0}{E_1}
\hln{0,0}{150}
\hln{100,0}{50}
\hln{0,50}{50}
\hln{100,50}{50}
%\du{57,0}{U_1}
%\ri{57,50}{I}
\end{picture}
\end{center}

}

%LUENTO8
\section{8. lecture}

\frame{
\frametitle{Homework 7 - Model Solution}
\begin{block}{Homework 7}
Find current $I_2$ by using the superposition principle.
\end{block}
\[
J_1=1\A \quad R_1=10 \ohm \quad R_2= 20 \ohm \quad R_3=30 \ohm
\quad E_1=5\V
\]

\begin{center}
\begin{picture}(150,50)(0,0)
\vj{0,0}{J_1}
\vz{50,0}{R_1}
\vz{100,0}{R_3}
\hz{50,50}{R_2}
%\hz{50,0}{R_4}
%\out{150,0}
%\out{150,50}
\ri{58,50}{I_2}
\vst{150,0}{E_1}
\hln{0,0}{150}
\hln{100,0}{50}
\hln{0,50}{50}
\hln{100,50}{50}
%\du{57,0}{U_1}
%\ri{57,50}{I}
\end{picture}
\end{center}

}

\frame{
\frametitle{Ratkaisu}
First, we find the effect of the current source:
\begin{center}
\begin{picture}(150,50)(0,0)
\vj{0,0}{J_1}
\vz{50,0}{R_1}
\vz{100,0}{R_3}
\hz{50,50}{R_2}
%\hz{50,0}{R_4}
%\out{150,0}
%\out{150,50}
\ri{58,50}{I_{21}}
%\vst{150,0}{E_1}
\vln{150,0}{50}
\hln{0,0}{150}
\hln{100,0}{50}
\hln{0,50}{50}
\hln{100,50}{50}
%\du{57,0}{U_1}
%\ri{57,50}{I}
\end{picture}
\end{center}
The voltage over $R_1$ and $R_2$ is the same (they are in parallel) and $R_2$ is twice as large as $R_1$
and therefore the current through $R_2$ is half of the current of $R_1$. Because the total current through the resistors
is $J_1=1\A$, the current through $R_1$:n is $2/3 \A$ and the current through $R_2$
is$I_{21}=1/3 \A$.
}

\frame{
\frametitle{Ratkaisu}
Next, we find out the effect of the voltage source:
\begin{center}
\begin{picture}(150,50)(0,0)
%\vj{0,0}{J_1}
\vz{50,0}{R_1}
\vz{100,0}{R_3}
\hz{50,50}{R_2}
%\hz{50,0}{R_4}
%\out{150,0}
%\out{150,50}
\ri{58,50}{I_{22}}
\vst{150,0}{E_1}
%\vln{150,0}{50}
\hln{0,0}{150}
\hln{100,0}{50}
\hln{0,50}{50}
\hln{100,50}{50}
%\du{57,0}{U_1}
%\ri{57,50}{I}
\end{picture}
\end{center}
The resistors $R_1$ and $R_2$ are now in series and the total voltage over them is $E=5\V$, and therefore
\[
I_{22}=-\frac{E}{R_1+R_2}=-\frac{5\V}{10\ohm+20\ohm}=-\frac{1}{6}\V .
\]
The minus sign comes from the fact that the direction of the current $I_{22}$ is upwards and the direction of the voltage $E$ is
downwards.\\
Finally, we sum the partial results:
\[
I_2=I_{21}+I_{22}=\frac{1}{3}\A-\frac{1}{6}\A=\frac{1}{6} \A.
\]
}





\frame{
\frametitle{Voltage Divider}
\begin{center}
\begin{picture}(150,50)(0,0)
\hz{0,50}{R_1}
\vz{50,0}{R_2}
\hln{0,0}{50}
\du{0,0}{U}
\du{60,0}{U_2}
\ru{0,60}{U_1}

\end{picture}
\end{center}
\begin{itemize}
\item $U_1=U\frac{R_1}{R_1+R_2}$ ja $U_2=U\frac{R_2}{R_1+R_2}$
\item It is quite common in electronic circuit design, that we need a reference voltage formed from another voltage in the circuit.
\item The formula is valid also for multiple resistors in series. The denominator is formed by summing all the resistances and
the resistor whose voltage is to be solved is in the numerator.
\end{itemize}

}

\frame{
\frametitle{Current Divider}
\begin{center}
\begin{picture}(150,50)(0,0)
\hln{0,50}{100}
%\hz{0,50}{R_1}
\vz{50,0}{R_1}
\vz{100,0}{R_2}
\hln{0,0}{100}
\ri{9,50}{I}
\di{50,42}{I_1}
\di{100,42}{I_2}

%\du{0,0}{U}
%\du{60,0}{U_2}
%\ru{0,60}{U_1}

\end{picture}
\end{center}

\begin{itemize}
\item $I_1=I\frac{G_1}{G_1+G_2}$ ja $I_2=I\frac{G_2}{G_1+G_2}$
\item  The formula is valid also for multiple resistors in parallel.
\item The formula for current divider is not used as frequently as the voltage divider, but it is natural to discuss it in this concept.
\end{itemize}

}

\frame{
\frametitle{Example 1}
\begin{center}
\begin{picture}(150,50)(0,0)
\hz{0,50}{R_4}
\hln{50,50}{100}
%\hz{0,50}{R_1}
\vz{50,0}{R_1}
\vz{100,0}{R_2}
\vz{150,0}{R_3}
\hln{0,0}{150}
%\ri{9,50}{I}
\di{50,42}{I_1}
\di{100,42}{I_2}
\di{150,42}{I_3}
\vj{0,0}{J}
\end{picture}
\end{center}
\pause
$I_1=J\frac{G_1}{G_1+G_2+G_3}\quad$
\pause
$I_2=J\frac{G_2}{G_1+G_2+G_3}\quad$
\pause
$I_3=J\frac{G_3}{G_1+G_2+G_3}$
}

\frame{
\frametitle{Example 2}
\begin{center}
\begin{picture}(150,50)(0,0)
\hz{0,50}{R_1}
%\hln{50,50}{100}
%\hz{0,50}{R_1}
\hz{50,50}{R_2}
\hz{100,50}{R_3}
\vz{150,0}{R_4}
\hln{0,0}{150}
%\ri{9,50}{I}
%\di{50,42}{I_1}
%\di{100,42}{I_2}
%\di{150,42}{I_3}
\vst{0,0}{E}
\color{red}
\rcuu{0,55}{U_1}
\rcuu{50,55}{U_2}
\rcuu{100,55}{U_3}
\dcru{155,0}{U_4}

\end{picture}
\end{center}
\pause
$U_1=E\frac{R_1}{R_1+R_2+R_3+R_4}\quad$
\pause
$U_2=E\frac{R_2}{R_1+R_2+R_3+R_4}\quad$
\pause
$U_3=E\frac{R_3}{R_1+R_2+R_3+R_4}\quad$
\pause
$U_4=E\frac{R_4}{R_1+R_2+R_3+R_4}\quad$
}


\begin{comment}
\frame{
\frametitle{Oppikirja}
Tällä luennolla käsiteltiin kirjan Kimmo Silvonen: {\em Sähkötekniikka ja elektroniikka} kappaleet:
\begin{description}
\item[1.5.3] Jännitteen jako
\item[1.5.4] Virran jako
\end{description}
Koska sivunumerointi saattaa vaihdella painoksittain, viittaan kappaleen numeroihin.

}
\end{comment}

\frame{
\frametitle{Homework 8  (released 24th Sep, to be returned 28th Sep)}
\begin{block}{Homework 8}
Find the voltage $U$ by applying the voltage divider formula.
\end{block}
\[
E_1=10\V \quad R_1=10 \ohm \quad R_2= 20 \ohm \quad R_3=30 \ohm
\]
\[
 R_4=40 \ohm \quad R_5=50 \ohm
\quad E_2=15\V
\]

\begin{center}
\begin{picture}(150,50)(0,0)
\vst{0,0}{E_1}
\hz{0,50}{R_1}
\vz{50,0}{R_2}
\vz{100,0}{R_3}
\hz{100,50}{R_4}
%\hz{50,0}{R_4}
%\out{150,0}
%\out{150,50}
%\ri{58,50}{I_2}
\ru{50,50}{U}
\vst{200,0}{E_2}
\hz{150,50}{R_5}
\hln{0,0}{200}
\hln{100,0}{50}
%\hln{100,50}{50}
%\du{57,0}{U_1}
%\ri{57,50}{I}
\end{picture}
\end{center}

}

%LUENTO9
\section{9. lecture}

\frame{
\frametitle{Homework 8 - Model Solution}
\begin{block}{Homework 8}
Find the voltage $U$ by applying the voltage divider formula.
\end{block}
\[
E_1=10\V \quad R_1=10 \ohm \quad R_2= 20 \ohm \quad R_3=30 \ohm
\]
\[
 R_4=40 \ohm \quad R_5=50 \ohm
\quad E_2=15\V
\]

\begin{center}
\begin{picture}(150,50)(0,0)
\vst{0,0}{E_1}
\hz{0,50}{R_1}
\vz{50,0}{R_2}
\vz{100,0}{R_3}
\hz{100,50}{R_4}
%\hz{50,0}{R_4}
%\out{150,0}
%\out{150,50}
%\ri{58,50}{I_2}
\ru{50,50}{U}
\vst{200,0}{E_2}
\hz{150,50}{R_5}
\hln{0,0}{200}
\hln{100,0}{50}
%\hln{100,50}{50}
%\du{57,0}{U_1}
%\ri{57,50}{I}
\color{red}
\du{60,0}{U_2}
\du{108,0}{U_3}

\end{picture}
\end{center}
$U_2=E_1\frac{R_2}{R_1+R_2}=10\V\frac{20\ohm}{10\ohm+20\ohm}=6\frac{2}{3}\V$
$U_3=E_2\frac{R_3}{R_3+R_4+R_5}=15\V\frac{30\ohm}{30\ohm+40\ohm+50\ohm}=3,75\V$
$U=U_2-U_3=2\frac{11}{12}\V\approx 2,92\V$.
}






\frame{
\frametitle{Inductors and Capacitors}
\begin{center}
\begin{picture}(200,50)(0,0)
\hz{0,0}{R} \ri{7,0}{i} \rcuu{0,5}{u}
\hl{75,0}{L} \ri{82,0}{i} \rcuu{75,5}{u}
\hc{150,0}{C} \ri{157,0}{i} \rcuu{150,5}{u}

\txt{25,-50}{u=Ri}
\txt{100,-50}{u=L\frac{{\rm d}i}{{\rm d}t}}
\txt{175,-50}{i=C\Du}

\end{picture}
\end{center}


}

\frame{
\frametitle{Inductors and Capacitors in DC Circuit}
\begin{center}
\begin{picture}(200,50)(0,-50)
\hz{0,0}{R} \ri{7,0}{i} \rcuu{0,5}{u}
\hl{75,0}{L} \ri{82,0}{i} \rcuu{75,5}{u}
\hc{150,0}{C} \ri{157,0}{i} \rcuu{150,5}{u}

\txt{25,-50}{u=Ri}
\txt{100,-50}{u=L\frac{{\rm d}i}{{\rm d}t}}
\txt{175,-50}{i=C\Du}

\end{picture}
\end{center}
DC voltage and current remain constant as function of time or the time derivatives of the voltage and current is zero. Therefore the voltage of an inductor and
the current of a capacitor is zero in a DC circuit.
}



\frame{
\frametitle{Exception 1}
The capacitor is fed with DC current so that the current has no other route.
\begin{center}
\begin{picture}(150,50)(0,0)
\vj{0,0}{J}
\vc{50,0}{C}
\hln{0,0}{50}
\hln{0,50}{50}
\end{picture}
\end{center}
$i=C\Du \Rightarrow J=C\Du \Rightarrow \Du=\frac{J}{C}$. The voltage of the capacitor rises at constant speed. 
}

\frame{
\frametitle{Exception 2}
A constant voltage source is connected to the terminals of an inductor.
\begin{center}
\begin{picture}(150,50)(0,0)
\vst{0,0}{E}
\vl{50,0}{L}
\hln{0,0}{50}
\hln{0,50}{50}
\end{picture}
\end{center}
$u=L\frac{{\rm d}i}{{\rm d}t} \Rightarrow E=L\frac{{\rm d}i}{{\rm d}t} \Rightarrow \frac{{\rm d}i}{{\rm d}t}=\frac{E}{L}$. The current of the inductor rises at constant speed. 
}


\frame{
\frametitle{Dealing with all other cases involving inductors and capacitors in DC circuits}
Inductors are replaced with short circuits and capacitors are replaced with open circuits.
}


\begin{comment}
\frame{
\frametitle{Oppikirja}
Kondensaattoria ja kelaa käsitellään kirjan kappaleissa
\begin{description}
\item[2.2] Kela, induktanssi ja permeabiliteetti
\item[2.3] Kondensaattori, kapasitanssi ja permittiivisyys 
\end{description}
{\bf mutta} siellä asiaa käsitellään huomattavasti laajemmin kuin tällä kurssilla!
Kalvoilla esitetyt tiedot riittävät; kirjasta voi tietenkin lukea, jos asia kiinnostaa enemmän.
Kelaan ja kondensaattoriin tutustutaan enemmän kurssilla Vaihtosähköpiirit.

}
\end{comment}

\frame{
\frametitle{Homework 9  (released 28th Sep, to be returned 1st Oct)}
\begin{block}{Homework 9}
Solve the voltage $U$ from this DC circuit.
\end{block}
\[
E_1=10\V \quad R_1=10 \ohm \quad R_2= 20 \ohm \quad R_3=30 \ohm
\]
\[
 R_4=40 \ohm \quad L=500\, \mbox{mH}\quad C=2\,{\rm F}
\quad E_2=15\V
\]

\begin{center}
\begin{picture}(150,50)(0,0)
\vst{0,0}{E_1}
\hz{0,50}{R_1}
\vz{50,0}{R_2}
\vz{100,0}{R_3}
\hl{100,50}{L}
%\hz{50,0}{R_4}
%\out{150,0}
%\out{150,50}
%\ri{58,50}{I_2}
\hc{50,50}{C}
\vst{200,0}{E_2}
\hz{150,50}{R_4}
\hln{0,0}{200}
\hln{100,0}{50}
%\hln{100,50}{50}
%\du{57,0}{U_1}
%\ri{57,50}{I}
\dcru{105,0}{U}
\end{picture}
\end{center}

}

%LUENTO10
\section{10. lecture}

\frame{
\frametitle{Homework 9 - Model Solution}
\begin{block}{Homework 9}
Solve the voltage $U$ from this DC circuit.
\end{block}
\[
E_1=10\V \quad R_1=10 \ohm \quad R_2= 20 \ohm \quad R_3=30 \ohm
\]
\[
 R_4=40 \ohm \quad L=500\, \mbox{mH}\quad C=2\,{\rm F}
\quad E_2=15\V
\]

\begin{center}
\begin{picture}(150,50)(0,0)
\vst{0,0}{E_1}
\hz{0,50}{R_1}
\vz{50,0}{R_2}
\vz{100,0}{R_3}
\hl{100,50}{L}
%\hz{50,0}{R_4}
%\out{150,0}
%\out{150,50}
%\ri{58,50}{I_2}
\hc{50,50}{C}
\vst{200,0}{E_2}
\hz{150,50}{R_4}
\hln{0,0}{200}
\hln{100,0}{50}
%\hln{100,50}{50}
%\du{57,0}{U_1}
%\ri{57,50}{I}
\dcru{105,0}{U}
\end{picture}
\end{center}


}

\frame{
\frametitle{Homework 9 - Model Solution}
Because there are no parallel connections of inductors and voltage sources and no serial connections of capacitors and current sources and the circuit is a DC circuit
(= constant voltages and currents), we can replace the inductors with short circuits and the capacitors with open circuits.

\begin{center}
\begin{picture}(150,50)(0,0)
\vst{0,0}{E_1}
\hz{0,50}{R_1}
\vz{50,0}{R_2}
\vz{100,0}{R_3}
\hln{100,50}{50}
%\hz{50,0}{R_4}
%\out{150,0}
%\out{150,50}
%\ri{58,50}{I_2}

\vst{200,0}{E_2}
\hz{150,50}{R_4}
\hln{0,0}{200}
\hln{100,0}{50}
%\hln{100,50}{50}
%\du{57,0}{U_1}
%\ri{57,50}{I}
\dcru{105,0}{U}
\end{picture}
\end{center}
in which case we obtain $U$ easily by applying the voltage divider formula:
\[
U=E_2\frac{R_3}{R_3+R_4}=6\frac{3}{7}\V\approx 6,4 \V
\]

}



\frame{
\frametitle{Controlled Sources}
\begin{itemize}
\item So far, all of our sources have been constant valued.
\item If the value of a source does not depend on any of the voltages or currents in the circuit, the source is an {independent source}. For example,
constant valued sources and sources varying as function of time (only) are independent sources.
\item If the value of a source is a function of a voltage and/or current in the circuit, the source is a {\em controlled source}.
\end{itemize}


}

\frame{
\frametitle{Voltage Controlled Voltage Source (VCVS)}
\begin{center}
\begin{picture}(150,50)(0,0)
\cn{0,0}
\cn{0,50}
\du{0,0}{u}
\vst{100,0}{e=Au}
\end{picture}
\end{center}
\begin{itemize}
\item The voltage $e$ of VCVS is dependent of some voltage $u$.
\item The multiplier $A$ is called {\em voltage gain}.
\item A real-world example: an audio amplifier.
\end{itemize}


}
\frame{

\frametitle{ Current Controlled Voltage Source (CCVS)}
\begin{center}
\begin{picture}(150,50)(0,0)
\cn{0,0}
\cn{0,50}
\vln{0,0}{50}
\di{0,25}{i}
\vst{100,0}{e=ri}
\end{picture}
\end{center}
\begin{itemize}
\item The voltage $e$ of CCVS  is dependent of some current $i$.
\item The multiplier $r$ is called {\em transresistance}.
\item There is no good everyday example of this source available (of course we can construct this kind of source by using an {\em operational amplifier}).
\end{itemize}
}

\frame{

\frametitle{Voltage Controlled Current Source (VCCS)}

\begin{center}
\begin{picture}(150,50)(0,0)
\cn{0,0}
\cn{0,50}
\du{0,0}{u}
\vj{100,0}{j=gu}
\end{picture}
\end{center}
\begin{itemize}
\item The current $j$ of VCCS  is dependent of some voltage $u$.
\item The multiplier $g$ is called {\em transconductance}.
\item A real-world example: a field-effect transistor (JFET or MOSFET).
\end{itemize}

}

\frame{
\frametitle{Current Controlled Current Source (CCCS)}
\begin{center}
\begin{picture}(150,50)(0,0)
\cn{0,0}
\cn{0,50}
\vln{0,0}{50}
\di{0,25}{i}
\vj{100,0}{j=\beta i}
\end{picture}
\end{center}
\begin{itemize}
\item The current $j$ of CCCS  is dependent of some current $i$.
\item The multiplier $\beta$ is called {\em current gain}.
\item A real-world example: a (bipolar junction) transistor.
\end{itemize}

}


\begin{comment}
\frame{
\frametitle{Oppikirja}
Ohjatut lähteet käsitellään kirjan kappaleissa
\begin{description}
\item[1.8.1] Ohjattu vai riippumaton lähde
\item[1.8.2] Jänniteohjattu jännitelähde (VCVS)
\item[1.8.3] Virtaohjattu jännitelähde (CCVS)
\item[1.8.4] Jänniteohjattu virtalähde (VCCS)
\item[1.8.5] Virtaohjattu virtalähde (CCCS)
 
\end{description}

}
\end{comment}

\frame{
\frametitle{Homework 10  (released 1st Oct, to be returned 5th Oct)}
\begin{block}{Homework 10}
Find the voltage $U$.
\end{block}
\[
E_1=10\V \quad R_1=10 \ohm \quad R_2= 20 \ohm \quad R_3=30 \ohm
\]
\[
 R_4=40 \ohm \quad
\quad r=2\ohm
\]

\begin{center}
\begin{picture}(150,50)(0,0)
\vst{0,0}{E_1}
\hz{0,50}{R_1}
\hz{50,50}{R_2}
\vz{100,0}{R_3}
\ri{50,50}{i}
%\hl{100,50}{L}
\hln{100,50}{50}
%\hz{50,0}{R_4}
%\out{150,0}
%\out{150,50}
%\ri{58,50}{I_2}
%\hc{50,50}{C}
\vst{200,0}{e_2=ri}
\hz{150,50}{R_4}
\hln{0,0}{200}
\hln{100,0}{50}
%\hln{100,50}{50}
%\du{57,0}{U_1}
%\ri{57,50}{I}
\dcru{105,0}{U}
\end{picture}
\end{center}
Note that the source on the right is a controlled source.

}

%LUENTO11
\section{11. lecture}

\frame{
\frametitle{Homework 10 - Model Solution}
\begin{block}{Homework 10}
Find the voltage $U$.
\end{block}
\[
E_1=10\V \quad R_1=10 \ohm \quad R_2= 20 \ohm \quad R_3=30 \ohm
\]
\[
 R_4=40 \ohm \quad
\quad r=2\ohm
\]

\begin{center}
\begin{picture}(150,50)(0,0)
\vst{0,0}{E_1}
\hz{0,50}{R_1}
\hz{50,50}{R_2}
\vz{100,0}{R_3}
\ri{50,50}{i}
%\hl{100,50}{L}
\hln{100,50}{50}
%\hz{50,0}{R_4}
%\out{150,0}
%\out{150,50}
%\ri{58,50}{I_2}
%\hc{50,50}{C}
\vst{200,0}{e_2=ri}
\hz{150,50}{R_4}
\hln{0,0}{200}
\hln{100,0}{50}
%\hln{100,50}{50}
%\du{57,0}{U_1}
%\ri{57,50}{I}
\dcru{105,0}{U}
\end{picture}
\end{center}
Note that the source on the right is a controlled source.

}

\frame{
\begin{center}
\begin{picture}(150,50)(0,0)
\vst{0,0}{E_1}
\hz{0,50}{R_1}
\hz{50,50}{R_2}
\vz{100,0}{R_3}
\ri{50,50}{i}
%\hl{100,50}{L}
\hln{100,50}{50}
%\hz{50,0}{R_4}
%\out{150,0}
%\out{150,50}
%\ri{58,50}{I_2}
%\hc{50,50}{C}
\vst{200,0}{e_2=ri}
\hz{150,50}{R_4}
\hln{0,0}{200}
\hln{100,0}{50}
%\hln{100,50}{50}
%\du{57,0}{U_1}
%\ri{57,50}{I}
\dcru{105,0}{U}
\end{picture}
\end{center}
Let's denote the total resistance of  $R_1$:n ja $R_2$ with symbol $R_{12}$ and write a nodal equation:
\[
UG_3=(E_1-U)G_{12}+(ri-U)G_4
\]
There are two unknowns in the circuit and therefore we need another equation with the same unknowns:
\[
i=(E_1-U)G_{12}
\]
Then we substitute $i$ to the first equation:
\[
E_1G_{12}-UG_{12}+rG_4G_{12}E_1-rG_4G_{12}U-UG_4=UG_3
\]
}

\frame{
\[
E_1G_{12}-UG_{12}+rG_4G_{12}E_1-rG_4G_{12}U-UG_4=UG_3
\]
from which we get
\[
G_{12}E_1(1+rG_4)=U(G_3+G_{12}+G_4+rG_4G_{12}).
\]
Then we substitute the component values and solve $U$:
\[
U=\frac{\frac{10}{30}(1+\frac{2}{40})}{\frac{1}{30}+\frac{1}{30}+\frac{1}{40}+\frac{2}{40\cdot 30}}
=3,75 \V
\]

}


\frame {
  \frametitle{Recapitulation}
On this lesson, we solve some refresher assignments. If you have solved all the circuits, solve the home assignment.
}


\frame{
\frametitle{Recap assignment 1} % Piiriarska 1 laskari 2. tehtävä
\begin{block}{Recap assignment 1}
Find $U$ and $I$ first by using the method of superposition and then by some other method of your choice.
\end{block}

\begin{center}
\[
R_1=1\ohm\quad R_2=2\ohm\quad J=1\A \quad E= 3\V
\]
\begin{picture}(150,50)(0,0)
\vst{0,0}{E}
\ri{25,50}{I}
\vz{50,0}{R_1}
\hz{50,50}{R_2}
\vdj{100,0}{J}
\du{112,0}{U}
\hln{0,0}{100}
\hln{0,50}{50}

\end{picture}
\end{center}
\tiny Vastaus: $I=4\A$ ja $U=1\V$.
}



\frame{
\frametitle{Recap assignment 2} % Piiriarska 1 laskari 3 teht. 2
\begin{block}{Recap assignment 2}
Form a Thévenin equivalent of the circuit on the left.
Then, compute the current $I_{\rm X}$, when the switches are closed and $R_{\rm X}$
is  a) $0\ohm$, b) $8\ohm$ ja c) $12\ohm$.
\end{block}

\[
R_1=5\ohm \quad R_2=3 \ohm \quad R_3=8\ohm \quad R_4=4 \ohm \quad E=16\V
\]

\begin{center}
\begin{picture}(150,100)(0,0)
\vz{0,0}{R_1}
\hz{0,100}{R_2}
\vz{50,50}{R_3}
\hz{50,100}{R_4}
\hso{100,100}{}
\hso{100,0}{}
\vst{50,0}{E}

\vln{0,50}{50}
\vz{150,25}{R_{\rm X}}
\hln{0,0}{100}
\vln{150,0}{25}
\vln{150,75}{25}
\di{150,10}{I_{\rm X}}


\end{picture}
\end{center}

\tiny: Vastaus: $R_{\rm T}=8\ohm$, $E_{\rm T}=8\V$. a) $1\A$ b) $0,5 \A$ $ c) 0,4 \A$.
}
\frame{
\frametitle{Recap assignment 3}
\begin{block}{Recap assignment 3}
Find $U_3$.
\end{block}
\[
G_1=1\,{\rm S}\quad G_2=2\,{\rm S} \quad G_3=3\,{\rm S}\quad G_4=4\,{\rm S}
\quad G_5=5\,{\rm S} \quad g=6\,{\rm S}\quad J=3\A
\]

\begin{center}
\begin{picture}(150,100)(0,0)
\vj{0,0}{J}
\vz{50,0}{G_1\hspace{-2mm}}
\vz{100,0}{G_2}
\vz{150,0}{G_3}
\hz{50,50}{G_4}
\hz{100,50}{G_5}
\hlj{100,100}{gU_1}
\du{15,0}{U_1}
\du{160,0}{U_3}
\hln{0,0}{150}
\hln{0,50}{50}

\vln{100,50}{50}
\vln{150,50}{50}
\cn{100,50}

\end{picture}
\end{center}
\tiny $U_3=-\frac{48}{115}\V \approx -417 \mV$
}

\frame{
\frametitle{Homework 11 (released 5th Oct, to be returned 8th Oct)}
\begin{block}{Homework 11}
We are given a fact that the current $I_3=0\A$. Find $E_1$.
\end{block}
\[
R_1=5 \ohm \quad R_2=4 \ohm \quad R_3=2 \ohm \quad R_4=5 \ohm \quad R_5=6 \ohm \quad
E_2=30 \V
\]

\begin{center}
\begin{picture}(150,50)(0,0)
\vst{0,0}{E_1}
\vst{150,0}{E_2}
\hz{0,50}{R_1}
\hz{50,50}{R_3}
\hz{100,50}{R_2}
\vz{50,0}{R_4}
\vz{100,0}{R_5}
\hln{0,0}{150}
\ri{97,50}{I_3}
\end{picture}
\end{center}
}

%LUENTO12
\section{12. lecture}

\frame{
\frametitle{Homework 11 - Model Solution}
\begin{block}{Homework 11}
We are given a fact that the current $I_3=0\A$. Find $E_1$.
\end{block}
\[
R_1=5 \ohm \quad R_2=4 \ohm \quad R_3=2 \ohm \quad R_4=5 \ohm \quad R_5=6 \ohm \quad
E_2=30 \V
\]

\begin{center}
\begin{picture}(150,50)(0,0)
\vst{0,0}{E_1}
\vst{150,0}{E_2}
\hz{0,50}{R_1}
\hz{50,50}{R_3}
\hz{100,50}{R_2}
\vz{50,0}{R_4}
\vz{100,0}{R_5}
\hln{0,0}{150}
\ri{97,50}{I_3}
\pause
\color{red}
\du{57,0}{U_4}
\du{107,0}{U_5}

\end{picture}
\end{center}

}

\frame{
Because $I_3=0\A$, the current through $R_1$ equals the current through $R_4$ and
 the current through $R_2$ equals the current through $R_5$.
Therefore, the resistors are in series\footnote{Because and only because we know that $I_3$ is zero.}
 and we may use the voltage divider formula to find voltages over $R_4$ and $R_5$.
The voltage over $R_5$ is $U_5=E_2\frac{R_5}{R_2+R_5}=18\V$. Therefore the voltage
over $R_4$ is $18 \V$ too. Now, by the voltage divider rule:
\[
U_4=E_1\frac{R_4}{R_1+R_4} \quad \Rightarrow \quad 18\V=E_1\frac{5\ohm}{5\ohm+5\ohm}
\]
from which we can solve $E_1=36\V$.
Note: it is completely correct to write nodal equations for the circuit and solve $E_1$ from them, too.
}



\frame {
  \frametitle{Recapitulation}
On this lesson, we solve some refresher assignments. I can also demonstrate some examples on the blackboard or to your booklets, too.
}

%Viimex oli
% Piiriarska 1 laskari 2. tehtävä
% Piiriarska 1 laskari 3 teht. 2
% Arska ykkönen, laskari 5, t. 2
% arska laskari 4, t. 3

\frame{
\frametitle{Recap assignment 4} % Arska 1 kotitehtävä 1 2009
\[
R_2=5\ohm \quad E_1=3\V \quad E_2=2\V
\]

\begin{center}
\begin{picture}(150,110)(0,0)
\vst{0,25}{E_1}
\vst{100,50}{E_2}
\vz{50,0}{R_2}
\vz{50,50}{R_1}
\ri{25,100}{I_1}
\li{75,100}{I_2}
\hln{0,100}{100}
\hln{0,0}{50}
\hln{50,50}{50}
\vln{0,0}{25}
\vln{0,75}{25}

\end{picture}
\end{center}
a) How should we choose $R_1$, if we want $I_2$ to be $ 0\A$?\\
b) How large is $I_1$ then?\\
\tiny a) $10 \ohm$ ja b) $0,2 \A$.
}

\frame{
\frametitle{Recap assignment 5} % Arska laskari 2 tehtävä 2
\[
R_1=100\ohm\quad R_2=500\ohm\quad R_3=1,5\kohm\quad R_4=1\kohm \quad E_1=5\V
\]
\[
\quad J_1=100\mA\quad J_2=150\mA
\]
\begin{center}
\begin{picture}(150,100)(0,0)
\vj{0,0}{J_1}
\vz{50,0}{R_1}
\vst{100,0}{E_1}
\vz{150,0}{R_4}
\hz{50,50}{R_2}
\hz{100,50}{R_3}
\hj{100,100}{J_2}
\hln{0,0}{150}
\hln{0,50}{50}
\vln{100,50}{50}
\cn{100,50}
\vln{150,50}{50}
\du{160,0}{U_4}
\end{picture}
\end{center}
Find $U_4$.\\
\tiny $U_4=92\V$
}

\frame{
\frametitle{Recap assignment 6} % http://users.tkk.fi/ksilvone/Lisamateriaali/teht100.pdf t. 114
\[
R_1=12\ohm\quad R_2=25\ohm\quad J=1 \A \quad E_1=1\V\quad
E_2=27\V
\]
\begin{center}
\begin{picture}(150,100)(0,0)
\vst{0,0}{E_1}
\vj{50,0}{J}
\hst{75,50}{E_2}
\hz{0,50}{R_1}
\vz{125,0}{R_2\hspace{-1.3cm}}
\hln{0,0}{125}
\hln{50,50}{25}
\du{70,0}{U}
\end{picture}
\end{center}
Find voltage $U$.\\
\tiny
Solution: $\frac{1}{37}\V \approx 27 \mV$
}



\frame{
\frametitle{Homework 12  (released 8th Oct, to be returned 12th Oct)} 
Write a {\bf short} essay on following subjects:
\begin{itemize}
\item What did you learn on the course?
\item Did the course suck or was it worthwhile?
\item What could the lecturer do better?
\item How should this course be improved?
\end{itemize}
The essay will not affect the grading of the exam --- please give honest feedback\footnote{I am really interested in how I could make the course better.}.

How to return this homework: Write the essay as a plain text email (no attachments) and send it to me no later than the exam day at 18:00.
{\bf The subject of the email message must be 'DC Circuits course feedback 2009 Firstname Surname'.}

}

\frame{
\frametitle{Final Notices on these Slides}
\begin{itemize}
\item The slides are licensed with CC By 1.0 \footnote{\url{http://creativecommons.org/licenses/by/1.0/}}. In short: you can use and modify the slides
freely as long as you mention my name (= Vesa Linja-aho) somewhere. 
\item Single examples and circuits can be of course used without any name mentioning, because they are not an object of copyright (legal term: "Threshold of originality"). 
\item The origin of these slides is the DC Circuits course in Metropolia polytechnic in Helsinki, Finland.
\item If you find typos, misspellings or errors in facts, please give me feedback.
\end{itemize}
}

\end{document}