% TODO Lisää tehokertoimen määritelmä (mikä on kosini fii)

\frame{
\frametitle{Vaihtovirtateho vastuksessa}
\begin{itemize}
\item Hetkelliselle teholle pätee $p=ui$. Vastuksessa $u=Ri$, joten $p=ui=Ri\cdot i=Ri^2$.
\item Sinimuotoinen vaihtovirta, jonka huippuarvo on esimerkiksi $10 \A$, lämmittää $2 \ohm$ vastusta
välillä $p=2\ohm \cdot (10\A)^2=200\,{\rm W}$ teholla, ja välillä $0 \,{\rm W}$ teholla.
\item Mikä sitten on keskimääräinen teho, jolla kyseinen vaihtovirta lämmittää vastusta? Jatkuva
keskiarvo saadaan laskettua integraalin avulla $P_{\rm av}=\frac{1}{T}\int_0^Tp(t){\rm d}t$.
\item Sijoitetaan sinimuotoinen virta, jonka huippuarvo on $\hat{i}$ ja kulmataajuus $\omega$,
tehon kaavaan, ja lasketaan integroimalla keskimääräinen teho.
\end{itemize}
\[
P_{\rm av}=\frac{1}{T}\int_0^Tp(t){\rm d}t=\frac{1}{T}\int_0^TR(\hat{i}\sin{\omega t})^2{\rm d}t
=\frac{R\hat{i}^2}{T}\int_0^T\frac{1}{2}(1-\cos2\omega t){\rm d}t
\]
}

\frame{
\frametitle{Integraali jatkuu}
\[
P_{\rm av}=\frac{1}{T}\int_0^Tp(t){\rm d}t=\frac{1}{T}\int_0^TR(\hat{i}\sin{\omega t})^2{\rm d}t
=\frac{R\hat{i}^2}{T}\int_0^T\frac{1}{2}(1-\cos2\omega t){\rm d}t
\]
Koska kulmataajuus $\omega = 2\pi f$ ja jaksonaika $T=\frac{1}{f}$, niin
\[
P_{\rm av}=\frac{R\hat{i}^2}{T}\int_0^T\frac{1}{2}(1-\cos2\frac{2\pi}{T} t){\rm d}t
=\frac{R\hat{i}^2}{T}\int_0^T\frac{1}{2}-\frac{1}{2}\cos\frac{4\pi}{T} t{\rm d}t
\]
\[
P_{\rm av}=\frac{R\hat{i}^2}{T}(\frac{1}{2}T-\frac{1}{2}\frac{T}{4\pi}(\sin\frac{4\pi}{T}T-\sin\frac{4\pi}{T}0))
=R\hat{i}^2(\frac{1}{2}-\frac{1}{8\pi}(0-0))=\frac{R\hat{i}^2}{2}
\]
Eli minkä suuruista tasavirtaa tällainen vaihtovirta vastaa tehon kaavassa?
\[
I_{\rm eff}=\frac{\hat{i}}{\sqrt{2}}
\]
}

\frame{
\frametitle{Kompleksinen teho}
Aivan kuten Ohmin laki voidaan yleistää vaihtosähkölle $U=RI \Rightarrow U=ZI$, voidaan
tehokin laskea kompleksilukujen avulla:
\[
S=UI^*, \mbox{missä}\ S=P+\jj Q
\]
Kompleksisen tehon reaaliosaa $P$ kutsutaan pätötehoksi (yksikkö: watti, W) ja imaginaariosaa $Q$ loistehoksi (yksikkö vari, var). Pätöteho
kuluu piirissä (esim. muuttuu vastuksessa lämmöksi), loisteholla tarkoitetaan tehoa,
joka heilahtelee edestakaisin piirissä, mutta ei varsinaisesti kulu mihinkään. $S$ on nimeltään näennäisteho. Sen yksikkö
on volttiampeeri (lyhenne: VA).

Huomaa kompleksisen tehon kaavassa merkintä $I^*$, joka tarkoittaa virran liittolukua
eli konjugaattia (= vaihda kulman etumerkki eli vaihda imaginaariosan etumerkki)!

}

\frame{
\frametitle{Perusteluja}
\[
S=UI^*, \mbox{missä}\ S=P+\jj Q
\]
Hetkelliselle teholle voidaan laskea kaava
\[
p(t)=u(t)i(t)=\hat{u}\hat{i}\sin(\omega t+\phi_u)\sin(\omega t+\phi_i)
\]
\[
=\hat{u}\hat{i}\frac{1}{2}[\cos(\phi_u-\phi_i)-\cos(2\omega t+\phi_u+\phi_i)].
\]
Ensimmäinen kosinitermi on vakio, toinen vaihtelee taajuudella, joka on kaksinkertainen
verrattuna piirin kulmataajuuteen. Kompleksisen tehon kaavassa on konjugaattimerkki,
jotta jännitteen ja virran tuloon saadaan kulmaksi $\phi_u-\phi_i$. (Jos konjugointia ei
tehtäisi, tehon kulmaksi tulisi $\phi_u+\phi_i$, joka ei merkitse yhtään mitään.)
}



\frame{
\frametitle{Esimerkki (jatkuu seuraavalla kalvolla)}
Tarkastellaan vaihtosähkögeneraattoria, jota saa kuormittaa enintään 10 ampeerin virralla eli jonka maksimikuormitus
on 2,3 kVA ($230 \V\cdot 10 A = 2,3\,{\rm kVA}$). Kytkemällä generaattoriin puhtaasti resistiivisen kuorman, esimerkiksi lämpöpatterin, saadaan kaikki teho käyttöön:
\begin{center}
\begin{picture}(50,50)(0,0)
\vst{0,0}{230 \V\ 50 \Hz}
\vz{50,0}{R=23\ohm\hspace{-2.5cm}}
\hln{0,0}{50}
\hln{0,50}{50}

\end{picture}
\end{center}
\[
I=\frac{230\V}{23\ohm}=10\A \qquad P=230\V\cdot 10 \A = 2,3\,{\rm kW}
\]
Koska vastus ei aiheuta vaihesiirtoa, tehon laskenta tapahtuu kuten tasasähköpiirissä.
}

\frame{
\frametitle{Esimerkki (jatkuu seuraavalla kalvolla)}
Nyt kytketään samaan generaattoriin induktiivinen kuorma, esimerkiksi sähkömoottori. Mallinnetaan kuormaa vastuksen ja kelan sarjaankytkennällä.
\begin{center}
\begin{picture}(50,50)(0,0)
\vst{0,0}{230 \V\ 50 \Hz}
\vz{100,0}{R=15\ohm\hspace{-2.5cm}}
\hl{50,50}{L=50\,{\rm mH}}
\hln{0,0}{100}
\hln{0,50}{50}

\end{picture}
\end{center}
Nyt virta ja teho ovat
\[
I=\frac{U}{Z}=\frac{U}{\jj\omega L + R}=\frac{230\V}{\jj\cdot 100\pi \cdot 50\cdot 10^{-3}+15}\approx 10,6\angle -46^\circ \A
\]
\[
S=UI^*=230\V \cdot 10,6\angle 46^\circ \A\approx 2440 \angle 46^\circ {\rm VA} \approx 1690+1750\jj
\]
\[
P=1690 \,{\rm W} \qquad Q=1750\, {\rm var}
\]
}

\frame{
\frametitle{Esimerkki: johtopäätökset}
\begin{itemize}
\item Kuorma (moottori) ottaa generaattorilta pätötehoa vain 1690 wattia, mutta sen ottama virta on jo hieman yli sallitun.
\item Käytännön haittana on se, että generaattoriin ei voida kytkeä muita kuormia ilman, että generaattori ylikuormittuu. Toisin sanoen käyttämättä jää 2300 W - 1690 W = 610 W.
\end{itemize}
}

\frame{
\frametitle{Loisteho sähkönjakelussa}
Loisteho on ei-toivottu ilmiö sähkönjakelussa, koska se kuormittaa verkkoa. Loistehon kulutus on pois verkon siirtokapasiteetista.

Esimerkiksi Fortum laskuttaa\footnote{Fortum Espoo Distribution Oy:n verkkopalveluhinnasto 1.1.2011, http://www.fortum.fi/} suurasiakkaita sähkön siirrosta seuraavasti (hinnat euroina, ei sis. alv.):

\begin{center}
\begin{tabular}{| l | l |  }
\hline
Perusmaksu €/kk & 31,50\\
Tehomaksu €/kW, kk & 1,55\\
\bf Loistehomaksu €/kVAr, kk & \bf 3,12\\
Päiväsiirto, talvi c/kWh & 2,30\\
Muun ajan siirto c/kWh & 1,12\\
\hline
\end{tabular}
\end{center}

Loistehomaksun perusteena on kuukausittainen loistehohuippu, josta on vähennetty 20 \% saman kuukauden pätötehohuipun määrästä.

}

\frame{
\frametitle{Esimerkki} % TODO: Lukuarvot mietintään - tämä on käsin kamala pyöriteltävä
\[
L=2\,{\rm H}\quad C=1\,{\rm F}\quad R=5 \ohm \quad E=10\angle 0^\circ \quad \omega=1
\]
\begin{center}
\begin{picture}(100,50)(0,0)
\vc{100,0}{C}
\vst{0,0}{E}
\hln{0,0}{100}
%\hln{50,50}{50}
\hz{50,50}{R}
\hl{0,50}{L}
%\du{115,0}{U}
%\ri{75,50}{I}

\end{picture}
\end{center}

Laske jokaisen neljän elementin ($E, L, R, C$) kompleksinen teho (jokainen erikseen!). Vihje: jos laskit oikein,
jännitelähteen teho on yhtä suuri mutta vastakkaismerkkinen kuin muiden komponenttien
tehojen summa.
}


\frame{
\frametitle{Esimerkki}
\[
L=2\,{\rm H}\quad C=1\,{\rm F}\quad R=5 \ohm \quad E=10\angle 0^\circ \quad \omega=1
\]
\begin{center}
\begin{picture}(100,50)(0,0)
\vc{100,0}{C}
\vst{0,0}{E}
\hln{0,0}{100}
%\hln{50,50}{50}
\hz{50,50}{R}
\hl{0,50}{L}
%\du{115,0}{U}
%\ri{75,50}{I}

\end{picture}
\end{center}

Laske jokaisen neljän elementin ($E, L, R, C$) kompleksinen teho (jokainen erikseen!). Vihje: jos laskit oikein,
jännitelähteen teho on yhtä suuri mutta vastakkaismerkkinen kuin muiden komponenttien
tehojen summa.
{\bf Ratkaisu:}
Lasketaan ensin kondensaattorin, kelan ja vastuksen impedanssit (yksikkö: $\ohm$):
\[
Z_C=\frac{1}{\jj \omega C}=\frac{1}{\jj}=-\jj \quad Z_L=\jj \omega L = 2\jj \quad Z_R=5
\]
}


\frame{
Komponentit ovat sarjassa, joten piirissä kiertävä virta on
\[
I=\frac{U}{Z}=\frac{E}{Z_C+Z_R+Z_L}=\frac{10}{-\jj + 5 +2\jj}=\frac{10}{5+\jj}
\]

Komponenttien jännitteet saadaan yleistetystä Ohmin laista $U=ZI$:
\[
U_C=IZ_C=\frac{10}{5\jj-1}\quad U_R=IZ_R=\frac{50}{5+\jj}\quad U_L=IZ_L=\frac{20\jj}{5+\jj}
\]
Tehon laskemista varten selvitetään virran konjugaatti (liittoluku):
\[
I^*=\left(\frac{10}{5+\jj}\right)^*=\frac{10}{5-\jj}
\]
Lasketaan tehot $S=U\cdot I^*$:
\[
S_C=U_CI^*=\frac{10}{5\jj-1}\frac{10}{5-\jj}=-\frac{100}{26}\jj\qquad S_R=U_Ri^*=\frac{50}{5+\jj}\frac{10}{5-\jj}=\frac{500}{26}
\]
\[
S_L=\frac{20\jj}{5+\jj}\frac{10}{5-\jj}=\frac{200}{26}\jj\qquad S_E=-10\frac{10}{5-\jj}=\frac{-100}{5-\jj}=\frac{-500-100\jj}{26}
\]
$S_E$:n kaavassa on jännitteen etumerkki vaihdettu, koska virta $I$ on erisuuntainen kuin 
jännite $E$.
}

\frame{
Jännitelähde {\bf luovuttaa} tehoa yhtä paljon kuin siihen kytketyt komponentit {\bf kuluttavat} tehoa,
eli summan $S_C+S_R+S_L$ tulee olla yhtä suuri mutta vastakkaismerkkinen kuin $S_E$.
Lasketaan
\[
S_C+S_R+S_L=-\frac{100}{26}\jj+\frac{500}{26}+\frac{200}{26}\jj=\frac{500+100\jj}{26}
\]
mikä on yhtä suuri mutta vastakkaismerkkinen kuin edellisellä kalvolla laskettu
\[
S_E=\frac{-500-100\jj}{26}.
\]

}

