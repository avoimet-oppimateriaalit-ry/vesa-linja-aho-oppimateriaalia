% Lähteenä käytetty mm. Autoteknillistä taskukirjaa (6. painos suomeksi)
% FIX: säädin -> saadin koska tuli Package pgf Error: No shape named `säädin' is known. }
% FIX: järjestelmä -> jarjestelma (sama kuin edellä)
\frame{
\frametitle{Säätötekniikan perusteita}
Mitä teille tulee mieleen termeistä
\begin{itemize}
\item Säädin / Säätöjärjestelmä
\item Takaisinkytkentä
\item Säätötekniikka
\end{itemize}
Arkipäivän esimerkkejä säätöjärjestelmistä?
}

\frame{
\frametitle{Säätötekniikka}
= ne keinot ja menetelmät, joilla jokin suure pidetään halutussa arvossa. 
}


 % Näiden perusteella väsätty:
 % % http://www.texample.net/tikz/examples/simple-flow-chart/
 % http://www.texample.net/tikz/examples/control-system-principles/ on myös hyvä
  \frame{
  \tikzstyle{decision} = [diamond, draw, fill=blue!20, text width=4.5em, text badly centered, node distance=3cm, inner sep=0pt]
\tikzstyle{block} = [rectangle, draw, fill=blue!20, text width=5em, text centered, rounded corners, minimum height=4em]
\tikzstyle{line} = [draw, -latex]
\tikzstyle{cloud} = [draw,circle ,fill=red!20, node distance=3cm,  minimum height=2em]

 \begin{center}
\begin{tikzpicture}[node distance = 3cm, auto,] % auto -> - ja säätösuure oikein, >=latex nuolet siistiksi
  \node [block] (saadin) {Säädin};
  \node [block, right of = saadin] (toimilaite) {Toimilaite};
  \node [block, right of = toimilaite] (jarjestelma) {Järjestelmä};
  \node[block, below of = toimilaite] (anturi) {Anturi};
  \node[cloud, left of = saadin, node distance=2cm] (summa) {$\sum$};
\node [coordinate, name=input, left of = summa, node distance=1cm] {};
\node [above of = jarjestelma, node distance=2cm] (häiriöt) {Häiriöt};

  \path  [line] (saadin) -- (toimilaite);
  \draw [line] (toimilaite) -- node {$y$} (jarjestelma);
  \path (jarjestelma) |- (anturi);
\path [line] (jarjestelma) |- (anturi);
%\path [line] (anturi) -| (summa);
\draw [line] (anturi) -| node[pos=0.99] {$-$} node [near end] {$x$} (summa);
\draw [line] (input) -- node {$w$} (summa) ;
\draw [line] (summa) -- node {$u$} (saadin);
\draw [line] (häiriöt) -- node {$z$} (jarjestelma) ;

\end{tikzpicture}
$u=w-x$
\end{center}

\begin{tabular}{l l  l l}
Ohjesuure & $w$ & Säätösuure & $x$\\
Toimisuure & $y$ & Häiriösuure & $z$\\
\end{tabular}



 } 

\frame{
\frametitle{Perusvaatimuksia säätimelle}
\begin{itemize}
\item Säätimen pitää tietää jotain säädettävästä prosessista.
\item Jos säätö on liian voimakasta, järjestelmä rupeaa värähtelemään.
\item Jos säätö on liian heikkoa, häiriötä ei saada hallintaan.
\end{itemize}
}

\frame{
\frametitle{Erilaisia säätimiä}
\begin{itemize}
\item Vakioarvosäätö
\item Seurantasäätö
\end{itemize}
}

\frame{
\frametitle{Säätöpiirin toiminnan hyvyys}
\begin{itemize}
\item Kuinka suuri on säätöpoikkeama?
\item Kauanko kestää, ennen kuin mitattu arvo on asettunut toivotulla tarkkuudella samaksi kuin ohjearvo?
\item Missä määrin häiriöitä pystytään korjaamaan?
\item Paljonko kuluu energiaa?
\end{itemize}
}

\frame{
\frametitle{Huonosti toimiva säätöpiiri}
\begin{itemize}
\item Pysyvä säätöpoikkeama
\item Värähtely
\item Liian suuri säätöpoikkeama
\item Liiallinen energiankulutus
\end{itemize}
}

% http://en.wikipedia.org/wiki/Control_system
\frame{
\frametitle{On-off -säädin}
\begin{itemize}
\item On-off -säädin kytkee nimensä mukaisesti järjestelmää pois ja päälle.
\end{itemize}
}

\frame{
\frametitle{Lineaariset säätimet}
\begin{itemize}
\item Säätimiä, joissa säätösuureelle suoritetaan lineaarisia laskutoimituksia, kutsutaan lineaarisiksi säätimiksi. Esimerkiksi
kertolasku, derivointi ja integrointi ovat lineaarisia laskutoimituksia.
\end{itemize}
\[
f(x + y) = f(x) + f(y)
\]
\[
f(ax) = af(x)
\]

}


\frame{
\frametitle{P-säädin}
\begin{itemize}
\item P-säädin (P = proportional, verrannollinen) korjaa ohjaussuuretta suoraan verrannollisesti virheen suuruuteen.
\end{itemize}
\[
y=Ku
\]
}

\frame{
\frametitle{I-säädin}
\begin{itemize}
\item I-säädin eli integroiva säädin lisää ohjaussuuretta jatkuvasti, jos virhettä esiintyy.
\end{itemize}
\[
y=K\int_0^t u {\rm d}t
\]
}

\frame{
\frametitle{D-säädin}
\begin{itemize}
\item D-säädin eli derivoiva säädin reagoi säätösuureen nopeisiin muutoksiin.
\end{itemize}
\[
y=K\frac{{\rm d}u}{{\rm d}t}
\]
}

\frame{
\frametitle{PID-säädin}
\begin{itemize}
\item PID-säätimessä on yhdistetty P, I ja D-säätimet. Kunkin säätimen lähtöarvot lasketaan yhteen ja näin muodostetaan koko säätimen lähtösignaali.
\end{itemize}
\[
y=K_{\rm p}u+K_{\rm i}\int_0^t u {\rm d}t+K_{\rm d}\frac{{\rm d}u}{{\rm d}t}
\]
}

\frame{
\frametitle{Muita säätimiä}
\begin{itemize}
\item Logiikkasäätö. Käyttämällä binaarilogiikkaportteja voidaan toteuttaa varsin monimutkaisiakin säätöprosesseja. Hyvä esimerkki logiikkasäädöstä
on hissin ohjaus.
\item PID-säätäjäkin voidaan toteuttaa -- ja yleensä nykyään toteutetaankin -- digitaalisesti. Säätösuure luetaan mikrokontrolleriin, tehdään
tarvittavat laskutoimitukset ja muodostetaan toimisuure.
\item Sumean logiikan käyttö, "sumea säätö". Sumeaa logiikka käytetään muun muassa liikennevalojen ohjauksessa. Sumean logiikan kantava ajatus on,
että perinteisen binaarilogiikan tilat 0 ja 1 korvataan totuusarvolla, joka voi vaihdella vapaasti välillä 0\ldots 1.
\end{itemize}
}

\frame{
\frametitle{Pulssinleveysmodulaatio (PWM)}
\begin{itemize}
\item Toimisuuretta voidaan muuttaa joko lineaarisesti tai pulssinleveyttä säätämällä
\item Erityisesti sähkötehon säätämiseen pulssinleveysmodulaatio sopii erinomaisesti.
\end{itemize}
\url{http://commons.wikimedia.org/wiki/File:PWM_duty_cycle_with_label.gif}
}