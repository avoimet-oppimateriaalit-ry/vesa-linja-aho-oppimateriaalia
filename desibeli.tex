\frame{
\frametitle{Desibeli}
\begin{itemize}
\item Desibeli on dimensioton yksikkö, joka kuvaa tehosuureiden suhdetta logaritmisella asteikolla.
\item Desibeleistä puhuttaessa pitää aina tietää vertailuun käytetty teho.
\item Teho desibeleinä:
\[
10\log_{10} \frac{\rm teho}{\rm vertailuteho}=10\log_{10} \frac{P}{P_{\rm ref}}
\]
\item Radiotekniikassa yleinen yksikkö on dBm, mikä tarkoittaa tehoa verrattuna yhden milliwatin tehoon.
\item Esimerkiksi 1 watin teho on $10\log_{10} \frac{1\, \rm W}{1\, \rm mW}=30\, \rm dBm$
\end{itemize}
}

\frame{
\frametitle{Amplitudisuureet desibeleinä}
\begin{itemize}
\item Teho on verrannollinen amplitudin neliöön (esimerkiksi ääniteho on verrannollinen äänenpaineen neliöön ja vastuksen teho on verrannollinen virran tai jännitteen neliöön).
\item Amplitudeista puhuttaessa desibeliyksikössä on kertoimena 20, ei 10. Esimerkiksi sähköteholle ja virralle:
\[
10\log_{10} \frac{P}{P_{\rm ref}}=10\log_{10} \frac{RI^2}{RI_{\rm ref}^2}=10\log_{10} \Bigg(\frac{I}{I_{\rm ref}}\Bigg)^2=20\log_{10} \frac{I}{I_{\rm ref}}
\]
\item Radiotekniikassa käytetään jännitetasojen ilmaisemiseen mm. desibelimikrovoltteja. Esimerkiksi 1 voltti on
\[
20\log_{10} \frac{1\,\rm V}{1\, \rm \mu V}= 120\, \rm dB\mu V
\]

\end{itemize}
}
