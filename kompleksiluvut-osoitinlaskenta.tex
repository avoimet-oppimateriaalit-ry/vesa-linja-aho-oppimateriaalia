% TODO Korosta, että vastaus (virta tai jännite) aina kulmamuodossa! RI-muoto ei kerro mitään.
% TODO Mieti asioiden esittelyjärjestystä (kompleksiluvut vs. niiden soveltaminen).

\frame{
\frametitle{Kompleksiluvut}
\begin{itemize}
\item Vaihtosähköpiirilaskuja on helppo laskea osoittimilla.
\item Osoittimia on puolestaan helppo käsitellä kompleksilukujen avulla.
\end{itemize}
}

\frame{
\frametitle{Kompleksiluvut}
Kompleksilukuaritmetiikka perustuu imaginaariyksikköön i, joka määritellään seuraavasti:
\[
{\rm i}^2=-1
\]
Perinteisessä reaalilukuaritmetiikassa mikään luku korotettuna toiseen ei ole -1, mutta
mikään ei estä määrittelemästä sellaista. Ettei pikku-i mene sekaisin virran kanssa, sähkötekniikassa
imaginaariyksikölle käytetään symbolia $\jj$:
\[
\jj^2=-1
\]
}

\frame{
\frametitle{Kompleksiluvut}
Kompleksiluku jakaantuu reaaliosaan ja imaginaariosaan. Esimerkiksi jos lasketaan yhteen reaaliluku 3 ja 
imaginaariyksikkö $\jj$, saadaan luku
\[
3+\jj
\]
 }

\frame{
\frametitle{Laskeminen kompleksiluvuilla}
Peruslaskutoimitusten  laskeminen kompleksiluvuilla ei käytännössä eroa reaaliluvuilla laskemisesta:
\[
3(\jj + 2)= 3\jj + 6
\]
\[
(1+2\jj)(1+\jj)=1+\jj+2\jj + 2\jj^2=1+3\jj -2=-1+3\jj
\]



}

\frame{
\frametitle{Eulerin kaava}
Kompleksiluvuille pätee Eulerin kaava. Sen todistaminen on mahdollista sarjakehitelmien avulla
ja kuuluu (yliopisto)matematiikan tunnille. Otamme kaavan käyttöön perustelematta:
\[
e^{\jj \phi}=\cos \phi + \jj \sin \phi
\]
Eulerin kaava on tärkeä, kun kompleksilukuja muunnetaan summamuodosta kulmamuotoon.
Kompleksitason piste $2+2\jj$ voidaan esittää joko summamuodossa:
\[
z=2+2\jj
\] 
tai kulmamuodossa (Eulerin kaavan avulla)
\[
2\sqrt{2}\cdot e^{\jj\frac{\pi}{4}}=2\sqrt{2}\left(\cos \frac{\pi}{4} +\jj\sin\frac{\pi}{4} \right)
\]
Kompleksiluvun $2+2\jj$ kulma eli argumentti on $\frac{\pi}{4}$ radiaania eli 45 astetta ja itseisarvo on $2\sqrt{2}$.
}

\frame{
\frametitle{Eulerin kaavan geometrinen tulkinta}
Kompleksiluvun itseisarvo tarkoittaa kompleksitason pisteen etäisyyttä origosta. Kompleksiluvun
kulma tarkoittaa pisteen suuntaa $x$-akselilta katsottuna.

Eulerin kaavan käyttö ei ole itsetarkoitus sähkötekniikassa. Sen takia kulmamuodossa olevaa kompleksilukua
ei kirjoiteta näkyviin muodossa $r e^{\phi \jj}$, vaan käytetään muotoa $r\angle \phi$. Kyseessä on vain
lyhennysmerkintä.
}


\frame{
\frametitle{Kompleksiluku}
Kompleksiluku $z$ voidaan esittää joko reaali-imaginaarimuodossa
\[
z=x+y\jj
\]
tai kulmamuodossa
\[
z=r\angle \phi
\]
Muunnoskaavat ovat
\[
r=\sqrt{x^2+y^2} \quad {\rm ja} \quad \phi=\arctan\frac{y}{x}
\]
ja toiseen suuntaan
\[
x=r\cos \phi \quad {\rm ja} \quad y=r\sin \phi
\]
Kyse ei ole sen monimutkaisemmasta asiasta, kuin että pisteen sijainti koordinaatistossa, jossa x-akseli
kuvaa reaaliosaa ja y-akseli imaginaariosaa.
}

\frame{
\frametitle{Älä tee näin!}
\begin{columns}
\column{5cm}
Joka vuosi kokeessa joku korottaa imaginaariyksikön vahingossa toiseen,
kun laskee kompleksiluvun itseisarvoa:
$z=3 + 4\jj \Rightarrow r=\sqrt{3^2 + (4{\color{red}\jj})^2}$
$\color{red}=\sqrt{9-16}=\sqrt{-7}$

Oikein on:
$z=3 + 4\jj \Rightarrow r=\sqrt{3^2 + 4^2}$
$=\sqrt{9+16}=\sqrt{25}=5$.

Kyse ei ole mistään monimutkaisemmasta kuin Pythagoraan lauseen soveltamisesta,
sinne ei imaginaariyksikköä sotketa!

\column{40mm}
\includegraphics[width=40mm]{mittaustekniikka_pics/epicfail.jpg}
\end{columns}
}


\frame{
\frametitle{Kompleksilukujen laskusääntöjä}
Nyrkkisääntö: yhteen- ja vähennyslaskut ovat helppoja reaali-imaginaarimuodossa
($x+y\jj$), jako- ja kertolasku ovat helppoja kulmamuodossa.

Kertolasku kulmamuodossa:
\[
r_1\angle\phi_1 \cdot r_2\angle\phi_2=(r_1r_2)\angle \phi_1+\phi_2
\]
eli itseisarvot kerrotaan keskenään, ja kulmat lasketaan yhteen.

Jakolasku kulmamuodossa:
\[
\frac{r_1\angle\phi_1}{r_2\angle\phi_2}=\frac{r_1}{r_2}\angle \phi_1-\phi_2
\]
eli itseisarvoille suoritetaan jakolasku ja kulmat vähennetään toisistaan.
}





\frame{
\frametitle{Osoitinlaskenta kompleksiluvuilla (johdanto)}
\begin{itemize}
\item Jos piirissä on yksi lähde ja yksi kela tai kondensaattori, lasku on helppo.
\item Jos niitä on useampi, laskeminen on mielettömän työlästä.
\item Onneksi voidaan käyttää kompleksilukuihin perustuvaa {\bf osoitinlaskentaa}.
\item Osoitinlaskennassa jännitteet ja virrat ovat kompleksilukuja, jotka sisältävät
sekä jännitteen tehollisarvon että vaihekulman.
\end{itemize}

}

\frame{
\frametitle{Tehollisarvon käsite}
\begin{itemize}
\item Osoitinlaskennassa käytetään tehollisarvon osoittimia, koska tällöin 
tehon kaava on siistimmän muotoinen kuin huippuarvon osoittimia käytettäessä.
\item Vaihtosähköteho käsitellään seuraavalla tunnilla: nyt riittää, että tiedämme, että:
\item Sinimuotoisen vaihtovirran ja -jännitteen huippuarvon ja tehollisarvon suhde on $\sqrt{2}$.
\item Tehollisarvolla tarkoitetaan sitä, kuinka suurta tasajännitettä vaihtojännite vastaa
lämmitysteholtaan, jos siihen kytketään resistiivinen kuorma. Esimerkiksi hehkulamppu
loistaa yhtä kirkkaasti, kytkipä sen 230 voltin akustoon (tasajännite) tai 230 voltin verkkosähköön
(vaihtojännite). Verkkojännitteen huippuarvo on $230\cdot \sqrt{2}\approx325$ volttia.
\end{itemize}

}

\frame{
\frametitle{Osoitinlaskenta kompleksiluvuilla}
\begin{itemize}
\item Uusi käsite: impedanssi ($Z$). Vastuksen impedanssi on $R$. Kelan
impedanssi $Z_L=\jj \omega L$ ja kondensaattorin impedanssi $Z_C=\frac{1}{\jj \omega C}$. 
\item $\jj$ on matematiikan tunnilta tuttu imaginaariyksikkö: $\jj^2=-1$.
Sähkötekniikassa ei käytetä lyhennettä $i$, koska se tarkoittaa virtaa.
\item Jännite ja virta ilmaistaan kompleksilukuna siten, että kompleksiluvun
itseisarvona on jännitteen/virran tehollisarvo ja kulmana (argumenttina)
jännitteen/virran vaihekulma.
\item $u= \hat{u}\sin(\omega t+ \phi) \Leftrightarrow U=\frac{\hat{u}}{\sqrt{2}}\angle\phi$
\item Merkintä $r\angle \phi$ tarkoittaa kompleksilukua, jonka itseisarvo on $r$ ja argumentti
$\phi$.
\end{itemize}

}

\frame{
\frametitle{Kondensaattori ja sinimuotoinen jännite osoitinlaskennalla}
Aiempi esimerkki osoitinlaskennalla.

\begin{center}
\begin{picture}(50,50)(0,0)
\vst{-25,0}{u(t)=\hat{u}\sin(\omega t+ \phi)}
\vc{25,0}{C}
\di{25,40}{i}
%\du{40,0}{u}
\hln{-25,0}{50}
\hln{-25,50}{50}

\end{picture}
\end{center}
Muunnetaan jännitelähteen arvo kompleksiluvuksi:
$u(t)=\hat{u}\sin(\omega t+ \phi) \Rightarrow U=\frac{\hat{u}}{\sqrt{2}}\angle \phi$.
Virta on Ohmin lain mukaan
\[
I=\frac{U}{Z}=\frac{\frac{\hat{u}}{\sqrt{2}}\angle \phi}{\frac{1}{\jj\omega C}}=\jj\omega C\cdot \frac{\hat{u}}{\sqrt{2}}\angle \phi
=\omega C\angle90^\circ \cdot \frac{\hat{u}}{\sqrt{2}}\angle \phi=
\frac{\hat{u}\omega C}{\sqrt{2}}\angle \phi+90^\circ
\]
Muunnetaan takaisin ajan funktioksi:
\[
i=C\omega\hat{u}\sin(\omega t+ \phi+\frac{\pi}{2})
\]
Eli jännitteen ja virran suhde on $\frac{1}{C\omega}$ ja niiden välillä on 90 asteen ($\frac{\pi}{2}$
radiaanin) vaihe-ero. Virta on 90 astetta jännitettä edellä.
}

\frame{
\frametitle{Osoitinlaskennan teoreettinen perusta}
\scriptsize
\begin{itemize}
\item Jokainen sinimuotoinen jännite ja virta voidaan ajatella pyörivänä viisarina. Viisari pyörii
kulmanopeudella $\omega$ ja sen pituus on jännitteen/virran huippuarvo (jos lasketaan huippuarvoilla)
tai tehollisarvo (jos lasketaan tehollisarvoilla).
\item Vaihekulma $\phi$ kertoo kohdan, josta viisari lähtee pyörimään.
\item Jos kelaan syötetään virta $I$, kelan jännite on 90 astetta $I$:tä edellä ja jännitteen suuruus
on $\omega L$-kertainen.
\item Kelan jänniteviisari saadaan virtaviisarista siis kääntämällä viisaria 90 astetta eteenpäin ja
kasvattamalla sen pituus $\omega L$-kertaiseksi.
\item Jos kondensaattoriin syötetään virta $I$, kondensaattorin jännite on 90 astetta virtaa $I$ jäljessä
ja jännite on $\frac{1}{\omega C}$-kertainen.
\item Kondensaattorin jänniteviisari saadaan virtaviisarista siis kääntämällä viisaria 90 astetta taaksepäin ja
kasvattamalla sen pituus $\frac{1}{\omega C}$-kertaiseksi.
\item Kompleksilukujen laskusäännöt sopivat tähän kuin nyrkki silmään: kertomalla virta kompleksisella
impedanssilla, muuttuu sekä pituus että vaihe. $\jj$:llä kertominenhan kääntää kompleksiluvun kulmaa
eteenpäin ja $\jj$:llä jakaminen taaksepäin 90 astetta.
\item Kompleksilukulaskennan voi perustella myös Fourier-muunnoksen avulla.
\end{itemize}
}

\frame{
\frametitle{Yksinkertainen laskuesimerkki}
\[
R_1=1\ohm\quad C=1\,{\rm F}\quad E=10\angle 20^\circ \quad \omega=1
\]
\begin{center}
\begin{picture}(100,50)(0,0)
\vc{100,0}{C}
\vst{0,0}{E}
\hln{0,0}{100}
\hln{50,50}{50}
\hz{0,50}{R}
\du{115,0}{U}
\ri{75,50}{I}

\end{picture}
\end{center}
Lasketaan $I$ ja $U$. Vastuksen impedanssi on $R$ ja kondensaattorin impedanssi on
$\frac{1}{\jj \omega C}$ ja niiden sarjaankytkennän impedanssi on tietysti $Z=R+\frac{1}{\jj \omega C}$.
Nyt virta $I$ on
\[
I=\frac{E}{Z}=\frac{10\angle 20^\circ}{1+1\frac{1}{\jj}}=\frac{10\angle 20^\circ}{1-\jj}
=\frac{10\angle 20^\circ}{\sqrt{2}\angle -45^\circ}=\frac{10}{\sqrt{2}}\angle 20^\circ-(-45^\circ)
\]
\[
\approx 7 \angle 65^\circ {\rm ampeeria}.
\]
}



\frame{
\frametitle{Esimerkki}

a) Muunna summamuotoon (pyöristä tulos tarvittaessa):
\begin{itemize}
\item $3\angle 30^\circ$
\item $5\angle 90^\circ$
\end{itemize}
Muunna kulmamuotoon (pyöristä tulos tarvittaessa):
\begin{itemize}
\item $1-\jj$
\item $-3+4\jj$
\end{itemize}
b) Laske, ja ilmoita lopputulos sekä kulma- että summamuodossa:
\begin{itemize}
\item $(1+\jj)(2-\jj)$
\item $\frac{1}{1-\jj}$
\end{itemize}

}


\frame{
\frametitle{Esimerkki}

Muunna summamuotoon (pyöristä tulos tarvittaessa):
\begin{itemize}
\item $3\angle 30^\circ=3(\cos 30^\circ+\jj \sin 30^\circ)\approx 2,6 + 1,5\jj$
\item $5\angle 90^\circ=5\jj$ (90 asteen kulma = puhdas imaginaariluku)
\end{itemize}
Muunna kulmamuotoon (pyöristä tulos tarvittaessa):
\begin{itemize}
\item $1-\jj=\sqrt{1^2+(-1)^2}\angle \arctan{\frac{-1}{1}}=\sqrt{2}\angle -45^\circ$
\item $-3+4\jj=\sqrt{(-3)^2+4^2}\angle \arctan{\frac{-4}{3}}=5\angle 127^\circ$\footnote{Varo! Laskin 
sanoo $-53^\circ$ -- sinun pitää itse siirtää kulma oikeaan neljännekseen!}
\end{itemize}
Laske, ja ilmoita lopputulos sekä kulma- että summamuodossa:
\begin{itemize}
\item $(1+\jj)(2-\jj)=2-\jj+2\jj-\jj^2=3+\jj=\sqrt{3^2+1^2}\angle \arctan\frac{1}{3}\approx 3,16\angle 18,4^\circ$
\item $\frac{1}{1-\jj}=\frac{1\cdot(1+\jj)}{(1-\jj)(1+\jj)}=\frac{1+\jj}{2}=0,5+0,5\jj\approx 0,7\angle 45^\circ$
\end{itemize}


}

\frame{
\frametitle{Esimerkki 2}

\begin{center}
\begin{picture}(50,50)(0,0)
\vst{-25,0}{U=12\angle 0^\circ}
\vc{25,0}{C}
\ri{15,50}{I}
\vz{60,0}{R}
%\du{40,0}{u}
\hln{-25,0}{85}
\hln{-25,50}{85}

\end{picture}
\end{center}
Ratkaise virta $I$ (kompleksilukuna -- jännite on ilmoitettu tehollisarvona, ilmoita myös virta
tehollisarvona).
\[
C=1\uF\quad  R=10\kohm \quad \omega=1000\frac{\rm rad}{s}
\]
}


\frame{
\frametitle{Esimerkki 2}
\begin{center}
\begin{picture}(50,50)(0,0)
\vst{-25,0}{U=12\angle 0^\circ}
\vc{25,0}{C}
\ri{15,50}{I}
\vz{60,0}{R}
%\du{40,0}{u}
\hln{-25,0}{85}
\hln{-25,50}{85}

\end{picture}
\end{center}
Ratkaise virta $I$ (kompleksilukuna -- jännite on ilmoitettu tehollisarvona, ilmoita myös virta
tehollisarvona).
\[
C=1\uF\quad  R=10\kohm \quad \omega=1000\frac{\rm rad}{s}
\]
Lasketaan rinnankytkennän impedanssi
\[
Z=\frac{1}{\frac{1}{Z_C}+\frac{1}{R}}=\frac{Z_C R}{Z_C+R}=\frac{\frac{1}{\jj \omega C}R}{\frac{1}{\jj \omega C}+R}
=\frac{R}{1+\jj\omega RC}
\]
Virta $I$ saadaan yleistetystä Ohmin laista
\[
I=\frac{U}{Z}=U\frac{1+\jj\omega RC}{R}=\frac{12}{10000}(1+\jj 1000\cdot 10^{-6}\cdot10000) 
=0,0012(1+10\jj)
\]
}

\frame{
\[
0,0012(1+10\jj)\approx 0,012\angle 84,3^\circ
\]
Eli virta on 12 milliampeeria kulmassa 84,3$^\circ$.
} 
