\frame{
\frametitle{Useita taajuuksia samanaikaisesti ("monitaajuusanalyysi", "harmoninen analyysi")}
\begin{itemize}
\item Jos piirissä on useita eri taajuuksia, kukin taajuus pitää analysoida erikseen.
\item Periaate on sama kuin kerrostamismenetelmässä.
\item {\bf Eritaajuisia kompleksilukuna olevia jännitteitä ei voi laskea yhteen!}
\item Lasketaan yksi taajuus kerrallaan niin, että muuntaajuiset jännitteet ja virrat on asetettu nollaan.
\end{itemize}
}

\frame{
\frametitle{Esimerkki}
\begin{center}
\begin{picture}(100,50)(0,0)
\vst{0,0}{e_1}
\vst{100,0}{e_2}
\hln{0,0}{100}
%\hln{50,50}{50}
%\hln{50,50}{50}
\ri{50,50}{i(t)}
\hl{50,50}{L}
\hz{0,50}{R}
%\du{115,0}{U}
%\ri{75,50}{I}
%\color{blue}
%\vc{50,0}{C}

\end{picture}
\end{center}
\[
e_1(t)=10+\sqrt{2}\cdot 20\sin\omega_1t
\]
\[
e_2(t)=\sqrt{2}\cdot 10\sin \omega_1 t + \sqrt{2}\cdot 30\sin \omega_2 t 
\]
\[
\omega_1=10 \quad \omega_2=20 \quad R= 10\ohm \quad L=1{\rm H}
\]
}

\frame{
\frametitle{Useita taajuuksia samanaikaisesti - lopputulos}
Eritaajuisia osoittimia ei voi laskea suoraan yhteen. Laskun lopputulos (jännite tai virta) pitää ilmoittaa joko
\begin{itemize}
\item Ajan funktiona.
\item Hetkellisarvona. Tämä tapahtuu laskemalla ensin jännite ajan funktiona ja sitten sijoittamalla jokin ajan arvo lausekkeeseen. 
\item Tehollisarvona. Tehollisarvo saadaan korottamalla jokaisen eritaajuisen sinijännitteen tehollisarvo toiseen, laskemalla
ne yhteen ja ottamalla tästä summasta neliöjuuri.
\end{itemize}
}








\frame{
\frametitle{Esimerkki}

\begin{center}
\begin{picture}(100,50)(0,0)
\vz{100,0}{R_{\rm L}}
\vst{0,0}{e}
\hln{0,0}{150}
%\hln{50,50}{50}
\hln{50,50}{100}
%\hl{0,50}{L}
\hl{0,50}{L}
\vj{150,0}{j}
%\du{115,0}{U}
%\ri{75,50}{I}
%\color{blue}
%\vc{50,0}{C}
\dcru{105,00}{U}
\end{picture}
\end{center}
Jännitelähteen tehollisarvo on $10 \V$ kulmataajuudella  $\omega_1=10$ ja virtalähteen tehollisarvo on
$1 \A$ kulmataajuudella $\omega_2=20$. Laske jännitteen $U$ tehollisarvo.

\[
L=1\, {\rm H}\quad R=10\ohm
\]

}


\frame{
\frametitle{Esimerkki}
\begin{center}
\begin{picture}(100,50)(0,0)
\vz{100,0}{R_{\rm L}}
\vst{0,0}{e}
\hln{0,0}{150}
%\hln{50,50}{50}
\hln{50,50}{100}
%\hl{0,50}{L}
\hl{0,50}{L}
\vj{150,0}{j}
%\du{115,0}{U}
%\ri{75,50}{I}
%\color{blue}
%\vc{50,0}{C}
\dcru{105,00}{U}
\end{picture}
\end{center}
Jännitelähteen tehollisarvo on $10 \V$ kulmataajuudella  $\omega_1=10$ ja virtalähteen tehollisarvo on
$1 \A$ kulmataajuudella $\omega_2=20$. Laske jännitteen $U$ tehollisarvo.
\[
L=1\, {\rm H}\quad R=10\ohm
\]
Ratkaisu: lasketaan taajuus kerrallaan. Ensin $\omega_1$ ja sitten $\omega_2$. Lopuksi
lasketaan jännitteiden yhteinen tehollisarvo.
}


\frame{
\frametitle{Piiri taajuudella $\omega_1$}
Virtalähde ei sisällä ollenkaan $\omega_2$-taajuista siniaaltoa, joten päällä on ainoastaan $e$.
\begin{center}
\begin{picture}(100,50)(0,0)
\vz{100,0}{R_{\rm L}}
\vst{0,0}{e}
\hln{0,0}{150}
%\hln{50,50}{50}
\hln{50,50}{100}
%\hl{0,50}{L}
\hl{0,50}{L}
%\vj{150,0}{j}
%\du{115,0}{U}
%\ri{75,50}{I}
%\color{blue}
%\vc{50,0}{C}
\dcru{105,00}{U_1}
\end{picture}
\end{center}
Jännitteenjakosäännön mukaan:
\[
U_1=E\frac{R_{\rm L}}{R_{\rm L}+Z_{\rm L}}=10\frac{10}{10+\jj10\cdot 1}=10\frac{1}{1+\jj}=10\frac{1}{\sqrt{2}\angle 45^\circ}=\frac{10}{\sqrt{2}}\angle -45^\circ
\]
}

\frame{
\frametitle{Piiri taajuudella $\omega_2$}
Jännitelähde ei sisällä ollenkaan $\omega_2$-taajuista siniaaltoa, joten päällä on ainoastaan $j$.
\begin{center}
\begin{picture}(100,50)(0,0)
\vz{100,0}{R_{\rm L}}
%\vst{0,0}{e}
\vln{0,0}{50}
\hln{0,0}{150}
%\hln{50,50}{50}
\hln{50,50}{100}
%\hl{0,50}{L}
\hl{0,50}{L}
\vj{150,0}{j}
%\du{115,0}{U}
%\ri{75,50}{I}
%\color{blue}
%\vc{50,0}{C}
\dcru{105,00}{U_2}
\end{picture}
\end{center}
Vastus ja kela ovat rinnakkain\footnote{Merkintä $||$ tarkoittaa rinnankytkennän impedanssia.}. Jännite on siis
\[
U_2=ZI=(R||Z_{\rm L})j=\frac{R\jj\omega_2 L}{R+\jj\omega_2 L}j=\frac{200\jj}{10+20\jj}\cdot 1=20\frac{\jj}{1+2\jj}=20\frac{1\angle 90^\circ}{\sqrt{5}\angle 63^\circ}
\]
Koko jännitteen $U$ tehollisarvo on
\[
|U|=\sqrt{|U_1|^2+|U_2|^2}=\sqrt{\left|\frac{10}{\sqrt{2}}\right|^2+\left|\frac{20}{\sqrt{5}}\right|^2}=\sqrt{130}\approx 11,4\ ({\rm volttia})
\]

}

