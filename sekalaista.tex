% TÄMÄ TIEDOSTO ON VÄLIVARASTO KALVOILLE, JOIDEN PAIKKAA PITÄÄ MIETTIÄ.







%%%%%%%%%%%%%%%%%%%%
%%%%%%%%%%%%%%%%%%%%
%%%%%%%%%%%%%%%%%%%%
%%%%%%%%%%%%%%%%%%%%
%%%%%%%%%%%%%%%%%%%%
%%%%%%%%%%%%%%%%%%%%
%Tämän alla on Elektroniikka 1 -kurssin kamaa jonka voi todennäköisesti heittää pois.


 \frame{
 \frametitle{Mitä on elektroniikka?}
\begin{block}{Kielitoimiston sanakirja 2.0}
Elektroniikka = vapaiden elektronien ja muiden varauksenkantajien tutkimus ja hyväksikäyttö (esim. puolijohde- ja näyttölaitteissa, mikropiireissä yms.). 
\end{block}

\begin{itemize}
 \item Kursseilla Tasasähköpiirit ja Vaihtosähköpiirien perusteet käsitellään piiriteoriaa ja sähkötekniikkaa.
 \item Raja sähkötekniikan ja elektroniikan välillä joskus häilyvä. Nyrkkisääntö: jos käytetään puolijohteita (tai radioputkia), kyse on elektroniikasta. Jos pelkkää sähkön lämpövaikutusta (esim. lämpöpatteri) tai sähkömagneettista voimavaikutusta (sähkömoottori), kyse on sähkötekniikasta.
\item Itse mm. mieltäisin kaiuttimen jakosuotimen elektroniikaksi, vaikkei siinä olisi yhtään puolijohdekomponenttia.
 \end{itemize}
 }


\frame{
\frametitle{Elektroniikka ja puolijohdekomponentit}
\begin{itemize}

\item Puolijohdetekniikka perustuu puolijohteiden ja/tai puolijohteen ja johteen
rajapinnassa tapahtuviin fysikaalisiin ilmiöihin.

\end{itemize}
}







\frame{ % TODO ihan hyödyllistä tietoa, mutta minkä alle?
\frametitle{Mikropiirit}
\begin{itemize}
\item Kymmeniä vuosia sitten suuri osa elektroniikasta toteutettiin erilliskomponenteista
kokoamalla.
\item 1970-luvulla mikropiirit yleistyivät rajusti. Mikropiirien etuja ovat edullisuus ja pieni koko.
\item Nykyään laitesuunnittelu kehittyy yhä enemmän siihen suuntaan, että valmistetaan standardimikropiirejä ja ajetaan sinne ohjelmisto sisään, jolloin saadaan ohjelmistosta riippuen aikaiseksi digiboksi tai tietokoneen äänikortti (kärjistetty esimerkki).

\end{itemize}
}




 \frame{ % TODO Kato onko tää mainittu
 \frametitle{Transistorivahvistimen piensignaalianalyysi}
Viime tunnilla analysoitiin transistorivahvistinta mallintamalla sen toimintaa virtalähteellä. Tämä malli on käytännössäkin epätarkka, koska valitsemalla emitterivastus $R_{\rm E}$ nollaksi (tai ohittamalla se suurella kondensaattorilla) vahvistuskerroin olisi ääretön (muka).
 }

\frame{
 \frametitle{Vakiovirtalähde} % TÄÄ taitaa olla jo tuolla omana kalvonaan!!
\begin{center}
\begin{picture}(180,100)(0,0)

\vz{0,0}{R_{\rm B}=2,2\kohm}
\zud{0,50}{5,1\V}

\vst{200,0}{E=12\V}
\vln{200,50}{75}
\vz{100,75}{R_{\rm E}=2,2\kohm}
\hln{100,125}{100}

\hln{0,125}{100}
\vln{0,100}{25}
\hln{0,50}{50}
\pnpc{50,50}{}
\hln{0,-50}{200}
\vln{0,-50}{50}
\vln{200,-50}{50}
\vln{100,0}{25}
\di{100,0}{I_{\rm L}}
\vz{100,-50}{R_{\rm L}}
\dcru{105,-50}{U_{\rm L}}
%\dcru{5,0}{U_{\rm B}}
\dcru{105,25}{U_{\rm EC}}
%\tx{70,25}{BC557}
\end{picture}

\end{center}
 }

\frame{ % Kato mikä tää on, löytyykö diodista tai piensignaalianalyysistä?
\frametitle{Diodipiirin ratkaisu "tarkasti"}
% Näytin tän esimerkin kuutostunnilla.
\begin{center}
\begin{picture}(180,100)(0,0)

\vst{0,0}{E=1\V}
\hz{25,50}{R=100\ohm}
\hln{75,50}{25}
\hln{0,50}{25}
\dd{100,0}{}
\hln{0,0}{100}
\du{120,0}{U}

\end{picture}
\end{center}

 }





%%%%%%%%%%%%%%%%%%%%
%%%%%%%%%%%%%%%%%%%%
%%%%%%%%%%%%%%%%%%%%
%%%%%%%%%%%%%%%%%%%%
%%%%%%%%%%%%%%%%%%%%
%%%%%%%%%%%%%%%%%%%%
%Tämän alla on digitaalitekniikan perusteet -kurssin kamaa jonka voi todennäköisesti heittää pois.


\frame{
\frametitle{Digitaalitekniikka}
\begin{itemize}
\item Analoginen signaali: signaali voi saada mitä tahansa arvoja, ja muuttua miten tahansa
ajan funktiona.
\item Digitaalinen signaali: signaalilla on määrätyt vakioarvot, ja se muuttuu ennalta sovitulla
tavalla (esim. 1000 000 kertaa sekunnissa).
\item Yleisesti käytössä oleva digitaalitekniikka käyttää binäärilogiikkaa, eli signaalilla on
kaksi sallittua tilaa, 0 ja 1.
\item Ensimmäisellä tunnilla tutustutaan logiikkaporttien toimintaan.
\end{itemize}
}



 \frame{
 \frametitle{Ensimmäinen laboratoriotyö 18.3.2011}
 \begin{itemize}
 \item Ensimmäisessä työssä harjoitellaan kytkentöjen rakentelua.
\
 \end{itemize}
 }

 \frame{
 \frametitle{Työ 1}
% Ohje kolumneihin: http://www.tug.org/pracjourn/2005-4/mertz/mertz.pdf
\begin{columns}[c]
\column{5cm}
\begin{center}
\begin{picture}(100,100)(-50,-50)
\vo{0,0}{741}{15}
%\txt{100,15}{\Uout=A(U_+-U_-)}
\hz{-50,20}{1\kohm}
\hz{0,55}{10\kohm}
\hgp{0,0}
\vln{0,20}{35}
\vln{50,10}{45}
\hln{50,10}{20}
\hgp{70,-40}
\du{70,-40}{\Uout}

\hgp{-50,-30}
\vst{-50,-30}{\Uin}

\end{picture}
\end{center}

\column{4cm}


\begin{center}
\begin{tabular}{|c|c|c|}
\hline
$\Uin$&\multicolumn{2}{|c|}{$\Uout$}\\
%\hline
%& \phantom{sdasdadas}\\
\hline
& Laske & Mittaa \\
\hline
$0\V$&&\\
\hline
$0,5\V$&&\\
\hline
$-0,5\V$&&\\
\hline
$1\V$&&\\
\hline
$-1\V$&&\\
\hline
$1,5\V$&&\\
\hline
$-1,5\V$&&\\
\hline
$2\V$&&\\
\hline
$-2\V$&&\\
\hline
\end{tabular}
\end{center}

\end{columns}
% Havaintoja: jännite heittää hieman, ja pääsee lähemmäs yläpäätä kuin alapäätä. Ehdoton raja on +-15 V.
% Kaksipuoleinen jännite tehdään käyttämällä kahta jännitelähdettä
 }

 \frame{
 \frametitle{Työ 2}
\begin{center}
\begin{picture}(100,50)(0,-40)
\hg{0,0}
\vst{0,0}{5\V}
\hso{0,50}{}
\hnot{50,38}
\hnot{100,38}

\vz{85,0}{1\kohm}
\ledd{85,-50}{}
\hg{85,-50}{}

\vz{135,0}{1\kohm}
\ledd{135,-50}{}
\hg{135,-50}{}

\hln{72,50}{28}
\hln{125,50}{10}


%\hln{-10,15}{10}
\end{picture}
\end{center}
Käytä CMOS-piiriä 4049. Käyttöjännite on 5 V. Rakenna piiri ja esittele sen toiminta opettajalle.
% Tulopuolella pitää olla alasvetovastus. Muuten se jää killumaan ykköseksi.

 }


 \frame{
 \frametitle{Työ 3}
 \begin{itemize}
 \item Rakenna piirin 4017 pohjalta piiri, jossa ledi siirtyy palkissa aina yhden pykälän eteenpäin, kun käyttäjä painaa nappia.
 \end{itemize}
% Reset ja enable maihin. Jos tulopuolella ei alasvetovastusta, niin ei tapahdu mitään (ei pulsseja). Jos ei tulopuolella konkkaa, niin
% ledi hyppii eteenpäin. Alasvetovastuksen puute aikaansaa myös vilistyksen 50 Hz kentästä.

 }


\frame{
 \frametitle{Mitä labrasta pitäisi jäädä mieleen}
 \begin{itemize}
\item Operaatiovahvistimen lähtöjännite ei voi ylittää käyttöjännitteitä -- eikä yleensä pääse edes käyttöjännitteisiin asti.
 \item Ilmassa roikkuvat tulot aiheuttavat ongelmia. Jos tuloa ohjataan kytkimellä, kytke tulo aina ylös- tai alasvetovastuksella ykköseen tai nollaan. Ilmassa roikkuva tulo voi esimerkiksi siepata 50 Hz kentän sähköverkosta.
\item Kytkinvärähtelyt aiheuttavat joskus ongelmia. Kun kytkintä painetaan, kytkinliuskat läpsyvät yhteen ja pulsseja tuleekin monta (laskuri hyppää eteenpäin monta pykälää). Tämän voi korjata esimerkiksi pienellä kondensaattorilla.

 \end{itemize}
 }


 \frame{
 \frametitle{Työ 5}
 \begin{itemize}
 \item Suunnittele ja rakenna piiri, joka laskee kahden 2-bittisen luvun summan.
 \end{itemize}
 }

 \frame{
 \frametitle{Työ 6 (haastava)}
 \begin{itemize}
 \item Suunnittele ja rakenna piiri, joka laskee kahden 2-bittisen luvun tulon.
 \end{itemize}
 }




 \frame{
 \frametitle{Työ 7}
\begin{center}


\begin{picture}(110,130)(50,-50)

\hz{0,20}{R_1\ 1\kohm}
\voi{50,0}{}{15}
\hz{50,60}{R_2\ 15\kohm}
\hgp{50,0}

\hz{100,10}{R_3\ 150 \kohm}
\vo{150,-10}{}{15}
\hgp{150,-10}
\hc{150,50}{C\ 100\ \mbox{nF}}

\vln{200,0}{75}
\hln{100,75}{100}
\vln{150,10}{40}
\vln{0,20}{55}
\hln{0,75}{100}

\vln{50,20}{40}
\vln{100,10}{50}

\hln{200,0}{10}
\du{210,-50}{\Uout}
\hg{210,-50}

\end{picture}

\end{center}
Rakenna piiri. Kuinka suuri on $\Uout$in amplitudi ja taajuus, ja millainen on sen aaltomuoto? Laske komponenttiarvojen perusteella, mitä amplitudin ja taajuuden pitäisi olla, jos komponentit olisivat ideaalisia. Mistä ero pääosin johtuu?
 }


 \frame{
 \frametitle{Työ 8}
 \begin{itemize}
 \item Suunnittele ja rakenna kytkentä, joka tutkii viittä kytkintä rivissä, ja jos rivin päissä olevilla kytkimillä on sama arvo ja välissä olevilla kytkimillä eri arvo kuin päissä olevilla,
niin ulostulo on 1, muuten 0.
 \end{itemize}
 }


 \frame{
 \frametitle{Työ 9}
 \begin{itemize}
 \item Rakenna kokosummaimia käyttämällä piiri, joka laskee kahden \_\_\_\_\_\_\_\_\ bittisen luvun summan\footnote{Tämä tehtävä tehdään riippuen osien saatavuudesta :-)}.
 \end{itemize}
 }





 \frame{
 \frametitle{Lopuksi}
 \begin{itemize}
 \item Kurssi jäi hieman (lue: pahasti) tyngäksi sairastumisten ja työnantajan määräämien menojen takia (kehittämispäivät sun muut on yleensä ajoitettu perjantaille). Ei oteta tavaksi :-).
\item Yksi ratkaisu olisi ollut ottaa teoriaa kiinni labrojen kustannuksella, mutta käytännön askartelu on erittäin opettavaista, vaikkei uusia teoriajuttuja niin paljon ehdi käydäkään.
\item Jos digitaalitekniikka kiinnostaa, niin netistä löytyy todella hyvin perusasioista materiaalia. Tutustu Boolen algebraan, Karnaugh'n karttaan, liukulukuihin, 2-komplementtiesitykseen, FPGA-piireihin, Hamming-koodaukseen, pariteettiin, multipleksereihin, laskureihin, AD- ja DA-muuntimiin tai vaikkapa Arduino-mikrokontrollerialustaan.
\item Suomenkielisestä kirjallisuudesta hyviä ovat ainakin:\\
{\bf Kimmo Silvonen}: Elektroniikka ja puolijohdekomponentit (luvut 12--13).\\
{\bf Panu-Kristian Poiksalo}: Digitaalitekniikan perusteet\\
{\bf Vesa Volotinen}: Digitaalitekniikka
 \end{itemize}
 }





%%%%%%%%%%%%%%%%%%%%
%%%%%%%%%%%%%%%%%%%%
%%%%%%%%%%%%%%%%%%%%
%%%%%%%%%%%%%%%%%%%%
%%%%%%%%%%%%%%%%%%%%
%%%%%%%%%%%%%%%%%%%%
%Tämän alla on elektroniikan komponentit -kurssin kamaa jonka voi todennäköisesti heittää pois.



% POHJANA OLI: S2010 Elektroniikan komponentit
% Vastukset, konkat sun muut heitetty pois filen loppuun
% Samoin tyristorikomponentit
% Diodiesimerkki pois, koska oli jo harkkatehtävänä


\frame{
\frametitle{Esimerkki 1}
\begin{center}
\begin{picture}(180,100)(0,0)

\vst{0,0}{E=12\V}

\hz{0,50}{R}
%\txt{75,70}{C\ 1\,\mathrm{nF}}
%\hso{0,50}{K}
\hln{50,50}{50}
\dd{100,0}{}
\put(87,20){\vector(-1,-1){10}}
\put(87,27){\vector(-1,-1){10}}


\hln{0,0}{100}
\du{120,0}{U_{\rm LED}}


\end{picture}

\end{center}

Valmistajan datalehden mukaan kuvan ledin nimellisjännite 10 mA virralla on 2,0 volttia.
Kuinka suuri vastuksen R on oltava, jotta ledin läpi kulkisi 10 mA virta?
}


\frame{
\frametitle{Esimerkki 1 - esimerkkiratkaisu}
\begin{center}
\begin{picture}(180,100)(0,0)

\vst{0,0}{E=12\V}

\hz{0,50}{R}
%\txt{75,70}{C\ 1\,\mathrm{nF}}
%\hso{0,50}{K}
\hln{50,50}{50}
\dd{100,0}{}
\put(87,20){\vector(-1,-1){10}}
\put(87,27){\vector(-1,-1){10}}


\hln{0,0}{100}
\du{120,0}{U_{\rm LED}}


\end{picture}

\end{center}

Valmistajan datalehden mukaan kuvan ledin nimellisjännite 10 mA virralla on 2,0 volttia.
Kuinka suuri vastuksen R on oltava, jotta ledin läpi kulkisi 10 mA virta?
{\bf Ratkaisu:} $R=\frac{12\V-2,0\V}{10\mA}=1\kohm$.
}


\frame{
\frametitle{Käytännön kelat ja kondensaattorit}
Aivan kuten ideaalista vastusta ei ole olemassa, ei ole olemassa ideaalisia keloja eikä kondensaattoreita.
}


\frame{
%\begin{center} % Sm2-junan tyristorikaappi (kuvattu syyskuussa 2010, Vesa Linja-aho)
\includegraphics[height=100mm]{kuvat/tyristorikaappi_IMG_1746.JPG}
%\end{center}
}


\frame{
\frametitle{Labratyö: Operaatiovahvistimen epäideaalisuudet}
\begin{itemize}
\item Mitataan laboratoriossa operaatiovahvistimen LM741 epäideaalisuudet: tulonsiirrosjännite, tulovirrat ja CMRR.
\item Tänään tutustutaan teoriaan, ensi torstaina mitataan. Maanantaina 29.11. ei ole tunteja, olen työnantajan määräämässä koulutuksessa.
\end{itemize}
}
