\frame {
\frametitle{Mitä moottorin ohjauksella tarkoitetaan?}
Moottorin ohjauksella tarkoitetaan moottorin tuottaman vääntömomentin säätämistä.

}

\frame {
\frametitle{Ottomoottorin ohjauksen menetelmät}
Ottomoottorin tuottamaa vääntömomenttia säädetään vaikuttamalla
\begin{itemize}
\item sylinterin täytökseen (vaikutetaan sylinterin ilmamäärään)
\item seoksen muodostukseen (kuinka paljon polttoainetta ja millä hetkellä)
\item sytytykseen (sytytyshetken säätö)
\end{itemize}
}

\frame {
\frametitle{Dieselmoottorin ohjauksen menetelmät}
Dieselmoottorin tuottamaa vääntömomenttia säädetään vaikuttamalla
\begin{itemize}
\item polttoaineen ruiskutukseen (millä hetkellä ja kuinka paljon)
\item dieselmoottorin imuilmaa ei kuristeta
\end{itemize}
}


\frame {
\frametitle{Sytytys}
Sytytysjärjestelmän perusperiaate pysynyt samana 1800-luvun lopulta asti: ilman ja polttoaineen seos
sytytetään sähkökipinällä. Moottorinohjauksella vaikutetaan {\bf sytytyshetkeen}.
Sytytyshetken valinta vaikuttaa seuraaviin asioihin
\begin{itemize}
\item moottorista ulos saatava teho
\item polttonesteen kulutus
\item nakuttaako moottori vai ei
\item kuinka puhtaita ovat pakokaasut
\end{itemize}
Kaikkia asioita ei saada parhaaseen mahdolliseen arvoon samanaikaisesti, vaan on tehtävä kompromissi.
Esimerkiksi sytytyksen aikaistaminen lisää moottorin tehoa ja vähentää kulutusta, mutta lisää hiilivety- (HC) ja
typenoksidipäästöjä (NO$\rm _x$).

}

\frame {
\frametitle{Dieselmoottorin polttoaineenruiskutus}
Nykyisin käytössä olevat ruiskutusjärjestelmät ovat
\begin{description}
\item[rivipumppu] Jokaisella sylinterillä on oma nokka-akselin käyttämä pumppuelementti.
\item[jakajapumppu] Yksi pumppumäntä, joka sekä kehittää paineen että jakaa sen sylintereille.
\item[pumppusuutin] Pumppu ja suutin on integroitu
\item[yhteispaineruiskutus] Erillinen korkeapainepumppu, jonka toiminta ei ole sidottu ruiskutushetkiin. Ruiskutusventtiilejä
ohjataan sähköisesti.
\end{description}
}

\frame {
\frametitle{Moottorinohjausjärjestelmän anturit}
Moottorinohjausjärjestelmä tarvitsee säätöä varten tietoa eri suureista, kuten 
\begin{itemize}
\item Lämpötila
\item Kierrosluku
\item Vaihe (nokka-akselin asento)
\item Ilmamäärä (tilavuus tai massa)
\item Nakutus
\item Paine
\item Happipitoisuus (Lambda)
\item Asento (esimerkiksi kaasupolkimen asento)
\end{itemize}
}


\frame {
\frametitle{Palaminen ottomoottorissa}

\begin{itemize}
\item Ilmakertoimella $\lambda$ (lambda) tarkoitetaan ilma-polttonestemassasuhteen poikkeamaa ideaalisesta
polttonestemassasuhteesta.
\item 1 kilogramma polttonestettä vaatii palaakseen noin 14,7 kilogrammaa ilmaa ($\lambda=1$). Tätä kutsutaan stökiometriseksi
seossuhteeksi.
\item Tilavuuksina ilmaistuna: 1 litra polttonestettä vaatii 9500 litraa ilmaa.
\item Imusarjasuihkutteisesta moottorista saadaan suurin teho kun $\lambda\approx 0,85\ldots 0,95$ (rikas seos). Pienin polttoaineenkulutus
taas saavutetaan arvolla $\lambda\approx 1,1\ldots 1,2$ (laiha seos).
\item Kolmitoimikatalysaattorin käyttö edellyttää erittäin tarkkaa seossuhteen säätöä ($\lambda=1 \pm 0,005$).

\end{itemize}
}


\frame {
\frametitle{Bensiinimoottorin seoksenmuodostusjärjestelmät}
Kaksi päätapaa
\begin{itemize}
\item kaasutin
\item suihkutusjärjestelmät
\end{itemize}
Kaasutin oli pitkään suosittu yksinkertaisuutensa vuoksi (=halpa, toimintavarma).
\begin{itemize}
\item Kiristyvät pakokaasupäästönormit vaativat yhä monimutkaisempia kaasutinrakenteita.
\item 1980-luvun lopussa suihkutusjärjetelmät syrjäyttivät kaasuttimen lähes kokonaan. 
\item Pakokaasujen puhdistaminen kolmitoimikatalysaattorilla vaatii erittäin tarkan seossuhteen
säädön.
\end{itemize}
}


\frame {
\frametitle{Suihkutusjärjestelmät}
Polttoaineensuihkutus ei ole uusi keksintö
\begin{itemize}
\item Käytössä mm. lentokoneissa toisen maailmansodan aikana (BMW 801). Kaasuttimet olivat
arkoja jäätymiselle ja g-voimille.
\item Henkilöautoihin 1952.
\item Kehitystä 1970-luvulla.
\item Vuonna 1987 julkaistu Mono-Jetronic oli riittävän edullinen korvaamaan kaasuttimen
myös edullisissa pikkuautoisssa.
\end{itemize}

}

\frame {
\frametitle{Suihkutusjärjestelmien päätyypit}
\begin{description}
\item[Yksipistesuihkutus] Polttoaine suihkutetaan kaasuläpälle.
\item[Yksittäissuihkutus] (tai monipistesuihkutus) Jokaisella sylinterillä oma suihkutusventtiili.
\item[Suorasuihkutus] Polttoaine suihkutetaan suoraan palotilaan.
\end{description}

}

\frame{
\frametitle{Boschin suihkutusjärjestelmien historia}
\begin{tabular}{ l l p{7cm} }
Nimi & Vuosi & Ominaisuudet\\
  D-Jetronic & -67 & Imusarjaohjattu ei-jatkuva monipistesuihkutus. Ohjattu analogiaelektroniikalla.\\
  K-Jetronic & -73 & Mekaanis-hydraulisesti ohjattu jatkuva monipistesuihkutus.\\
  L-Jetronic & -73 & Ilmanvirtausohjattu, elektroninen ei-jatkuva monipistesuihkutus. \\
  LH-Jetronic & -81 & Ilmamassaohjattu, elektroninen ei-jatkuva monipistesuihkutus. \\
  KE-Jetronic & -82 & Elektronisesilla lisäohjauksilla täydennetty K-Jetronic.  \\
  Mono-Jetronic & -87 & Ei-jatkuva yksipistesuihkutus. Ohjattu moottorin nopeuden ja kaasuläpän asennon avulla.\\

\end{tabular}
}

\frame {
\frametitle{D-Jetronic}
\begin{itemize}
\item Vuodelta 1967, imusarjaohjattu ei-jatkuva monipistesuihkutus. Ohjattu analogiaelektroniikalla.
\item D = drücksensorsteuert = paineanturiohjattu
\item Ohjausyksikkö seuraa imusarjan painetta, imuilman lämpötilaa, moottorin lämpötilaa ja 
moottorin kierroslukua.
\item Imusarjan paine = pääohjaussuure. % http://www.scribd.com/doc/7303245/D-Jetronic
\item Käytössä  mm. Saab (99E), Volvo (1800E, 1800ES, ), Porsche (914), MB, BMW \ldots % http://www.pyhimyskerho.com/v2/index.php/huolto/51-d-jetronic
\end{itemize}
}

\frame {
\frametitle{K-Jetronic} % http://www.autotieto.net/polttonestelaitteet/kjetronic_periaate.htm
\begin{itemize}
\item Vuodelta 1973, mekaanis-hydraulisesti ohjattu jatkuva monipistesuihkutus. Käytössä 1995 asti.
\item K = kontinueirlich = jatkuva. Polttoainetta suihkutetaan jatkuvasti; kun imuventtiili aukeaa, se imaisee 
venttiilin etupuolella olevan polttoainepilven sisään.
\item Ilmamäärää mitataan mekaanisen levyanturin avulla. Levy ohjaa polttoaineen määränjakajaa. Vastakkaiseen suuntaan
määränjakajan mäntää painaa {\bf ohjauspaine}, jonka avulla säädetään seosta rikkaammaksi lämpiämisvaiheessa.
\item Voidaan käyttää myös Lambda-säädön kanssa.
\item Edullisempi ja luotettavampi kuin D-Jetronic
\begin{itemize}
\item Vielä 1970-luvulla elektroniikka oli kallista ja mekaniikka halpaa (nykyään tilanne on päinvastoin). Myös elektroniikan
luotettavuus on kehittynyt huomattavasti. 
\end{itemize} 
\end{itemize}
}

\frame {
\frametitle{L-Jetronic}
\begin{itemize}
\item Vuodelta 1973, ilmanvirtausohjattu, elektroninen ei-jatkuva monipistesuihkutus.
\item Järjestelmässä on kehitetty D-Jetronicista saatujen kokemusten perusteella. Suora ilmanvirtausmittaus
mahdollistaa moottorin muutoksiin reagoimisen (toisin kuin D-Jetronicin paineeseen perustuva säätö), kuten
palotilan karstoittuminen, kuluminen, venttiilivälysten muutokset.
\item L3-Jetronic = kuten L-Jetronic mutta toteutettu digitaalielektroniikalla.
\item L = Luft = ilma.
\item Lambdalla tai ilman.
\end{itemize}
}

\frame {
\frametitle{LH-Jetronic}
\begin{itemize}
\item Vuodelta 1981, ilmamassaohjattu, elektroninen ei-jatkuva monipistesuihkutus.
\item Erona L-Jetroniciin on se, että ilmanvirtauksen (l/h) sijasta mitataan ilmamassaa (kg/h).
\item Anturina on kuumakalvo- tai kuumalankavastus. Lankaan syötetään vakiovirta, jolloin
langan jännitettä mittaamalla voidaan päätellä sisään virtaava ilmamassa.
\item LH-Jetronicin digitaaliseen ohjainlaitteeseen on tallennettu säätökäyrästö
(seoksen säätö kuormituksen ja pyörintänopeuden perusteella).
\item Joissain malleissa on ilman pyörteilyyn perustuva Kauman-Vortex -tilavuusvirtausmittari.

\end{itemize}
}

\frame {
\frametitle{KE-Jetronic}
\begin{itemize}
\item Vuodelta 1982, elektronisesilla lisäohjauksilla täydennetty K-Jetronic. Ohjauspainetta
säädetään elektronisesti.
\end{itemize}
}

\frame {
\frametitle{Mono-Jetronic}
\begin{itemize}
\item Vuodelta 1987, ei-jatkuva yksipistesuihkutus. Ohjattu mikroprosessorilla.
\item Ohjaussuureina voidaan käyttää kaasuläpän asentoa, kierroslukua, tuloilman
lämpötilaa, pakokaasun jäännöshappipitoisuutta ($\lambda$-säätö), ilmastointikompressorin
käyntitietoa ym.
\item Mikroprosessoriohjaus mahdollistaa "oppivan"\ järjestelmän. Pakokaasupäästöt ja optimaalinen
suorituskyky kohtuullisina koko auton käyttöiän ajan.
\end{itemize}
}

\frame{
\frametitle{Jetronicit nykyään}
\begin{itemize}
\item K-Jetronic oli tuotannossa vuoteen 1995 asti, LH vielä vuoteen 1998 asti.
\item Uudet Boschin ohjausjärjestelmät kulkevat Motronic-nimen alla.
\end{itemize}

}


\frame{
\frametitle{Motronic}
\begin{itemize}
\item Tuotantoon vuonna 1979, kehitetty jatkuvasti.
\item Yhdistetty sytytyksen ja suihkutuksen ohjausjärjestelmä.
\item M-Motronic, mekaaninen kaasupoljin.
\item ME-Motronic, elektroninen kaasupoljin.
\item ME7-Motronic, ohjausjärjestelmä yhdelle mikrosirulle (1998).
\item MED-Motronic, suorasuihkutusjärjestelmä (2000).
\item Bifuel-Motronic (maakaasu \& bensiini).
\end{itemize}

}

\frame{
\frametitle{Nykyaikainen MED-Motronic}
\begin{itemize}
\item Suorasuihkutuksen avulla voidaan polttoaineen kulutusta vähentää noin 20 \%.
\item Perusajatus: syöttämällä polttoaine suoraan palotilaan (eikä imukanavaan) voidaan
kontrolloida, muodostuuko palotilaan homogeeninen (tasainen) seos vai ei.
\item Täyskuormalla käytetään homogeenista seosta.
\item Osakuormalla käytetään {\bf kerrossyöttöä}. Kerrossyötössä riittää, että sytytystulpan tienoilla on
syttymiskelpoinen seos. Muualla sylinterissä on jäännöskaasuja ja ilmaa (sylinterissä ilmaylimäärä). Tämä saavutetaan suihkuttamalla
polttoaine juuri ennen sytytyshetkeä.
\end{itemize}

}

\frame{
\frametitle{Kerrossyöttö}
\begin{itemize}
\item Kerrossyötössä moottori käy ilmaylimäärällä ($\lambda>1$), mikä vähentää polttoaineenkulutusta mutta 
lisää NO$_{\rm X}$-päästöjä. Kerrossyöttötilassa kaasuläppä on täysin auki, moottoria ei kuristeta.
\item Tämä edellyttää tehokkaampaa pakokaasujen puhdistusta.
\item Pakokaasujen takaisinkierrätyksen (EGR) avulla NO$_{\rm X}$-päästöjä saadaan vähennettyä
jopa 70 \%.
\item Nykyiset pakokaasumääräykset ovat niin tiukkoja, että tämäkään ei riitä, vaan on käytettävä
NO$_{\rm X}$-varaajakatalysaattoria.
\end{itemize}

}


\frame{
\frametitle{Polttoaineensyöttöjärjestelmä}
Polttoaineensyöttöjärjestelmän tehtävä on kuljettaa polttoainetta suihkusuuttimille
\begin{itemize}
\item riittävällä virtaamalla
\item oikeansuuruisella paineella.
\end{itemize}
}

\frame{
\frametitle{Imusarjasuihkutuksen ja suorasuihkutuksen vaatimukset}
\begin{itemize}
\item Imusarjasuihkutuksessa ei tarvita korkeaa painetta. Riittää, että yksi pumppu pumppaa
polttoaineen polttoainesäiliöltä suihkusuuttimille.
\item Suorasuihkutuksessa suoraan palotilaan vaaditaan korkea paine. Tällöin käytetään
kahta pumppua. Pienpainepumppu tuo polttoaineen polttoainesäiliöltä moottoritilaan, jossa
korkeapainepumpun avulla kehitetään suorasuihkutuksen vaatima paine.

\end{itemize}
}

\frame{
\frametitle{Imusarjasuihkutus}
\begin{itemize}
\item Paine tyypillisesti $<$ 5 baaria.
\item Riittävä paine estää höyrykuplien muodostumisen polttoainelinjaan.
\item Pumppuun integroitu takaiskuventtiili pitää paineen polttoainelinjassa vielä moottorin
sammuttamisen jälkeen, jolloin linjaan ei muodostu höyrykuplia.
\end{itemize}
}

\frame{
\frametitle{Paluukierrolla varustettu järjestelmä}
\begin{itemize}
\item Pumppu pumppaa polttoainetta jakoputkelle.
\item Jakoputkella on paineentasaaja (regulaattori), joka pitää suihkusuuttimien ja imusarjan välisen paineen vakiona.
\item Säätö tapahtuu paluuvirtausta kontrolloimalla.
\item Mitä huonoja puolia tässä järjestelmässä on?
\end{itemize}
}

\frame{
\frametitle{Paluukierroton järjestelmä}
\begin{itemize}
\item Jatkuva polttoaineen kierrättäminen moottoritilan läpi takaisin polttoainesäiliöön
nostaa säiliössä olevan polttoaineen lämpötilaa. Tämä taas johtaa suurempiin polttoainehöyrypäästöihin.
\item Ratkaisu: paluukierroton järjestelmä.
\item Paluukierrottomassa järjestelmässä paineentasaaja sijaitsee polttoainesäiliöllä. Paluukiertoa ei tarvita.
\item Koska paineentasaaja ei ole jakoputkella vaan polttoainesäiliöllä eikä pidä suhteellista suihkutuspainetta (=imusarjan ja jakoputken paine-eroa) vakiona,
 tulee moottorinohjausyksikön huolehtia imusarjan paineen huomioon ottamisesta ja muuttaa suihkutuksen kestoa tarpeen mukaan.
\end{itemize}
}

\frame{
\frametitle{Tarveohjattu järjestelmä}
\begin{itemize}
\item Mekaanisen paineentasaajan sijasta järjestelmässä on polttoainepaineanturi.
\item Moottorinohjausyksikkö säätää pumpun käyntiä 
\end{itemize}
Järjestelmän edut:
\begin{itemize}
\item Ei pumpun hukkakäyntiä $\to$ vielä vähemmän säiliön lämpenemistä.
\item Suihkutuspaineen vapaampi säätö (esim. ahdetuissa moottoreissa).
\item Tarkempi säätö ylipäätään (kun tiedetään polttoainelinjan ja imusarjan paine tarkasti).
\end{itemize}

}

\frame{
\frametitle{Polttoaineensyöttö suorasuihkutuksessa}
\begin{itemize}
\item Suorasuihkutus suoraan palotilaan vaatii korkean paineen.
\item Miksi?
\end{itemize}
}

\frame{
\frametitle{Polttoaineensyöttö suorasuihkutuksessa}
\begin{itemize}
\item Suorasuihkutus suoraan palotilaan vaatii korkean paineen, koska
aikaikkuna polttoaineen suihkuttamiselle on erittäin lyhyt.
\item Korkea paine mahdollistaa riittävän polttoainemäärän suihkuttamisen
nopeasti.
\item Polttoainejärjestelmä jakautuu {\bf matalapainepiiriin} ja {\bf korkeapainepiiriin}.
\item Miksi tarvitaan erillinen matalapainepiiri?
\end{itemize}
}

\frame{
\frametitle{Polttoaineensyöttö suorasuihkutuksessa}
\begin{itemize}
\item Korkeapainepumppu tarvitsee riittävän korkean syöttöpaineen, ettei
tuloletkuun synny höyrykuplia.
\item Matalapainepiirin syöttöjärjestelmä ei juurikaan eroa imusarjasuihkutteisten järjestelmien
polttoainejärjestelmästä. Tarveohjattuja pienpainepumppuja käytetään yleisesti.
\end{itemize}
}

\frame{
\frametitle{Polttoaineensyöttö suorasuihkutuksessa}
\begin{itemize}
\item Uusissa järjestelmissä suihkutuspaine voi olla jopa 200 baaria, vanhemmissa 50-12 baaria.
\end{itemize}
}

\frame{
\frametitle{Jatkuvasyöttöinen järjestelmä (vanhempi tekniikka)}
\begin{itemize}
\item Korkeapainepumppu syöttää vakiovirtauksella polttoainetta jakoputkeeen.
\item Jakoputken päässä on paineenalennusventtiili, joka syöttää ylimääräisen polttoaineen takaisin
pienpainepiiriin.
\item Turha pumppaaminen kuluttaa energiaa ja polttoaineen kierrättäminen nostaa polttoainepiirin lämpötilaa.
\end{itemize}
}

\frame{
\frametitle{Tarveohjattu järjestelmä}
\begin{itemize}
\item Moottorinohjausyksikkö ohjaa pumpun säätöventtiiliä niin, että jakoputkessa pysyy moottorin
toimintatilaan nähden oikea paine.
\end{itemize}
}

\frame{
\frametitle{Polttoainepumput}
\begin{itemize}
\item Korkeapainepumppu on tyypillisesti nokka-akselilta tehonsa saava mäntäpumppu.
\item Pienpainepiirissä käytetään sähkömoottoripumppua, joka on toimintaperiaatteeltaan joko
tilavuuspumppu (positive-displacement pump) tai virtauspumppu.
\item Tilavuuspumppu eli syrjäytyspumppu siirtää nestemääriä suljetuissa kammioissa eteenpäin. Tilavuuspumppuja
ovat esimerkiksi keskipakorullapumppu ja sisähammasrataspumppu.
\item Virtauspumppu on toinen pumpputyyppi. Polttonestepumppuina käytetään mm. ulkokehä- ja sivukanavapumppuja, jotka
ovat virtauspumppuja.  
\end{itemize}
}

\frame{
\frametitle{Polttoaineensuodattimet}
\begin{itemize}
\item Polttoaineensuodattimen tehtävä on poistaa polttoaineen seasta hiukkasia, jotka voivat a) tukkia järjestelmän b) aiheuttavat
eroosiota.
\item Uusin suuntaus on, että polttoainesuodatin suunnitellaan koko auton eliniän kestäväksi ja se integroidaan säiliöasennusyksikköön.
\item Edelleen käytetään myös polttoainelinjaan kiinnitettäviä suodattimia.
\item Miksi polttoaineensuihkutusjärjestelmän käyttö asettaa tiukemmat vaatimukset polttoaineensuodattimille kuin kaasuttimen käyttö?
\end{itemize}
}

\frame{
\frametitle{Polttoaineensuodattimet}
\begin{itemize}
\item Materiaalina käytetään mikrokuitupaperia ja keinokuitukudoksia.
\item Polttoaineensuodattimen huokoskoko tulee olla luokkaa 10 $\mu$m (imusarjasuihkutus) tai 5 $\mu$m
(suorasuihkutus).
\item Polttoaineensuodattimien vaihtoväli on tyypillisesti luokkaa 30 000 km - 90 000 km. Polttoainesäiliöön asennettavissa
suodattimissa vaihtoväli on 160 000 km tai pidempi (jopa 250 000 km). 
\end{itemize}
}

\frame{
\frametitle{Boschin korkeapainepumput}
Korkeapainepumppu = Hochdruckpumpe = HDP
\begin{itemize}
\item HDP1 (kolme mäntää)
\item HDP2 (yksi mäntä)
\item HDP5 (yksi mäntä; uusinta tekniikkaa)
\end{itemize}
}

\frame{
\frametitle{Jakoputki}
\begin{itemize}
\item (Korkeapaine)pumppu syöttää polttoaineen jakoputkeen ({\em fuel rail}). Suihkutussuuttimet ovat kiinni suoraan
jakoputkessa.
\item Jakoputken tehtävä on taata jokaiselle suuttimelle yhtäläinen polttoaineen jakelu (jokaisella suuttimella sama paine).
\item Lisäksi jakoputkeen on kiinnitetty paineanturi, 
\item Putken koko on mitoitettava käytettävän moottorin mukaan. Suuttimien toiminta ei saa aiheuttaa
(merkittäviä) paineen vaihteluja putkeen.
\item Boschin KSZ-HD -jakoputki on valmistettu alumiinista (HDP1 ja HDP2) tai ruostumattomasta teräksestä HDP5.
\end{itemize}
}

% Sytytys käsitellään omassa filessään, sytytysjarjestelma.tex



\frame{
\frametitle{Suorasuihkutus bensiinimoottorissa (GDI)}
\begin{itemize}
\item Määritelmä: polttoaine suihkutetaan suoraan palotilaan, eikä esim. imusarjaan.
\item Toisin sanoen seoksenmuodostus tapahtuu sylinterin sisällä (nk. sisäinen seoksenmuodostus).
\end{itemize}
}

\frame{
\frametitle{Historia}
\begin{itemize}
\item Tekniikka ei ole mitenkään uusi.
\begin{itemize}
\item Hesselman-moottori vuonna 1925.
\item Toimi bensiinillä ja petrolilla, matala puristussuhde.
\item Käytössä vuoteen 1947 asti.
\end{itemize}
\item Lentokonemoottorissa vuonna 1937.
\item Gutbrod -henkilöauto vuonna 1951.
\item Mercedes 300 SL vuonna 1954.
\item "Tavallisiin"\ autoihin vasta edellisten 15 vuoden aikana.
\item Mitsubishi GDI (1996).
\end{itemize}
}

\frame{
\frametitle{Haasteet}
\begin{itemize}
\item Moottorinohjauksen haasteet (ohjaus vaikeaa toteuttaa mekaanisesti).
\item Materiaali- ja valmistustekniikka.
\item Liittyen edellisiin: huollontarve.
\item Elektroniikan ja materiaalitekniikan kehitys johtanut suorasuihkutuksen yleistymiseen. 
\end{itemize}
}

\frame{
\frametitle{Sisäisen seoksenmuodostuksen edut}
\begin{itemize}
\item Kun polttoaine voidaan suihkuttaa puristustahdin aikana, voidaan suihkutushetken valinnalla vaikuttaa erittäin
paljon siihen, millainen seos palotilaan muodostuu.
\item Edellyttää palotilan ja imukanavien tarkkaa muotoilua.
\item Pienempi polttoaineenkulutus, koska moottoria ei tarvitse kuristaa.
\item Polttoaineen suihkutus suoraan palotilaan aikaansaa sisäisen jäähdytyksen $\to$ puristussuhdetta
voidaan nostaa.
\end{itemize}
}

\frame{
\frametitle{Toimintatilat}
\begin{itemize}
\item Suorasuihkutusmoottorin toiminta voidaan jakaa kahteen päätoimintatilaan.
\item Homogeenikäyttö. Suihkutus tapahtuu aikaisessa vaiheessa, jolloin palotilaan muodostuu tasainen seos.
\item Kerrossyöttö. Suihkutus tapahtuu puristustahdin lopussa niin, että polttoainepilvi saavuttaa sytytystulpan, mutta
ympärille jää puhdasta ilmaa.
\item Perustiloja voidaan yhdistellä ja muuttaa eri käyttötiloihin sopiviksi.
\end{itemize}
}

\frame{
\frametitle{Homogeenikäyttö}
\begin{itemize}
\item Polttoaine suihkutetaan imutahdin aikana.
\item Seossuhde yleensä $\lambda=1$, täyskuormalla myös hieman alle 1.
\item Koko sylinteri saadaan käyttöön, maksimiteho ulos moottorista.
\item Käytännössä samanlainen toiminta kuin imusarjasuihkutuksessa.
\end{itemize}
}

\frame{
\frametitle{Kerrossyöttö}
\begin{itemize}
\item Polttoaine suihkutetaan puristustahdin lopussa juuri ennen sytytystä. 
\item Suuttimesta tuleva polttoaine(ilmaseos)pilvi ohjataan tulpalle.
\item Muualla sylinterissä on (teoriassa) pelkkää ilmaa.
\item Koko palotilassa on ilmaylimäärä. Tämä nostaa NO$_{\rm x}$-päästöjä.
\item Hyöty: moottoria ei tarvitse kuristaa (kaasuläppä auki), lämpö ei karkaa seinämien kautta, pienempi polttoaineen kulutus. 
\item Ei toimi suurilla ($>3000$ rpm) moottorin nopeuksilla (polttoainepilvi ei ehdi homogenisoitua sisäisesti).
\item Pieni kierrosluku suurella kuormalla ei toimi myöskään tässä tilassa (nokeaminen).
\end{itemize}
}

\frame{
\frametitle{Homogeeninen laihaseoskäyttö}
\begin{itemize}
\item Käytetään kerrossyötön ja homogeenikäytön "välitilassa".
\item Kaasuläppä täysin auki, pienempi polttoaineenkulutus.
\item Haittana NO$_{\rm x}$-päästöt.
\end{itemize}
}

\frame{
\frametitle{Homogeeninen kerrossyöttö}
\begin{itemize}
\item Ensin suihkutetaan 75 \% polttoaineesta imutahdissa, ja juuri ennen sytytystä loput 25 \%.
\item Käytetään siirryttäessä kerrossyötön ja homogeenikäytön välillä. Syy: tasaisempi muutos väännössä.
\item Voidaan käyttää myös pienillä kierrosnopeuksilla nokemisen estoon.
\end{itemize}
}

\frame{
\frametitle{Homogeeninen split-mode}
\begin{itemize}
\item Suihkutuksen ajoitus kuten homogeenisessa kerrossyötössä, mutta sytytys annetaan myöhässä (YKK:n jälkeen).
\item Tällöin vääntö ei kasva, mutta pakokaasujen lämpötila nousee.
\item Käytetään katalysaattorin nopeaan lämmitykseen käynnistyksen jälkeen.
\end{itemize}
}

\frame{
\frametitle{Homogeeninen nakutuksenestotila}
\begin{itemize}
\item Käytetään matalilla kierroksilla täyskuormalla ehkäisemään nakutusta.
\item Seoksen kerrostuminen ehkäisee nakutusta.
\item Suihkutus kuten homogeenisessa kerrossyötössä.
\item Kun nakutus ehkäistään syöttöä muuttamalla eikä sytytystä myöhentämällä, väännössä ei menetetä.
\end{itemize}
}


\frame{
\frametitle{Katalysaattorin lämmitys kerrossyötöllä}
\begin{itemize}
\item Normaalin kerrossyötön lisäksi polttoainetta suikutetaan myös työtahdin aikana.
\item Nostaa pakokaasujen lämpötilaa.
\item Ei toimi kylmällä moottorilla yhtä hyvin kuin split-tila.
\item Käytetään myös varaajakatalysaattorin puhdistukseen (rikinpoisto), koska tällä menetelmällä on helppo
saavuttaa erittäin korkea pakokaasujen lämpötila (yli 650 astetta) laajalla kuormitusalueella.
\end{itemize}
}

\frame{
\frametitle{Käynnistys kerrossyötöllä}
\begin{itemize}
\item Polttoaine suihkutetaan puristustahdin aikana, jolloin ilma on kuumempaa kuin imutahdin aikana. Tällöin
polttoaine höyrystyy paremmin.
\item Vähentää polttoainekalvon muodostumista sylinterin reunoille.
\item Vähentää hiilivetypäästöjä käynnistyksen aikana.
\end{itemize}
}

\frame{
\frametitle{Seoksenmuodostusjärjestelmät}
\begin{itemize}
\item Kerrossyötön seoksenmuodostus voidaan toteuttaa kolmella eri tavalla: suihkuohjattu, seinämäohjattu ja ilmaohjattu seoksenmuodostus.
\item Tavoitteena on saada syttymiskelpoinen seospilvi sytytystulpalle oikealla hetkellä.
\item Suihkuohjatussa seoksenmuodostuksessa suihkusuutin on sijoitettu keskelle sylinterin kattoa. Polttoaine suihkutetaan suoraan tulppaan.
Huono puoli: erittäin lyhyt aika seoksenmuodostumiselle, vaatii korkean suihkutuspaineen. Myös tulpan kemiallinen rasitus on suuri. Hyvä puoli:
tällä tekniikalla saavutetaan pienin polttoaineenkulutus.
\item Seinämäohjatussa seoksenmuodostuksessa polttoaine suihkutetaan männässä olevaan koloon, josta suihku ohjautuu tulpalle.
\item Ilmaohjatussa seoksenmuodostuksessa palotilan ilmavirtaukset ohjaavat seospilven tulpalle.
\end{itemize}
}

\frame{
\frametitle{Sytytys}
\begin{itemize}
\item Homogeenikäytössä sytytys ei käytännössä eroa mitenkään imusarjasuihkutteisesta moottorista.
\item Kerrossyötöllä seinämä- ja ilmaohjatussa seoksenmuodostuksessa sytytyshetken valinnassa ei ole juurikaan
vapautta: seos sytytetään silloin, kun polttoainepilvi on tulpan kohdalla.
\item Suihkuohjatussa seoksenmuodostuksessa polttoaineen suihkuttaminen suoraan tulpalle takaa sen, että myös myöhäisellä suihkutushetkellä
tulpalla on riittävästi seosta, vaikka suihkutettu määrä olisi pienikin.
\end{itemize}
}

\frame{
\frametitle{Suihkusuuttimen toiminta suorasuihkutusmoottorissa}
\begin{itemize}
\item Suihkusuuttimen tehtävä on suihkuttaa palotilaan oikea määrä polttoainetta oikealla hetkellä. 
\item Suorasuihkutus vaatii korkean suihkutuspaineen, koska suihkutuksen on tapahduttava nopeasti ja 
polttoaineen on höyrystyttävä nopeasti.
\item Rakenteeltaan suihkusuutin on magneettiventtiili.
\end{itemize}
}

\frame{
\frametitle{Suihkusuutin imusarjasuihkutuksessa}
\begin{itemize}
\item Perustekniikka sama kuin suorasuihkutuksessa.
\item Suuttimen suihkun muoto riippuu moottorin rakenteesta. Esimerkiksi kahdella imuventtiilillä
varustettuihin sylintereihin voidaan käyttää kaksoissuihkun tuottavaa suutinta.
\end{itemize}
}

\frame{
\frametitle{Pakokaasujen takaisinkierrätys (EGR)}
\begin{itemize}
\item NO$_{\rm x}$-päästöjen määrä kasvaa eksponentiaalisesti palamislämpötilan noustessa.
\item Kierrättämällä pakokaasuja imuilman sekaan, voidaan palamislämpötilaa laskea ja näin vähentää NO$_{\rm x}$-päästöjä jopa 70 \%.
\item Voidaan toteuttaa sisäisesti ja/tai ulkoisesti.
\item Sisäinen pakokaasujen takaisinkierrätys tapahtuu avaamalla imuventtiili poistotahdin loppupuolella.
\item Ulkoinen kierrätys tapahtuu pako- ja imusarjan välisen kierrätyslinjan kautta.
\item Pakokaasujen takaisinkierrätys vähentää myös polttoaineenkulutusta, koska se vähentää moottorin kuristamistarvetta.
\end{itemize}
}


\frame{
\frametitle{Diagnoositekniikka}
\begin{itemize}
\item Nykyaikainen moottorinohjausjärjestelmä tallentaa tiedot vioista ja poikkeavuuksista moottorinohjausyksikön
muistiin.
\item Tiedot voidaan lukea sopivalla työkalulla diagnoosiliitännästä.
\item Liitäntä on standardoitu (ISO 15031-1).
\item 1990-luvulla eri automerkeillä oli omat diagnoosiliitäntänsä.
\end{itemize}
}

\frame{
\frametitle{Mitä diagnostiikkajärjestelmä valvoo?}
\begin{itemize}
\item Loogisuustarkastukset (esimerkiksi: kampiakselin pyörintänopeustunnistimen ja nokka-akselin asemantunnistimen kesken).
\item Reagointitarkastus (esimerkiksi: tapahtuuko mitään, kun EGR-venttiilin asemaa muutetaan).
\item Signaalialueen tarkastus (esimerkiksi: antaako lämpötila-anturi mielekkään lämpötilan).
\item Moottorin sammuttamisen jälkeen tehdään aikaavievät testit (esimerkiksi EPROM-muistin tarkastus).
\end{itemize}
}

\frame{
\frametitle{Vikaan reagoiminen}
\begin{itemize}
\item Vikaan reagoidaan tapauskohtaisesti.
\item Signaalipiiri luokitellaan vialliseksi, kun vika säilyy päällä tietyn ajanjakson.
\item Järjestelmä käyttää viimeiseksi saatua loogista arvoa, kunnes vika on todettu pysyväksi.
\item Kun vika on todettu pysyväksi, käytetään jotain sopivaa oletusarvoa (esimerkiksi
oletetaan, että moottorin lämpötila on 90 $^\circ$C).
\item Jos signaali palautuu itsestään, se hyväksytään, kun se on pysynyt loogisena tietyn ajanjakson.
Signaalin puuttuminen tallennetaan vikamuistiin.
\item Vikamuistiin tallennetaan myös olosuhteet, joilla vika on ilmentynyt (kierrosluku, lämpötila ym.).
\item Kuljettajalle sytytetään vikamerkkivalo (MIL = malfunction indicator light).
\end{itemize}
}

\frame{
\frametitle{Vikatietojen lukeminen}
\begin{itemize}
\item Vikatiedot luetaan erillisellä diagnoosilaitteella.
\item Kommunikointi tapahtuu K-linjan tai CAN-protokollan avulla.
\item Testilaitteita on erilaisia. Anturitietojen lukemiseen sekä vikakoodien lukemiseen ja nollaamiseen kykeneviä laitteita saa
alle sadalla eurolla. Merkkikohtaiset ja/tai monipuoliset korjaamotesterit, joilla voidaan ajaa eri testejä (esimerkiksi puristuspainetesti)
maksavat muutamia tuhansia euroja.
\end{itemize}
}

\frame{
\frametitle{OBD = On-Board Diagnosis}
\begin{itemize}
\item Vuodesta 1988 alkaen CARB (California Air Resources Board) vaati kaikkiin Kaliforniassa rekisteröityihin autoihin pakollisen OBD-järjestelmän (OBD I).
\item OBD I:ssä riitti, että pakokaasujenpuhdistusjärjestelmää valvottiin oikosulkujen ja katkosten varalta.
\item Viasta ilmoitettiin kuljettajalle vikamerkkivalolla. Vian sijainti piti pystyä lukemaan ilman ulkoisia
apuvälineitä esimerkiksi diagnoosivalon vilkkukoodina.
\end{itemize}
}

\frame{
\frametitle{OBD II}
\begin{itemize}
\item Vuonna 1994 Kaliforniassa voimaan tiukempi OBD II.
\item Voimaan myös muissa osavaltioissa, hieman lievemmillä rajoituksilla.
\item OBD II vaatii, että kaikkia sellaisia järjestelmiä valvotaan, joiden rikkoutumisella on merkitystä pakokaasupäästöjen takia.
\item Ei valvota pelkästään anturien toimintaa, vaan puututaan järjestelmän epäloogiseen toimintaan. Esimerkiksi, jos moottorin
käyntilämpötila on kauan alle 10 $^\circ C$, annetaan vikahälytys.
\item Vikakoodien lukeminen tapahtuu standardoidun diagnoosiliitännän kautta. Avoin standardi mahdollistaa koodien lukemisen
vapaasti ostettavan testerin avulla.
\end{itemize}
}

\frame{
\frametitle{EOBD}
\begin{itemize}
\item EOBD = eurooppalaisiin olosuhteisiin sovitettu OBD II. Pakollinen bensiinimoottorilla varustetuissa henkilöautoissa
vuodesta 2001 ja dieselmoottorilla varustetuissa henkilöautoissa vuodesta 2004.
\item EOBD poikkeaa OBD II:sta mm. siten, että päästöraja-arvot ovat absoluuttisia. OBD II:ssa raja-arvot ovat 1,5-kertaiset
kyseisen päästökategorian raja-arvoihin verrattuna. 
\end{itemize}
}

\frame{
\frametitle{Järjestelmätestit}
\begin{itemize}
\item Varsinaisen vikatunnistuksen lisäksi diagnostiikka tekee useita eri kuntotestejä
ajoneuvon järjestelmille.
\item EOBD vaatii mm. katalysaattorin kuntotestin, palamiskatkosten ja lambda-anturien kuntotestit.
\item CARB-OBD (Kalifornia) on huomattavasti tiukempi, sisältäen vaatimukset mm. polttonestesäiliön
vuotodiagnoosille ja kampikammion tuuletuksen diagnoosille.  
\end{itemize}
}

\frame{
\frametitle{Protokollat}
\begin{itemize}
\item Testeri voi kommunikoida diagnoosiliitännän kautta K-linjan tai CAN-väylän välityksellä.
\item Koska ajoneuvon sisällä käytetään CAN-väylää viestintään, tämä on yleistyvä tapa.
\item Liittimessä on nastat eri protokollille.
\item Nastat 2 ja 10: SAE J1850 (PWM ja VPWM)
\item Nastat 7 (K-linja) ja 15 (L-linja): ISO 14230-(1-4) (KWP 2000), KWP 71 ja ISO 9141-2
\item Nastat 6 ja 14: ISO 15765-4 (CAN) 
\end{itemize}
Sovelluskerroksen protokollana käytetään McMess-protokollaa (K-linja) ja CCP-protokollaa (CAN).
}

%http://www.kbmsystems.net/obd_tech.htm
\frame{
\frametitle{Protokollat}
Protokollien käyttö (karkeasti) eri valmistajilla:
\begin{itemize}
\item CAN-väylä käytössä mm. uusissa: Ford, Mazda, Volvo.
\item VPWM: GM, Chrysler
\item PWM: ennen vuotta 2004 valmistetut Fordit ja Mazdat.
\item KWP2000: Useimmat eurooppalaiset ja aasialaiset autot.
\item Bittien käsitteleminen on helppoa: halpakin koodinlukija tukee usein kaikkia protokollia.
\end{itemize}
}


\frame{
\frametitle{Motronic}
\begin{itemize}
\item Tuotantoon vuonna 1979, kehitetty jatkuvasti.
\item Yhdistetty sytytyksen ja suihkutuksen ohjausjärjestelmä.
\item M-Motronic, mekaaninen kaasupoljin.
\item ME-Motronic, elektroninen kaasupoljin.
\item ME7-Motronic, ohjausjärjestelmä yhdelle 16-bittiselle mikrosirulle (1998).
\item ME9-Motronic, 32-bittinen suoritin.
\item MED-Motronic, suorasuihkutusjärjestelmä (2000).
\item Bifuel-Motronic (maakaasu \& bensiini).
\item MEG-Motronic (integroitu vaihteistonohjaukseen), MP-Motronic (integroitu MAP-anturi) \ldots
\end{itemize}
Laajasti käytössä eurooppalaisissa autoissa.
}

\frame{
\frametitle{Nimeäminen}
\begin{itemize}
\item Kirjainkoodi kertoo päätyypin (ME, MED jne.)
\item Numero kertoo version ("sukupolven"), esim. M1, M3, M7, ME7.
\item Ajoneuvovalmistajien vaatimusten perusteella kehitetyt variaatiot
nimetään versionumeron jälkeisellä pisteellä (esim. M4.3 (Volvo), MED9.1 (VAG: VW, Seat, Audi\ldots)).
\item Huom! Pisteen jälkeinen numero vaihtelee eri sarjoissa: esim. Motronic 4.1 käytettiin Opelissa ja Peugeotissa,
ja MED9.1 on käytössä VAG:n autoissa.
\end{itemize}
}

\frame{
\frametitle{Motronic}
Ohjausjärjestelmä säätää
\begin{itemize}
\item Kaasuläppää
\item Sytytysennakkoa
\item Suihkutuksen ajankohtaa
\item Hukkaporttia (ahdetut moottorit)
\item Venttiileitä (muuttuva venttiilien ajoitus, Valvetronic)
\end{itemize}
}

\frame{
\frametitle{M-Motronic}
\begin{itemize}
\item Vanhat M-motronicit käsittelivät vääntömomenttitoiveen erikseen joka järjestelmässä:
säätämällä sytytystä, muuttamalla suihkutusaikaa.
\item Uusissa (7-versiosta alkaen) järjestelmissä vääntömomenttitoive käsitellään keskitetysti.
\end{itemize}
}

\frame{
\frametitle{ME-Motronic}
\begin{itemize}
\item Erona M-Motroniciin: elektroninen kaasuläpän ohjaus, vääntömomenttiohjaukseen perustuva ohjelmisto.
\end{itemize}
}

\frame{
\frametitle{MED-Motronic}
\begin{itemize}
\item Keskeinen ero ME-Motroniciin verrattuna on varaajakatalysaattorin ohjausmahdollisuus sekä suorasuihkutuksen ohjaus.
\end{itemize}
}

\frame{
\frametitle{Alijärjestelmät päätoimintoineen}
\begin{itemize}
\item Järjestelmädokumentointi ja -ohjaus (SD, SC)
\item Vääntömomenttitoive (TD)
\item Vääntömomenttiohjaus (TS)
\item Täytöksen säätö (AS)
\item Polttoainejärjestelmä (FS)
\item Sytytysjärjestelmä (IS)
\item Pakokaasujärjestelmä (ES)
\item Toimintotiedot (OD) ja  Tiedonsiirto (CO)
\item Oheislaiteohjaus (AC)
\item Valvonta (MO) ja Diagnostiikka (DS)
\end{itemize}
}

\frame{
\frametitle{Järjestelmädokumentointi ja -ohjaus (SD, SC)}
\begin{itemize}
\item Järjestelmädokumentointiin on tallennettu ajoneuvon ja moottorin tiedot.
\item Järjestelmäohjaus sisältää ohjausyksikön "käyttöjärjestelmän": mitä tehdään
järjestelmän käynnistyksessä, mitä normaalitilassa ja mitä sytytysvirran katkaisun
jälkeen (esim. puhaltimen jälkikäyttö, laitteistotestit).
\end{itemize}
}

\frame{
\frametitle{Vääntömomenttitoive (TD)}
Järjestelmä käsittelee kaasupolkimelta tulevan momenttipyynnön.
\begin{itemize}
\item Molempien potentiometrien asennot luetaan ja esikäsitellään.
\item Polkimen asennosta lasketaan ohjearvo vääntömomentille.
\item Vääntömomenttipyynnölle lasketaan toimintatilan mukaiset rajoitteet (pyörintänopeuden rajoitus, värähtelynvaimennus).
\item Tyhjäkäynnille lasketaan sopiva vääntömomentti mm. moottorin lämpötilan, akkujännitteen ja ilmastoinnin toimintatilan mukaan.
\item Vakionopeudensäädin voi myös pyytää vääntömomenttia.
\end{itemize}
}

\frame{
\frametitle{Vääntömomenttiohjaus (TS)}
\begin{itemize}
\item Yksikkö laskee sopivan sylinterintäytöksen (ilmamäärän) ja sytytyshetken.
\item Vääntömomenttimallinnusyksikkö laskee täytöksen, lambdan, sytytyshetken, suihkutuksen ja
kierrosluvun perusteella optimaalisen vääntömomentin.
\item Viestit välitetään eteenpäin sytytys-, täytöksen säätö- ja polttonestejärjestelmälle.
\end{itemize}
}

\frame{
\frametitle{Täytöksen säätöjärjestelmä (AS)}
\begin{itemize}
\item Järjestelmä laskee kaasuläpän kulman ja mahdollisen pyörteytysläpän asennon. Jos moottorissa on
säädettävä venttiilinajoitus, lasketaan ohjearvot myös niille.
\item Järjestelmä säätää myös pakokaasujen takaisinkierrätysventtiiliä ja ahtopainetta.
\end{itemize}
}

\frame{
\frametitle{FS, IS ja ES}
\begin{itemize}
\item Polttonestejärjestelmä laskee suihkutushetken/hetket ja määrän vääntömomenttiohjaukselta tulevan tiedon perusteella.
\item Järjestelmä reagoi vikoihin ja kulumisiin. Esimerkiksi imusarjan vuoto otetaan huomioon. FS huolehtii myös polttonestehöyryjen
 talteenottojärjestelmän aktiivihiilisäiliön tuuletuksesta.
\item Sytytysjärjestelmä laskee sytytyshetken toimintatilan perusteella, ottaen huomioon mahdollisen muuttuvan venttiilienajoituksen sekä 
nakutustunnistimilta saatavan tiedon.
\item Pakokaasujärjestelmä huolehtii lambda-säädöstä ja varaajakatalysaattorin ohjauksesta. 
\end{itemize}
}

\frame{
\frametitle{Toimintotiedot (OD)}
\begin{itemize}
\item OD-järjestelmä valvoo kaikkien antureilta saatavien parametrien loogisuutta, ja antaa tarvittaessa korvaavan arvon väärän arvon tilalle.
\end{itemize}
}

\frame{
\frametitle{Tiedonsiirto (CO)}
\begin{itemize}
\item Tiedonsiirtoyksikkö huolehtii kommunikaatiosta muiden järjestelmien kanssa.
\item Esimerkiksi ajonestolaitteen, vaihteiston ohjausyksikön ja testerin kanssa kommunikoidaan keskitetysti CO-yksikön kautta.
\end{itemize}
}

\frame{
\frametitle{Oheislaiteohjaus (AC)}
\begin{itemize}
\item Oheislaiteohjaus ohjaa nimensä mukaisesti oheislaitteita: ilmastointi, puhallin, vesipumppu, laturi\ldots
\end{itemize}
}

\frame{
\frametitle{Valvonta (MO) ja Diagnostiikka (DS)}
\begin{itemize}
\item Valvontapiiri valvoo moottorinohjausyksikön sisäisten järjestelmien toimintaa. Esimerkiksi jumiin
mennyt mikrokontrolleri käynnistetään välittömästi uudelleen.
\item Diagnostiikkajärjestelmä tallentaa viat ja esiintymisolosuhteet ja ohjaa vikamerkkivaloa.
\end{itemize}
}



\frame {
\frametitle{Dieselmoottorin ohjauksen menetelmät}
Dieselmoottorin tuottamaa vääntömomenttia säädetään vaikuttamalla
\begin{itemize}
\item polttoaineen ruiskutukseen (millä hetkellä ja kuinka paljon)
\item dieselmoottorin imuilmaa ei kuristeta
\end{itemize}
}

\frame{
\frametitle{Dieselmoottorin historia}
\begin{itemize}
\item Rudolf Diesel $\to$ patentti vuonna 1892
\item Dieselin tavoitteena oli kehittää kilpailija höyrykoneelle.
\item Alkuperäinen tarkoitus oli käyttää polttoaineena hiilipölyä.
\item Toimiva prototyyppi vuonna 1897
\begin{itemize}
\item Paino 4500 kg
\item Korkeus 3 m
\item Käytti raskasta polttoöljyä
\item Hyötysuhde 26 \%.
\item Polttoaine ruiskutettiin paineilman avulla.
\item Samaa ruiskutustekniikkaa käytettiin myöhemmin Daimlerin kuorma-autoissa.
\end{itemize}


\end{itemize}
}

\frame{
\frametitle{Historia: esikammio}
\begin{itemize}
\item Seuraava edistysaskel oli esikammio, johon polttoaine ruiskutettiin suurella paineella.
\item Osa polttoaineesta palaa esikammiossa. Lämpötila ja paine nousevat ja seos syöksyy palotilaan.
\item Esikammion tehtävä on seoksenmuodostus. Ruiskutuspaineet aluksi 100-300 baaria.
\item Vasta moderni korkeapaineruiskutustekniikka teki esikammion tarpeettomaksi.
\item Suoraruiskutustekniikka rupesi vähitellen yleistymään 1960-luvulta alkaen.
\item Henkilöautoissa esikammiotekniikka oli käytössä vielä 1990-luvun alkuun asti. 
\end{itemize}
}

\frame{
\frametitle{Historia: yleistyminen}
\begin{itemize}
\item Dieselmoottorit kuorma-autoihin 1920-luvulla.
\item Aluksi polttoaineena ruskohiilestä tislattu öljy (oli halvempaa kuin kovasti verotettu bensiini :).
\item Daimler käytti paineilmaruiskutusta, Benz esikammiota ja MAN suoraruiskutusta.
\item Henkilöautoon 1936 (Mercedes 260D, 45 hv).
\item Suosio henkilöautoissa kasvoi vasta turboahtamisen ja korkeapaineruiskutuksen myötä 1990-luvulla.
\item Nykyään Euroopassa (noin) joka toinen rekisteröity henkilöauto on diesel.
\end{itemize}
}

\frame{
\frametitle{Käyttö muualla}
\begin{itemize}
\item Dieselmoottorilaiva (1903).
\item Dieselveturi (1913).
\item Käytettiin ilmalaivoissa, kokeiltiin lentokoneissa.
\item \ldots
\end{itemize}
}



\frame {
\frametitle{Dieselmoottorin polttoaineenruiskutus}
Nykyisin käytössä olevat ruiskutusjärjestelmät ovat
\begin{description}
\item[rivipumppu] Jokaisella sylinterillä on oma nokka-akselin käyttämä pumppuelementti.
\item[jakajapumppu] Yksi pumppumäntä, joka sekä kehittää paineen että jakaa sen sylintereille.
\item[pumppusuutin] Pumppu ja suutin on integroitu
\item[yhteispaineruiskutus] Erillinen korkeapainepumppu, jonka toiminta ei ole sidottu ruiskutushetkiin. Ruiskutusventtiilejä
ohjataan sähköisesti.
\end{description}
}

\frame{
\frametitle{Eroja bensiinimoottoriin verrattuna}
\begin{itemize}
\item Ei kuristusläppää.
\item Dieselmoottoria käytetään ilmaylimäärällä ($\lambda>1$) nokeamisen estämiseksi.
\item Syttyminen tapahtuu itsestään (ei sytytystulppia).
\item Ahtaminen käytännössä "pakollista"\ (parantaa hyötysuhdetta, tekee ylipäätään mielekkääksi).
\item Vedentunnistin.
\item Usein monimutkaisempi pakokaasujen puhdistus.
\item Käynnistysapulaitteet
\end{itemize}
}

\frame{
\frametitle{Samankaltaisuuksia}
\begin{itemize}
\item Perustoimintaidea.
\item Pakokaasujen takaisinkierrätys (alempi palamislämpötila, vähemmän NO$_{\rm X}$).
\end{itemize}
}


\frame{
\frametitle{Nykyaikainen henkilöauton dieselmoottori}
\begin{itemize}
\item Yhteispaineruiskutus (common rail) mahdollistaa tarkan säädön. Suurempi teho, pienemmät päästöt ja polttoaineenkulutus.
\item Pietsosuuttimet.
\item Ruiskutuspaineet melkein tai yli 2000 baaria.
\item Suoraruiskutus, ei käytetän esi- tai swirl-kammiota.
\end{itemize}
}

\frame{
\frametitle{Common rail: etuja}
\begin{itemize}
\item Ruiskutuspaineen säätö pyörimisnopeudesta riippumatta. Mahdollistaa tarkan säädön ja riittävän ruiskutuspaineen
kaikille kierrosluvuilla.
\item Vapaa ruiskutushetken valinta. Useat ruiskutukset työtahtia kohden helppo toteuttaa.
\end{itemize}
}


\frame{
\frametitle{Pakokaasujen puhdistus}
\begin{itemize}
\item Tavallinen hapetuskatalysaattori + DPF
\item Denoxtronic (urearuiskutus hapetuskatalysaattorin jälkeen + SCR-katalysaattori)
\item Varaava katalysaattori.
\end{itemize}
}


\frame{
\frametitle{DPF (Diesel Particulate Filter)}
\begin{itemize}
\item Kerää nokihiukkasia pakokaasuista.
\item Suodatin tulee regeneroida (puhdistaa) noin 500 kilometrin välein. Puhdistus kestää noin 10 minuuttia.
\item Noen pois polttaminen vaatisi 600 asteen lämpötilan. 
\item Käyttämällä hapetuskatalysaattoria ennen suodatinta, voidaan typpimonoksidi hapettaa typpidioksidiksi. 
Typpidioksidi puolestaan reagoi noen kanssa (muodostaen hiilidioksidia ja typpimonoksidia) jo noin 300 asteen lämpötilassa.
\item Regenerointi voidaan myös toteuttaa ruiskuttamalla polttoaineeseen lisäainetta, joka nostaa pakokaasujen
lämpötilaa.
\end{itemize}
}

\frame{
\frametitle{SCR-katalysaattori}
\begin{itemize}
\item Ruiskuttamalla ammoniakkia tai ureaa pakokaasujen joukkoon, voidaan typenoksidit pelkistää
typeksi. 
\end{itemize}
}


\frame{
\frametitle{Dieselmoottorin polttoaineensyöttö}
\begin{itemize}
\item Perusperiaate sama kuin bensiinimoottorissa.
\item Syöttöjärjestelmän rakenne riippuu erittäin paljon käytettävästä ruiskutusjärjestelmästä.
\end{itemize}
}

\frame{
\frametitle{Dieselmoottorin polttoaineensuodatin}
\begin{itemize}
\item Polttoaineensuodattimen tehtävänä on erottaa hiukkasepäpuhtaudet ja vesi pois polttoaineesta.
\item Dieselpolttoaineessa on tyypillisesti enemmän epäpuhtauksia kuin bensiinissä.
\item Korkeat ruiskutuspaineet (2000 baaria) vaativat erittäin tarkan polttoaineen puhdistuksen.
\item Usein käytetään esisuodatinta polttoainesäiliössä; etenkin maissa, joissa polttoaineen laatu on huono.
\item Joissain suodattimissa on erillinen vedenpoistoventtiili.
\item On mahdollista myös käyttää polttoaineen esilämmitystä; tämä estää muodostuvia parafiinikiteitä
tukkimasta suodatinta talvella.
\end{itemize}
}

\frame{
\frametitle{Siirtopumppu}
\begin{itemize}
\item Kuten (suora)suihkutusbensiinimoottoreissa, nykyaikaisessa dieselmoottorissa on kaksi pumppua:
siirtopumppu (matalapainepumppu) ja ruiskutuspumppu (korkeapainepumppu).
\item Useissa jakajapumppujärjestelmissä siirtopumppu on integroitu korkeapainepumppuun.
\item Dieselmoottorin siirtopumppu on perinteisesti ollut keskipakorullapumppu.
\item Polttoaine virtaa pumpun sähkömoottorin läpi jäähdyttäen ja voidellen sitä.
\end{itemize}
}

\frame{
\frametitle{Hammaspyöräpumppu ja siipilevypumppu}
\begin{itemize}
\item Uusissa yhteispaineruiskutusjärjestelmissä käytetään usein hammaspyöräpumppua.
\item Pumppu voi olla erillinen imupumppu moottoritilassa, tai se on integroitu korkeapainepumppuun.
\item Pumppusuuttimien kanssa käytetään myös siipilevypumppua.
\end{itemize}
}

\frame{
\frametitle{Tandempumppu}
\begin{itemize}
\item Tandempumppuun on yhdistetty samaan yksikköön pumppusuuttimien matalapainepumppu sekä
jarrutehostimen alipainepumppu.
\item Itse polttoainepumppuosana on joko hammaspyöräpumppu tai siipilevypumppu.
\item Pumpun polttonestekanavat on suunniteltu niin, että jos tankin ajaa tyhjäksi, pumppu ei
tyhjene.
\end{itemize}
}

\frame{
\frametitle{Muita osia}
\begin{itemize}
\item Paineensäätöventtiili. Käytetään pumppu(putki)suuttimien paluupiirissä vakioimaan syöttöpainetta.
\item Moottorinohjausyksikön jäähdytin. Ohjausyksikön jäähdytys tapahtuu tavallisesti polttoaineen avulla.
\item Palaavan polttoaineen jäähdytin. Palaava polttoaine on usein niin kuumaa, että sitä on pakko jäähdyttää
ennen palautusta polttoainetankkiin. Tämä tehdään erillisen jäähdytinpiirin avulla.
\end{itemize}
}


\frame{
\frametitle{Rivipumppu}
\begin{itemize}
\item "Perinteinen"\ dieselmoottorin polttoaineenruiskutusjärjestelmä.
\item Yksinkertainen rakenne $\to$ kestävä $\to$ tarvitsee vähän huoltoa.
\item Käytössä nykyäänkin suurissa dieselmoottoreissa. Ei asenneta enää uusiin henkilöautoihin.
\item Pumppuyksikön voitelu tapahtuu moottoriöljyllä, ei dieselöljyllä.
\item Ruiskutuspaine noin 500-1300 baaria.
\end{itemize}
}

\frame{
\frametitle{Rivipumpun säätäminen}
\begin{itemize}
\item Rivipumpussa pyörintänopeutta säädetään mekaanisella tai elektronisella säätimellä.
\item Usein käytetään myös erillistä ennankonsäädintä (engl. timing device).
\item Luistiohjatussa rivipumpussa voidaan määrän lisäksi säätää myös syötön alkuhetkeä, jolloin
erillistä ennakonsäädintä ei tarvita.
\item Vääntömomentti on suurin piirtein suoraan verrannollinen ruiskutusmäärään.
\end{itemize}
}

\frame{
\frametitle{Mekaaninen pyörintänopeussäädin}
\begin{itemize}
\item Mekaaninen säädin on toiminnaltaan keskipakosäädin.
\item Idea on vanha: käytetty Wattin höyrykoneessa ja sitä ennen tuulimyllyissä.
\item Kaksi painoa siirtyvät sitä kauemmas toisistaan, mitä suurempi on pyörimisnopeus.
\end{itemize}
}

\frame{
\frametitle{Mekaaninen säädin}
\begin{itemize}
\item Yksinkertaisimmillaan mekaaninen säädin rajoittaa vain maksimipyörintänopeuden
kaasuvivun asennon mukaan. (RQ)
\item RQU-säädin säätää vakioi myös tyhjäkäyntinopeuden.
\item Nykyaikaiset säätimet säätävät ruiskutusmäärää koko pyörimisnopeusalueella.
\end{itemize}
}

\frame{
\frametitle{Ennakonsäädin}
\begin{itemize}
\item Ennakonsäädin aikaistaa ruiskutushetkeä pyörimisnopeuden kasvaessa.
\end{itemize}
}

\frame{
\frametitle{Elektroninen säätöjärjestelmä}
\begin{itemize}
\item Keskipakosäädin ja ennakonsäädin voidaan helposti korvata elektronisella säädöllä.
\end{itemize}
}

\frame{
\frametitle{Rivipumpun polttoaineensyöttö}
\begin{itemize}
\item Rivipumpun siirtopumppu on yleensä moottoritilaan asennettu, nokan käyttämä mäntäimupumppu.
\item Joissain tapauksissa siirtopumppua ei tarvita (esimerkiksi traktoreissa, joissa polttoainesäiliö
on moottorin yläpuolella).
\item Siirtopumpun yhteydessä on käsipumppu, jolla polttoainelinja ja siirtopumppu saadaan täytettyä
huollon tai säiliön kuivaksiajamisen jälkeen. 
\end{itemize}
}

\frame{
\frametitle{Bosch PE-rivipumpun toiminta}
\begin{itemize}
\item PE-rivipumpun ruiskutusmäärää säädetään muuttamalla männän tehollisen iskun pituutta.
\item Männässä on viisto ohjausreuna, joka mahdollistaa tehollisen iskun pituuden säätämisen
mäntää kiertämällä.
\item Maksimiruiskutuspaine tällä pumpputyypillä on tavallisesti yli 400-1350 baaria.
\item Sylinterin yläpäässä on paineventtiili, joka aukeaa syötön alkaessa.
\end{itemize}
}

\frame{
\frametitle{Paineventtiilityypit}
Pumpun paineventtiili sulkeutuu nopeasti, kun syöttö loppuu. Tämä nopea paineen aleneminen aikaansaa
ruiskusuuttimen nopean sulkeutumisen, jolloin palotilaan ei tihku ylimääräistä polttoainetta.
\begin{itemize}
\item Vakiotilavuusventtiili sallii paineenvaihtelut korkeapainelinjassa, ja päästää läpi vakiotilavuuden polttoainetta.
\item Vakiopaineventtiili pitää linjan paineen vakiona. Tätä venttiilityyppiä käytetään nopeissa dieselmoottoreissa,
joissa on korkea ruiskutuspaine.
\end{itemize}
}

\frame{
\frametitle{Ennakonsäädin}
\begin{itemize}
\item Mekaaninen ennakonsäädin perustuu keskipakosäätöön (kuten pyörimisnopeudensäädin).
\item Kierrosluvun kasvaessa säädin kääntää rivipumpun nokka-akselia niin, että ruiskutus alkaa aikaisemmin.
\end{itemize}
}

\frame{
\frametitle{Luistiohjattu rivipumppu}
\begin{itemize}
\item Luistiohjatussa rivipumpussa ennakonsäätö tapahtuu pumpun sisäisellä ohjausluistilla, ei nokka-akselia kääntämällä
\item Luistiohjauttua rivipumppua säädetään elektronisesti.
\end{itemize}
}


\frame{
\frametitle{Jakajapumppu}
\begin{itemize}
\item Pienikokoinen, soveltuu pieniin 3-6-sylinterisiin nopeakäyntisiin dieselmoottoreihin.
\item Käyttöön vuonna 1962.
\item Voitelu tapahtuu polttoaineella, ei moottoriöljyllä.
\item Käytössä vielä 2000-luvun alussa; 2000-luvulla pumppusuuttimet ja yhteispaineruiskutus korvasivat nopeasti.
\end{itemize}
}

\frame{
\frametitle{Pumpputyypit}
Männän toteutuksen perusteella:
\begin{itemize}
\item Aksiaalimäntäjakajapumppu
\item Säteismäntäjakajapumppu
\end{itemize}
Ruiskutusmäärän annostelun perusteella:
\begin{itemize}
\item Porttiohjaus
\item Solenoidiventtiiliohjaus
\end{itemize}
Säädön perusteella
\begin{itemize}
\item Mekaaninen säätö
\item Elektroninen säätö
\end{itemize}

}

\frame{
\frametitle{Siirtopumppu}
\begin{itemize}
\item Jakajapumppujärjestelmä voi käyttää joko pumppuun integroitua siirtopumppua tai polttoainesäiliössä olevaa sähköistä pumppua.
\end{itemize}
}
\frame{
\frametitle{Jakajapumpun toiminta}
\begin{itemize}
\item Jakajapumpussa yksi mäntä kehittää ruiskutuspaineen ja jakaa sen sylintereille.
\item Tavallaan sama toiminta-ajatus kuin bensiinimoottorin virranjakajassa.
\item Mäntä kehittää paineen edestakaisella liikkeellä ja kohdesylinterin valinta tapahtuu männän
pyörivällä liikkeellä.
\end{itemize}
}


\frame{
\frametitle{Magneettiventtiiliohjattu jakajapumppu}
\begin{itemize}
\item Magneettiventtiiliohjatussa jakajapumpussa magneettiventtiilillä valitaan ruiskutuksen aloitushetki.
\item Mahdollistaa tarkan säädön
\end{itemize}
}

\frame{
\frametitle{Säteismäntäjakajapumppu}
\begin{itemize}
\item Säteismäntäjakajapumpulla on mahdollista saavuttaa korkeampi (1950 baaria) ruiskutuspaine kuin aksiaalimäntäjakajapumpulla (n. 1500 baaria).
\item Korkea paine saavutetaan käyttämällä useampaa mäntää.
\item Jakaminen tapahtuu pyörivällä männällä kuten aksiaalipumpussakin.
\item Säteismäntäpumput ovat aina elektronisesti ohjattuja.
\end{itemize}
}

\frame{
\frametitle{Jakajapumpun ohjaus}
\begin{itemize}
\item Kuten rivipumpussa (tai missä tahansa laitteessa), ohjaus voidaan toteuttaa joko mekaanisesti tai elektronisesti.
\end{itemize}
}

\frame{
\frametitle{Jakajapumpun mekaaninen säätö}
\begin{itemize}
\item Ruiskutusennakkoa säädetään kääntämällä pumpun rullarengasta.
\item Ruiskutusmäärää säädetään säätöluistin avulla.
\end{itemize}
}

\frame{
\frametitle{Jakajapumpun elektroninen säätö}
\begin{itemize}
\item Jakajapumppua voidaan säätää elektronisesti yksinkertaisesti ohjaamalla solenoidin
avulla ennakonsäätölaitetta ja säätöluistia.
\item Magneettiventtiiliohjatussa jakajapumpussa ruiskutuksen alkuhetki ja kesto säädetään
magneettiventtiilin avulla.
\item Erillistä säätöluistia ei tarvita, ei myöskään rullarenkaan kääntämiseen perustuvaa ennakon säätöä.
\end{itemize}
}

\frame{
\frametitle{Elektronisen säädön edut}
Pääetu: tarkempi säätö.
\begin{itemize}
\item matalampi polttoaineen kulutus
\item pienemmät päästöt
\item suurempi teho
\item tasaisempi tyhjäkäynti
\item mahdollisuus sovittaa tyhjäkäynti apulaitteisiin (esim. ilmastointi)
\item diagnostiikkatoiminnot
\item tietojenvaihto muiden ajoneuvon järjestelmien (esimerkiksi automaattivaihteisto) kanssa
\end{itemize}
}





\frame{
\frametitle{Ruiskusuuttimet}
Suuttimen perustehtävinä on taata 
\begin{itemize}
\item optimaalinen polttoaineen purkautuminen sylinteriin (tasainen paineen nousu)
\item optimaalinen polttoaineen sumuuntuminen
\item sekä eristää palotila polttoainejärjestelmästä.
\end{itemize}
}

\frame{
\frametitle{Vaatimukset}
\begin{itemize}
\item Ruiskusuutin altistuu tärinälle ja nopeille paineiskuille, sekä korkealle lämpötilalle. 
\end{itemize}
}

\frame{
\frametitle{Suutintyypit}
\begin{itemize}
\item Suutin voi olla joko itsenäinen yksikkö, tai kiinni suutinpitimessä. Yhteispaineruiskutusjärjestelmissä ja pumppusuuttimissa
ei erillistä pidintä käytetä.
\item Jaetulla palotilalla (= esi- tai pyörrekammio) varustetussa moottorissa käytetään (kuristin)tappisuuttimia,
suoraruiskutteisissa moottoreissa taas reikäsuuttimia.
\end{itemize}
}

\frame{
\frametitle{Kuristintappisuutin}
\begin{itemize}
\item Kuristintappisuutin muodostaa kapean, suuttimen akselin suuntaisen suihkun.
\item Suutinneulan nousun määrä vaikuttaa ruiskutusmäärään.
\item Suuttimen painekammion alaosa on muotoiltu siten, että kun neula nousee, polttonesteen määrä kasvaa jyrkästi.
\item Alussa virtaa polttonestettä vähän, lopussa paljon $\to$ positiivinen vaikutus käyntiääneen. 
\end{itemize}
}

\frame{
\frametitle{Reikäsuutin}
\begin{itemize}
\item Reikäsuuttimessa on suutinneula, joka toimii venttiilinä polttoaineelle. Polttoaine suihkuaa useista rei'istä.
\end{itemize}
}

\frame{
\frametitle{Suutinpidin}
\begin{itemize}
\item Suutinpitimen tarkoituksena on tarjota jousivoima suuttimen neulalle, joka avautuu paineen vaikutuksesta.
\item Esiruiskutuksella voidaan vaikuttaa suotuisasti moottorin käyntimeluun. Esiruiskutus voidaan aikaansaada
käyttämällä kaksijousisuutinpidintä: paineen noustessa ensin toinen jousi päästää suuttimen avautumaan
vähän, ja hetkeä myöhemmin toinen jousi avaa suutinta lisää, jolloin tapahtuu pääruiskutus.
\end{itemize}
}

\frame{
\frametitle{Pumppusuutin ja yhteispaineruiskutus}
\begin{itemize}
\item Edellä kuvatut suuttimet avautuvat paineen vaikutuksesta.
\item Pumppusuutin- ja yhteispaineruiskutuksessa suuttimen avautumista ohjataan elektronisesti (sähkömagneetilla tai pietsosähköisesti).
\end{itemize}
}



\frame{
\frametitle{Yksikköpumput}
\begin{itemize}
\item pumppusuutin
\item pumppuputkisuutin.
\end{itemize}
Yksikköpumppujen ruiskutuspaineiden suuruusluokka on noin 2000 baaria.
}

\frame{
\frametitle{Pumppusuutin}
\begin{itemize}
\item Pumppusuuttimessa on nokka-akselin keinuvivun välityksellä käyttämä pumppu ja elektronisesti
ohjattu suutin integroitu samaan pakettiin.
\item Mahdollistaa erittäin vapaan säädön. 
\end{itemize}
}

\frame{
\frametitle{Pumppuputkisuutin}
\begin{itemize}
\item Kuten pumppusuutin, mutta suuttimen ja pumpun välillä on lyhyt korkeapaineputki.
\item Pumppu voidaan sijoittaa suoraan nokka-akselille, joten keinuvipuja ei tarvita. 
\item Pumppuputkisuuttimia käytetään myös raskaalla polttoöljyllä käyvissä dieselmoottoreissa.
\end{itemize}
}

\frame{
\frametitle{Käyttökohteet}
\begin{itemize}
\item Pumppusuuttimia käytetään henkilöautoissa ja hyötyajoneuvoissa.
\item Pumppuputkisuuttimia käytetään raskaissa kuorma-autoissa, työkoneissa,
vetureissa ja laivoissa.
\end{itemize}
}

\frame{
\frametitle{Magneettiventtiilin toiminta}
\begin{itemize}
\item Kun magneettiventtiili suljetaan, ruiskutus alkaa.
\end{itemize}
}

\frame{
\frametitle{Yksikköpumpun historia}
\begin{itemize}
\item Ajatus yksikköpumpusta jo Rudolf Dieselin patenttidokumentissa vuonna 1905.
\item 1994 pumppusuutin hyötyajoneuvoihin.
\item 1995 pumppuputkisuutin kuorma-autoihin.
\item 1998 pumppusuutin henkilöautoihin.
\end{itemize}
}

\frame{
\frametitle{Yhteispaineruiskutus (common rail)}
\begin{itemize}
\item Paineentuotto on riippumaton kierrosnopeudesta.
\item Kaikki suuttimet on kytketty samaan korkeapainelinjaan.
\item Ruiskutushetki ja määrä ovat täysin vapaasti valittavissa.
\item Ohjausjärjestelmä samantyyppinen kuin yksikköpumpuissa.
\item Ruiskutuspaineet 2000 baaria tai enemmän.
\end{itemize}
}

\frame{
\frametitle{Korkeapainepumppu}
\begin{itemize}
\item Henkilöautoissa käytetään säteismäntäpumppua, ja hyötyajoneuvoissa rivityyppistä pumppua.
\end{itemize}
}

\frame{
\frametitle{Esi- ja jälkiruiskutus}
\begin{itemize}
\item Esi- ja jälkiruiskutuksen avulla voidaan tasata moottorin käyntiääntä (vähentää terävää "nakutusta").
\item Common rail -järjestelmä mahdollistaa myös myöhäisen jälkiruiskutuksen. Esiruiskutus on mahdollista
myös muissa elektronisesti ohjatuissa järjestelmissä.
\end{itemize}
}



\frame{
\frametitle{Korkeapainejärjestelmä}
\begin{itemize}
\item Paineensäätö tapahtuu ohjaamalla korkeapainepumppua ja paluulinjan venttiiliä.
\end{itemize}
}

\frame{
\frametitle{Solenoidisuutin}
\begin{itemize}
\item Yhteispaineruiskutuksessa käytetään nopeita reikäsuuttimia.
\item Nopea avautuminen tarvitsee erittäin suuren virtapiikin.
\item Virtapiikki kehitetään hakkuriteholähteellä.
\end{itemize}
}

\frame{
\frametitle{Pietsosuuttimet}
\begin{itemize}
\item Perustuu pietsosähköiseen ilmiöön.
\item Kun pietsokidettä puristetaan, siihen muodostuu jännite (hyödynnetään esimerkiksi tupakansytyttimissä).
\item Toimii myös kääntäen: kun pietsokiteeseen kytketään jännite, se vetäytyy kasaan.
\end{itemize}
}

\frame{
\frametitle{Korkeapainepumppu}
\begin{itemize}
\item Dieselmoottorin korkeapainepumpun perusperiaate on sama kuin bensiinimoottoreissakin.
\item Kehitettävä paine on luonnollisesti korkeampi.
\item Henkilöautoissa käytetään tyypillisesti kolmesylinteristä säteismäntäpumppua.
\item Hyötyajoneuvoissa käytetään myös rivityyppisiä mäntäpumppuja.
\end{itemize}
}

\frame{
\frametitle{Jakoputki}
\begin{itemize}
\item Jakoputken suunnittelussa on olennaista suunnitella putken rakenne sellaiseksi, että 
yksittäiset ruiskutukset eivät aiheuta merkittäviä paineaaltoja putkeen.
\end{itemize}
}

\frame{
\frametitle{Käynnistysapulaitteet}
\begin{itemize}
\item Dieselpolttoaine vaatii (itse)syttyäkseen noin 250 asteen lämpötilan.
\item Tämä lämpötila saavutetaan suoraruiskutusmoottoreissa vielä noin
nollan asteen lämpötiloissa.
\item Jaetulla palotilalla varustettu moottori vaatii apulaitteiden käyttöä
lämpimämmässäkin.
\item Esikammiomoottori vaatii apulaitteiden käyttöä jo alle +40 asteen lämpötiloissa ja
pyörrekammiolla varustettu jo alle +20 asteen lämpötiloissa.
\end{itemize}
}

\frame{
\frametitle{Hehkutus}
\begin{itemize}
\item Sylinterissä oleva hehkutulppa lämmittää sylinterissä olevaa imuilmaa niin, että polttoaineen ja ilman seos syttyy.
\item Hehkutulppa on käytännössä metallitupen sisällä oleva lämmitysvastus.
\item Nykyaikaisissa moottoreissa käynnistyminen matalilla akkujännitteillä on varmistettu siten, että hehkutulpat on mitoitettu
akkujännitettä matalammalle jännitteelle, ja niitä ohjataan elektronisesti.
\end{itemize}
}

\frame{
\frametitle{Liekkihehkutus}
\begin{itemize}
\item Suurissa moottoreissa imuilma lämmitetään liekkihehkutulpan avulla.
\item Imusarjassa olevaan hehkutulppaan suihkutetaan polttoainetta niin, että imusarjassa palaa liekki,
joka lämmittää imusarjassa olevan ilman.
\end{itemize}
}

\frame{
\frametitle{Elektroninen dieselsäätö (EDC)}
\begin{itemize}
\item Käytännössä kaikki nykyaikaiset dieselmoottorit ovat elektronisesti ohjattuja.
\item Säätösuureita ovat ainakin
\begin{itemize}
\item Kampiakselin kulma ja kierrosluku.
\item Jakoputken paine.
\item Imusarjan paine.
\item Polttonesteen, jäähdytysnesteen ja imuilman lämpötila.
\item Mitattu ilmamassavirtaus.
\item Auton nopeus.
\end{itemize}
\end{itemize}

}

\frame{
\frametitle{Elektronisen dieselsäädön edut}
\begin{itemize}
\item Tarkempi säätö $\to$ pienemmät päästöt, suurempi teho\ldots
\item Samat hyödyt siis kuin bensiinimoottorilla.
\item Elektronisella dieselsäädöllä voidaan toteuttaa lisäsäätöjä, jotka olisivat vaikeita tai mahdottomia toteuttaa mekaanisilla säätimillä.
\end{itemize}
}

\frame{
\frametitle{Tärinänvaimennus ja kulumisen kompensointi}
\begin{itemize}
\item Nopeat vääntömomentin vaihtelut rasittavat voimansiirtolinjaa ja aiheuttavat epämiellyttäviä
värähtelyjä.
\item Esimerkiksi jos nostaa kuorma-autossa kytkimen liian nopeasti, tai henkilöautossa painaa matalilla
kierroksilla kaasun pohjaan.
\item Elektronisella ohjauksella voidaan lieventää tällaisia värähtelyjä aktiivisesti.
\item Dieselmoottori kuluu epätasaisesti. Tällöin joku sylintereistä saattaa tarvita hieman enemmän tai
vähemmän ruiskutusta kuin muut.
\item Säätämällä ruiskutusmäärät kulumisen mukaan, saadaan moottori käymään tasaisemmin, etenkin tyhjäkäynnillä.
\end{itemize}
}

\frame{
\frametitle{Sylinterien poiskytkentä}
\begin{itemize}
\item Hyötyajoneuvoissa (ja joissain henkilöautoissa) kytketään usein puolet sylintereistä pois
päältä, jos kuormitus on erittäin pieni.
\end{itemize}
}

\frame{
\frametitle{Moottorinohjausyksikkö ja anturit}
\begin{itemize}
\item Käytännössä sama mikä pätee bensiinimoottorin ohjausyksikköön ja antureihin, pätee myös dieselmoottoreihin.
\end{itemize}
}



