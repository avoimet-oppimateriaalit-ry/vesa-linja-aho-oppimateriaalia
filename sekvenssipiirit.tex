 \frame{
 \frametitle{Sekvenssipiirit}
 \begin{itemize}
 \item Logiikkaporttipiirit eli kombinaatiopiirit eivät ota huomioon aikaulottuvuutta: lähtöön vaikuttavat senhetkiset tulojen tilat.
\item Kombinaatiopiirien lisäksi tärkeitä digitaalipiirejä ovat {\bf sekvenssipiirit}. Sekvenssipiireissä lähdön tilaan vaikuttaa tulojen muuttumisen ajallinen järjestys sekä piirin senhetkinen tila.
 \end{itemize}
 }

 \frame{
 \frametitle{Dynaaminen tulo}
 \begin{itemize}
 \item Dynaaminen tulo vaikuttaa piirin toimintaan vain muutoshetkellä.
\item Eli "jotain tapahtuu"\ vain silloin, kun tulo muuttuu nollasta ykköseksi tai ykkösestä nollaksi.
\item Nollasta ykköseen -muutoksessa toimiva tulo on {\bf nousevalla reunalla toimiva} tulo. Sitä merkitään pienellä kolmiolla (alla vasemmalla).
\item Ykkösestä nollaan -muutoksessa toimiva tulo on {\bf laskevalla reunalla toimiva} tulo. Sitä merkitään pienellä kolmiolla, jonka edessä on pallo (alla oikealla).
 \end{itemize}
\begin{center}
\begin{picture}(50,50)(0,0)
\dff{0,0}
\jk{100,0}
\end{picture}
\end{center}

 }

 \frame{
 \frametitle{Kiikku vai veräjä?}
 \begin{itemize}
 \item Kiikku on sekvenssipiiri, jossa on kellotulo.
\item Veräjässä ei ole kellotuloa, vaan ohjaustuloissa tapahtuva muutos vaikuttaa piirin toimintaan välittömästi.
 \end{itemize}
 }

 \frame{
 \frametitle{SR-veräjä ja SR-kiikku}
 \begin{itemize}
\item Joskus kirjaimet ovat toisin päin (RS-kiikku/veräjä). Kyseessä on sama komponentti.
 \item Kun kellotuloon tuodaan positiivinen pulssi, lähdön $Q$ tila muuttuu ykköseksi jos $S$-tulon arvo on 1, ja nollaksi, jos $R$-tulon arvo on 1.
\item $S$=set, $R$=reset.
\item Jos molemmat tulot ovat nollia, tila ei muutu.
\item Tila, jossa $S$- että $R$-tuloihin syötetään ykkönen, on kielletty tila. Silloin piirin toimintaa ei ole määritelty (piiri voi toimia jotenkin arvaamattomasti).
\item SR-veräjä on kuten SR-kiikku, mutta erillinen kellotulo puuttuu, ja muutos tapahtuu välittömästi, kun tulojen tila muuttuu.
 \end{itemize}

\begin{center}
\begin{picture}(50,50)(0,0)
\sr{0,0}
\end{picture}
\end{center}
 }

 \frame{
 \frametitle{JK-kiikku}
 \begin{itemize}
 \item Melkein kuin SR-kiikku: J-tulo vastaa set-tuloa ja K-tulo reset-tuloa.
\item Parannus verrattuna SR-kiikkuun: JK-kiikussa ei ole "kiellettyä tilaa". Jos sekä $J$- että $K$-tulossa on arvo 1, niin lähtö vaihtaa tilaansa, kun kellopulssi tulee.
\item Monessa JK-kiikussa on lisäksi pakko-ohjaustulot $S$ ja $R$, joilla lähdön tilaa voidaan muuttaa kellosta riippumatta.
\item Kuvassa negatiivisella kellopulssin reunalla toimiva JK-kiikku.
 \end{itemize}
\begin{center}
\begin{picture}(50,50)(0,0)
\jk{0,0}
\txt{15,30}{J}
\txt{15,0}{K}

\txt{35,30}{Q}
\txt{35,0}{\bar{Q}}

\txt{25,52}{\bar{S}}
\txt{25,-25}{\bar{R}}

\end{picture}
\end{center}
 }

 \frame{
 \frametitle{D-kiikku}
 \begin{itemize}
 \item Toiminta hyvin yksinkertainen: lähtö $Q$ saa arvon $D$, kun kellopulssi tulee.
\item Kiikussa voi olla myös erilliset pakko-ohjaustulot ($S$ ja $R$).
 \end{itemize}
\begin{center}
\begin{picture}(50,50)(0,0)
\dff{0,0}
\txt{-10,30}{D}
\txt{-10,0}{clk}

\txt{35,30}{Q}
\txt{35,0}{\bar{Q}}

%\txt{25,52}{\bar{S}}
%\txt{25,-25}{\bar{R}}

\end{picture}
\end{center}
 }
