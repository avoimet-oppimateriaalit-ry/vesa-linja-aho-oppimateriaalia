\frame {
  \frametitle{Kerrostamismenetelmä}
\begin{itemize}
\item Vastuksista ja vakioarvoisista virta- ja jännitelähteistä koostuva piiri on lineaarinen.
\item Jos piiri on lineaarinen, voidaan vastusten jännitteet ja virrat selvittää laskemalla
kunkin lähteen vaikutus erikseen.
\item Tätä ratkaisumenetelmää kutsutaan {\bf kerrostamismenetelmäksi}.
\end{itemize}
}

\frame {
  \frametitle{Kerrostamismenetelmä}
Kerrostamismenetelmää sovelletaan seuraavasti
\begin{itemize}
\item Lasketaan kunkin lähteen aiheuttama(t) virta/virrat ja/tai jännite/jännitteet erikseen siten,
että muut lähteet ovat sammutettuina.
\item Sammutettu jännitelähde = oikosulku (suora johdin), sammutettu virtalähde = avoin piiri (katkaistu johdin).
\item Lopuksi lasketaan osatulokset yhteen.
\end{itemize}
}


\frame {
\frametitle{Esimerkki kerrostamismenetelmän soveltamisesta}
Ratkaise virta $I_3$ kerrostamismenetelmällä.
\begin{center}
\begin{picture}(200,50)(0,0)
\vst{0,0}{E_1}
\vst{100,0}{E_2}
\vz{50,0}{R_3}
\di{50,1}{I_3}
\hz{0,50}{R_1}
\hz{50,50}{R_2}
\hln{0,0}{100}
\end{picture}
\end{center}
Sammutetaan oikeanpuoleinen jännitelähde:
\begin{center}
\begin{picture}(200,50)(0,0)
\vst{0,0}{E_1}
%\vst{100,0}{E_2}
\vln{100,0}{50}
\vz{50,0}{R_3}
\di{50,1}{I_{31}}
\hz{0,50}{R_1}
\hz{50,50}{R_2}
\hln{0,0}{100}
\txl{100,25}{$I_{31}=\frac{E_1}{R_1+\frac{1}{G_2+G_3}}\frac{1}{G_2+G_3}G_3$}
\end{picture}
\end{center}
Sammutetaan vasemmanpuoleinen jännitelähde:
\begin{center}
\begin{picture}(200,50)(-100,0)
%\vst{0,0}{E_1}
\vst{100,0}{E_2}
\vln{0,0}{50}
\vz{50,0}{R_3}
\di{50,1}{I_{32}}
\hz{0,50}{R_1}
\hz{50,50}{R_2}
\hln{0,0}{100}
\txl{-125,25}{$I_{32}=\frac{E_2}{R_2+\frac{1}{G_1+G_3}}\frac{1}{G_1+G_3}G_3$}
\end{picture}
\end{center}
}

\frame {
\frametitle{Esimerkki kerrostamismenetelmän soveltamisesta}
Virta $I_3$ saadaan laskemalla osavirrat $I_{31}$ ja $I_{32}$.
\[
I_3=I_{31}+I_{32}=\frac{E_1}{R_1+\frac{1}{G_2+G_3}}\frac{1}{G_2+G_3}G_3+\frac{E_2}{R_2+\frac{1}{G_1+G_3}}\frac{1}{G_1+G_3}G_3
\]
}

\frame {
\frametitle{Milloin kerrostamismenetelmä on kätevä?}
\begin{itemize}
\item Kun laskija pitää enemmän piirin sormeilemisesta kuin yhtälöryhmien pyörittelemisestä.
\item Jos piirissä on paljon lähteitä ja vähän vastuksia, kerrostamismenetelmä on usein nopea.
\item Jos piirissä on useita eritaajuisia lähteitä, piirin analysointi perustuu kerrostamismenetelmään.
\end{itemize}
}


\frame {
\frametitle{Lineaarisuus ja kerrostamismenetelmän teoriatausta}
\begin{itemize}
\item Kerrostamismenetelmä perustuu piirin lineaarisuuteen, eli siihen,
että jokainen lähde vaikuttaa jokaiseen jännitteeseen vakiokertoimella.
\item Sama kaavana: jos piirissä on lähteet $E_1$, $E_2$, $E_3$, $J_1$,
$J_2$, niin jokainen piirin jännite ja virta on muotoa
$k_1E_1+k_2E_2+k_3E_3+k_4J_1+k_5J_2$, missä vakiot $k_n$ ovat
reaalilukuja.
\item Jos kaikkien lähteiden arvo on nolla, ovat piirin vastusten virrat ja
jännitteet nolla; eli nollaamalla kaikki lähteet paitsi yksi, voidaan laskea
kyseisen lähteen vaikutuskerroin.
\end{itemize}
}


\begin{comment}
\frame{
\frametitle{Oppikirja}
Kimmo Silvonen: {\em Sähkötekniikka ja piiriteoria}:
\begin{description}
\item[1.9.5] Superpositioperiaate eli kerrostamismenetelmä
\end{description}

Kimmo Silvonen: {\em Sähkötekniikka ja elektroniikka}:
\begin{description}
\item[1.9.5] Superpositioperiaate eli kerrostamismenetelmä
\end{description}
}
\end{comment}


\frame{
\begin{block}{Esimerkki}
Ratkaise virta $I_2$ kerrostamismenetelmällä.
\end{block}
\[
J=1\A \quad R_1=10 \ohm \quad R_2= 20 \ohm \quad R_3=30 \ohm
\quad E=5\V
\]

\begin{center}
\begin{picture}(150,50)(0,0)
\vj{0,0}{J}
\vz{50,0}{R_1}
\vz{100,0}{R_3}
\hz{50,50}{R_2}
%\hz{50,0}{R_4}
%\out{150,0}
%\out{150,50}
\ri{58,50}{I_2}
\vst{150,0}{E}
\hln{0,0}{150}
\hln{100,0}{50}
\hln{0,50}{50}
\hln{100,50}{50}
%\du{57,0}{U_1}
%\ri{57,50}{I}
\end{picture}
\end{center}

}

%LUENTO8

\frame{
\begin{block}{Ratkaisu}
Ratkaise virta $I_2$ kerrostamismenetelmällä.
\end{block}
\[
J=1\A \quad R_1=10 \ohm \quad R_2= 20 \ohm \quad R_3=30 \ohm
\quad E=5\V
\]

\begin{center}
\begin{picture}(150,50)(0,0)
\vj{0,0}{J}
\vz{50,0}{R_1}
\vz{100,0}{R_3}
\hz{50,50}{R_2}
%\hz{50,0}{R_4}
%\out{150,0}
%\out{150,50}
\ri{58,50}{I_2}
\vst{150,0}{E}
\hln{0,0}{150}
\hln{100,0}{50}
\hln{0,50}{50}
\hln{100,50}{50}
%\du{57,0}{U_1}
%\ri{57,50}{I}
\end{picture}
\end{center}

}

\frame{
\frametitle{Ratkaisu}
Lasketaan ensin virtalähteen vaikutus:
\begin{center}
\begin{picture}(150,50)(0,0)
\vj{0,0}{J}
\vz{50,0}{R_1}
\vz{100,0}{R_3}
\hz{50,50}{R_2}
%\hz{50,0}{R_4}
%\out{150,0}
%\out{150,50}
\ri{58,50}{I_{21}}
%\vst{150,0}{E_1}
\vln{150,0}{50}
\hln{0,0}{150}
\hln{100,0}{50}
\hln{0,50}{50}
\hln{100,50}{50}
%\du{57,0}{U_1}
%\ri{57,50}{I}
\end{picture}
\end{center}
Vastusten $R_1$ ja $R_2$ yli on sama jännite (ne ovat rinnan) ja vastus $R_2$ kaksinkertainen verrattuna
vastukseen $R_1$ joten $R_2$:n läpi kulkee puolet pienempi virta kuin $R_1$:n. Koska vastusten
läpi kulkee yhteensä $J=1\A$:n suuruinen virta, kulkee $R_1$:n läpi $2/3 \A$ ja $R_2$:n
läpi $I_{21}=1/3 \A$.
}

\frame{
\frametitle{Ratkaisu}
Lasketaan seuraavaksi jännitelähteen vaikutus:
\begin{center}
\begin{picture}(150,50)(0,0)
%\vj{0,0}{J_1}
\vz{50,0}{R_1}
\vz{100,0}{R_3}
\hz{50,50}{R_2}
%\hz{50,0}{R_4}
%\out{150,0}
%\out{150,50}
\ri{58,50}{I_{22}}
\vst{150,0}{E}
%\vln{150,0}{50}
\hln{0,0}{150}
\hln{100,0}{50}
\hln{0,50}{50}
\hln{100,50}{50}
%\du{57,0}{U_1}
%\ri{57,50}{I}
\end{picture}
\end{center}
Vastukset $R_1$ ja $R_2$ ovat nyt sarjassa ja niiden yli on yhteensä $E=5\V$ jännite, joten
\[
I_{22}=-\frac{E}{R_1+R_2}=-\frac{5\V}{10\ohm+20\ohm}=-\frac{1}{6}\A .
\]
Negatiivinen etumerkki johtuu siitä, että virran $I_{22}$ suunta on alhaalta ylös ja
vastusten jännitteen suunta ylhäältä alas.\\
Lopuksi yhdistetään tulokset:
\[
I_2=I_{21}+I_{22}=\frac{1}{3}\A-\frac{1}{6}\A=\frac{1}{6} \A.
\]
}

\frame{
\begin{block}{Esimerkki} % Piirianalyysi 1 laskari 2. tehtävä
Ratkaise $U$ ja $I$ ensin kerrostamismenetelmällä ja sitten jollain muulla
menetelmällä.
\end{block}

\begin{center}
\[
R_1=1\ohm\quad R_2=2\ohm\quad J=1\A \quad E= 3\V
\]
\begin{picture}(150,50)(0,0)
\vst{0,0}{E}
\ri{25,50}{I}
\vz{50,0}{R_1}
\hz{50,50}{R_2}
\vdj{100,0}{J}
\du{112,0}{U}
\hln{0,0}{100}
\hln{0,50}{50}

\end{picture}
\end{center}
\tiny Vastaus: $I=4\A$ ja $U=1\V$.
}

\frame{
\begin{block}{Esimerkki}
Ratkaise kerrostamismenetelmällä virta $I$ ja jännite $U$.
\end{block}
\begin{center}
\begin{picture}(50,80)(0,0)
\vz{0,0}{R_1}
\hln{0,50}{50}
\vj{50,0}{J}
\vst{100,0}{E}
\hln{0,0}{100}
\hz{50,50}{R_2}
\di{0,2}{I}
\rcuu{50,53}{U}
%\vst{0,0}{1,5 \V}
%\vst{0,50}{1,5 \V}
%\vz{50,25}{R=20\ohm\hspace{-2.5cm}}
%\vln{50,0}{25}
%\vln{50,75}{25}
%\hln{0,100}{50}
%\hln{0,0}{50}
%\di{50,25}{I}
\end{picture}
\[
E=3\V\qquad R_1=R_2=1\ohm \qquad J=1\A
\]
\end{center}

}





\frame{
%\frametitle{Kotitehtävä 1b}
\begin{block}{Ratkaisu}
Ratkaise kerrostamismenetelmällä virta $I$ ja jännite $U$.
\end{block}
Ratkaistaan ensin virtalähteen vaikutus.
\begin{center}
\begin{picture}(50,80)(0,0)
\vz{0,0}{R_1}
\hln{0,50}{50}
\vj{50,0}{J}
%\vst{100,0}{E}
\vln{100,0}{50}
\hln{0,0}{100}
\hz{50,50}{R_2}
\di{0,2}{I_1}
\rcuu{50,53}{U_1}
\end{picture}
\[
E=3\V\qquad R_1=R_2=1\ohm \qquad J=1\A
\]
\end{center}
\[
U_1=J\frac{R_1R_2}{R_1+R_2}=0,5\V \qquad I_1=J\frac{G_1}{G_1+G_2}=0,5\A
\]
}

\frame{
\begin{block}{Ratkaisu}
Ratkaise kerrostamismenetelmällä virta $I$ ja jännite $U$.
\end{block}
Ratkaistaan seuraavaksi jännitelähteen vaikutus.
\begin{center}
\begin{picture}(50,80)(0,0)
\vz{0,0}{R_1}
\hln{0,50}{50}
%\vj{50,0}{J}
\vst{100,0}{E}
%\vln{50,0}{50}
\hln{0,0}{100}
\hz{50,50}{R_2}
\di{0,2}{I_2}
\rcuu{50,53}{U_2}
\end{picture}
\[
E=3\V\qquad R_1=R_2=1\ohm \qquad J=1\A
\]
\end{center}
\[
U_2=-E\frac{R_2}{R_1+R_2}=-1,5\V \qquad I_2=\frac{E}{R_1+R_2}=1,5\A
\]
}


\frame{
\begin{block}{Ratkaisu}
Ratkaise kerrostamismenetelmällä virta $I$ ja jännite $U$.
\end{block}
Yhdistetään tulokset:
\[
U=U_1+U_2=0,5\V-1,5\V=-1\V
\]
\[
I=I_1+I_2=0,5\A+1,5\A=2\A
\]
}
