\frame{
\frametitle{Kondensaattori}
\begin{itemize}
\item Ideaalisen kondensaattorin toiminta noudattaa yhtälöä
\[
i=C\frac{{\rm d}u}{{\rm d}t}
\]
\item Yksinkertaisen tasokondensaattorin kapasitanssi riippuu levyjen pinta-alasta, niiden välisestä etäisyydestä sekä eristeaineen permittiivisyydestä:
\[
C=\epsilon_0\epsilon_{\rm r}\frac{A}{d}
\]
\item Kondensaattori valmistetaan pinoamalla ohuita levyjä eristeineen päällekkäin, ja kiertämällä näin saatu levypino rullalle.
\item Eristettä ohentamalla kapasitanssi kasvaa, mutta jännitteenkesto pienenee.
\end{itemize}
}


\frame{
\frametitle{Käytännön kondensaattorien ominaisuudet}
\begin{itemize}
\item Kapasitanssi toleransseineen
\item Jännitteenkesto
\item Sarjaresistanssi
\end{itemize}
}

\frame{
\frametitle{Kondensaattorien merkinnät}
\begin{itemize}
\item Aikaisemmin käytettiin värikoodeja.
\item Suuriin kondensaattoreihin kapasitanssi merkitään yleensä selkokielellä.
\item Hyvin yleinen on merkintätapa, jossa on kolme numeroa ja yksi kirjain. Kaksi ensimmäistä numeroa
kertoo kapasitanssin pikofaradeina, kolmas nollien lukumäärän ja kirjain kertoo toleranssin (K = $\pm 10\%$, M = $\pm 20 \%$ ja Z = $-20\ldots +80 \%$).
\item Esimerkiksi 474M on 470000 pF eli 470 nF kondensaattori, jonka toleranssi on $\pm 20 \%$. 
\item Merkintä 100V tarkoittaa yksinkertaisesti, että kondensaattorin jännitteenkesto on 100 volttia. 
\end{itemize}
}

\frame{
\frametitle{Päätyypit}
\begin{itemize}
\item Keraamiset kondensaattorit
\item Muovikondensaattorit
\item Elektrolyyttikondensaattorit ("tavallinen"\ ja tantaalikondensaattori)
\end{itemize}
}

\frame{
\frametitle{Elektrolyyttikondensaattorien kytkeminen}
\begin{itemize}
\item Elektrolyyttikondensaattoreissa eristekerros muodostuu kemiallisen reaktion seurauksena
\item Jos elektrolyyttikondensaattori (= "elko") kytketään väärin päin piiriin, eristekerrosta ei muodostu ja kondensaattori käyttäytyy kuin matalaresistanssinen vastus ja yleensä tuhoutuu.
\item Huom! Virta saa kulkea välillä myös väärään suuntaan. Pääasia on, että kun laite kytketään päälle, kondensaattorissa kulkee ensin virta oikeaan suuntaan.
\end{itemize}
}

\frame{
\frametitle{Sovelluksia}
\begin{itemize}
\item Energiavarasto
\item Suodatus signaalinkäsittelyssä
\item Loistehon kompensointi ja tehosovitus
\item Läheisyysanturi
\item Kosketusnäytöt
\end{itemize}
}
