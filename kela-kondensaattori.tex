\frame{
\frametitle{Kela ja kondensaattori}
\begin{center}
\begin{picture}(200,50)(0,0)
\hz{0,0}{R} \ri{7,0}{i} \rcuu{0,5}{u}
\hl{75,0}{L} \ri{82,0}{i} \rcuu{75,5}{u}
\hc{150,0}{C} \ri{157,0}{i} \rcuu{150,5}{u}

\txt{25,-50}{u=Ri}
\txt{100,-50}{u=L\frac{{\rm d}i}{{\rm d}t}}
\txt{175,-50}{i=C\Du}

\end{picture}
\end{center}


}

\frame{
\frametitle{Kela ja kondensaattori tasasähköpiirissä}
\begin{center}
\begin{picture}(200,50)(0,-50)
\hz{0,0}{R} \ri{7,0}{i} \rcuu{0,5}{u}
\hl{75,0}{L} \ri{82,0}{i} \rcuu{75,5}{u}
\hc{150,0}{C} \ri{157,0}{i} \rcuu{150,5}{u}

\txt{25,-50}{u=Ri}
\txt{100,-50}{u=L\frac{{\rm d}i}{{\rm d}t}}
\txt{175,-50}{i=C\Du}

\end{picture}
\end{center}
Tasajännite ja -virta pysyvät ajan suhteen vakiona eli jännitteiden ja virtojen aikaderivaatat ovat nollia. Eli kelan jännite on tasasähköpiirissä nolla ja kondensaattorin virta on tasasähköpiirissä nolla.

}



\frame{
\frametitle{Poikkeus 1}
Kondensaattoriin syötetään väkisin tasavirtaa.
\begin{center}
\begin{picture}(150,50)(0,0)
\vj{0,0}{J}
\vc{50,0}{C}
\hln{0,0}{50}
\hln{0,50}{50}
\end{picture}
\end{center}
$i=C\Du \Rightarrow J=C\Du \Rightarrow \Du=\frac{J}{C}$ eli kondensaattorin jännite kasvaa vakionopeudella. 
}

\frame{
\frametitle{Poikkeus 2}
Kelaan kytketään tasajännitelähde.
\begin{center}
\begin{picture}(150,50)(0,0)
\vst{0,0}{E}
\vl{50,0}{L}
\hln{0,0}{50}
\hln{0,50}{50}
\end{picture}
\end{center}
$u=L\frac{{\rm d}i}{{\rm d}t} \Rightarrow E=L\frac{{\rm d}i}{{\rm d}t} \Rightarrow \frac{{\rm d}i}{{\rm d}t}=\frac{E}{L}$ eli kelan virta kasvaa vakionopeudella. 
}


\frame{
\frametitle{Kelan ja kondensaattorin käsittely tasasähköpiirilaskuissa}
Kela korvataan oikosululla (=johtimella) ja kondensaattori korvataan katkoksella (eli irrotetaan
piiristä).
}

\frame{
\begin{block}{Esimerkki}
Ratkaise jännite $U$ oheisesta tasasähköpiiristä.
\end{block}
\[
E_1=10\V \quad R_1=10 \ohm \quad R_2= 20 \ohm \quad R_3=30 \ohm
\]
\[
 R_4=40 \ohm \quad L=500\, \mbox{mH}\quad C=2\,{\rm F}
\quad E_2=15\V
\]

\begin{center}
\begin{picture}(150,50)(0,0)
\vst{0,0}{E_1}
\hz{0,50}{R_1}
\vz{50,0}{R_2}
\vz{100,0}{R_3}
\hl{100,50}{L}
%\hz{50,0}{R_4}
%\out{150,0}
%\out{150,50}
%\ri{58,50}{I_2}
\hc{50,50}{C}
\vst{200,0}{E_2}
\hz{150,50}{R_4}
\hln{0,0}{200}
\hln{100,0}{50}
%\hln{100,50}{50}
%\du{57,0}{U_1}
%\ri{57,50}{I}
\dcru{105,0}{U}
\end{picture}
\end{center}

}

%LUENTO10

\frame{
\begin{block}{Ratkaisu}
Ratkaise jännite $U$ oheisesta tasasähköpiiristä.
\end{block}
\[
E_1=10\V \quad R_1=10 \ohm \quad R_2= 20 \ohm \quad R_3=30 \ohm
\]
\[
 R_4=40 \ohm \quad L=500\, \mbox{mH}\quad C=2\,{\rm F}
\quad E_2=15\V
\]

\begin{center}
\begin{picture}(150,50)(0,0)
\vst{0,0}{E_1}
\hz{0,50}{R_1}
\vz{50,0}{R_2}
\vz{100,0}{R_3}
\hl{100,50}{L}
%\hz{50,0}{R_4}
%\out{150,0}
%\out{150,50}
%\ri{58,50}{I_2}
\hc{50,50}{C}
\vst{200,0}{E_2}
\hz{150,50}{R_4}
\hln{0,0}{200}
\hln{100,0}{50}
%\hln{100,50}{50}
%\du{57,0}{U_1}
%\ri{57,50}{I}
\dcru{105,0}{U}
\end{picture}
\end{center}


}

\frame{
\frametitle{Ratkaisu jatkuu}
Koska piirissä ei ole jännitelähde-kela-rinnankytkentöjä eikä virtalähde-kondensaattori-sarjaankytkentökä ja kyseessä on tasasähköpiiri (jännitteet ja virran pysyvät vakiona), voidaan
kelat korvata oikosuluilla ja kondensaattorit katkoksilla

\begin{center}
\begin{picture}(150,50)(0,0)
\vst{0,0}{E_1}
\hz{0,50}{R_1}
\vz{50,0}{R_2}
\vz{100,0}{R_3}
\hln{100,50}{50}
%\hz{50,0}{R_4}
%\out{150,0}
%\out{150,50}
%\ri{58,50}{I_2}

\vst{200,0}{E_2}
\hz{150,50}{R_4}
\hln{0,0}{200}
\hln{100,0}{50}
%\hln{100,50}{50}
%\du{57,0}{U_1}
%\ri{57,50}{I}
\dcru{105,0}{U}
\end{picture}
\end{center}
jolloin jännite $U$ saadaan näppärästi jännitteenjakosäännöllä:
\[
U=E_2\frac{R_3}{R_3+R_4}=6\frac{3}{7}\V\approx 6,4 \V
\]

}

