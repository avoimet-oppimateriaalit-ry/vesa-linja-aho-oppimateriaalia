\frame{
\frametitle{Virtalähde}
\begin{itemize}
\item Puhekielessä sanaa virtalähde käytetään varsin monimerkityksellisesti.
Esimerkiksi {\em tietokoneen virtalähde hajosi}.
\item Virtalähteellä tarkoitetaan piiriteoriassa elementtiä, jonka läpi kulkee
jokin tietty virta (se voi olla vakio tai muuttua jonkin säännön mukaan).
\item Aivan kuten jännitelähde pitää napojensa välillä aina jonkin tietyn jännitteen riippumatta siitä, mitä lähteeseen on kiinnitetty, virtalähde syöttää siis lävitseen jonkun tietyn virran, riippumatta siitä mitä lähteeseen on kiinnitetty.
\end{itemize}
\begin{center}
\begin{picture}(100,50)(0,0)
\vj{0,0}{J}
\vz{50,0}{R}
\hln{0,0}{50}
\hln{0,50}{50}
\end{picture}
\end{center}
}

\frame{
\frametitle{Virtalähde}
\begin{itemize}
\item Kun jossain johtimen haarassa on virtalähde, tiedät johtimen virran.
\end{itemize}
\begin{center}
\begin{picture}(100,50)(0,0)
\vj{0,0}{J=1\A}
\vz{50,0}{R_1}
\hln{0,0}{100}
\hln{0,50}{100}
\ri{25,50}{I=1\A}
\vz{100,0}{R_2}
\end{picture}
\end{center}
}