\frame{
\frametitle{Jännitteenjakosääntö}
\begin{center}
\begin{picture}(150,50)(0,0)
\hz{0,50}{R_1}
\vz{50,0}{R_2}
\hln{0,0}{50}
\du{0,0}{U}
\du{60,0}{U_2}
\ru{0,60}{U_1}

\end{picture}
\end{center}
\begin{itemize}
\item $U_1=U\frac{R_1}{R_1+R_2}$ ja $U_2=U\frac{R_2}{R_1+R_2}$
\item Elektroniikkapiirissä tarvitaan usein vertailujännite, joka muodostetaan jostain suuremmasta jännitteestä.
\item Kaava toimii myös useamman vastuksen sarjaankytkennälle. Nimittäjään tulee kaikkien vastusten summa
ja osoittajaan se vastus, jonka yli olevaa jännitettä kysytään.
\end{itemize}

}

\frame{
\frametitle{Virranjakosääntö}
\begin{center}
\begin{picture}(150,50)(0,0)
\hln{0,50}{100}
%\hz{0,50}{R_1}
\vz{50,0}{R_1}
\vz{100,0}{R_2}
\hln{0,0}{100}
\ri{9,50}{I}
\di{50,42}{I_1}
\di{100,42}{I_2}

%\du{0,0}{U}
%\du{60,0}{U_2}
%\ru{0,60}{U_1}

\end{picture}
\end{center}

\begin{itemize}
\item $I_1=I\frac{G_1}{G_1+G_2}$ ja $I_2=I\frac{G_2}{G_1+G_2}$
\item Kaava pätee myös monen vastuksen rinnankytkennälle.
\item Tätä ei tarvita yhtä tavallisesti kuin jännitteenjakosääntöä, mutta on luontevaa ottaa se esille jännitteenjakosäännön yhteydessä.
\end{itemize}

}

\frame{
\frametitle{Esimerkki 1}
\begin{center}
\begin{picture}(150,50)(0,0)
\hz{0,50}{R_4}
\hln{50,50}{100}
%\hz{0,50}{R_1}
\vz{50,0}{R_1}
\vz{100,0}{R_2}
\vz{150,0}{R_3}
\hln{0,0}{150}
%\ri{9,50}{I}
\di{50,42}{I_1}
\di{100,42}{I_2}
\di{150,42}{I_3}
\vj{0,0}{J}
\end{picture}
\end{center}
\pause
$I_1=J\frac{G_1}{G_1+G_2+G_3}\quad$
\pause
$I_2=J\frac{G_2}{G_1+G_2+G_3}\quad$
\pause
$I_3=J\frac{G_3}{G_1+G_2+G_3}$
}

\frame{
\frametitle{Esimerkki 2}
\begin{center}
\begin{picture}(150,50)(0,0)
\hz{0,50}{R_1}
%\hln{50,50}{100}
%\hz{0,50}{R_1}
\hz{50,50}{R_2}
\hz{100,50}{R_3}
\vz{150,0}{R_4}
\hln{0,0}{150}
%\ri{9,50}{I}
%\di{50,42}{I_1}
%\di{100,42}{I_2}
%\di{150,42}{I_3}
\vst{0,0}{E}
\color{red}
\rcuu{0,55}{U_1}
\rcuu{50,55}{U_2}
\rcuu{100,55}{U_3}
\dcru{155,0}{U_4}

\end{picture}
\end{center}
\pause
$U_1=E\frac{R_1}{R_1+R_2+R_3+R_4}\quad$
\pause
$U_2=E\frac{R_2}{R_1+R_2+R_3+R_4}\quad$
\pause
$U_3=E\frac{R_3}{R_1+R_2+R_3+R_4}\quad$
\pause
$U_4=E\frac{R_4}{R_1+R_2+R_3+R_4}\quad$
}


\frame{
\begin{block}{Esimerkki}
Ratkaise jännite $U$ jännitteenjakosääntöä hyväksikäyttämällä.
\end{block}
\[
E_1=10\V \quad R_1=10 \ohm \quad R_2= 20 \ohm \quad R_3=30 \ohm
\]
\[
 R_4=40 \ohm \quad R_5=50 \ohm
\quad E_2=15\V
\]

\begin{center}
\begin{picture}(150,50)(0,0)
\vst{0,0}{E_1}
\hz{0,50}{R_1}
\vz{50,0}{R_2}
\vz{100,0}{R_3}
\hz{100,50}{R_4}
%\hz{50,0}{R_4}
%\out{150,0}
%\out{150,50}
%\ri{58,50}{I_2}
\ru{50,50}{U}
\vst{200,0}{E_2}
\hz{150,50}{R_5}
\hln{0,0}{200}
\hln{100,0}{50}
%\hln{100,50}{50}
%\du{57,0}{U_1}
%\ri{57,50}{I}
\end{picture}
\end{center}

}

%LUENTO9

\frame{
\begin{block}{Ratkaisu}
Ratkaise jännite $U$ jännitteenjakosääntöä hyväksikäyttämällä.
\end{block}
\[
E_1=10\V \quad R_1=10 \ohm \quad R_2= 20 \ohm \quad R_3=30 \ohm
\]
\[
 R_4=40 \ohm \quad R_5=50 \ohm
\quad E_2=15\V
\]

\begin{center}
\begin{picture}(150,50)(0,0)
\vst{0,0}{E_1}
\hz{0,50}{R_1}
\vz{50,0}{R_2}
\vz{100,0}{R_3}
\hz{100,50}{R_4}
%\hz{50,0}{R_4}
%\out{150,0}
%\out{150,50}
%\ri{58,50}{I_2}
\ru{50,50}{U}
\vst{200,0}{E_2}
\hz{150,50}{R_5}
\hln{0,0}{200}
\hln{100,0}{50}
%\hln{100,50}{50}
%\du{57,0}{U_1}
%\ri{57,50}{I}
\color{red}
\du{60,0}{U_2}
\du{108,0}{U_3}

\end{picture}
\end{center}
$U_2=E_1\frac{R_2}{R_1+R_2}=10\V\frac{20\ohm}{10\ohm+20\ohm}=6\frac{2}{3}\V$
$U_3=E_2\frac{R_3}{R_3+R_4+R_5}=15\V\frac{30\ohm}{30\ohm+40\ohm+50\ohm}=3,75\V$
$U=U_2-U_3=2\frac{11}{12}\V\approx 2,92\V$.
}

\frame{
\begin{block}{Esimerkki}
Tiedetään, että virta $I_3=0\A$. Laske $E_1$.
\end{block}
\[
R_1=5 \ohm \quad R_2=4 \ohm \quad R_3=2 \ohm \quad R_4=5 \ohm \quad R_5=6 \ohm \quad
E_2=30 \V
\]

\begin{center}
\begin{picture}(150,50)(0,0)
\vst{0,0}{E_1}
\vst{150,0}{E_2}
\hz{0,50}{R_1}
\hz{50,50}{R_3}
\hz{100,50}{R_2}
\vz{50,0}{R_4}
\vz{100,0}{R_5}
\hln{0,0}{150}
\ri{97,50}{I_3}
\end{picture}
\end{center}
}

%LUENTO12

\frame{
\begin{block}{Ratkaisu}
Tiedetään, että virta $I_3=0\A$. Laske $E_1$.
\end{block}
\[
R_1=5 \ohm \quad R_2=4 \ohm \quad R_3=2 \ohm \quad R_4=5 \ohm \quad R_5=6 \ohm \quad
E_2=30 \V
\]

\begin{center}
\begin{picture}(150,50)(0,0)
\vst{0,0}{E_1}
\vst{150,0}{E_2}
\hz{0,50}{R_1}
\hz{50,50}{R_3}
\hz{100,50}{R_2}
\vz{50,0}{R_4}
\vz{100,0}{R_5}
\hln{0,0}{150}
\ri{97,50}{I_3}
\pause
\color{red}
\du{57,0}{U_4}
\du{107,0}{U_5}

\end{picture}
\end{center}

}

\frame{
Koska $I_3=0\A$,  vastusten $R_1$ ja $R_4$ läpi kulkee sama virta, ja samoin vastusten
$R_2$ ja $R_5$ läpi kulkee sama virta. Näin ollen ne ovat sarjassa\footnote{Siksi ja vain siksi
että tiedetään, että $I_3$ on nolla.} ja niihin voidaan soveltaa jännitteenjakosääntöä. Vastuksen $R_5$ yli oleva jännite on $U_5=E_2\frac{R_5}{R_2+R_5}=18\V$. Tällöin vastuksen
$R_4$ yli on myös $18 \V$. Nyt jännitteenjakosäännön mukaan:
\[
U_4=E_1\frac{R_4}{R_1+R_4} \quad \Rightarrow \quad 18\V=E_1\frac{5\ohm}{5\ohm+5\ohm}
\]
josta ratkeaa $E_1=36\V$.

Huom! Aivan yhtä oikein olisi ollut kirjoittaa solmujänniteyhtälöt piirille ja ratkaista niistä
$E_1$.
}

\frame{
\begin{block}{Esimerkki}
Ratkaise virta $I$ ja jännite $U$.
\end{block}
\begin{center}
\begin{picture}(50,68)(0,0)
\vst{0,0}{E}
\hz{0,50}{R_1}
\vz{50,0}{R_2}
\vz{100,0}{R_3}
\hln{0,0}{100}
\hln{50,50}{50}
\di{100,2}{I}
\rcuu{0,53}{U}
%\vst{0,0}{1,5 \V}
%\vst{0,50}{1,5 \V}
%\vz{50,25}{R=20\ohm\hspace{-2.5cm}}
%\vln{50,0}{25}
%\vln{50,75}{25}
%\hln{0,100}{50}
%\hln{0,0}{50}
%\di{50,25}{I}
\end{picture}
\[
E=10\V\qquad R_1=7,5\kohm \qquad R_2=R_3=5\kohm
\]
\end{center}

}


\frame{
%\frametitle{Kotitehtävä 1a }
\begin{block}{Esimerkki}
Ratkaise virta $I$ ja jännite $U$.
\end{block}
\begin{center}
\begin{picture}(50,68)(0,0)
\vst{0,0}{E}
\hz{0,50}{R_1}
\vz{50,0}{R_2}
\vz{100,0}{R_3}
\hln{0,0}{100}
\hln{50,50}{50}
\di{100,2}{I}
\rcuu{0,53}{U}
\end{picture}
\[
E=10\V\qquad R_1=7,5\kohm \qquad R_2=R_3=5\kohm
\]
\end{center}
Jännitteenjakosäännön mukaan
\[
U=E \frac{R_1}{R_1+R_2||R_3}=7,5\V.
\]
Ratkaistaan vastuksen $R_3$ jännite Kirchhoffin jännitelailla, ja sitten virta Ohmin lailla:
\[
I=\frac{E-U}{R_3}=0,5\mA
\]
}

