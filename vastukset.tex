\frame{
\frametitle{Käytännön vastukset}

\begin{itemize}
\item Massavastukset
\item Hiilikalvovastukset
\item Metallikalvovastukset
\item Lankavastukset
\end{itemize}
Mikropiireissä vastukset vievät paljon tilaa $\to$ pyritään korvaamaan muilla komponenteilla.
}

\frame{
\frametitle{Ominaisuudet}
\begin{itemize}
\item Nimellisarvo
\item Tehonkesto
\item Toleranssi
\item Lämpötilariippuvuus
\end{itemize}
}

\frame{
\frametitle{E-sarjat}
\begin{itemize}
\item Komponentteja ei ole käytännössä järkevää (eikä mahdollistakaan) valmistaa jokaista mahdollista lukuarvoa.
\item Yleinen sarja on E12: 10, 12, 15, 18, 22, 27, 33, 39, 47, 56, 68, 82

\end{itemize}
}

\frame{
\frametitle{Värikoodit}
\begin{itemize}
\item Tehovastuksissa nimellisarvo ja tehonkesto on merkitty numeroin vastuksen kylkeen.
\item Pienillä vastuksilla arvo on värikoodattu.
\item Välillä on hankala tietää, kummasta päästä lukeminen aloitetaan ja mitkä arvot kuuluvat lukuarvoon.
\end{itemize}
}

\frame{
\frametitle{Loisilmiöt}
\begin{itemize}
\item Ideaalista vastusta ei ole olemassakaan.
\item Tarkka mallintaminen vaikeaa.
\item Rinnakkaiskapasitanssi (pikofaradeja tai vähemmän).
\item Sarjainduktanssi: kymmeniä nano- tai korkeintaan muutama mikrohenryjä.
\item Kohina (mikrovoltteja).
\end{itemize}
}

\frame{
\frametitle{Erikoisvastuksia}
\begin{itemize}
\item Säätövastus eli potentiometri.
\item VDR
\item NTC ja PTC
\end{itemize}
}
