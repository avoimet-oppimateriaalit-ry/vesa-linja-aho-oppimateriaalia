\frame {
\frametitle{Sarjaankytkentä ja rinnankytkentä}
\begin{alertblock}{Määritelmä: sarjaankytkentä}
Piirielementit ovat sarjassa, jos niiden läpi kulkee sama virta.
\end{alertblock}
\begin{alertblock}{Määritelmä: rinnankytkentä}
Piirielementit ovat rinnan, jos niiden yli on sama jännite.
\end{alertblock}

Sama tarkoittaa samaa, ei samansuuruista.

}

\frame {
\frametitle{Sarjaankytkentä ja rinnankytkentä}
\begin{exampleblock}{Sarjaankytkentä}
\begin{center}
\begin{picture}(100,20)(0,-10)
\hz{0,0}{}
\hz{50,0}{}
\ri{-10,0}{I}
\ri{110,0}{I}
\hln{-50,0}{50}
\hln{100,0}{50}
\end{picture}
\end{center}
\end{exampleblock}

\begin{exampleblock}{Rinnankytkentä}
\begin{center}
\begin{picture}(50,75)(0,-5)
\hz{0,0}{}
\hz{0,50}{}
%\hln{-50,25}{50}
%\hln{50,25}{50}
\vln{0,0}{50}
\vln{50,0}{50}
\ru{0,10}{U}
\ru{0,60}{U}
\end{picture}
\end{center}
\end{exampleblock}
}

\frame {
\frametitle{Vastusten sarjaankytkentä ja rinnankytkentä}
\begin{exampleblock}{Sarjaankytkentä}
\begin{center}
\begin{picture}(100,20)(50,-15)
\hz{0,0}{R_1}
\hz{50,0}{R_2}
%\ri{-10,0}{I}
%\ri{110,0}{I}
%\hln{-50,0}{50}
%\hln{100,0}{50}
\txt{125,0}{\Longleftrightarrow}
\hz{150,0}{R=R_1+R_2}
\end{picture}
\end{center}
\end{exampleblock}
\begin{exampleblock}{Rinnankytkentä}
\begin{center}
\begin{picture}(100,75)(0,-20)
\hz{0,0}{R_1}
\hz{0,50}{R_2}
\hln{-25,25}{25}
\hln{50,25}{25}
\vln{0,0}{50}
\vln{50,0}{50}
%\ru{0,10}{U}
%\ru{0,60}{U}
\txt{110,25}{\Longleftrightarrow}
\hz{130,25}{\vspace{-1cm}R=\frac{1}{\frac{1}{R_1}+\frac{1}{R_2}}}
\end{picture}
\end{center}
\end{exampleblock}
Tai sama kätevämmin konduktansseilla $G=G_1+G_2$.
}



\frame{
\frametitle{Vastusten sarjaankytkentä ja rinnankytkentä}
\begin{itemize}
\item Edellisen kalvon kaavat soveltuvat myös mielivaltaisen monelle vastukselle. Esimerkiksi 
viiden resistanssin sarjaankytkennän resistanssi on $R=R_1+R_2+R_3+R_4+R_5$.
\end{itemize}
}

\frame{
\frametitle{Jännitelähteiden sarjaankytkentä}
\begin{itemize}
\item Jännitelähteiden sarjaankytkennässä jännitteet voidaan laskea yhteen (mutta etumerkeissä
pitää olla tarkkana).
\item Jännitelähteiden rinnankytkentä on piiriteoriassa kielletty (kahden pisteen välillä
ei voi olla yhtäaikaa kaksi eri jännitettä).
\end{itemize}
\begin{center}
\begin{picture}(100,35)(0,-30)
\hst{0,0}{E_1}
\hlst{50,0}{E_2}
\hst{100,0}{E_3}
\cn{0,0}
\cn{150,0}
\end{picture}

\begin{picture}(100,25)(0,0)
\txt{75,23}{\Longleftrightarrow}
\hst{50,0}{E=E_1-E_2+E_3}
\cn{50,0}
\cn{100,0}
\end{picture}
\end{center}
}

\frame{
\frametitle{Mitä sarjaan- ja rinnankytkentä eivät ole}
\begin{itemize}
\item Pelkkä se, että komponentit "näyttävät olevan vierekkäin" ei tarkoita, että kyseessä on rinnankytkentä.
\item Pelkkä se, että komponentit "näyttävät olevan peräkkäin" ei tarkoita, että kyseessä on sarjaankytkentä.
\item Mitkä kuvan vastuksista ovat keskenään sarjassa ja mitkä rinnan?
\end{itemize}
\begin{center}
\begin{picture}(100,50)(0,0)
\vst{0,0}{E_1}
\vst{100,0}{E_2}
\vz{50,0}{R_3}
\hz{0,50}{R_1}
\hz{50,50}{R_2}
\hln{0,0}{100}
\end{picture}
\end{center}
\pause
\vspace{-0.2cm}
\begin{alertblock}{Vastaus}
\scriptsize
Eivät mitkään! $E_1$ ja $R_1$ ovat sarjassa keskenään, samoin $E_2$ ja $R_2$. Nämä sarjaankytkennät
ovat puolestaan molemmat rinnan $R_3$:n kanssa. Sen sijaan mitkään vastukset eivät ole keskenään rinnan
eivätkä sarjassa. 
\end{alertblock}
}

\frame{
\begin{block}{Esimerkki}
Ratkaise virta $I$.
\end{block}
\begin{center}
\begin{picture}(150,50)(0,0)
\vst{0,0}{E}
\hz{0,50}{R_1}
\hz{50,50}{R_3}
\hz{100,50}{R_5}
\vz{50,0}{R_2}
\vz{100,0}{R_4}
\vz{150,0}{R_6}
\hln{0,0}{150}
\ri{8,50}{I}
\end{picture}
$R_1=R_2=R_3=R_4=R_5=R_6=1\ohm\qquad E=9\V$
\end{center}

}

%LUENTO3

%\frame{\tableofcontents}

%\subsection{2. Kotitehtävän ratkaisu}

\frame{

\begin{block}{Ratkaisu}
Ratkaise virta $I$.
\end{block}
\begin{center}
\begin{picture}(150,50)(0,0)
\vst{0,0}{E}
\hz{0,50}{R_1}
\hz{50,50}{R_3}
\hz{100,50}{R_5}
\vz{50,0}{R_2}
\vz{100,0}{R_4}
\vz{150,0}{R_6}
\hln{0,0}{150}
\ri{8,50}{I}
\end{picture}
$R_1=R_2=R_3=R_4=R_5=R_6=1\ohm\qquad E=9\V$
\end{center}
\begin{itemize}
\item $R_5$ ja $R_6$ ovat sarjassa. Tämän sarjaankytkennän resistanssi on $R_5+R_6=2\ohm$.
\item Tämä sarjaankytkentä puolestaan on rinnan $R_4$:n kanssa. Tämän rinnankytkennän
resistanssi on $\frac{1}{\frac{1}{1}+\frac{1}{2}}\ohm=\frac{2}{3}\ohm$.
\end{itemize}
}
\frame{
\frametitle{Ratkaisu jatkuu}
\begin{itemize}
\item $R_3$ taas on sarjassa edellisen kanssa. Sarjaankytkennän resistanssi on $R_3+\frac{2}{3}\ohm
=\frac{5}{3}\ohm$.
\item Ja tämä sarjaankytkentä on rinnan $R_2$:n kanssa. Tämän rinnankytkennän resistanssi
on $\frac{1}{(\frac{5}{3})^{-1}+\frac{1}{1}}=\frac{5}{8}\ohm$.
\item Ja tämän kanssa on sarjassa vielä $R_1$. Jännitelähteen $E$ näkemä kokonaisresistanssi on
siis $\frac{5}{8}\ohm + R_1=\frac{13}{8}\ohm$.
\item Virta $I$ on Ohmin lain mukaan $I=\frac{E}{\frac{13}{8}\ohm}=\frac{72}{13}\A\approx5,5\A$.
\end{itemize}
}

