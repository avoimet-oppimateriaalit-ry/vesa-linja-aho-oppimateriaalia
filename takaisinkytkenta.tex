\frame{
\frametitle{Takaisinkytkentä (engl. feedback)}
\begin{itemize}
\item Takaisinkytkentä: piirin lähtösignaali summataan piirin tulosignaaliin, kertoimella $B$ painotettuna.
\item Jos lähtösignaali kytketään takaisin niin, että lähtösignaalin kasvu aikaansaa lähtösignaalin pienenemisen, kyseessä on negatiivinen takaisinkytkentä.
\item Jos lähtösignaali kytketään takaisin niin, että lähtösignaalin kasvu aikaansaa lähtösignaalin kasvamisen, kyseessä on positiivinen takaisinkytkentä.
\end{itemize}
}

  \frame{
  \tikzstyle{decision} = [diamond, draw, fill=blue!20, text width=4.5em, text badly centered, node distance=3cm, inner sep=0pt]
\tikzstyle{block} = [rectangle, draw, fill=blue!20, text width=5em, text centered, rounded corners, minimum height=4em]
\tikzstyle{line} = [draw, -latex]
\tikzstyle{cloud} = [draw,circle ,fill=red!20, node distance=3cm,  minimum height=2em]

 \begin{center}
\begin{tikzpicture}[node distance = 3cm, auto,] 
  \node [block] (aa) {$A$};
  \node[block, below of = aa] (bee) {$B$};
  \node[cloud, left of = aa, node distance=2cm] (summa) {$\sum$};
\node [coordinate, name=input, left of = summa, node distance=1cm] {};
\node [coordinate, name=output, right of = aa] {};
\draw [line] (aa) -- node {$\Uout$} (output) ;
\draw [line] (bee) -| node[pos=0.99] {$-$} node [near end] {} (summa);
\draw [line] (input) -- node {$\Uin$} (summa) ;
\draw [line] (summa) -- node {} (aa);
\draw [line] (output) |- node {} (bee);

\end{tikzpicture}
\[
\Uout = A(\Uin-B\Uout)\quad \Rightarrow \quad \Uout=\frac{A}{1+AB}\Uin
\]
\end{center}
 } 

\frame{
\frametitle{Takaisinkytkennän käyttö}
\begin{itemize}
\item Takaisinkytkentä on tärkeä säätötekniikassa.
\item Takaisinkytkennän avulla voidaan vahvistimen ominaisuuksia stabiloida.
\item Esimerkki: ei-invertoiva operaatiovahvistin.
\end{itemize}
}

\frame{
\frametitle{Ei-invertoiva vahvistin} 
\begin{center}
\begin{picture}(100,100)(0,-100)
\voi{0,0}{}{15}
\hln{-50,20}{50}

\hln{0,-40}{50}
\vln{0,-40}{40}
\vz{50,-40}{R_2\hspace{-1.1cm}}
\vz{50,-90}{R_1}
\hg{50,-90}

\hgp{70,-40}
\du{70,-40}{\Uout}

\hgp{-50,-30}
\du{-50,-30}{\Uin}
\hln{50,10}{20}

\end{picture}
\end{center}
\[
\Uout=\left(1+\frac{R_2}{R_1}\right)\Uin
\]
}

\frame{
\frametitle{Ei-invertoiva vahvistin: takaisinkytkennän analyysi}
\begin{center}
\begin{picture}(100,100)(0,-90)
\voi{0,0}{}{15}
\hln{-50,20}{50}

\hln{0,-40}{50}
\vln{0,-40}{40}
\vz{50,-40}{R_2\hspace{-1.1cm}}
\du{0,-90}{U_-}
\vz{50,-90}{R_1\hspace{-1.1cm}}
\hg{50,-90}

\hgp{70,-40}
\du{70,-40}{\Uout}

\hgp{-50,-30}
\du{-50,-30}{\Uin}
\hln{50,10}{20}

\end{picture}
\end{center}
\[
\Uout=A(U_+-U_-)=A(\Uin-\frac{R_1}{R_1+R_2}\Uout)
\]
Kun verrataan tätä takaisinkytkennän kaaviokuvaan, nähdään, että takaisinkytkentäkerroin $B=\frac{R_1}{R_1+R_2}$.
}