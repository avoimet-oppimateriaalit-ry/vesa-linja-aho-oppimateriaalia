\frame {
\frametitle{Kuinka suunnitella tuote, jolla tehdään rahaa?}
\begin{itemize}
\item Moni tuote voi "olla kiva", mutta kuinka saada asiakas maksamaan siitä niin, että sen valmistaminen on kannattavaa?
\item Miksi asiakas valitsee juuri tämän tuotteen?
\end{itemize}
}

\frame {
\frametitle{Esimerkkejä tuotteista}
\begin{itemize}
\item iPhone
\item Ruuvimeisseli
\item Kirjanpitopalvelu
\item Baby Safe Feeder
\item Matkakortinlukulaite
\item Vedenkeitin
\item Videopuhelin
\item Siivouspalvelu
\item Lastenhoito
\item Koulutuspalvelu
\item Power Balance -ranneke
\item Katsastuspalvelu
\end{itemize}
}



\frame {
\frametitle{Onnistunut tuotekehityksen tulos}
\begin{itemize}
\item Tuotteen laatu
\item Tuotteen valmistuskustannukset
\item Kehitysaika
\item Tuotekehityskulut
\item Tuotekehityskyvykkyys
\end{itemize}
}


\frame {
\frametitle{Ketkä osallistuvat?}
\begin{itemize}
\item Markkinointi
\item Suunnittelu
\item Valmistus
\end{itemize}
}


\frame {
\frametitle{Mikä {\em prosessi}?}
\begin{quote}
“Meillä on tässä yrityksessä vain yksi prosessi. Se on myyntiprosessi.

Jos tästä talosta sattuu löytymään joitain muita prosesseja, ne ovat sitten myynnin tukiprosesseja.”
\end{quote}
-- Cisco Systemsin toimitusjohtaja John Chambers

\begin{quote}
Accounting is a department. Marketing isn't.
\end{quote}
-- Kirjasta Rework


}


\frame {
\frametitle{Paljonko kuluu aikaa ideasta valmiiksi tuotteeksi?}
\begin{itemize}
\item Stanley Jobmaster -ruuvimeisseli
\item Rollerblade-rullaluistimet
\item HP DeskJet -tulostin
\item VW New Beetle -auto
\item Boeing 777 -lentokone
\end{itemize}
}


\frame {
\frametitle{Tuotekehitysprosessin pääkulku}
\begin{itemize}
\item Suunnittelu
\item Konseptisuunnittelu
\item Järjestelmätason suunnittelu
\item Yksityiskohtien suunnittelu
\item Testaus ja parantelu
\item Tuotannon aloittaminen
\end{itemize}
}


\frame {
\frametitle{Suunnittelu}
\begin{itemize}
\item Markkinatavoitteet
\item Reunaehdot
\item Rooli koko yrityksen strategiassa
\end{itemize}
}


\frame {
\frametitle{Konseptisuunnittelu}
\begin{itemize}
\item Markkinatarpeiden kartoittaminen
\item Eri tuotevaihtoehtojen ideointi ja vertailu
\item Tuotteen muoto, ominaisuudet ja toiminta.
\item Vertailu kilpaileviin tuotteisiin.
\end{itemize}
}

\frame {
\frametitle{Järjestelmätason suunnittelu}
\begin{itemize}
\item Tuotearkkitehtuuri
\item Alijärjestelmät ja komponentit
\item Kokoonpanojärjestelmän vaatimukset
\end{itemize}
}

\frame {
\frametitle{Yksityiskohtien suunnittelu}
\begin{itemize}
\item Yksityiskohtainen muotoilu
\item Käytettävät materiaalit
\item Valmistustoleranssit
\item Yksityiskohtainen dokumentointi
\item Ostettavien osien tekniset tiedot
\item Tuotantokustannukset
\item Luotettavuusanalyysi
\end{itemize}
}

\frame {
\frametitle{Testaus ja parantelu}
\begin{itemize}
\item Alfa-prototyypit (yleensä lähimpiä vastaavia valmisosia käyttäen)
\item Beta-prototyypit (sama valmistusprosessi kuin lopullisessa tuotteessa)
\item Tavoitteena luotettavuuden ja käytettävyyden arviointi
\end{itemize}
}

\frame {
\frametitle{Tuotannon aloittaminen}
\begin{itemize}
\item Valmistusprosessi käynnistetään varovasti
\item Korjataan valmistusprosessin viat ja koulutetaan työntekijöitä
\item Lopulta tuote päästetään markkinoille ({\em launch})
\end{itemize}
}


\frame {
\frametitle{Tuotesuunnittelu}
Neljä eri päätyyppiä:
\begin{itemize}
\item Uudet tuotealustat
\item Olemassaolevien tuotealustojen johdannaiset
\item Olemassaolevan tuotteen jatkuva parantaminen
\item Kokonaan uudet tuotteet
\end{itemize}
}

\frame {
\frametitle{Tuotesuunnitteluprosessi}
\begin{itemize}
\item Tunnista mahdollisuudet
\item Arvioi ja priorisoi
\item Resurssien allokointi ja aikataulutus
\item Vedä tulokset yhteen
\item Arvioi tuotoksia ja prosessia
\end{itemize}
}

\frame {
\frametitle{Tunnista mahdollisuudet}
\begin{itemize}
\item Henkilökunta, nykyiset ja potentiaaliset asiakkaat, asiakaspalaute \ldots
\end{itemize}
}

\frame {
\frametitle{Arvioi ja priorisoi}
\begin{itemize}
\item Kilpailustrategia (hinnalla, laadulla, palvelulla)
\item Markkinasegmentointi
\item S-käyrä ja tekniikan kehittyminen
\end{itemize}
}

\frame {
\frametitle{Resurssien allokointi ja aikataulutus}
\begin{itemize}
\item Paljonko tarvitaan henkilöstöä?
\item Milloin tuote on valmis markkinoille?
\end{itemize}
}

\frame {
\frametitle{Vedä tulokset yhteen}
\begin{itemize}
\item Yhden rivin kuvaus tuotteesta
\item Liiketavoitteet (markkinaosuus ym.)
\item Kohdemarkkinat
\item Reunaehdot
\end{itemize}
}

%\frame {
%\frametitle{Arvioi tuotoksia ja prosessia}
%\begin{itemize}
%\item 
%\end{itemize}
%}

\frame {
\frametitle{Mielenkiintoista luettavaa}
\url{http://gurumarkkinointi.fi/}
}



\frame {
\frametitle{Asiakastarpeiden tunnistaminen}
Esimerkkejä asiakastarpeista
\begin{itemize}
\item Taulun kiinnittäminen seinälle
\item Paikasta toiseen siirtyminen
\item Yhteydenpito kavereihin
\item Sosiaalisen statuksen kohottaminen
\item Vieraskielisen tekstin lukeminen
\item \ldots
\end{itemize}
}

\frame {
\frametitle{Asiakastarpeiden kartoitusprosessi}
\begin{itemize}
\item Raakadatan kerääminen
\item Raakadatan tulkinta asiakastarpeiksi
\item Tarpeiden järjesteleminen ensisijaisiin ja toissijaisiin
\item Tarpeiden suhteellisen tärkeyden todentaminen
\item Tulosten yhteenveto
\end{itemize}
}

\frame {
\frametitle{Raakadatan kerääminen}
\begin{itemize}
\item Haastattelut
\item Kohderyhmähaastattelut
\item Tuotteen käytön havainnointi
\end{itemize}
}

\frame {
\frametitle{Lead users}
\begin{itemize}
\item Lead users, {\em edelläkävijät}: henkilöt, jotka kokevat asiakastarpeen ennen suurta yleisöä.
\end{itemize}
}

\frame {
\frametitle{Haastattelu}
\begin{itemize}
\item Käytä virikkeitä (kuvat, laitteet)
\item Anna haastateltavan demonstroida 
\item Tarkkaile sanatonta viestintää
\item Käytä apuvälineitä (nauhoitus, videokuvaus, muistiinpanot)
\end{itemize}
}

\frame {
\frametitle{Raakadatan tulkinta asiakastarpeiksi}
\begin{itemize}
\item Keskity siihen, mitä tuotteen pitää tehdä, ei siihen, miten se tehdään.
\end{itemize}
}

\frame {
\frametitle{Tarpeiden järjesteleminen ensisijaisiin ja toissijaisiin}
\begin{itemize}
\item Laaja haastatteluprosessi tuottaa kymmeniä tai satoja tarpeita.
\item Poista päällekkäiset tarpeet.
\item Ryhmittele tarpeet.
\end{itemize}
}

\frame {
\frametitle{Tarpeiden suhteellisen tärkeyden todentaminen}
\begin{itemize}
\item Kun tarpeet on ryhmitelty, voidaan tärkeyttä selvittää esimerkiksi kyselytutkimuksella.
\end{itemize}
}

\frame {
\frametitle{Tulosten yhteenveto}
\begin{itemize}
\item Mitä uutta selvisi priosessin aikana?
\item Ketkä haastatelluista voivat olla avuksi myöhemmissä vaiheissa?
\item Haastattelimmeko riittävän monenlaisia asiakkaita?
\end{itemize}
}



\frame {
\frametitle{Tuotemäärittelyt}
\begin{itemize}
\item Määrittelyt = spesifikaatiot = "speksit".
\end{itemize}
}

\frame {
\frametitle{Mihin speksejä tarvitaan}
\begin{itemize}
\item Asiakastarpeet ilmaistaan yleensä asiakkaan kielellä.
\item Määrittelyt ovat {\em tarkkoja kuvauksia} siitä, mitä tuotteen tulee toimia. 
\item Vrt. {leikkimökki on helppo koota} ja {\em leikkimökin kokoaminen kestää keskimäärin 1,5 tuntia}.
\item Myös termiä {\em tuotevaatimukset} käytetään.
\item Spesifikaatioiden täytyy olla mitattavissa. Spesifikaatiolla on aina yksikkö ja arvo.
\item Tuotteen suunnittelu ja toteuttaminen ei ole käytännössä mahdollista ilman speksejä. Sama koskee laadunvalvontaa.
\end{itemize}
}

\frame {
\frametitle{Pohdittavaa}
Mieti spesifikaatioita seuraaville tuoteideoille.
\begin{itemize}
\item Kuulakärkikynä, jolla kirjoittaminen on sulavaa ja miellyttävää.
\item Kattomateriaali, joka on kestävää (mittayksikkö ja testi).
\end{itemize}
Jos kannettavan tietokoneen yksi asiakastarve on, että "se näyttää tyylikkäältä", miten mittaisit sitä?
}

\frame {
\frametitle{Tavoitespesifikaatiot vs. lopulliset spesifikaatiot}
\begin{itemize}
\item Asiakastarpeiden kartoituksen jälkeen asetetaan tavoitespesifikaatiot.
\item Kun tuotekonsepteja kehitetään, vertaillaan ja karsitaan, esiin nousee aina uusia reunaehtoja, jotka voivat olla teknisiä tai taloudellisia.
\item Konseptin valinnan ja testauksen jälkeen asetetaan lopulliset spesifikaatiot.
\end{itemize}
}

\frame {
\frametitle{Tavoitespesifikaatiot}
\begin{itemize}
\item Valitse mittarit.
\item Kerää informaatiota kilpailevista tuotteista.
\item Aseta ideaaliset vaatimukset ja minimivaatimukset.
\end{itemize}
}


\frame {
\frametitle{Lopulliset spesifikaatiot}
\begin{itemize}
\item Kehitä tekninen malli tuotteelle.
\item Selvitä mallin kustannusrakenne.
\item Jalosta speksejä, tee tarvittaessa kompromisseja.
\item Madalla vaatimuksia, jos on pakko.
\end{itemize}
}


\frame {
\frametitle{Konseptin valinta}
Valintaan käytetään aina jotain menetelmää, vaikkei näin olisi erikseen päätetykään.
\begin{itemize}
\item Ulkoinen päätös (asiakas valitsee).
\item Tuotemestari valitsee (tiimin vaikutusvaltaisin henkilö tekee valinnan omiin mieltymyksiinsä perustuen).
\item Intuitio (valitaan se, joka tuntuu parhaalta).
\item Äänestäminen
\item Hyvien ja huonojen puolien listaaminen
\item Prototyypin rakentaminen ja testaaminen
\item Päätösmatriisi
\end{itemize}
}

\frame {
\frametitle{Jäsennelty tapa tehdä valinta}
\begin{itemize}
\item Tee valintamatriisi (konseptit sarakkaisiin, ominaisuudet riveille).
\item Pisteytä eri konseptien ominaisuudet (0, + tai -).
\item Laita konseptit paremmuusjärjestykseen kokonaispisteiden perusteella.
\item Parantele ja yhdistele konsepteja.
\item Valitse yksi tai useampi konsepti jatkoon.
\end{itemize}
}

\frame {
\frametitle{Seuraava vaihe}
Uusitaan edellisen kalvon prosessi, mutta käytetään hienompaa asteikkoa (esimerkiksi viisiportaista). Lisäksi eri ominaisuuksien tärkeyttä painotetaan.

}

\frame {
\frametitle{Huomioitavaa}
\begin{itemize}
\item Ominaisuudet vaikuttavat usein toisiinsa.
\item Ulkonäköön ja käyttömukavuuteen vaikuttavat kriteerit ovat subjektiivisia.
\end{itemize}
}

\frame {
\frametitle{Arkinen sovellus}
Jäsenneltyä valintatapaa voi soveltaa myös arkielämässä valmiiden tuotteiden valintaan.
}

\frame {
\frametitle{Esimerkki: akkutekniikan valinta kannettavaan tietokoneeseen}
\begin{itemize}
\item Mitä eri valintakriteerejä?
\end{itemize}
}


\frame {
\frametitle{Immateriaalioikeudet}
\begin{itemize}
\item Patentti
\item Hyödyllisyysmalli
\item Mallisuoja
\item Tavaramerkki
\item Liikesalaisuus
\item Tekijänoikeus

\end{itemize}
}


\frame {
\frametitle{Patentti}
\begin{itemize}
\item = yhteiskunnan keksijälle myöntämä yksinoikeus keksinnön ammattimaiseen hyödyntämiseen.
\item Kesto enintään 20 vuotta (lääke- ja kasvinsuojeluaineet voivat saada 5 vuoden lisäsuojan).
\item Patentit ovat julkisia. Kantava idea on, että julkistamalla keksinnön, keksijä palkitaan määräaikaisilla yksinoikeuksilla.
\item Suomessa patentteja hallinnoi Patentti- ja rekisterihallitus (PRH).
\end{itemize}
}

\frame {
\frametitle{Mitä saa patentoida?}
\begin{itemize}
\item Keksinnön on oltava uusi ja aiemmin julkistamaton.
\item Jotta keksintö olisi patentoitavissa, sen pitää ylittää keksinnöllisyyskynnys. Mikäli keksintö olisi ilmeinen kenelle tahansa keskinkertaiselle ammattilaiselle, patenttia ei voida myöntää.
\end{itemize}
Patenttilaki: \url{http://www.finlex.fi/fi/laki/ajantasa/1967/19670550}
}

\frame {
\frametitle{Patentti on aluekohtainen}
\begin{itemize}
\item Patentit ovat aluekohtaisia. Esimerkiksi yhdysvaltalainen patentti ei estä hyödyntämästä tuotetta Suomessa ja päinvastoin.
\item Todella "mullistava"\ keksintö kannattaakin patentoida useissa maissa.
\end{itemize}
}

\frame {
\frametitle{Mitä patentointi maksaa?}
\begin{itemize}
\item Patenttihakemuksen käsittely maksaa useita satoja euroja (käytännössä vähintään 450+450=900 euroa).
\item Hakemuksen tekeminen ei ole yhtä helppoa kuin yrityksen perustaminen tai opintotuen hakeminen: usein on mielekästä käyttää patenttiasiamiestä. Tällöin kulut voivat olla tuhansia tai kymmeniä tuhansia euroja.
\item Kalliiksi tulee myös ulkomaisten patenttien hakeminen. Maailmanlaajuista patenttia ei ole olemassa.
\item EPC ja PCT. \url{http://www.prh.fi/fi/patentit/hakuulkom.html}
\end{itemize}
}

\frame {
\frametitle{Patentin voimassaolo}
\begin{itemize}
\item Patentti pysyy voimassa maksamalla vuosittaiset ylläpitomaksut.
\item Vuosimaksu on sitä suurempi, mitä kauemmin patentti on ollut voimassa.
\item \url{http://www.prh.fi/fi/patentit/hinnastot/vuosmaks.html}
\item Taustalla on tarkoitus, että jos patentti ei tuota haltijalleen rahaa, hakemusta ei kannata uudistaa ja muut pääsevät hyödyntämään keksintöä.
\end{itemize}
}

\frame {
\frametitle{Patentin mitätöiminen}
\begin{itemize}
\item Jos voidaan osoittaa, että patentoitu keksintö ei olekaan ollut uusi, voidaan patentti mitätöidä.
\item Esimerkki: Molok-jäteastia. \url{http://www.taloussanomat.fi/arkisto/2004/03/19/jatelaitevalmistaja-molok-menetti-patenttinsa/200432898/12}
\end{itemize}
}

\frame {
\frametitle{Hyödyllisyysmalli}
\begin{itemize}
\item Ikään kuin patentin light-versio :-).
\item Keksinnöltä ei vaadita yhtä suurta keksinnöllisyyttä kuin patentilta, ja yksinoikeuden voi saada 10 vuodeksi.
\end{itemize}
Laki hyödyllisyysmallioikeudesta: \url{http://www.finlex.fi/fi/laki/ajantasa/1991/19910800}
}

\frame {
\frametitle{Mallisuoja}
\begin{itemize}
\item Mallisuoja koskee tuotteen muotoilua, ei teknisiä ratkaisuja.
\item Mallin pitää olla uusi ja yksilöllinen.

\end{itemize}
Mallioikeuslaki: \url{http://www.finlex.fi/fi/laki/ajantasa/1971/19710221}
}

\frame {
\frametitle{Tavaramerkki}
\begin{itemize}
\item Tunnusmerkki, joka erottaa yrityksen valmistamat tai tuottamat tavarat ja palvelut muiden yritysten vastaavista.
\item Eskimo, Lännen, Saarioisten, Hugo Boss, Lexus, \ldots
\item Oikeuden tavaramerkkiin voi saada joko rekisteröimällä tai vakiinnuttamalla.
\item Tavaramerkin tulee olla erottuva: esimerkiksi omenoille ei voi rekisteröidä Apple-tavaramerkkiä, tietokoneille voi.
\end{itemize}
Tavaramerkkilaki: \url{http://www.finlex.fi/fi/laki/ajantasa/1964/19640007}
}

\frame {
\frametitle{Tekijänoikeus}
{\em Sillä, joka on luonut kirjallisen tai taiteellisen teoksen, on tekijänoikeus teokseen, olkoonpa se kaunokirjallinen tahi selittävä kirjallinen tai suullinen esitys, sävellys- tai näyttämöteos, elokuvateos, valokuvateos tai muu kuvataiteen teos, rakennustaiteen, taidekäsityön tai taideteollisuuden tuote taikka ilmetköönpä se muulla tavalla.}

Tekijänoikeuslaki: \url{http://www.finlex.fi/fi/laki/ajantasa/1961/19610404}
}

\frame {
\frametitle{Mikä on teos?}
\begin{itemize}
\item Jotta teos saisi tekijänoikeussuojan, pitää teoskynnyksen ylittyä. Teokselta edellytetään riittävää omaperäisyyttä.
\item Teoskynnys vaihtelee aloittain. Esimerkiksi mustavalkopiirrokset yleisesti tunnetuista teknisistä laitteista eivät yleensä ylitä teoskynnystä. Sen sijaan laulun säe tai lyhyt runo saa yleensä suojaa teoksena.
\item Tekijänoikeus on voimassa automaattisesti, teosta ei siis tarvitse rekisteröidä mihinkään.
\item Tekijänoikeussuoja on voimassa hyvin pitkään. Esimerkiksi kirjallisen teoksen suoja-aika päättyy vasta, kun on kulunut 70 vuotta tekijän kuolemasta.
\end{itemize}
}

\frame {
\frametitle{Esimerkkejä}
Poimintoja Tekijänoikeusneuvoston (TN) lausunnoista. 
\begin{itemize}
\item Sana "sauvakävely"\ ei ole teos. (TN 2010:1)
\item Sen sijaan ilmaisut "heikun keikun"\ ja "vinksin vonksin"\  nauttivat tekijänoikeussuojaa (TN 2010:2)
\item Yksi kirjan lause ei saanut tekijänoikeussuojaa (TN 2003:14).
\end{itemize}
}

\frame {
\frametitle{Iskulauserekisteri}
\begin{itemize}
\item Ei viranomaisten ylläpitämä. Rekisteriä ylläpitää Suomen markkinointiliitto.
\item \url{http://www.iskulauserekisteri.fi/}
\end{itemize}
}


\frame {
\frametitle{Tuotearkkitehtuuri}
\begin{itemize}
\item Tuotteen rakenne on yleensä organisoitu niin, että tuote koostuu muutamista suurista lohkoista.
\item Jokainen lohko koostuu pienemmistä komponenteista.
\item Tapaa, jolla tuote on jaettu lohkoihin ja jolla lohkot vuorovaikuttavat toistensa kanssa, kutsutaan {\em tuotearkkitehtuuriksi}.
\end{itemize}
}

\frame {
\frametitle{Intergroitu vai modulaarinen?}
\begin{itemize}
\item Integroitu arkkitehtuuri = yksi lohko tekee paljon asioita.
\item Modulaarinen arkkitehtuuri = lohko tekee vain yhden tai muutaman asian ja lohkojen välinen vuorovaikutus on tarkkaan määritelty.
\end{itemize}
}

\frame {
\frametitle{Modulaarisen arkkitehtuurin pääedut}
\begin{itemize}
\item Yhden osan muokkaaminen ei vaadi koko tuotteen toiminnan uudelleenmiettimistä.
\item Suunnittelutyö on helppo jakaa ja tarvittaessa ulkoistaa.
\end{itemize}
}

\frame {
\frametitle{Modulaarisen arkkitehtuurin tyyppejä}
\begin{itemize}
\item Slot-modular (kolomodulaarinen\footnote{Suomennokset Vesan omia.})
\item Bus-modular (väylämodulaarinen)
\item Section-modular (palamodulaarinen)
\end{itemize}
}

\frame {
\frametitle{Lisää modulaarisen arkkitehtuurin etuja}
\begin{itemize}
\item Päivitettävyys
\item Lisäosat
\item Mukauttaminen
\item Kuluminen (wear \& consumption)
\item Joustavuus
\item Uudelleenkäyttö
\end{itemize}
Tuotteen muunneltavuus markkinatilanteen mukaan on tärkeää.
}

\frame {
\frametitle{Tuotearkkitehtuurin laatiminen}
\begin{itemize}
\item Laadi lohkokaavio.
\item Kokoa lohkokaavion osat sopiviksi ryppäiksi.
\item Laadi karkea piirroskuva lohkojen fyysisestä sijainnista.
\item Tunnista lohkojen vuorovaikutussuhteet.
\end{itemize}
}


\frame {
\frametitle{Teollinen muotoilu}
\begin{itemize}
\item Muotoilulla on tärkeä rooli tuotteen valmistettavuudessa ja markkinoitavuudessa.
\end{itemize}
}

\frame {
\frametitle{Onnistuneen muotoilun ominaisuuksia}
\begin{itemize}
\item Hyödyllisyys: tuote on turvallinen ja helppo käyttää. Tuotteen ulkonäkö kertoo, mitä sillä on tarkoitus tehdä.
\item Ulkonäkö: tuotteen muoto ja väri muodostavat miellyttävän kokonaisuuden.
\item Huollon helppous: tuotteesta näkee, kuinka sitä on tarkoitus huoltaa.
\item Matalat valmistuskustannukset: muotoilulla on suuri vaikutus valmistuskustannuksiin.
\item Viestintä: tuote viestii yrityksen suunnittelufilosofiasta.
\end{itemize}
}

\frame {
\frametitle{Tavoitteet}
\begin{itemize}
\item Ergonomia
\item Estetiikka
\end{itemize}
}

\frame {
\frametitle{Ergonomiset tarpeet}
\begin{itemize}
\item Helppokäyttöisyys
\item Helppohuoltoisuus
\item Montako käyttäjäinteraktiota tarvitaan tuotteen käyttämiseen
\item Joutuuko käyttäjä opettelemaan uudenlaisia vuorovaikutustapoja
\item Turvallisuusnäkökohdat
\end{itemize}
}

\frame {
\frametitle{Esteettiset tarpeet}
\begin{itemize}
\item Tuleeko visuaalisen ilmeen erottaa tämä tuote muista vastaavista?
\item Kuinka tärkeitä ovat muotoilun luoma imago, omistusylpeys, muoti?
\item Motivoiko tuotteen ulkonäkö kehitystiimiä?
\end{itemize}
}

\frame {
\frametitle{Muotoiluprosessi}
\begin{itemize}
\item Asiakastarpeiden tutkinta
\item Tarpeiden tulkinta konsepteiksi
\item Konseptien jalostaminen
\item Jatkojalostaminen ja lopullinen konseptinvalinta
\item Mallipiirustusten tekeminen
\item Prosessin koordinoiminen tuotevalmistuksen, teknisen suunnittelun ja alihankkijoiden kanssa.
\end{itemize}
}

\frame {
\frametitle{Muotoilun laadun arviointi}
\begin{itemize}
\item Käyttäjärajapinnan laatu
\item Tuotteen vetovoima
\item Huollettavuus ja korjattavuus
\item Järkevä resurssien käyttö
\item Erottuuko tuote muista tuotteista?
\end{itemize}
}

\frame {
\frametitle{Kirjallisuutta}
Jos tuotekehitys kiinnostaa enemmän, tutustu esimerkiksi seuraaviin kirjoihin:
\begin{itemize}
\item Ulrich, Eppinger: Product design and development. 3rd edition. 2004. (Alan perusteos.)
\item Henry Petroski (suom. K. Pietiläinen): Ideasta tuotteeksi -- miten insinöörit keksivät suunnitelmallisesti. 1997.
\item Dariush Rafinejad: Innovation, product development and commercialization. 2007.
\end{itemize}
Kirjat ovat saatavilla esimerkiksi Metropolian kirjastosta.
}
