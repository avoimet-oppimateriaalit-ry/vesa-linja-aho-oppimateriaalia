
\frame{
\frametitle{Sarja- ja rinnakkaisresonanssi}
\begin{itemize}
\item Sarjaresonanssissa kelan ja kondensaattorin sarjaankytkennän impedanssi on nolla.
\item Rinnakkaisresonanssissa kelan ja kondensaattorin rinnankytkennän impedanssi on ääretön.
\end{itemize}
Yleisemmin: resonanssitaajuudella piirin impedanssin imaginaariosa on nolla (sarjaresonanssi) tai
admittanssin imaginaariosa on nolla (rinnakkaisresonanssi).
Jos piirissä on kela ja kondensaattori sarjassa tai rinnan, resonanssikulmataajuus $\omega_0=\frac{1}{\sqrt{LC}}$.
}

\frame{
\frametitle{Suodatin}
Suodatin on elektroninen piiri, jonka avulla signaalia muokataan halutunlaiseksi.
\begin{itemize}
\item Alipäästösuodatin: vaimentaa korkeita taajuuksia, päästää läpi matalat taajuudet.
\item Ylipäästösuodatin: vaimentaa matalia taajuuksia, päästää läpi korkeat taajuudet.
\item Kaistanestosuodatin: päästää läpi korkeat ja matalat taajuudet, mutta vaimentaa tietyllä välillä olevia taajuuksia.
\item Kaistanpäästösuodatin: vaimentaa liian matalia ja liian korkeita taajuuksia, mutta päästää läpi tietyllä välillä olevat taajuudet.
\end{itemize}
Suodattimen asteluku kertoo suodattimen siirtofunktion (yleensä: jännitevahvistus) nimittäjäpolynomin asteluvun.
}

\frame{
\frametitle{Alipäästösuodatin}
Alipäästösuodatin voidaan toteuttaa yksinkertaisesti vastuksen ja kondensaattorin sarjaankytkennällä.
Ensimmäisen asteen siirtofunktion perusmuoto on alipäästösuodattimella
\[
\frac{\Uout}{\Uin}=\frac{1}{\frac{\jj \omega}{\omega_0}+1}
\]
ja ylipäästösuodattimella
\[
\frac{\Uout}{\Uin}=\frac{\frac{\jj \omega}{\omega_0}}{\frac{\jj \omega}{\omega_0}+1}
\]
Ensimmäisen asteen ali- tai ylipäästösuodattimen jännitevahvistus on ominaiskulmataajuudella $\frac{1}{\sqrt{2}}$ eli
tehovahvistus on 0,5.
}



\frame{
\frametitle{Esimerkki}

\begin{center}
\begin{picture}(100,50)(0,0)
\vl{100,0}{L}
\vst{0,0}{\Uin}
\hln{0,0}{150}
%\hln{50,50}{50}
\hln{50,50}{100}
%\hl{0,50}{L}
\hz{0,50}{R}
\vc{150,0}{C}
%\du{115,0}{U}
%\ri{75,50}{I}
%\color{blue}
%\vc{50,0}{C}
\dcru{155,00}{\Uout}
\end{picture}
\end{center}
Laske piirin jännitevahvistus $\frac{\Uout}{\Uin}$. Hahmottele jännitevahvistuksen amplitudivaste (vaaka-akselille $\omega$, pystyakselille $|\frac{\Uout}{\Uin}|$). Onko piiri alipäästö-, ylipäästö-, kaistanpäästö- vai kaistanestosuodatin?

\[
L=1\, {\rm H}\quad C = 1\, {\rm F}\quad R=1\ohm
\]

{\tiny Oikeita lopputuloksia: $\frac{\Uout}{\Uin}=\frac{1}{\jj(\omega-\frac{1}{\omega})+1}$ $\left| \frac{\Uout}{\Uin}\right|=\frac{1}{\sqrt{1+(\omega-\frac{1}{\omega})^2}}$}

}


\frame{
\frametitle{Esimerkki}

\begin{center}
\begin{picture}(100,50)(0,0)
\vl{100,0}{L}
\vst{0,0}{\Uin}
\hln{0,0}{150}
%\hln{50,50}{50}
\hln{50,50}{100}
%\hl{0,50}{L}
\hz{0,50}{R}
\vc{150,0}{C}
%\du{115,0}{U}
%\ri{75,50}{I}
%\color{blue}
%\vc{50,0}{C}
\dcru{155,00}{\Uout}
\end{picture}
\end{center}
Laske piirin jännitevahvistus $\frac{\Uout}{\Uin}$. Hahmottele jännitevahvistuksen amplitudivaste (vaaka-akselille $\omega$, pystyakselille $|\frac{\Uout}{\Uin}|$). Onko piiri alipäästö-, ylipäästö-, kaistanpäästö- vai kaistanestosuodatin?
\[
L=1\, {\rm H}\quad C = 1\, {\rm F}\quad R=1\ohm
\]
Lasketaan kelan ja kondensaattorin rinnankytkennän impedanssi ("tulo jaettuna summalla"\ -kaavalla) ja
sijoitetaan lukuarvot:
\[
Z_{\rm LC}=\frac{\jj\omega L \frac{1}{\jj\omega C}}{ \jj\omega L +\frac{1}{\jj\omega C} }=\frac{\frac{L}{C}}{ \jj\omega L -\jj\frac{1}{\omega C} }=\frac{1}{\jj \omega-\jj\frac{1}{\omega}}
\]
}

\frame{
Nyt $\Uout$ ratkeaa jännitteenjakosäännöllä:
\[
\Uout=\Uin\frac{Z_{\rm LC}}{R+Z_{\rm LC}}\Rightarrow \frac{\Uout}{\Uin}=\frac{1}{\frac{R}{Z_{\rm LC}}+1}
\]
Sijoitetaan $R=1$ sekä edellisellä kalvolla laskettu $Z_{\rm LC}$:
\[
\frac{\Uout}{\Uin}=\frac{1}{\jj(\omega-\frac{1}{\omega})+1}
\]
Lasketaan itseisarvo (amplitudivaste):
\[
\left| \frac{\Uout}{\Uin}\right|=\frac{1}{\sqrt{1+(\omega-\frac{1}{\omega})^2}}
\]
Amplitudivasteen kulkua voi tarkastella matemaattisesti, mutta sitä ei vaadita tehtävässä. Käyrällä on maksimi
kohdassa $\omega=1$. Raja-arvot nollassa sekä äärettömyydessä ovat nollia. Kyseessä on siis kaistanpäästösuodatin.

}

