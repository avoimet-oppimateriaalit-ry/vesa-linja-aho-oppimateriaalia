\frame{
\frametitle{Komponenttien jäähdytys}
\begin{itemize}
\item Lämmöntuotanto on ongelma monessa elektronisessa laitteessa (audiovahvistimet, tietokoneet, sähkömoottorin ohjauselektroniikka).
\item Nyrkkisääntö: TO-3-kotelo kestää jäähdyttämättömänä noin 3 W ja TO-220-kotelo noin 2 W tehon.
\item Suurtehosovelluksissa käytetään pakotettua ilmajäähdytystä (=tuuletinta) tai nestejäähdytystä.
\item Muutamien kymmenien wattien lämpöteho voidaan käsitellä passiivisella ilmajäähdytyksellä, mikäli laitteen kotelo pääsee tuulettumaan hyvin.
\item Lämmön johtumista pois komponentista ympäristön ilmaan mallinnetaan {\bf lämpöresistanssilla}.
\end{itemize}
}

\frame{
\frametitle{Lämpöresistanssi}
\begin{itemize}
\item Lämmön johtumista pois komponentista voidaan mallintaa melko hyvin lineaarisella mallilla, jossa komponentin ja jäähdytyselementin muodostaman lämmönsiirtopiirin lämmönjohtavuutta (tai oikeastaan sen käänteislukua) mallinnetaan lämpöresistanssilla.
\item Yhden watin tehonlisäys komponentissa kasvattaa komponentin ja ympäristön lämpötilaeroa koko lämmönsiirtopiirin lämpöresistanssin osoittamalla määrällä
\[
\Delta T=\theta \cdot P
\]
\item $\Delta T$=lämmönlähteen (=puolijohteen) ja ympäristön (=ilman) lämpötilaero.
\item $\theta$=Lämmönsiirtopiirin lämpöresistanssi, joka koostuu komponentin kotelon, kotelon ja jäähdytyselementin välisen liitoksen sekä jäähdytyselementin lämpöresistanssista.
\item $P=$ komponentin lämpenemisteho.
\end{itemize}
}

\frame{
\frametitle{Lämpöresistanssi: laskuesimerkki}
TO-3-kotelossa oleva puolijohdekomponentti kuumenee 15 watin teholla. Lämpöresistanssi puolijohteesta TO-3-koteloon on $\theta_{\rm JC}=2\ ^\circ \rm C /W$, komponentin ja jäähdytyslevyn liitoksen lämpöresistanssiksi oletetaan $\theta_{\rm CS}=1\ ^\circ \rm C /W$ ja jäähdytyselementin lämpöresistanssi on $\theta_{\rm SA}=5\ ^\circ \rm C /W$. Kuinka kuumaksi puolijohde lämpenee, jos ympäristön lämpötila pysyy 30 asteessa?
\[
\Delta T=\theta \cdot P = (\theta_{\rm JC}+\theta_{\rm CS}+\theta_{\rm SA}) \cdot P = 8\ ^\circ \rm C /W \cdot 15\,W=120\ ^\circ \rm C
\]
Puolijohteen lämpötila on täten
\[
T=T_{\rm A}+\Delta T=30 \ ^\circ \rm C + 120 \ ^\circ \rm C=150\ ^\circ \rm C
\]

\begin{alertblock}{Huomaa!}
Jos ympäristön lämpötila nousee (esimerkiksi elektroniikkalaitteen huonosti tuulettuvan kotelon takia) 40 asteella, nousee myös puolijohteen lämpötila 40 asteella!
\end{alertblock}
}
