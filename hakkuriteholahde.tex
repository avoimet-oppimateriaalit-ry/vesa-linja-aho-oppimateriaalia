



\frame{
\frametitle{Hakkuritekniikka}
\begin{itemize}
\item Tavallisten verkkomuuntajaan perustuvien teholähteiden ongelmia ovat rajallinen tehontuotto, suuri fyysinen koko sekä paino.
\item Lisäksi suurilla virroilla hurinajännitteen eliminointi vaatisi todella suuren suodatuskondensaattorin.
\item Tasasuuntaamalla verkkojännite sellaisenaan, ja syöttämällä se suurella taajuudella katkottuna muuntajaan, voidaan käyttää fyysisesti pienikokoista suurtaajuusmuuntajaa.
\item Tällaista verkkolaitetta kutsutaan hakkuriteholähteeksi tai hakkuriverkkolaitteeksi.
\item Hakkuritekniikka soveltuu myös tasajännitteen jännitetason laskemiseen sekä -- toisin kuin lineaarinen regulaattoritekniikka -- tasajännitteen jännitetason nostamiseen.
\item Elektroniikan tuotantomenetelmien kehittyminen ja tuotantomäärien kasvaminen on johtanut hintojen laskuun niin, että lähes kaikissa laitteissa on nykyään hakkuriteholähde.

\end{itemize}
}


\frame{
\frametitle{Hakkuriteholähde lyhyesti}
\begin{itemize}
\item Hakkuriteholähteessä jännitteen muuntaminen perustuu nopeaan elektroniseen kytkimeen (transistoriin).
\item Hakkuriteholähteeseen perustuvassa verkkolaitteessa verkkojännite tasasuunnataan, jonka jälkeen se syötetään nopean kytkimen välityksellä suurtaajuusmuuntajaan.
\item Nopean katkomisen ansiosta saadaan muuntajaan suuri magneettivuon muutosnopeus $\to$ pärjätään pienikokoisella muuntajalla.
\item Suuren taajuuden ansiosta myös tarvittavat suodatuskondensaattorit ovat pieniä. Koska jännitteen säätö tapahtuu on-off-kytkimellä, lämpöhäviöt ovat pieniä.
\item Pieni koko ja hyvä hyötysuhde ovat hakkuriteholähteen tärkein etu. Haittapuolia ovat monimutkaisuus ja nopean virran katkomisen tuottamat häiriöt.

\end{itemize}
}

\frame{

\frametitle{Buck- eli step-down-hakkuri}
Jännitettä laskeva hakkuri:

\begin{center}
\begin{picture}(100,100)(0,0)
\pnpb{0,0}{}
\hln{25,50}{25}
\hl{50,50}{}
\ud{50,0}{}
\vc{100,0}{}
\hln{50,0}{50}
\end{picture}
\end{center}

Toteutettavissa esimerkiksi piirillä LM2574.
}

\frame{
\frametitle{Boost- eli step-up-hakkuri}
Jännitettä nostava hakkuri:
\begin{center}
\begin{picture}(100,100)(0,0)
\hl{0,50}{L}
\rd{50,50}{}
\npnnc{5,25}{}
\vc{100,0}{}
\hln{0,0}{100}
\end{picture}
\end{center}

}

\frame{
\frametitle{Buck-boost-hakkuri}
Napaisuuden kääntävä hakkuri:
\begin{center}
\begin{picture}(100,100)(0,0)
\pnpb{0,0}{}
\hln{25,50}{25}
\ld{50,50}{}
\vl{50,0}{}
\vc{100,0}{}
\hln{50,0}{50}
\end{picture}
\end{center}

Toteutettavissa esimerkiksi piirillä LM2574.

}