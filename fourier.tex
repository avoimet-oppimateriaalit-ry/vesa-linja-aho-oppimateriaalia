\frame{
\frametitle{Laplace-muunnos ja Fourier-muunnos}
Laplace-muunnos:
\[
\lap (f(t))=\int_0^\infty f(t)e^{-st}{\rm d}t
\]
Fourier-muunnos:
\[
{\cal F} (f(t))=\int_{-\infty}^\infty f(t)e^{-j\omega t}{\rm d}t
\]
Kaavoissa ainoa ero on alempi integrointiraja, sekä muuttuja $s=\jj \omega$. Osoitinlaskenta perustuu teoreettisesti
differentiaaliyhtälöiden ratkaisuun Fourier-muunnoksen avulla. Tämän takia Laplace-muunnettujen ja Fourier-muunnettujen
komponenttien ainoa ero impedanssin kaavoissa on $s=\jj \omega$.
}

\frame{
\frametitle{Fourier-sarja jaksolliselle funktiolle}
Jos halutaan laskea, miten piiri käyttäytyy esimerkiksi jos sinne syötetään kolmioaaltoa, voidaan käyttää
Fourier-sarjaa. Fourierin teoreeman mukaan mikä tahansa jaksollinen signaali voidaan hajottaa siniaaltojen
summaksi. Tällöin muodostuvan jokaisen siniaallon taajuus on alkuperäisen signaalin monikerta.

Fourier-sarjan laskeminen ei ole varsinaisesti vaikeaa, mutta jätämme sen rajoitetun ajan vuoksi pois kurssilta. Laskeminen vaatii
hieman integrointia. Kiinnostuneet voivat tutustua aiheeseen kirjan kappaleesta 3.3.1 {\em Fourier-sarjan laskeminen}.
}

