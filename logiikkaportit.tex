\frame{
\frametitle{Ei-portti}
\begin{center}
\begin{picture}(100,50)(0,0)
\hln{-10,15}{10}
\hnot{0,0}
\end{picture}
\end{center}
Jos tulo on 0, lähtö on 1. Jos tulo on 1, lähtö on 0.
}

\frame{
\frametitle{Ja-portti}
\begin{center}
\begin{picture}(100,50)(0,0)
\hand{0,0}
\hln{-10,0}{10}
\hln{-10,20}{10}

\end{picture}
\end{center}
Lähtö on 1 jos ja vain jos kaikki tulot ovat 1, muuten lähtö on 0.
}


\frame{
\frametitle{Tai-portti}
\begin{center}
\begin{picture}(100,50)(0,0)
\hor{0,0}
\hln{-10,0}{10}
\hln{-10,20}{10}

\end{picture}
\end{center}
Lähtö on 0 jos ja vain jos kaikki tulot ovat 0, muuten lähtö on 1.
}

\frame{
\frametitle{Yksinomainen tai -portti (XOR)}
\begin{center}
\begin{picture}(100,50)(0,0)
\hxor{0,0}
\hln{-10,0}{10}
\hln{-10,20}{10}

\end{picture}
\end{center}
Lähtö on 1 jos jompikumpi (mutta ei molemmat) tuloista on 1, muuten lähtö on 0.
}

\frame{
\frametitle{Esimerkki}
Suunnittele logiikkaporttipiiri, joka antaa lähtösignaalin 1 (antaa äänimerkin), jos auton avaimet eivät ole virtalukossa,
ovi on auki ja ajovalot ovat päällä. Tulosignaalit ovat:\\
$A$ = "Avaimet ovat virtalukossa"\\
$B$ = "Ovi on auki"\\
$C$ = "Ajovalot ovat päällä"\\

{\bf Yksi esimerkki toimivasta ratkaisusta}:
\begin{center}
\begin{picture}(100,100)(0,0)
\hnot{20,50}
\vln{50,20}{43}
\hln{50,20}{20}

\hand{70,0}
\txt{0,60}{A}
\txt{0,30}{B}
\txt{0,0}{C}
\hln{0,60}{20}

\hln{0,0}{70}
\hln{0,30}{30}
\vln{30,10}{20}
\hln{30,10}{40}


\end{picture}
\end{center}


}