\frame{
\frametitle{Moottoriajoneuvojen valot}
Moottoriajoneuvon valojen päätehtävä on 
\begin{itemize}
\item varmistaa, että ajoneuvo havaitaan ja
\item varmistaa, että kuljettaja näkee ajoradan.
\end{itemize}
Valoja käytetään myös valaisemaan sisätilaa, rekisterikilpiä ym. sekä tietenkin merkkivaloina kojelaudassa. Valoja käytetään myös viestintään (suuntavalot).
}

\frame{
\frametitle{Valaistusteknistä sanastoa}
Termistö on hieman erilaista fysiikassa, lakitekstissä ja arkikielessä.
\begin{itemize}
\item Lampulla tarkoitetaan valoa tuottavaa laitetta. 
\item Valo tarkoittaa fysiikassa sähkömagneettista säteilyä, jonka silmä aistii. Valaisimella tarkoitetaan lampun ja apulaitteiden (varjostimet, heijastimet ym.) muodostamaa kokonaisuutta. Auton valaisimia kutsutaan sekä lakitekstissä, arkikielessä että autotekniikassa valoiksi (etuvalot, ajovalot jne.). 

\end{itemize}
}

\frame{
\frametitle{Valaistustekniikan suureita}
\begin{itemize}
\item Valovirta $\Phi$, yksikkö lumen (lm) = valonlähteen kokonaissäteilyteho.
\item Valon voimakkuus $L$, yksikkö kandela (cd), tiettyyn suuntaan säteilevä valovirta.
\item Valaistusvoimakkuus $E$, yksikkö luksi (lx), tietylle pinnalle säteilevä valovirta. 1 lx = kun 1 lumenin valovirta osuu
tasaisesti 1 m$^2$ pinnalle.
\item Värilämpötila $T$, yksikkö Kelvin. Kuuma kappale säteilee sinertävää valoa, kylmä kappale punertavaa. Hehkulampun
värilämpötila on noin 2700 K. Päivänvalon ja kaasupurkausvalon värilämpötila on noin 5600 K.
\end{itemize}
}

\frame{
\frametitle{Valaistustekniikan suureita}
 \begin{itemize}
\item 1 lx riittää juuri ja juuri kirjan lukemiseen.
\item 120 lx = auton valojen kirkkain valaisema alue.
\item 3000 lx = pilvinen talvipäivä.
\item 1000000 lx = aurinkoinen kesäpäivä.
\end{itemize}
}

\frame{
\frametitle{Hehkulamput}
\begin{itemize}
\item 1900-luvun alun autoissa käytettiin öljy- ja kaasulamppuja.
\item Hehkulamput yleistyivät 1920-luvulla.
\item 1960-luvulla halogeenilamput.

\end{itemize}
}

\frame{
\frametitle{Hehkulamput}
\begin{itemize}
\item Wolframlanka kuumennetaan sähkövirralla yli 2000-asteiseksi.
\item Lampun suojakuvusta on pumpattu ilma pois, ja lisätty hieman suojakaasua.
\item Wolfram höyrystyy ja ajan myötä tummentaa lampun kuvun.
\end{itemize}
}

\frame{
\frametitle{Halogeenilamput}
\begin{itemize}
\item Suojakaasuun lisätty halogeeni (yleensä jodi, bromi tai fluori) aikaansaa kiertoreaktion, jossa wolframatomit palaavat takaisin hehkulankaan.
\item Langan lämpötila voidaan nostaa 3000-asteeseen $\to$ parempi valontuotto ja hyötysuhde.
\item Lasi ei juurikaan tummene käytön aikana, koska wolframatomit palaavat hehkulangalle.
\end{itemize}
}

\frame{
\frametitle{Kaasupurkauslamput ("ksenonlamput")}
\begin{itemize}
\item Valon synnyttää ksenon-jalokaasussa tapahtuva sähköpurkaus, "valokaari".
\item Pelkkä akkujännitteen kytkeminen lampun napoihin ei riitä. Ksenonlamppu vaatii liitäntälaitteen kuten loistelamppu.
\item Ksenonlamppu antaa samalla sähköteholla noin 2,5-kertaisen valontuoton verrattuna halogeenilamppuun.
\end{itemize}
}

\frame{
\frametitle{Ksenonlampun toiminta}
\begin{itemize}
\item Sytytys: ohjainlaite antaa noin 20 kV:n suuruisen jännitepulssin, joka sytyttää valokaaren.
\item Lamppuun syötetään suuri virta, joka aikaansaa ksenonkaasun lämpenemisen ja siihen seostettujen metallien höyrystymisen.
\item Lamppu vaatii palaessaan noin 85 voltin jännitteen.
\end{itemize}
}

\frame{
\frametitle{Käsittelyohjeita}
\begin{itemize}
\item Älä koske lamppuihin paljain sormin. Sormista tarttuva rasva palaa kiinni lamppuun tummentaen sitä.
\item Irrota kaasupurkausvalaisimen virransyöttökaapeli ennen lampun käsittelyä (suurjännitevaara).
\item Hehku- ja halogeenilamput ovat sekajätettä. Ksenonlamput ovat ongelmajätettä. Jos rikot ksenonlamput, älä hengitä siitä vapautuvia höyryjä ja tuuleta tila.
\end{itemize}
}

\frame{
\frametitle{Led-lamput}
\begin{itemize}
\item Ledit ovat tällä hetkellä voimakkaasti kehittyvää teknologiaa.
\item Hyvää: Iskunkestävyys, pitkä käyttöikä, hyvä hyötysuhde ($\to$ matala lämmöntuotto), nopea syttyminen.
\item Huonoa: hinta. Todellla tehokkaita ledivalaisimia ei vielä osata valmistaa.
\item Sisävalaistuskäyttöa ajatellen huonoa on myös ledien kalsea valo.
\end{itemize}
}

\frame{
\frametitle{Led-lamput}
\begin{itemize}
\item Ledejä käytetty 1980-luvulta lähtien merkkivaloina.
\item 2000-luvulla ledijarruvalot ja -huomiovalot (suuntavalot, hälytysajoneuvojen valot\ldots).
\item Ledejä on helppo ohjata pulssisuhdemodulaatiolla.
\end{itemize}
}

\frame{
\frametitle{Tehojen suuruusluokkia}
\begin{itemize}
\item Halogeeniajovalopolttimo: 60 wattia.
\item Ksenonajovalopolttimo: 35 wattia.
\item Suuntavalot, jarruvalot ym. 20 wattia. 
\item Rekisterikilven valot: 10 wattia.
\item Merkkilamput: muutama watti.
\end{itemize}
}

\frame{
\frametitle{Auton valojen rakenteet}
\begin{itemize}
\item Paraboliset ja elliptiset heijastinvalot.
\item Projektorivalot.
\item Monimuotoheijastimet.
\item Säädettävällä lampun tai varjostimen sijainnilla varustetut järjestelmät.
\end{itemize}
}

\frame{
\frametitle{Paraboliset ja elliptiset heijastinvalot}
\begin{itemize}
\item Paraabelin tai ellipsin osan muotoinen heijastin ohjaa polttimon valonsäteet suoraan eteenpäin.
\item Jotta lähivalo ei häikäisisi vastaantulijoita, ohjataan säteet alaviistoon joko asettamalla polttimo muualle kuin paraabelin polttopisteeseen tai käyttämällä
polttimossa pientä metallikauhaa, joka suuntaa säteet yläviistoon (jolloin ne heijastuvat alaviistoon).
\item Heijastimen edessä oleva hajotinlasi suuntaa valokuvion tien reunoille.
\item Muotoilemalla polttimon kauha sopivaksi, saadaan valokuvio epäsymmetriseksi niin, että tien oikea reuna valaistaan pitkältä matkalta.
\end{itemize}
}

\frame{
\frametitle{Projektorivalot}
\begin{itemize}
\item Projektorivaloissa ei käytetä hajotinlasia, vaan valokuvio muotoillaan linssien avulla.
\end{itemize}
}

\frame{
\frametitle{Monimuotoheijastimeen perustuvat valot}
\begin{itemize}
\item Monimuotoheijastimessa valokuvio muotoillaan tarkkaan lasketun heijastimen avulla. Linssejä ei tarvita.
\end{itemize}
}

\frame{
\frametitle{Säädettävällä lampun (tai varjostimen) sijainnilla varustetut valot}
\begin{itemize}
\item Samaa valonlähdettä voidaan käyttää sekä lähi- että kaukovaloille, jos valokuvio muotoillaan liikuttamalla polttimoa
tai varjostinta sopivasti.
\end{itemize}
}

\frame{
\frametitle{Pesulaitteet}
\begin{itemize}
\item Likainen lasi sekä heikentää valotehoa että hajottaa valokuviota.
\item Kaasupurkauslamppuja käytettäessä pesulaitteet ovat pakolliset (1996).
\item Vanhemmissa autoissa on perinteiset pyyhkimeen ja vesisuihkuun perustuvat laitteet (jos niitäkään).
\item Uusissa autoissa valojen pesu perustuu tehokkaaseen vesisuihkuun.
\end{itemize}
}

\frame{
\frametitle{Kaarrevalot ja kulmavalot}
\begin{itemize}
\item Elektroninen ohjausjärjestelmä kääntää valoja korkeintaan 15 astetta niin, että valot eivät osoita metsään vaan tielle.
\item Joissain autoissa on kulmavalot, jotka osoittavat jyrkästi sivulle. Ne aktivoituvat matalissa nopeuksissa suuntavaloja käytettäessä ja/tai ohjauspyörää käännettäessä.
\end{itemize}
}

\frame{
\frametitle{Valojen suuntaus}
\begin{itemize}
\item Valot suunnataan erityisen suuntauslaitteen avulla.
\item Perussäätö suoritetaan mekaanisesti.
\end{itemize}
}

\frame{
\frametitle{Automaattiset säätöjärjestelmät}
\begin{itemize}
\item Staattinen säätöjärjestelmä säätää valojen pystysuuntauksen auton akselikuormien mukaan.
\item Dynaaminen säätöjärjestelmä valvoo jatkuvasti auton asentoa, ja pitää valot oikein suunnattuna
myös tilapäisten kallistumien (kaarreajo, jarrutukset ja kiihdytykset) aikana.
\end{itemize}
}

\frame{
\frametitle{Tulevaisuuden näkymiä}
\begin{itemize}
\item Ledivalaistuksen kehittyminen.
\item Keskusvalojärjestelmä: yksi valonlähde, josta valo siirretään kuituja pitkin kohteisiin.
\end{itemize}
}