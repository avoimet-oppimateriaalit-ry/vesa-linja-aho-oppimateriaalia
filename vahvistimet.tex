\frame{
\frametitle{Vahvistimet}
\begin{block}{Vahvistin} % Määritelmä Silvonen sivu 395
= elektroninen piiri, jolla signaalilähteestä tuleva signaali vahvistetaan, ja kytketään kuormaan tai ketjun seuraavalle laitteelle.
\end{block}
}

\frame{
\frametitle{Lineaarinen vahvistin}
\begin{center}
\begin{picture}(150,50)(0,0)
\cn{0,0}
\cn{0,50}
\hln{0,0}{50}
\hln{0,50}{50}
\vz{50,0}{R_{\rm in}}

\du{65,0}{u_{\rm in}}
\vst{150,0}{Au_{\rm in}}
\hln{150,0}{50}
\hz{150,50}{R_{\rm out}}
\cn{200,0}
\cn{200,50}
\end{picture}
\end{center}
}

\frame{
\frametitle{Vahvistimen toteuttaminen}
Halutunlainen vahvistin voidaan rakentaa esimerkiksi
\begin{itemize}
\item Operaatiovahvistimen avulla
\item Käyttämällä erilliskomponentteja, kuten bipolaaritransistoria tai MOSFET-transistoria
\end{itemize}
}