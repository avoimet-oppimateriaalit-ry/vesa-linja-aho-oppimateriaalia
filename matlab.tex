% TODO: siivoa vähän ja mieti järkevä paikka demotiedostolle

\frame{
\frametitle{MATLAB-harjoitukset}
MATLAB = MATrix LABoratory. Laajasti käytössä oleva numeerisen laskennan työkaluohjelmisto.
\begin{itemize}
\item Ohjelmalla suoritetaan laskutoimituksia skalaareille, vektoreille ja matriiseille.
\item Erittäin laaja valikoima valmiita funktioita.
\end{itemize}
Harjoitukset suoritetaan vapaamuotoisesti. Kokeile annettuja komentoja, kokeile omia komentoja. Pyri ymmärtämään, mitä kukin komento tekee!\\[1cm]
\scriptsize
Linux-käyttäjien on helppo kokeilla komentoja kotona vapaan Octave-ohjelmiston kanssa. Esim. Ubuntu 10.04:ssa: asenna paketit octave3.2, octave-audio ja octave-signal. Lisäksi pitää luoda kotihakemistoon tiedosto .octaverc ja luoda sinne rivi\\
global sound\_play\_utility = 'aplay';\\
jotta soundsc-funktio toistaa äänet.
}




\begin{frame}[fragile]
Tutki, mitä tapahtuu seuraavilla komennoilla:
\frametitle{MATLAB-harjoitukset}
\small
\begin{verbatim}
1+1
x=3
y=9
x+y
\end{verbatim}
Mitä eroa on seuraavilla kahdella komennolla
\begin{verbatim}
z=x+y
z=x+y;
\end{verbatim}
\end{frame}


\begin{frame}[fragile]
Vektorilaskutoimitukset
\frametitle{MATLAB-harjoitukset}
\small
\begin{verbatim}
x=[1,2,3]
2*x
x.^2
y=[3 4 5]
x.*y
\end{verbatim}
\end{frame}


\begin{frame}[fragile]
Käyrien piirtely
\frametitle{MATLAB-harjoitukset}
\small
\begin{verbatim}
x=linspace(0,100);
y=2*x
help plot
plot(x,y)
f=4
x=linspace(0,1,101);
y=2*sin(2*pi*f*sin(x));
plot(x,y)
\end{verbatim}
\end{frame}

\begin{frame}[fragile]
\frametitle{MATLAB-harjoitukset}
\small Lataa tiedosto \url{http://users.metropolia.fi/~vesali/aani.dat} ja tallenna se esim. työpöydälle.
Valitse MATLABin työhakemistoksi (vasemmasta ylänurkasta "Current folder") sama kansio, johon
tallensit tiedoston.
\begin{verbatim}
[y,Fs]=wavread('aani.dat'');
%typistetään vähän äänitiedostoa
z=y(1:308700);
soundsc(z)
%Mitä kyseinen äänitiedosto sisältää?
Kokeillaanpa uudestaan:
soundsc(z,Fs)
clear playsnd % Keskeyttää äänentoiston, jos tarvis
%ja pelleillään
soundsc(z,3*Fs)
soundsc(fliplr(z),Fs)
\end{verbatim}
\end{frame}




\begin{frame}[fragile]
\frametitle{MATLAB-harjoitukset}
Häiriön lisääminen
\small
\begin{verbatim}
N=length(z) % Tallennetaan näytemäärä
x=linspace(0,7,N);
h=0.1*sin(2*pi*1000*x);
e=h+z;
soundsc(e,Fs)
%wavwrite(e,Fs,'output.wav'); % Tällä voisi tallentaa tiedoston.
\end{verbatim}
\end{frame}

\begin{frame}[fragile]
\frametitle{MATLAB-harjoitukset}
\small
\begin{verbatim}
E=fft(e)/N; % Fourier-muunnos täytyy skaalata jakamalla näytemäärällä
plot(abs(E))
% Häiriöpiikin korkeus on puolet häiriön amplitudista. Syynä on se, että
% Matlabin Fourier-muunnosfunktio tekee muunnoksen kaksipuoleiseen
% taajuusalueeseen. Oikean korkuisena häiriö (ja muu signaali) näkyy
% komennolla
plot(2*abs(E))
% Piirroksessa on vaaka-akselina näytteiden määrä N. Kuinka saada näkymään
% häiriöpiikki oikealla taajuudella?
% Fourier-muunnoksessa taajuuden askellusväli (=kuinka suurta taajuutta
% vastaa yksi pykälä vaaka-akselilla):
f_step = Fs/N;
f = [0:f_step:(N-1)*f_step];
plot(f,2*abs(E))
\end{verbatim}
\end{frame}

\begin{frame}[fragile]
\frametitle{MATLAB-harjoitukset}
Fourier-käänteismuunnoksen voi kuunnella:
\small
\begin{verbatim}
soundsc(ifft(E),Fs)
\end{verbatim}
Tehtävä: poista häiriösignaali taajuustason esityksestä (Fourier-muunnetusta signaalista E). Vihje: komennosta \verb+find+
saattaa olla apua. Häiriö näkyy (kahtena) korkeana piikkinä taajuusspektrissä. Poista piikit.
\end{frame}


\begin{frame}[fragile]
\frametitle{MATLAB-harjoitukset}
Signaalin suodattaminen. Kokeile erilaisia suodattimia!
\small
\begin{verbatim}
[b,a]=butter(5,[600/22050,1000/22050]);
suod=filtfilt(b,a,z);
soundsc(suod,Fs)
help butter
\end{verbatim}
\end{frame}
