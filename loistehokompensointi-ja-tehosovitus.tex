\frame{
\frametitle{Loistehokompensointi}
\begin{itemize}
\item Loisteho on  ei-toivottu ilmiö. Esimerkiksi sähkömoottori ottaa sähköverkosta loistehoa, koska
siinä on käämejä (=induktanssia).
\item Tehdas, jossa on satoja tai tuhansia sähkömoottoreita, kuormittaa sähköverkkoa tarpeettomasti.
\item Loisteho sykkii moottorien ja voimalaitoksen välillä kuormittaen johtimia turhaan.
\item Teollisuuslaitoksilta peritään loistehosta maksua, joka on usein suurempi kuin pätötehomaksu!
\item Tämän takia loisteho pyritään kompensoimaan pois.
\item Kompensointi tapahtuu yleensä rinnakkaiskondensaattorilla.
\item Perustapa: laitetaan induktiivisen kuorman rinnalle kondensaattori, joka kumoaa
loistehon niin, että kuorma näyttää sähköverkkoon päin (lähes) pelkältä vastukselta.
\item Kompensoinnin voisi tehdä myös sarjakondensaattorilla, mutta se ei ole käytännössä järkevää. Miksi?
\end{itemize}
}

\frame{
\frametitle{Induktiivisen loistehon kompensointi rinnakkaiskondensaattorilla}
\begin{itemize}
\item Sarjakompensoinnin haittana on, että koko kuorman ottama virta kulkee silloin kondensaattorin läpi (kuormittaa
kondensaattoria).
\item Rinnakkaiskondensaattorin läpi kulkee vain loistehon kompensointiin vaadittava virta.
\item Kondensaattorin koko valitaan siten, että kuorman (kela ja vastus) sekä kondensaattorin
muodostaman kokonaisuuden loisteho on nolla, eli impedanssin imaginaariosa on nolla!
\end{itemize}
\begin{center}
\begin{picture}(100,50)(0,0)
\vl{100,0}{L}
\vst{0,0}{E}
\hln{0,0}{100}
%\hln{50,50}{50}
\hz{50,50}{R}
%\hl{0,50}{L}
\hln{0,50}{50}
%\du{115,0}{U}
%\ri{75,50}{I}
\color{blue}
\vc{50,0}{C}

\end{picture}
\end{center}

}

\frame{
\frametitle{Esimerkki}
Loistehokompensointi rinnakkaiskondensaattorilla perustuu siihen, että kondensaattori ottaa
sähköverkosta (jännitelähteestä) yhtä suuren mutta vastakkaismerkkisen loistehon kuin
induktiivinen kuorma. Kondensaattorin mitoituksen voi laskea kahdella tavalla:
\begin{description}
\item[Tapa 1] Lasketaan kuorman (kuvassa vastus + kela) ottama loisteho. Sitten lasketaan, kuinka
suuri kondensaattorin täytyy olla, jotta se ottaa yhtä suuren mutta vastakkaismerkkisen loistehon.
\item[Tapa 2] Lasketaan kondensaattorin, kelan ja vastuksen muodostaman kokonaisuuden
impedanssi, ja sitten valitaan kondensaattori niin, että tämän impedanssin imaginaariosa on nolla.
\end{description}
Tapa 2 on usein helpompi.


% (C=14\uF)
}

\frame{
\frametitle{Esimerkki - Tapa 1}

\[
U=230 \V \quad L=0,5\ {\rm H} \quad \omega=100\pi \quad R=100\ohm
\]
\begin{center}
\begin{picture}(100,50)(0,0)
\vl{100,0}{L}
\vst{0,0}{U}
\hln{0,0}{100}
%\hln{50,50}{50}
\hz{50,50}{R}
%\hl{0,50}{L}
\hln{0,50}{50}
%\du{115,0}{U}
%\ri{75,50}{I}
\color{blue}
\vc{50,0}{C}

\end{picture}
\end{center}
% (C=14\uF)

Kela ja vastuksen sarjaankytkennän näennäisteho on {\small
\[
S=UI^*=U\left(\frac{U}{Z_{\rm R}+Z_{\rm L}}\right)^*=U\frac{U^*}{(R+\jj \omega L)^*}=\frac{|U|^2}{R-\jj \omega L}=\frac{|U|^2(R+\jj \omega L)}{R^2+(\omega L)^2}
\]
}
Näennäistehon reaaliosa on pätöteho ($P$, watteja) ja imaginaariosa on loisteho ($Q$, vareja). Loistehon suuruus on
\[
Q=\frac{|U|^2\omega L}{R^2+(\omega L)^2}\approx 239,647\ldots\quad (\rm var)
\]
}

\frame{
\frametitle{Esimerkki - Tapa 1 jatkuu}
Seuraavaksi mitoitetaan kondensaattori siten, että se kuluttaa yhtä suuren mutta vastakkaismerkkisen loistehon kuin äsken laskettu
loisteho. Kondensaattorin näennäisteho on
\[
S_{\rm C}=UI^*=U\left(\frac{U}{Z_{\rm C}}\right)^*=U\frac{U^*}{\left(\frac{1}{\jj \omega C}\right)^*}=\frac{|U|^2}{\left(-\jj\frac{1}{\omega C}\right)^*}=\frac{|U|^2}{\jj\frac{1}{\omega C}}=
-\jj|U|^2 \omega C 
\]
eli sen loisteho on $-|U|^2 \omega C$ (kondensaattori ei koskaan kuluta pätötehoa!). Tämän pitää olla yhtä suuri (mutta vastakkaismerkkinen) kuin edellisellä kalvolla
laskettu $Q$:
\[
|U|^2 \omega C=\frac{|U|^2\omega L}{R^2+(\omega L)^2} \quad \Rightarrow C=\frac{L}{R^2+(\omega L)^2}\approx 14,4\cdot 10^{-6}
\]
eli tarvitaan 14,4 mikrofaradin kondensaattori.
}


\frame{
\frametitle{Esimerkki - Tapa 2}

\[
U=230 \V \quad L=0,5\ {\rm H} \quad \omega=100\pi \quad R=100\ohm
\]
\begin{center}
\begin{picture}(100,50)(0,0)
\vl{100,0}{L}
\vst{0,0}{U}
\hln{0,0}{100}
%\hln{50,50}{50}
\hz{50,50}{R}
%\hl{0,50}{L}
\hln{0,50}{50}
%\du{115,0}{U}
%\ri{75,50}{I}
\color{blue}
\vc{50,0}{C}

\end{picture}
\end{center}
% (C=14\uF)

Kela ja vastus ovat keskenään sarjassa, ja niiden muodostama sarjaankytkentä on rinnan kondensaattorin kanssa:
\[
Z=\frac{1}{\frac{1}{Z_{\rm C}}+\frac{1}{R+Z_{\rm L}}}=\frac{1}{\jj\omega C+\frac{1}{R+\jj\omega L}}
\]
Jotta piiri ei kuluttaisi loistehoa, tulee impedanssin olla reaalinen (=reaaliluku). Osoittajassa
on reaaliluku (ykkönen), joten impedanssi on reaalinen, jos ja vain jos nimittäjä on reaalinen. 
}

\frame{
\frametitle{Esimerkki - Tapa 2 jatkuu}
Tutkitaan nimittäjää:{\small
\[
\jj\omega C+\frac{1}{R+\jj\omega L}=\jj\omega C + \frac{R-\jj\omega L}{R^2+(\omega L)^2}=\underbrace{\jj\omega C}_{\rm imaginaarinen} + \underbrace{\frac{R}{R^2+(\omega L)^2}}_{\rm reaalinen}+\underbrace{\frac{-\jj\omega L}{R^2+(\omega L)^2}}_{\rm imaginaarinen}
\]
}
Jotta luku olisi reaalinen, täytyy imaginaariosan olla nolla. Ratkaistaan, millä kapasitanssin arvolla imaginaariosa saadaan nollaksi:
\[
\jj\omega C + \frac{-\jj\omega L}{R^2+(\omega L)^2}=0 \quad \Rightarrow C = \frac{ L}{R^2+(\omega L)^2} \approx 14,4\cdot 10^{-6}
\]
Vastaus on sama kuin mitä saatiin edellisellä tavalla laskemalla.
}


\frame{
\frametitle{Tehosovitus} % TODO Painota, että tehosovituksessa jos kuorma tiedetään, niin lähderesistanssia ei tule säätää samaksi!
\begin{itemize}
\item Eri asia kuin loistehokompensointi!
\item Tehosovituksessa pyritään valitsemaan kuorman impedanssi siten, että kuormaan kulkeva
pätöteho on suurimmillaan.
\item Esimerkiksi jos vahvistimen lähtöimpedanssi tiedetään, valitaan kaiuttimen impedanssi siten, että
teho on mahdollisimman suuri. Toisin sanoen: jos $Z_{\rm S}$ tiedetään, kuinka $Z_{\rm L}$ tulee
valita, jotta $Z_{\rm L}$:ään siirtyvä teho maksimoituu.
\item Toinen esimerkki: radioantennin kytkeminen lähetinvahvistimeen.
\end{itemize}
\begin{center}
\begin{picture}(100,50)(0,0)
\vz{100,0}{Z_{\rm L}}
\vst{0,0}{E}
\hln{0,0}{100}
%\hln{50,50}{50}
\hln{50,50}{50}
%\hl{0,50}{L}
\hz{0,50}{Z_{\rm S}}
%\du{115,0}{U}
%\ri{75,50}{I}
%\color{blue}
%\vc{50,0}{C}

\end{picture}
\end{center}

}



\frame{
\frametitle{Tehosovitus vaihtosähköllä}
\begin{itemize}
\item Vaihtosähköpiirissä tulee $Z_{\rm L}$ valita siten, että $Z_{\rm L}$:n imaginaariosa
kumoaa $Z_{\rm S}$:n imaginaariosan. Tällöin virta on suurin ja pätöteho kuormassa on suurin.
\item Lähteestä, jonka $Z_{\rm S}$ tunnetaan, suurinta mahdollista ulos tulevaa tehoa kutsutaan {\bf yltötehoksi}.
\item Perustelu on samanlainen kuin seuraavassa esimerkissä suoritettava perustelu; lasketaan vain kompleksiluvuilla.
\item Voidaan perustella myös maalaisjärjellä: resistansseille suoritetun tehosovituksen lisäksi pitää hankkiutua
impedanssin imaginaariosasta eroon, jolloin kuormaan kulkee niin suuri virta kuin mahdollista.
\end{itemize}
\[
R_{\rm L}=R_{\rm S}\quad \mbox{ja vaihtosähköllä}\quad  Z_{\rm L}=Z_{\rm S}^*
\]

}








\frame{
\frametitle{Esimerkki}

\begin{center}
\begin{picture}(100,50)(0,0)
\vz{100,0}{R_{\rm L}}
\vst{0,0}{E}
\hln{0,0}{100}
%\hln{50,50}{50}
\hln{50,50}{50}
%\hl{0,50}{L}
\hz{0,50}{R_{\rm S}}
%\du{115,0}{U}
%\ri{75,50}{I}
%\color{blue}
%\vc{50,0}{C}

\end{picture}
\end{center}

Osoita, että\footnote{P.S. Tämä on ihan puhdas tasasähköpiiritehtävä, eli nyt et tarvitse kompleksilukuja.} kuormavastuksen $R_{\rm L}$ teho on suurimmillaan silloin, kun $R_{\rm L}$
on arvoltaan yhtä suuri kuin $R_{\rm S}$. {Ohje:} Muodosta lauseke $R_{\rm L}$ teholle. 
Arvot $E$ ja $R_{\rm S}$ ovat {\bf vakioita}. Sitten derivoi lauseke $R_{\rm L}$:n suhteen
ja etsi tehon maksimi derivaatan avulla. 
}


\frame{
\frametitle{Esimerkki}
\begin{center}
\begin{picture}(100,50)(0,0)
\vz{100,0}{R_{\rm L}}
\vst{0,0}{E}
\hln{0,0}{100}
%\hln{50,50}{50}
\hln{50,50}{50}
%\hl{0,50}{L}
\hz{0,50}{R_{\rm S}}
%\du{115,0}{U}
%\ri{75,50}{I}
%\color{blue}
%\vc{50,0}{C}

\end{picture}
\end{center}

Osoita, että\footnote{P.S. Tämä on ihan puhdas tasasähköpiiritehtävä, eli nyt et tarvitse kompleksilukuja.} kuormavastuksen $R_{\rm L}$ teho on suurimmillaan silloin, kun $R_{\rm L}$
on arvoltaan yhtä suuri kuin $R_{\rm S}$. {Ohje:} Muodosta lauseke $R_{\rm L}$ teholle. 
Arvot $E$ ja $R_{\rm S}$ ovat {\bf vakioita}. Sitten derivoi lauseke $R_{\rm L}$:n suhteen
ja etsi tehon maksimi derivaatan avulla.

Ratkaisu: teho vastuksessa $R_{\rm L}$ on
\[
P_{\rm L}=U_{\rm L}I=R_{\rm L}I^2=R_{\rm L}\left(\frac{E}{R_{\rm S}+R_{\rm L}}\right)^2= \frac{E^2R_{\rm L}}{R_{\rm S}^2+2R_{\rm S}R_{\rm L}+R_{\rm L}^2}.
\]

}


\frame{
Osamäärän derivaatta on
\[
\left(\frac{f}{g}\right)'=\frac{f'g-fg'}{g^2}
\]
Derivoidaan äsken saatu lauseke $R_{\rm L}$:n suhteen. Muut muuttujat ovat vakioita:
\[
P_{\rm L}=\frac{E^2R_{\rm L}}{R_{\rm S}^2+2R_{\rm S}R_{\rm L}+R_{\rm L}^2}
\]
\[
P_{\rm L}'=E^2\frac{R_{\rm S}^2+2R_{\rm S}R_{\rm L}+R_{\rm L}^2-R_{\rm L}(2R_{\rm S}+2R_{\rm L})}{(R_{\rm S}^2+2R_{\rm S}R_{\rm L}+R_{\rm L}^2)^2}
\]
Derivaatan nollakohta
\[
R_{\rm S}^2+2R_{\rm S}R_{\rm L}+R_{\rm L}^2-R_{\rm L}(2R_{\rm S}+2R_{\rm L})=0
\]
\[
R_{\rm L}=\pm R_{\rm S}
\]

}

\frame{
Hylätään negatiivinen vastaus, koska negatiivinen vastus ei kuluta vaan luovuttaa tehoa. Varmistetaan vielä, että löydetty derivaatan
nollakohta todella on maksimi. Jatkuvan ja derivoituvan funktion maksimi voi sijaita vain välin päätepisteissä ja derivaatan nollakohdissa.
Päätepisteissä (nolla, ääretön) tehon raja-arvo on nolla. Löydetty derivaatan nollakohta on maksimi, koska jos $R_{\rm L}$ on suurempi kuin 
$R_{\rm S}$, derivaatta on negatiivinen ja jos pienempi kuin $R_{\rm S}$, derivaatta on positiivinen. Siis kyseessä on maksimi.

Kuormavastuksen teho on siis suurimmillaan, kun se valitaan yhtä suureksi kuin $R_{\rm S}$.
}

