\frame{
\frametitle{Ohjattu lähde}
\begin{itemize}
\item Tähän mennessä (jännite- ja virta)lähteet ovat olleet vakioarvoisia.
\item Jos lähteen arvo ei riipu piirin muista jännitteistä, lähdettä kutsutaan
{\bf riippumattomaksi}. Vakioarvoiset tai ajan funktiona muuttuvat lähteet
ovat riippumattomia lähteitä.
\item Jos lähteen arvo riippuu jonkin toisen piirin osan virrasta tai jännitteestä,
lähde on {\bf ohjattu lähde}.
\end{itemize}


}

\frame{
\frametitle{Jänniteohjattu jännitelähde (VCVS)}
\begin{center}
\begin{picture}(150,50)(0,0)
\cn{0,0}
\cn{0,50}
\du{0,0}{u}
\vst{100,0}{e=Au}
\end{picture}
\end{center}
\begin{itemize}
\item VCVS:n jännite $e$ riippuu jostain toisesta jännitteestä $u$.
\item Kerrointa $A$ kutsutaan jännitevahvistukseksi.
\item Käytännön esimerkki: audiovahvistin.
\end{itemize}


}
\frame{

\frametitle{ Virtaohjattu jännitelähde (CCVS)}
\begin{center}
\begin{picture}(150,50)(0,0)
\cn{0,0}
\cn{0,50}
\vln{0,0}{50}
\di{0,25}{i}
\vst{100,0}{e=ri}
\end{picture}
\end{center}
\begin{itemize}
\item CCVS:n jännite $e$ riippuu jostain virrasta $i$.
\item Kerrointa $r$ kutsutaan siirto- tai transresistanssiksi.
\item Käytännössä harvinainen (voidaan rakentaa {\em operaatiovahvistimen} avulla).
\end{itemize}
}

\frame{

\frametitle{Jänniteohjattu virtalähde (VCCS)}

\begin{center}
\begin{picture}(150,50)(0,0)
\cn{0,0}
\cn{0,50}
\du{0,0}{u}
\vj{100,0}{j=gu}
\end{picture}
\end{center}
\begin{itemize}
\item VCCS:n virta $j$ riippuu jostain toisesta jännitteestä $u$.
\item Kerrointa $g$ kutsutaan siirto- tai transkonduktanssiksi.
\item Käytännön esimerkki: kanavatransistori.
\end{itemize}

}

\frame{
\frametitle{Virtaohjattu virtalähde (CCCS)}
\begin{center}
\begin{picture}(150,50)(0,0)
\cn{0,0}
\cn{0,50}
\vln{0,0}{50}
\di{0,25}{i}
\vj{100,0}{j=\beta i}
\end{picture}
\end{center}
\begin{itemize}
\item CCCS:n virta $j$ riippuu jostain virrasta $i$.
\item Kerrointa $\beta$ kutsutaan virtavahvistukseksi.
\item Käytännön esimerkki: bipolaaritransistori.
\end{itemize}

}

\frame{
\begin{block}{Esimerkki}
Ratkaise jännite $U$ oheisesta tasasähköpiiristä.
\end{block}
\[
E_1=10\V \quad R_1=10 \ohm \quad R_2= 20 \ohm \quad R_3=30 \ohm
\]
\[
 R_4=40 \ohm \quad
\quad r=2\ohm
\]

\begin{center}
\begin{picture}(150,50)(0,0)
\vst{0,0}{E_1}
\hz{0,50}{R_1}
\hz{50,50}{R_2}
\vz{100,0}{R_3}
\ri{50,50}{i}
%\hl{100,50}{L}
\hln{100,50}{50}
%\hz{50,0}{R_4}
%\out{150,0}
%\out{150,50}
%\ri{58,50}{I_2}
%\hc{50,50}{C}
\vst{200,0}{e_2=ri}
\hz{150,50}{R_4}
\hln{0,0}{200}
\hln{100,0}{50}
%\hln{100,50}{50}
%\du{57,0}{U_1}
%\ri{57,50}{I}
\dcru{105,0}{U}
\end{picture}
\end{center}
Huomaa, että oikeanpuoleinen lähde on ohjattu lähde.

}

%LUENTO11

\frame{
\begin{block}{Ratkaisu}
Ratkaise jännite $U$ oheisesta tasasähköpiiristä.
\end{block}
\[
E_1=10\V \quad R_1=10 \ohm \quad R_2= 20 \ohm \quad R_3=30 \ohm
\]
\[
 R_4=40 \ohm \quad
\quad r=2\ohm
\]

\begin{center}
\begin{picture}(150,50)(0,0)
\vst{0,0}{E_1}
\hz{0,50}{R_1}
\hz{50,50}{R_2}
\vz{100,0}{R_3}
\ri{50,50}{i}
%\hl{100,50}{L}
\hln{100,50}{50}
%\hz{50,0}{R_4}
%\out{150,0}
%\out{150,50}
%\ri{58,50}{I_2}
%\hc{50,50}{C}
\vst{200,0}{e_2=ri}
\hz{150,50}{R_4}
\hln{0,0}{200}
\hln{100,0}{50}
%\hln{100,50}{50}
%\du{57,0}{U_1}
%\ri{57,50}{I}
\dcru{105,0}{U}
\end{picture}
\end{center}
Huomaa, että oikeanpuoleinen lähde on ohjattu lähde.

}

\frame{
\begin{center}
\begin{picture}(150,50)(0,0)
\vst{0,0}{E_1}
\hz{0,50}{R_1}
\hz{50,50}{R_2}
\vz{100,0}{R_3}
\ri{50,50}{i}
%\hl{100,50}{L}
\hln{100,50}{50}
%\hz{50,0}{R_4}
%\out{150,0}
%\out{150,50}
%\ri{58,50}{I_2}
%\hc{50,50}{C}
\vst{200,0}{e_2=ri}
\hz{150,50}{R_4}
\hln{0,0}{200}
\hln{100,0}{50}
%\hln{100,50}{50}
%\du{57,0}{U_1}
%\ri{57,50}{I}
\dcru{105,0}{U}
\end{picture}
\end{center}
Merkitään $R_1$:n ja $R_2$:n sarjaankytkentää symbolilla $R_{12}$ ja kirjoitetaan solmuyhtälö:
\[
UG_3=(E_1-U)G_{12}+(ri-U)G_4
\]
Yhtälössä on kaksi tuntematonta, joten kirjoitetaan toinen yhtälö, jossa esiintyvät samat
tuntemattomat:
\[
i=(E_1-U)G_{12}
\]
Sijoitetaan $i$ ylempään yhtälöön:
\[
E_1G_{12}-UG_{12}+rG_4G_{12}E_1-rG_4G_{12}U-UG_4=UG_3
\]
}

\frame{
\[
E_1G_{12}-UG_{12}+rG_4G_{12}E_1-rG_4G_{12}U-UG_4=UG_3
\]
josta
\[
G_{12}E_1(1+rG_4)=U(G_3+G_{12}+G_4+rG_4G_{12})
\]
sijoitetaan lukuarvot ja ratkaistaan $U$:
\[
U=\frac{\frac{10}{30}(1+\frac{2}{40})}{\frac{1}{30}+\frac{1}{30}+\frac{1}{40}+\frac{2}{40\cdot 30}}
=3,75 \V
\]

}


\frame{
\begin{block}{Esimerkki}
Laske jännite $U_3$.
\end{block}
\[
G_1=1\,{\rm S}\quad G_2=2\,{\rm S} \quad G_3=3\,{\rm S}\quad G_4=4\,{\rm S}
\quad G_5=5\,{\rm S} \quad g=6\,{\rm S}\quad J=3\A
\]

\begin{center}
\begin{picture}(150,100)(0,0)
\vj{0,0}{J}
\vz{50,0}{G_1\hspace{-2mm}}
\vz{100,0}{G_2}
\vz{150,0}{G_3}
\hz{50,50}{G_4}
\hz{100,50}{G_5}
\hlj{100,100}{gU_1}
\du{15,0}{U_1}
\du{160,0}{U_3}
\hln{0,0}{150}
\hln{0,50}{50}

\vln{100,50}{50}
\vln{150,50}{50}
\cn{100,50}

\end{picture}
\end{center}
\tiny $U_3=-\frac{48}{115}\V \approx -417 \mV$
}