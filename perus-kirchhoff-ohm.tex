\frame {
\frametitle{Sähkövirta}
\begin{itemize}
\item Sähkövirta on varauksenkuljettajien liikettä.
\item Yksikkö on ampeeri (A).
\item Suureen lyhenne on $I$.
\item Sähkövirtaa voidaan verrata letkussa kulkevaan veteen.
\item Virta kiertää aina jossain silmukassa (se ei puristu kasaan eikä häviä olemattomiin).
\item Virtapiirissä virta merkitään nuolella johtimeen:
\end{itemize}

\begin{center}
\begin{picture}(100,25)(0,0)

\hln{0,0}{100}
\ri{50,0}{I=2\mA}

\end{picture}
\end{center}
}

\frame {
\frametitle{Kirchhoffin virtalaki}
\begin{itemize}
\item Kuten edellä todettiin, sähkövirta ei häviä mihinkään.
\end{itemize}
\begin{block}{Kirchhoffin virtalaki (myös: Kirchhoffin ensimmäinen laki)}
Virtapiirin jollekin alueelle tulevien virtojen summa on yhtä suuri kuin sieltä lähtevien
virtojen summa.
\end{block}

\begin{center}
\begin{picture}(100,50)(0,0)

\hln{0,0}{100}
\hln{0,50}{100}
\vln{0,0}{50}
\vln{50,0}{50}
\vln{100,0}{50}

\ri{25,0}{I_1=3\mA}
\ri{75,0}{I_2=2\mA}
\ui{50,25}{I_3=1\mA}
\end{picture}
\end{center}
Piirsitpä ympyrän mihin tahansa kohtaan piiriä, ympyrän sisään menee yhtä paljon virtaa
kuin mitä tulee sieltä ulos!

}

\frame {
\frametitle{Ole tarkka etumerkkien kanssa!}
\begin{itemize}
\item Voidaan sanoa: "pankkitilin saldo on -50 euroa" tai "olen 50 euroa velkaa pankille".
\item Voidaan sanoa: "Yrityksen tilikauden tulos oli -500000 euroa" tai "firma teki tappiota 500000 euroa".
\item Jos mittaat johtimen virtaa virtamittarilla ja se näyttää $-15 \mA$, niin kääntämällä mittarin
toisin päin se näyttää $15\mA$.
\item Aivan samalla tavalla voidaan virran suunta ilmoittaa etumerkillä. Alla on kaksi täysin samanlaista piiriä.
\end{itemize}

\begin{picture}(100,50)(-25,0)
\hln{0,0}{100}
\hln{0,50}{100}
\vln{0,0}{50}
\vln{50,0}{50}
\vln{100,0}{50}
\ri{25,0}{I_1=3\mA}
\ri{75,0}{I_2=2\mA}
\ui{50,25}{I_3=1\mA}
\end{picture}
\begin{picture}(100,50)(-75,0)
\hln{0,0}{100}
\hln{0,50}{100}
\vln{0,0}{50}
\vln{50,0}{50}
\vln{100,0}{50}
\li{25,0}{I_{\rm a}=-3\mA}
\li{75,0}{I_{\rm b}=-2\mA\hspace{-1cm}}
\ui{50,25}{I_3=1\mA}
\end{picture}


}

\frame {
\frametitle{Jännite}
\begin{itemize}
\item Jännite on kahden pisteen välinen potentiaaliero.
\item Suureen lyhenne on $U$.
\item Virtapiirianalyysissä ei oteta kantaa siihen, miten potentiaaliero on luotu.
\item Jännitteen yksikkö on voltti (V).
\item Jännitettä voi verrata paine-eroon putkessa tai korkeuseroon.
\item Jännitettä merkitään pisteiden välille piirretyllä nuolella.
\end{itemize}

\begin{center}
\begin{picture}(50,50)(0,0)
\vst{0,0}{12 \V}
\du{20,0}{U=12 \V}
\hln{0,0}{20}
\hln{0,50}{20}
\end{picture}
\end{center}
}

\frame {
\frametitle{Kirchhoffin jännitelaki}
\begin{itemize}
\item Kahden pisteen välillä vaikuttaa sama jännite tarkastelureitistä riippumatta.
\item Tämä on helpoin hahmottaa rinnastamalla jännite korkeuseroihin.
\end{itemize}
\begin{block}{Kirchhoffin jännitelaki (myös: Kirchhoffin toinen laki)}
{\bf Silmukan jännitteiden summa on }etumerkit huomioon ottaen {\bf nolla}.
\end{block}


\begin{center}
\begin{picture}(100,50)(0,0)
\hst{0,0}{1,5 \V}
\hst{50,0}{1,5 \V}
\hst{100,0}{1,5 \V}
\vln{0,0}{50}
\vln{150,0}{50}
\hln{0,50}{50}
\hln{100,50}{50}
\lu{50,50}{4,5 \V}
\cn{50,50}
\cn{100,50}
\end{picture}
\end{center}
}

\frame{
\frametitle{Ohmin laki}
\begin{itemize}
\item Mitä suurempi virta, sitä suurempi jännite -- ja päinvastoin.
\item Resistanssilla tarkoitetaan kappaleen kykyä vastustaa sähkövirran kulkua.
Resistanssi on jännitteen ja virran suhde.
\item Resistanssin tunnus on $R$ ja yksikkö ohmi ($\ohm$).
\end{itemize}

\begin{center}
$U=RI$

\begin{picture}(50,50)(0,0)

\hz{0,0}{R}
\ru{0,10}{U}
\ri{5,0}{I}
\hln{50,0}{15}
\hln{-15,0}{15}

\end{picture}
\end{center}
}

\frame {
  \frametitle{Käsitteitä}
\begin{description}
\item[Virtapiiri] Elektronisista komponenteista koostuva järjestelmä, jossa sähkövirta kulkee.
\item[Tasasähkö] Sähköiset suureet (jännite, virta) eivät muutu - tai muuttuvat vain vähän - ajan kuluessa.
\item[Tasasähköpiiri] Virtapiiri, jossa jännitteet ja virrat ovat ajan suhteen vakioita.
\end{description}
\begin{exampleblock}{Esimerkki}
Taskulampussa on tasasähköpiiri (paristo, kytkin ja polttimo). Polkupyörän dynamo ja lamppu
puolestaan muodostavat vaihtosähköpiirin.
\end{exampleblock}

}

\frame {
  \frametitle{Vaihtoehtoinen tasasähkön määritelmä}
Tasajännitteellä ja -virralla voidaan tarkoittaa myös jännitettä ja virtaa, jonka suunta (etumerkki)
pysyy samana, mutta suuruus voi vaihdella. Esimerkiksi tavallinen lyijyakkujen laturi tuottaa yleensä
nk. {\em sykkivää tasajännitettä}, jonka suuruus vaihtelee välillä 0 V ... $\approx$ 18 V. Tätäkin
kutsutaan yleensä tasajännitteeksi.
\begin{alertblock}{Sopimus}
Piiriteoriassa tasajännitteellä (virralla) tarkoitetaan vakiojännitettä (virtaa). Sekä
suunta että suuruus pysyvät ajan suhteen vakiona.
\end{alertblock}
}


\frame{
\frametitle{Yksinkertainen virtapiiri}
\begin{itemize}
\item Akkuun kiinnitetty hehkulamppu. Hehkulangan resistanssi on $10 \ohm$. 
\end{itemize}

\begin{center}
\begin{picture}(50,50)(0,0)
\vst{0,0}{12 \V}
\uncover<-1>{\vlamp{50,0}{}}
\uncover<2->{\vz{50,0}{10\ohm}}
\uncover<-3>{\ri{25,50}{I=?}}
\uncover<4->{\ri{25,50}{I=1,2\A}}
\hln{0,0}{50}
\hln{0,50}{50}
\uncover<3->{\du{65,0}{12\V}}
\end{picture}
\end{center}


\uncover<4->{$U=RI$\\
$I=\frac{U}{R}=\frac{12 \V}{10 \ohm}=1,2 \A$}
}

\frame{
\begin{block}{Esimerkki}
Ratkaise jännite $E$.
\end{block}
\begin{center}
\begin{picture}(50,100)(0,0)
\vst{0,0}{E}
\vst{0,50}{1,5 \V}
\vz{50,25}{R=20\ohm\hspace{-2.5cm}}
\vln{50,0}{25}
\vln{50,75}{25}
\hln{0,100}{50}
\hln{0,0}{50}
\di{50,25}{I=50\mA}
\end{picture}
\end{center}

}


%LUENTO2

%\frame{\tableofcontents}

%\subsection{1. Kotitehtävän ratkaisu}

\frame{
\begin{block}{Ratkaisu}
Ratkaise jännite $E$.
\end{block}
\begin{center}
\begin{picture}(50,100)(0,0)
\vst{0,0}{E}
\vst{0,50}{1,5 \V}
\vz{50,25}{R=20\ohm\hspace{-2.5cm}}
\vln{50,0}{25}
\vln{50,75}{25}
\hln{0,100}{50}
\hln{0,0}{50}
\di{50,25}{I=50\mA}
\uncover<2->{\du{12,0}{E} } 
\uncover<2->{\du{12,50}{U} }
\uncover<2->{\du{41,25}{\hspace{-0.65cm}U_{\rm R}} }
\end{picture}
\end{center}%\uncover<3->{\[ U_1+U_2-U_{\rm R}=0 \Leftrightarrow U_{\rm R}=U_1+U_2 \]}\uncover<4->{\[ U=RI \Rightarrow U_{\rm R}=RI \Rightarrow I=\frac{U_{\rm R}}{R}=\frac{U_1+U_2}{R}=\frac{1,5 \V+1,5 \V}{20\ohm}=150 \mA \]}
\uncover<2->{\[ E+U-U_{\rm R}=0 \Leftrightarrow U_{\rm R}=E+U\quad U_{\rm R}=RI=20\ohm\cdot50\mA=1\V\]}
\uncover<3->{\vspace{-\baselineskip}
\[  \Rightarrow  U_{\rm R}=E+U \Rightarrow 1\V=E+1,5\V \Rightarrow E=-0,5\V  \]}
}


\frame{
\frametitle{Esimerkki} % Arska 1 kotitehtävä 1 2009
\[
R_2=5\ohm \quad E_1=3\V \quad E_2=2\V
\]

\begin{center}
\begin{picture}(150,110)(0,0)
\vst{0,25}{E_1}
\vst{100,50}{E_2}
\vz{50,0}{R_2}
\vz{50,50}{R_1}
\ri{25,100}{I_1}
\li{75,100}{I_2}
\hln{0,100}{100}
\hln{0,0}{50}
\hln{50,50}{50}
\vln{0,0}{25}
\vln{0,75}{25}

\end{picture}
\end{center}
a) Millä $R_1$:n arvolla $I_2=0\A$?\\
b) Mikä on silloin virta $I_1$?\\
\tiny a) $10 \ohm$ ja b) $0,2 \A$.
}

