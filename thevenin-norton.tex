\frame {
  \frametitle{Théveninin ja Nortonin teoreemat}
\begin{itemize}
 \item Olemme käsitelleet seuraavat piirimuunnokset: jännitelähteiden sarjaankytkentä, virtalähteiden rinnankytkentä, vastusten rinnankytkentä
 sekä vastusten sarjaankytkentä sekä jännitelähde-virtalähdemuunnos.
\item Théveninin ja Nortonin teoreemat liittyvät nekin piirimuunnoksiin.
 \item Théveninin ja Nortonin teoreemojen nojalla mikä tahansa jännitelähteistä, virtalähteistä ja vastuksista
 koostuva piiri voidaan esittää jännitelähteen ja vastuksen sarjaankytkentänä tai virtalähteen ja vastuksen
 rinnankytkentänä. 
\end{itemize}
}


\frame {
\frametitle{Esimerkki piirimuunnoksesta}

\begin{exampleblock}{Théveninin teoreema}
Mikä tahansa lineaarinen piiri voidaan esittää yhdestä portista katsottuna jännitelähteen ja vastuksen
sarjaankytkentänä. Tätä sarjaankytkentää kutsutaan Théveninin lähteeksi.
\end{exampleblock}

\begin{exampleblock}{Portti}
Portti = napapari = kaksi napaa eli sellaista solmua, johon voidaan kytkeä joku toinen
piiri (esimerkiksi auton akun navat ovat hyvä esimerkki napaparista).
\end{exampleblock}

\begin{exampleblock}{Nortonin teoreema}
Mikä tahansa lineaarinen piiri voidaan esittää yhdestä portista katsottuna virtalähteen ja vastuksen
rinnankytkentänä. Tätä rinnankytkentää kutsutaan Nortonin lähteeksi.
\end{exampleblock}

}

\frame {
\frametitle{Théveninin lähteen muodostaminen}
\begin{center}
\begin{picture}(200,55)(0,0)
\vst{0,0}{E}
\hz{0,50}{R_1}
\vz{50,0}{R_2}
\hln{0,0}{60}
\hln{50,50}{10}
\out{60,0}
\out{60,50}

\txt{100,25}{\Longleftrightarrow}

\vst{150,0}{E_{\rm T}}
\hz{150,50}{R_{\rm T}}
\hln{150,0}{50}
\out{200,0}
\out{200,50}

\end{picture}
\end{center}
Théveninin lähteen $E_{\rm T}$ selvitetään yksinkertaisesti laskemalla portin jännite.
$R_{\rm T}$ voidaan selvittää kahdella tavalla:
\begin{itemize}
\item Sammuttamalla kaikki piirin riippumattomat (ei-ohjatut) lähteet ja
laskemalla portista näkyvä resistanssi.
\item Selvittämällä portin oikosulkuvirta ja soveltamalla Ohmin lakia.
\end{itemize}
Ohjattu lähde on lähde, jonka arvo riippuu piirin jostain toisesta jännitteestä
tai virrasta. Nämä käydään kurssin loppupuolella.

}


\frame {
\frametitle{Théveninin lähteen muodostaminen}
\begin{center}
\begin{picture}(200,55)(0,0)
\vst{0,0}{E}
\hz{0,50}{R_1}
\vz{50,0}{R_2}
\hln{0,0}{60}
\hln{50,50}{10}
\out{60,0}
\out{60,50}

\txt{100,25}{\Longleftrightarrow}

\vst{150,0}{E_{\rm T}}
\hz{150,50}{R_{\rm T}}
\hln{150,0}{50}
\out{200,0}
\out{200,50}

\end{picture}
\end{center}
Portin jännite saadaan (tässä tapauksesa) laskemalla vastusten läpi kulkeva virta ja kertomalla se
$R_2$:lla. Tämä portin jännite, niin sanottu {\bf tyhjäkäyntijännite}, on sama kuin $E_{\rm T}$
\[
E_{\rm T}=\frac{E}{R_1+R_2}R_2
\]

}
\frame {
\frametitle{Théveninin lähteen muodostaminen}
$R_{\rm T}$ voidaan ratkaista kahdella tavalla. Tapa 1: sammutetaan piirin kaikki lähteet, ja lasketaan
napojen välinen resistanssi. Sammutettu jännitelähde on jännitelähde, jonka jännite on nolla volttia, eli
toisin sanoen pelkkä johdin:
\begin{center}
\begin{picture}(200,55)(0,0)
%\vst{0,0}{E}
\vln{0,0}{50}
\hz{0,50}{R_1}
\vz{50,0}{R_2}
\hln{0,0}{60}
\hln{50,50}{10}
\out{60,0}
\out{60,50}

\txt{100,25}{\Longleftrightarrow}

%\vst{150,0}{E_{\rm T}}
\vln{150,0}{50}
\hz{150,50}{R_{\rm T}}
\hln{150,0}{50}
\out{200,0}
\out{200,50}

\end{picture}
\end{center}
Nyt napojen välinen resistanssi on helppo laskea: $R_1$ ja $R_2$ ovat rinnan, joten resistanssiksi saadaan
\[
R_{\rm T}=\frac{1}{G_1+G_2}=\frac{R_1R_2}{R_1+R_2}.
\]
Tämä tapa on yleensä helpompi kuin oikosulkuvirran käyttäminen!

}
\frame {
\frametitle{$R_{\rm T}$:n selvittäminen oikosulkuvirran avulla}
$R_{\rm T}$ voidaan ratkaista kahdella tavalla. Tapa 2: oikosuljetaan navat, ja
lasketaan oikosulun läpi kulkeva virta eli {\bf oikosulkuvirta}:
\begin{center}
\begin{picture}(200,55)(0,0)
\vst{0,0}{E}
\vln{60,0}{50}
\hz{0,50}{R_1}
\vz{50,0}{R_2}
\hln{0,0}{60}
\hln{50,50}{10}
\out{60,0}
\out{60,50}

\txt{100,25}{\Longleftrightarrow}

\vst{150,0}{E_{\rm T}}
\vln{200,0}{50}
\hz{150,50}{R_{\rm T}}
\hln{150,0}{50}
\out{200,0}
\out{200,50}

\di{60,25}{I_{\rm K}}
\di{200,25}{I_{\rm K}}


\end{picture}
\end{center}
Oikosulkuvirran suuruus on
\[
I_{\rm K}=\frac{E}{R_1}
\]
ja vastuksen $R_{\rm T}$ arvoksi saadaan (soveltamalla Ohmin lakia oikeanpuoleiseen kuvaan):
\[
R_{\rm T}=\frac{E_{\rm T}}{I_{\rm K}}=\frac{E_{\rm T}}{\frac{E}{R_1}}=\frac{\frac{E}{R_1+R_2}R_2}{\frac{E}{R_1}}=
\frac{R_1R_2}{R_1+R_2}
\]

}


\frame {
\frametitle{Nortonin lähde}
Nortonin lähde on yksinkertaisesti Théveninin lähde johon on sovellettu lähdemuunnosta (tai päinvastoin).
Resistanssi on sama molemmissa lähteissä. Nortonin lähteessä virtalähteen virta on sama kuin portin
oikosulkuvirta.

\begin{center}
\begin{picture}(100,50)(0,0)
\vj{0,0}{J_{\rm N}}
\vz{50,0}{R_{\rm N}}
\hln{0,0}{60}
\hln{0,50}{60}
\out{60,0}
\out{60,50}
\end{picture}
\end{center}
}

\frame {
\frametitle{Esimerkki 1}
Muodosta Théveninin lähde. Kaikki komponenttiarvot = 1.
\begin{center}
\begin{picture}(150,50)(0,0)
\vj{0,0}{J_1}
\vz{50,0}{R_1}
\vz{100,0}{R_2}
%\hz{50,50}{R_2}
\hst{50,50}{E}
%\hz{50,0}{R_4}
\out{150,0}
\out{150,50}

\hln{0,0}{150}
\hln{100,0}{50}
\hln{0,50}{50}
\hln{100,50}{50}
%\du{57,0}{U_1}
%\ri{57,50}{I}
\end{picture}
\end{center}
}

\frame {
\frametitle{Esimerkki 2}
Muodosta Théveninin lähde. Kaikki komponenttiarvot = 1.
\begin{center}
\begin{picture}(150,50)(0,0)
\vst{0,0}{E}
\vz{50,0}{R_2}
\vj{100,0}{J}
%\hz{50,50}{R_2}
\hz{50,50}{R_3}
%\hz{50,0}{R_4}
\out{150,0}
\out{150,50}

\hln{0,0}{150}
\hln{100,0}{50}
%\hln{0,50}{50}
\hz{0,50}{R_1}
\hln{100,50}{50}
%\du{57,0}{U_1}
%\ri{57,50}{I}
\end{picture}
\end{center}
}

\frame{
\begin{block}{Esimerkki}
Muodosta kuvan piiristä Théveninin lähde. Kaikki komponenttiarvot ovat 1.
(Vastukset ovat jokainen $1 \ohm$ ja virtalähde $J_1=1\A$.)
\end{block}

\begin{center}
\begin{picture}(150,50)(0,0)
\vj{0,0}{J_1}
\vz{50,0}{R_1}
\vz{100,0}{R_3}
\hz{50,50}{R_2}
%\hz{50,0}{R_4}
\out{150,0}
\out{150,50}

\hln{0,0}{150}
\hln{100,0}{50}
\hln{0,50}{50}
\hln{100,50}{50}
%\du{57,0}{U_1}
%\ri{57,50}{I}
\end{picture}
\end{center}

}

%LUENTO7


\frame{
\begin{block}{Ratkaisu}
Muodosta kuvan piiristä Théveninin lähde. Kaikki komponenttiarvot ovat 1.
(Vastukset ovat jokainen $1 \ohm$ ja virtalähde $J_1=1\A$.)
\end{block}

\begin{center}
\begin{picture}(150,50)(0,0)
\vj{0,0}{J_1}
\vz{50,0}{R_1}
\vz{100,0}{R_3}
\hz{50,50}{R_2}
%\hz{50,0}{R_4}
\out{150,0}
\out{150,50}

\hln{0,0}{150}
\hln{100,0}{50}
\hln{0,50}{50}
\hln{100,50}{50}
%\du{57,0}{U_1}
%\ri{57,50}{I}
\end{picture}
\end{center}
Ratkaistaan ensin Théveninin jännite $E_{\rm T}$. Tämän voi tehdä esimerkiksi lähdemuunnoksen avulla:
\begin{center}
\begin{picture}(150,50)(0,0)
\vst{0,0}{J_1R_1}
\hz{0,50}{R_1}
\vz{100,0}{R_3}
\hz{50,50}{R_2}
%\hz{50,0}{R_4}
\out{150,0}
\out{150,50}

\hln{0,0}{150}
\hln{100,0}{50}
%\hln{0,50}{50}
\hln{100,50}{50}
%\du{57,0}{U_1}
%\ri{57,50}{I}
\du{110,0}{E_{\rm T}=\frac{J_1R_1}{R_1+R_2+R_3}R_3=\frac{1}{3}\V}
\end{picture}
\end{center}

}

\frame{
\frametitle{Ratkaisu jatkuu}

Ratkaistaan seuraavaksi Théveninin lähteen resistanssi $R_{\rm T}$. Helpoiten tämä onnistuu sammuttamalla
lähteet ja laskemalla portista näkyvä resistanssi (toinen tapa olisi oikosulkuvirran selvittäminen). Resistanssin
voi laskea joko alkuperäisestä tai muunnetusta piiristä, lopputulos on sama. Lasketaan muunnetusta piiristä,
eli sammutetaan jännitelähde:
\begin{center}
\begin{picture}(150,50)(0,0)
%\vst{0,0}{J_1R_1}
\vln{0,0}{50}
\hz{0,50}{R_1}
\vz{100,0}{R_3}
\hz{50,50}{R_2}
%\hz{50,0}{R_4}
\out{150,0}
\out{150,50}

\hln{0,0}{150}
\hln{100,0}{50}
%\hln{0,50}{50}
\hln{100,50}{50}
%\du{57,0}{U_1}
%\ri{57,50}{I}
\txt{170,25}{R_{\rm T}=\frac{1}{\frac{1}{R_1+R_2}+\frac{1}{R_3}}=\frac{2}{3}\ohm}
\end{picture}
\end{center}
Vastukset $R_1$ ja $R_2$ ovat sarjassa, ja tämä sarjaankytkentä on rinnan $R_3$:n kanssa.
Nyt $E_{\rm T}$ ja $R_{\rm T}$ tiedetään, joten voimme muodostaa Théveninin lähteen
(ks. seuraava kalvo).
}

\frame{
\frametitle{Lopullinen ratkaisu}

\begin{center}
\begin{picture}(150,50)(0,0)
\vst{0,0}{E_{\rm T}=\frac{1}{3}\V}
\hln{0,0}{50}
\hz{0,50}{R_{\rm T}=\frac{2}{3}\ohm}
\out{50,50}
\out{50,0}
\end{picture}
\end{center}

}

\frame{
\begin{block}{Esimerkki} % Piiriarska 1 laskari 3 teht. 2
Muodosta kytkimien vasemmalla puolella olevasta piiristä Théveninin lähde.
Laske sitten, kuinka suuri on virta $I_{\rm X}$, kun kytkimet suljetaan ja $R_{\rm X}$
on a) $0\ohm$, b) $8\ohm$ ja c) $12\ohm$.
\end{block}

\[
R_1=5\ohm \quad R_2=3 \ohm \quad R_3=8\ohm \quad R_4=4 \ohm \quad E=16\V
\]

\begin{center}
\begin{picture}(150,100)(0,0)
\vz{0,0}{R_1}
\hz{0,100}{R_2}
\vz{50,50}{R_3}
\hz{50,100}{R_4}
\hso{100,100}{}
\hso{100,0}{}
\vst{50,0}{E}

\vln{0,50}{50}
\vz{150,25}{R_{\rm X}}
\hln{0,0}{100}
\vln{150,0}{25}
\vln{150,75}{25}
\di{150,10}{I_{\rm X}}


\end{picture}
\end{center}

\tiny Vastaus: $R_{\rm T}=8\ohm$, $E_{\rm T}=8\V$. a) $1\A$ b) $0,5 \A$  c) $0,4 \A$.
}
