\frame{
\frametitle{Operaatiovahvistin}
\begin{itemize}
\item Ammattislangilla opari, englanniksi operational amplifier tai lyhyesti opamp.
\item Operaatiovahvistimeen kytketään käyttöjännite, josta se saa energiansa (kuvassa $\pm 15$ volttia). Operaatiovahvistin mittaa tulonapojensa välistä jännite-eroa ja muuttaa lähtöjännitettä sen mukaisesti.
\end{itemize}


\begin{center}
\begin{picture}(100,50)(0,0)
\vo{0,0}{}{15}
\txt{100,15}{\Uout=A(U_+-U_-)}
\end{picture}
\end{center}

\begin{itemize}
\item $A$ on operaatiovahvistimen vahvistuskerroin, $U_+, U_-, \Uout$ ovat jännitteitä
{\bf maasolmua vasten ilmoitettuna}.

\end{itemize}


}

\frame{
\frametitle{Ideaalinen operaatiovahvistin}
Ideaaliselle operaatiovahvistimelle pätee
\begin{itemize}
\item Vahvistuskerroin $A$ on ääretön (käytännön oparilla se on $>100 000$).
\item Tulonapoihin ei mene virtaa (käytännön oparilla niihin menee mikro- tai nanoampeereja).
\item Lähtöjännite voi vaihdella käyttöjännitteiden välillä (näin voi tapahtua käytännössäkin, jos operaatiovahvistimen datalehdessä lukee "rail-to-rail-operation").
\item Ideaalinen operaatiovahvistin on äärettömän nopea.
\end{itemize}

\begin{center}
\begin{picture}(100,50)(0,-10)
\vo{0,0}{}{15}
\txt{100,15}{\Uout=A(U_+-U_-)}
\end{picture}
\end{center}



}


\frame{
\frametitle{Operaatiovahvistinkytkennät}
\begin{itemize}
\item Operaatiovahvistinta ei käytännössä koskaan käytetä sellaisenaan, vaan kytkemällä
siihen muita komponentteja saadaan aikaiseksi käytännöllinen piiri.
\item Operaatiovahvistimen peruskytkentöjä ovat muun muassa \begin{itemize}
\item Invertoiva vahvistin
\item Invertoiva summain
\item Ei-invertoiva vahvistin
\item Jännitteenseuraaja
\end{itemize}
\end{itemize}
}

\frame{
\frametitle{Invertoiva vahvistin}
\begin{center}
\begin{picture}(100,100)(0,-50)
\vo{0,0}{}{15}
\hz{-50,20}{R_1}
\hz{0,55}{R_2}
\hgp{0,0}
\vln{0,20}{35}
\vln{50,10}{45}
\hln{50,10}{20}
\hgp{70,-40}
\du{70,-40}{\Uout}

\hgp{-50,-30}
\du{-50,-30}{\Uin}

\end{picture}
\end{center}
\[
\Uout=-\frac{R_2}{R_1}\Uin
\]

}


\frame{
\frametitle{Laskutekniikkaa}
\begin{itemize}
\item Jos ideaalinen operaatiovahvistin ($A\to \infty$) on kytketty siten, että $\Uout$:n nousu
kasvattaa jännitettä $U_-$ (tai pienentää jännitettä $U_+$), kyseessä on {\bf negatiivinen
takaisinkytkentä}.
\item Negatiivinen takaisinkytkentä {\bf pakottaa molempiin tulonapoihin saman jännitteen}
eli $U_+=U_-$.
\item Laskutekniikka negatiivisessa takaisinkytkennässä: selvitä toisen tulonavan jännite;
siitä seuraa, että toisessakin tulonavassa on sama jännite.
\item Tarkista lopuksi, että lähtöjännite on käyttöjännitteiden asettamissa rajoissa!
\end{itemize}
}

\frame{
\frametitle{Invertoiva summain}
\begin{center}
\begin{picture}(100,100)(0,-50)
\vo{0,0}{}{15}
\hz{-50,20}{R_1}
\hz{-50,50}{R_2}
\hz{-50,80}{R_3}

\hln{-70,50}{20}
\hln{-90,80}{40}
\du{-70,0}{U_2}
\du{-90,30}{U_3}

\hgp{-70,0}
\hgp{-90,30}


\hz{0,55}{R}
\hgp{0,0}
\vln{0,20}{60}
\vln{50,10}{45}
\hln{50,10}{20}
\hgp{70,-40}
\du{70,-40}{\Uout}

\hgp{-50,-30}
\du{-50,-30}{U_1}


\end{picture}
\end{center}
\[
\Uout=-R\left(\frac{1}{R_1}U_1+\frac{1}{R_2}U_2+\frac{1}{R_3}U_3\right)
\]

}

\frame{
\frametitle{Ei-invertoiva vahvistin}
\begin{center}
\begin{picture}(100,100)(0,-100)
\voi{0,0}{}{15}
\hln{-50,20}{50}

\hln{0,-40}{50}
\vln{0,-40}{40}
\vz{50,-40}{R_2\hspace{-1.1cm}}
\vz{50,-90}{R_1}
\hg{50,-90}

\hgp{70,-40}
\du{70,-40}{\Uout}

\hgp{-50,-30}
\du{-50,-30}{\Uin}
\hln{50,10}{20}

\end{picture}
\end{center}
\[
\Uout=\left(1+\frac{R_2}{R_1}\right)\Uin
\]

}

\frame{
\frametitle{Jännitteenseuraaja}
\begin{center}
\begin{picture}(100,100)(0,-100)
\voi{0,0}{}{15}
\hln{-50,20}{50}

\hln{0,-20}{50}
\vln{0,-20}{20}
\vln{50,-20}{30}

\hgp{70,-40}
\du{70,-40}{\Uout}

\hgp{-50,-30}
\du{-50,-30}{\Uin}
\hln{50,10}{20}

\end{picture}
\end{center}
\[
\Uout=\Uin
\]
Kytkennän hyöty: jännitettä $\Uout$ voidaan kuormittaa (esimerkiksi kytkemällä siihen
jokin mittalaite) ilman, että $\Uin$-puolelle kytketty laite huomaa mitään.
}





\frame{
\[
R_1=1\kohm\quad R_2=2\kohm\quad R_3=1\kohm\quad R_4=3\kohm\quad
\]

\begin{center}
\begin{picture}(100,100)(0,-50)
\vo{0,0}{}{15}
\hz{-50,20}{R_1}
\hz{0,55}{R_2}
\hgp{0,0}
\vln{0,20}{35}
\vln{50,10}{45}
\hz{50,10}{R_3}
\hgp{170,-50}
\du{170,-50}{\Uout}

\hgp{-50,-30}
\du{-50,-30}{\Uin}


\vo{100,-10}{}{15}
\hz{100,45}{R_4}
\hgp{100,-10}
\vln{100,10}{35}
\vln{150,00}{45}
\hln{150,0}{20}


\end{picture}
\end{center}

\begin{exampleblock}{Esimerkki}
Laske $\Uout$, kun $\Uin=2\V$.
\end{exampleblock}

%\tiny $\Uout=12\V$
}



\frame{
\frametitle{Esimerkki}
\[
R_1=1\kohm\quad R_2=2\kohm\quad R_3=1\kohm\quad R_4=3\kohm\quad
\]

\begin{center}
\begin{picture}(100,100)(0,-50)
\vo{0,0}{}{15}
\hz{-50,20}{R_1}
\hz{0,55}{R_2}
\hgp{0,0}
\vln{0,20}{35}
\vln{50,10}{45}
\hz{50,10}{R_3}
\hgp{170,-50}
\du{170,-50}{\Uout}

\hgp{-50,-30}
\du{-50,-30}{\Uin}


\vo{100,-10}{}{15}
\hz{100,45}{R_4}
\hgp{100,-10}
\vln{100,10}{35}
\vln{150,00}{45}
\hln{150,0}{20}


\end{picture}
\end{center}

Laske $\Uout$, kun $\Uin=2\V$. Ratkaisu: piirissä on kaksi peräkkäin kytkettyä invertoivaa
 vahvistinta. Ensimmäisen vahvistuskerroin on $-\frac{R_2}{R_1}=-2$ ja toisen
$-\frac{R_4}{R_3}=-3$. Koko piirin vahvistuskerroin on näiden tulo $-2\cdot -3=6$,
eli kysytty jännite $\Uout=2\V\cdot 6=12 \V$.

}

 \frame{
 \frametitle{Positiivinen takaisinkytkentä}
 \begin{itemize}
 \item Negatiivisessa takaisinkytkennässä lähtöjännitettä kasvattava häiriö kasvattaa myös invertoivan tulon jännitettä $\to$ välitön korjausliike $\to$ piiri pysyy tasapainoasemassa, eli $U_+=U_-$.
\item Entä jos vaihdetaan invertoivan ja ei-invertoivan tulon paikkaa? Tällöin käy juuri päinvastoin: pieni häiriö tasapainoasemasta voimistaa häiriötä $\to$ lähtöjännite ajautuu toiseen äärilaitaan.
 \end{itemize}
 }

\frame{
\frametitle{Positiivinen takaisinkytkentä}
\begin{center}
\begin{picture}(100,100)(0,-100)
\vo{0,0}{}{15}
%\txt{100,15}{\Uout=A(U_+-U_-)}
\hln{-50,20}{50}

\hln{0,-40}{50}
\vln{0,-40}{40}
\vz{50,-40}{R_2\hspace{-1.1cm}}
\vz{50,-90}{R_1}
\hg{50,-90}

\hgp{70,-40}
\du{70,-40}{\Uout}

\hgp{-50,-30}
\du{-50,-30}{\Uin}
\hln{50,10}{20}

\end{picture}
\end{center}


}

\frame{
\frametitle{Positiivinen takaisinkytkentä}
\begin{center}
\begin{picture}(100,100)(0,-50)
\voi{0,0}{}{15}
\hz{-50,20}{R_1}
\hz{0,55}{R_2}
\hgp{0,0}
\vln{0,20}{35}
\vln{50,10}{45}
\hln{50,10}{20}
\hgp{70,-40}
\du{70,-40}{\Uout}

\hgp{-50,-30}
\du{-50,-30}{\Uin}

\end{picture}
\end{center}


}

\frame{
\frametitle{Esimerkki}
\begin{center}
\begin{picture}(100,100)(0,-50)
\voi{0,0}{}{15}
\hz{-50,20}{R_1=7,5\kohm}
\hz{-50,80}{R_2=15\kohm}

\hln{-90,80}{40}
\txt{-105,80}{-15\V}



\hz{0,55}{R=15\kohm}
\hgp{0,0}
\vln{0,20}{60}
\vln{50,10}{45}
\hln{50,10}{20}
\hgp{70,-40}
\du{70,-40}{\Uout}

\hgp{-50,-30}
\du{-50,-30}{\Uin}


\end{picture}
\end{center}
Piirrä piirin ominaiskäyrä (vaaka-akselille $\Uin$ ja pystyakselille $\Uout$). Operaatiovahvistin on rail-to-rail-tyyppinen, eli lähtöjännite voi vaihdella käyttöjännitteiden välillä.
}


\frame{
\frametitle{Esimerkki}
Kumpikin hystereesisraja saadaan laskemalla, milloin operaatiovahvistimen tulonavoissa on sama jännite. Tutkitaan ensin tapaus, kun $\Uout=15\V$. Jos ei-invertoivassa tulossa on $0\V$, on vastuksen $R$ virta $1\mA$ (oikealta vasemmalle) ja 
vastuksen $R_2$ virta myös $1\mA$ oikealta vasemmalle. Tällöin virta $R_1$:n läpi on $0\mA$, eli $\Uin=0\V$. Alempi hystereesisraja on siis $0\V$.

Toinen raja: nyt $\Uout=-15\V$. Jos ei-invertoivassa tulossa on $0\V$, on vastuksen $R$ virta $1\mA$ (vasemmalta oikealle) ja  vastuksen $R_2$ virta $1\mA$ oikealta vasemmalle. Tällöin virta $R_1$:n läpi on $1\mA+1\mA=2\mA$ vasemmalta oikealle, eli $\Uin=15\V$. Ylempi hystereesisraja on siis $+15\V$.


}

 \frame{
 \frametitle{Integraattori}
\begin{center}
\begin{picture}(100,100)(0,-50)
\vo{0,0}{}{15}
\hz{-50,20}{R}
\hc{0,55}{C}
\hgp{0,0}
\vln{0,20}{35}
\vln{50,10}{45}
\hln{50,10}{20}
\hgp{70,-40}
\du{70,-40}{\Uout}

\hgp{-50,-30}
\du{-50,-30}{\Uin}

\end{picture}
\end{center}
\[
u_{\rm out}=-\frac{1}{RC}\int_0^t u_{\rm in}{\rm d}t+U_0
\]

Lähtöjännitteen muutosnopeus on verrannollinen tulojännitteeseen.

 }

 \frame{
 \frametitle{Derivaattori}
\begin{center}
\begin{picture}(100,100)(0,-50)
\vo{0,0}{}{15}
\hc{-50,20}{C}
\hz{0,55}{R}
\hgp{0,0}
\vln{0,20}{35}
\vln{50,10}{45}
\hln{50,10}{20}
\hgp{70,-40}
\du{70,-40}{\Uout}

\hgp{-50,-30}
\du{-50,-30}{\Uin}

\end{picture}
\end{center}
\[
u_{\rm out}=-RC\frac{{\rm d}u_{\rm in}}{{\rm d}t}
\]

Lähtöjännite on verrannollinen tulojännitteen muutosnopeuteen.

 }

\frame{
\frametitle{Komparaattori}
\scriptsize
\begin{itemize}
\item Jos negatiivista takaisinkytkentää ei ole, operaatiovahvistimen lähtöjännite ajautuu aina jompaan kumpaan äärilaitaan.
\item Jos plustulossa on suurempi jännite kuin miinustulossa, operaatiovahvistimen lähtöjännite on maksimiäärilaidassa.
\item Jos plustulossa on pienempi jännite kuin miinustulossa, operaatiovahvistimen lähtöjännite on minimiäärilaidassa.
\item Operaatiovahvistinta voidaan siis käyttää kahden jännitteen vertailijana eli {\em komparaattorina}. Kaikki operaatiovahvistinmallit eivät toimi stabiilisti
komparaattorina – kannattaa ostaa komparaattoriksi suunniteltu operaatiovahvistin, jos tarvitsee sellaista.
\end{itemize}
\begin{center}
\begin{picture}(200,50)(0,0)
\voi{0,0}{}{15}
\stx{-20,20}{10,2 V}
\stx{-20,0}{10,1 V}
\stx{75,10}{$\Uout=+15\V$}

\voi{150,0}{}{15}

\stx{130,20}{10,0 V}
\stx{130,0}{10,1 V}
\stx{225,10}{$\Uout=-15\V$}

\end{picture}
\end{center}


}


 \frame{
 \frametitle{Tarkkuustasasuuntaaja}  %LH5 T1 elepruju
\begin{center}
\begin{picture}(100,100)(0,-200)
% X -350 Y -200
\vo{0,-150}{}{15}
\hz{-50,-130}{R}
\hgp{0,-150}
\rd{0,-100}{}
\hz{0,-70}{R}

\ud{75,-140}{}
\vln{50,-140}{40}
\hln{50,-140}{25}
\vln{0,-130}{60}
\vln{75,-90}{20}
\hln{50,-70}{70}

\du{-50,-180}{U_\mathrm{in}}
\hg{-50,-180}

\vz{120,-120}{R_\mathrm{L}}
\hg{120,-120}
\du{130,-120}{U_\mathrm{out}}
\end{picture}
\end{center}
 }

% \frame{
 %\frametitle{Kokoaaltotasasuuntaaja} % TODO: Lisää tämä
% \begin{itemize}
 %\item
 %\end{itemize}
 %}

\frame{
%\[
%R_1=1\kohm\quad R_2=2\kohm\quad R_3=1\kohm\quad R_4=3\kohm\quad
%\]

\begin{center}
\begin{picture}(200,150)(0,-50)

\hgp{0,100}
\hz{0,100}{R}
\hz{50,100}{R}
\hz{100,100}{R}
\hz{150,100}{R}
\ho{50,25}{}
%\hg{50,25}
\ho{150,25}{}
%\hg{150,25}

\vln{50,75}{25}
\vln{100,50}{50}

\vln{150,75}{25}
\vln{200,50}{50}

\hln{0,25}{50}
\hln{0,-25}{150}
\vln{150,-25}{50}

\du{0,-25}{\Uin}

\hln{200,50}{20}

\hgp{220,0}
\du{220,0}{\Uout}


\end{picture}
\end{center}

\begin{exampleblock}{Esimerkki}
Laske miten $\Uout$ riippuu jännitteestä $\Uin$.
\end{exampleblock}
}

\frame{
\frametitle{Esimerkki}
%\[
%R_1=1\kohm\quad R_2=2\kohm\quad R_3=1\kohm\quad R_4=3\kohm\quad
%\]

\begin{center}
\begin{picture}(200,150)(0,-20)

\hgp{0,100}
\hz{0,100}{R}
\hz{50,100}{R}
\hz{100,100}{R}
\hz{150,100}{R}
\ho{50,25}{}
%\hg{50,25}
\ho{150,25}{}
%\hg{150,25}

\vln{50,75}{25}
\vln{100,50}{50}

\vln{150,75}{25}
\vln{200,50}{50}

\hln{0,25}{50}
\hln{0,-25}{150}
\vln{150,-25}{50}

\du{0,-25}{\Uin}

\hln{200,50}{20}

\hgp{220,0}
\du{220,0}{\Uout}

\lu{0,110}{U_{\rm x}}
\lu{50,110}{U_{\rm x}}
\lu{100,110}{U_{\rm y}}
\lu{150,110}{U_{\rm y}}


\end{picture}
\end{center}
Kaksi vasemmanpuoleisinta vastusta ovat keskenään sarjassa joten niiden molempien yli on sama jännite ($U_{\rm x}$). Kaksi oikeanpuoleisinta
vastusta ovat keskenään sarjassa joten niidenkin yli on sama jännite kummallakin ($U_{\rm y}$).


}

\frame{
\frametitle{Esimerkki}

\begin{center}
\begin{picture}(200,150)(0,-20)

\hgp{0,100}
\hz{0,100}{R}
\hz{50,100}{R}
\hz{100,100}{R}
\hz{150,100}{R}
\ho{50,25}{}
%\hg{50,25}
\ho{150,25}{}
%\hg{150,25}

\vln{50,75}{25}
\vln{100,50}{50}

\vln{150,75}{25}
\vln{200,50}{50}

\hln{0,25}{50}
\hln{0,-25}{150}
\vln{150,-25}{50}

\du{0,-25}{\Uin}

\hln{200,50}{20}

\hgp{220,0}
\du{220,0}{\Uout}

\lu{0,110}{U_{\rm x}}
\lu{50,110}{U_{\rm x}}
\lu{100,110}{U_{\rm y}}
\lu{150,110}{U_{\rm y}}


\end{picture}
\end{center}
Kirjoitetaan Kirchhoffin jännitelain mukaan kaksi yhtälöä:
\begin{eqnarray*}
\Uin + U_{\rm x} + U_{\rm y} &=& 0\\
\Uout - U_{\rm y} - U_{\rm y} - U_{\rm x} - U_{\rm x} &=& 0
\end{eqnarray*}


}

\frame{
\frametitle{Esimerkki}

Tarkastellaan äskeisiä yhtälöitä:
\begin{eqnarray*}
\Uin + U_{\rm x} + U_{\rm y} &=& 0\\
\Uout - U_{\rm y} - U_{\rm y} - U_{\rm x} - U_{\rm x} &=& 0
\end{eqnarray*}
Ratkaistaan jälkimmäisestä $\Uout$:
\[
\Uout = 2(U_{\rm x}+U_{\rm y})
\]
Ensimmäisestä yhtälöstä saadaan:
\[
 U_{\rm x} + U_{\rm y} = -\Uin 
\]
Sijoitetaan yllä oleva edelliseen yhtälöön:
\[
\Uout=-2\Uin
\]

}
