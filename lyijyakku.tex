\frame{
\frametitle{Akku}
\begin{itemize}
\item Akun päätehtävä on tarjota riittävästi energiaa käynnistykseen sekä varmistaa sähkön saanti silloin, kun moottori ei käy.
Akku myös tasaa ajoneuvon sähköjärjestelmän jännitettä.

\item Yleisin akkutyyppi ajoneuvokäytössä on lyijyakku.

\item Jos huoltoväli halutaan saada mahdollisimman pitkäksi ja akun hinta on pieni verrattuna ajoneuvon hintaan, saatetaan käyttää myös nikkelikadmium- eli lipeäakkuja. Vaativimmissa käyttökohteissa, kuten lentokoneissa, käytetään hopeasinkkiakkuja.
\end{itemize}

}

\frame{
\frametitle{Triviatietoa lyijyakuista}
\begin{itemize}
\item Lyijyakku on vanha keksintö. Lyijyakun kehitti ranskalainen fyysikko Gaston Planté (1834–1889) vuonna 1859.
\item Lyijyakku oli ensimmäinen kaupalliseen käyttöön soveltunut ladattava akku.
\item Lyijyakku on suosittu edullisen hinnan ja hyvän pakkaskestävyyden takia.
\item Täyteen varatun akun jännite kuormittamattomana on noin 12.6 V. Kun
jännite on laskenut noin 10,5 volttiin, akku on käytännössä tyhjä. Välittömästi latauksen jälkeen
akun jännite voi olla jopa 15 volttia.
\item Täyden akun pluslevy on lyijyoksidia ja miinuslevy lyijyä. Tyhjän akun molemmat levyt ovat lyijysulfaattia.
\end{itemize}
}

\frame{
\frametitle{Sähköiset ominaisuudet}
Tärkein vaatimus akulle on se, että akun tulee tarjota riittävä käynnistysvirta käynnistimelle ilman, että akun napajännite
laskee alle sytytysjärjestelmän tarvitseman jännitteen.\\[1cm]

Suomen olosuhteissa tämä vaatimus korostuu, koska lämpötilan laskiessa sekä akun käynnistysteho heikkenee että
tarvittava käynnistysteho kasvaa. 
}

\frame{
\frametitle{Sähköiset ominaisuudet}
Akun olennaisimmat sähköiset ominaisuudet ovat
\begin{itemize}
\item Nimellisjännite (voltteina, V)
\item Varauskyky (ampeeritunteina, Ah)
\item Käynnistysvirta (ampeereina, A)
\end{itemize}
}

\frame{
\frametitle{Ominaisuuksien ilmoittaminen}
Käytössä kaksi normia, DIN ja SAE.
\begin{block}{DIN}
Akusta ilmoitetaan nimellisjännite, 20 tunnin purkausaikaa vastaava varauskyky sekä
kylmäkäynnistysvirta.
\end{block}
\begin{block}{SAE}
Akusta ilmoitetaan nimellisjännite, varakapasiteetti minuutteina sekä
kylmäkäynnistysvirta.
\end{block}
Varakapasiteetti kertoo, kuinka kauan akkua voidaan kuormittaa 25 ampeerin virralla, ennen kuin jännite alenee 10,5 volttiin.


}

\frame{
\frametitle{Kylmäkäynnistysvirran määrittäminen}
\begin{itemize}
\item DIN-kylmäkäynnistysvirta määritellään seuraavasti: akku ladataan täyteen, jäähdytetään -18, $^\circ$C lämpötilaan, varastoidaan vähintään 24 tuntia, minkä
jälkeen akkua kuormitetaan merkityllä kylmäkäynnistysvirralla. 30 sekunnin kuormituksen jälkeen jännitteen tulee ollut vähintään 8,4 V ja 3 minuutin jälkeen vähintään
6 V.
\item SAE-kylmäkäynnistysvirta mitataan muuten kuten DIN, mutta jännitteestä mitataan ainoastaan, että 30 sekunnin kuluttua sen tulee olla vähintään 7,2 V. 
\end{itemize}
}

\frame{
\frametitle{Esimerkkejä merkinnöistä}
\begin{itemize}
\item 12 V 60 Ah/20h 220A/-18\,$^\circ$C DIN
\item 12 V 72 min 420A/-18\,$^\circ$C SAE
\end{itemize}
}

\frame{
\frametitle{Turvallisuus}
\begin{itemize}
\item Akkuun saa lisätä vain tislattua tai ionivaihdettua vettä.
\item Täyteen varatun akun nesteestä noin 39 \% on rikkihappoa.
\item Akusta vapautuu vetyä, etenkin jos se varataan nopeasti. Kipinöinti voi sytyttää vetykaasun.
\item Akkua ajoneuvosta irrotettaessa irrotetaan ensin maajohto.
\pause
\item Tällöin mikään työkalu ei vahingossa oikosulje akkua runkoon.
\item Myös kaikki sähkölaitteet tulee kytkeä pois päältä (kipinöinnin välttämiseksi).
\end{itemize}
}

\frame{
\frametitle{Lyijyhyytelöakut}
\begin{itemize}
\item Lyijyhyytelöakkuja käytetään kevyissä laitteissa (sähköpolkupyörät) sekä laitteissa, joissa akku ei aina pysy vaakasuorassa (veneet). Ne eivät vaadi erityistä huoltoa.
\item Lyijyhyytelöakut ovat umpinaisia, joten liian voimakkaan varausvirran vapauttama vety ei pääse poistumaan akusta.
Lyijyhyytelöakkua on varattava erityisellä suljettujen akkujen varaukseen kehitetyllä varaajalla, ei koskaan tavallisella
perinteisellä akkuvaraajalla.
\end{itemize}
}

\frame{
\frametitle{Akun huolto}
Perinteinen lyijyakku huolletaan seuraavasti:
\begin{itemize}
\item Lisätään tislattua tai ionivaihdettua vettä akun kyljessä olevien merkkien korkeudelle. Ellei merkkejä ole, lisätään
vettä niin, että kennolevyt jäävät noin 5--10 mm veden alle.
\item Puhdistetaan akku tarvittaessa.
\item Tarkastetaan liitäntäjohtojen kunto.
\item Tarkastetaan akun varaus (optisella) tiheysmittarilla.
\end{itemize}
Heikko varaustaso kertoo akun tai latausjärjestelmän viasta. Tyhjähkö akku voi myös jäätyä (täyteen varattu akku kestää -50 asteen
ja pahempiakin pakkasia).
}

\frame{
\frametitle{Huoltovapaat ja vähähuoltoiset akut}
\begin{itemize}
\item Madaltamalla lyijyn antimoniseostusta tai käyttämällä lisäaineena antimonin sijasta kalsiumia, varausvirta pienenee ja veden hajoaminen kaasuiksi vähenee.
\item Tällöin kennotulppia ei tarvita.
\item Akun varaustila tutkitaan napajännitteen mittauksella
\end{itemize}
\begin{table}
\begin{tabular}{ l l}
Jännite & Varaus\\
12,6 V & 100 \%\\
12,4 V & 75 \%\\
12,2 V & 50 \%\\
12,0 V & 25 \%\\

\end{tabular}
\caption{Akun napajännitteen riippuvuus varaustilasta.}
\end{table}
}


\frame{
\frametitle{Akun varaaminen}
\begin{itemize}
\item Perinteinen akkuvaraaja koostuu muuntajasta, tasasuuntaajasta ja virtamittarista.
\item Akun kennotulpat avataan löysälle ja kytketään varaaja akkuun. Sopiva latausvirta on 5--10 ampeeria.
\item Tyhjän akun varaaminen kestää tyypillisesti 5--15 tuntia. Akun nestemäärää on tarkkailtava varauksen aikana.
\end{itemize}
}

\frame{
\frametitle{Pikavaraus}
\begin{itemize}
\item Pikavarausta käytetään vain, kun on pakko, ja se voidaan tehdä vain hyväkuntoiselle akulle.
\item Noudata pikavaraajan käyttöohjeita.
\item Pikavarausta varten akku tulee irroittaa ajoneuvosta, ellei auton ja varaajan ohjeissa toisin sanota.
\item Muuten auton latausjärjestelmä voi vaurioitua.
\end{itemize}
}

\frame{
\frametitle{Kuormitustestit}
\begin{itemize}
\item Yksinkertainen tapa selvittää akun kunto on kuormituskoe.
\item Kuormituskoe on helpointa tehdä erillisellä testerillä, joka kertoo suoraan akun kunnon.
\item Manuaalinen kuormitustesti suoritetaan säädettävän tehovastuksen ja virta- ja jännitemittarin avulla.
\item DIN-testissä akkua kuormitetaan virralla, joka on kolminkertainen verrattuna sen ampeerituntikapasiteettiin. Esimerkiksi
60 Ah akkua kuormitetaan 180 ampeerin virralla.
\item Kuorma kytketään akkuun, säädetään virta kohdalleen ja jatketaan kuormitusta reilu 10 sekuntia. Tämän jälkeen
mitataan napajännite ja poistetaan kuorma.
\item Jos 12 voltin akun jännite on kuormitustestissä yli 9,9 volttia, on akku hyväkuntoinen. Alle 9 voltin jännite kertoo heikkokuntoisesta akusta.
\end{itemize}
}

\frame{
\frametitle{Kuormitustestit}
\begin{itemize}
\item SAE-testissä kuormitusvirta on puolet akun ilmoitetusta kylmäkäynnistysvirrasta. 15 sekunnin kuluttua jännitteen
on oltava vähintään 9,6 volttia (jos testi tehdään huoneenlämpötilassa).
\end{itemize}
}

\frame{
\frametitle{Esimerkki}
Autossa on 60 Ah akku ja 12 voltin sähköjärjestelmä. Kuljettaja unohtaa päälle radion, joka kuluttaa 50 wattia. Kuinka pitkän ajan kuluttua 
akun varauksesta on kulutettu puolet? Vastaus: 7,2 tunnissa.
}
