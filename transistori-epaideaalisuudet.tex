\frame{
\frametitle{Käytännön bipolaaritransistorit}
Valintaan vaikuttavat muun muassa
\begin{itemize}
\item Virtavahvistus
\item Suurin sallittu kollektorivirta
\item Kollektori-emitteri -liitoksen jännitteenkesto
\item Yksikkövahvistuksen rajataajuus $f_{\rm T}$
\end{itemize}
}

\frame{
\frametitle{Virtavahvistus}
\begin{itemize}
\item Transistorin valmistaja takaa yleensä jonkin vähimmäisvirtavahvistuksen tietyllä kollektorivirralla.
\item Toleranssit ovat suuria.
\item (Muuten) samaa transistoria myydään usein eri virtavahvistuksilla, esim. BC547A, BC547B ja BC547C.
\end{itemize}
}

\frame{
\frametitle{Suurin sallittu kollektorivirta ja SOA}
\begin{itemize}
\item Kuumenemisteho rajoittaa transistorin virrankestoa.
\item SOA = Safe Operating Area on valmistajan ilmoittama alue sallituille kollektori-emitteri -jännitteille ja kollektorivirroille.
\end{itemize}
}

\frame{
\frametitle{Jännitteenkesto}
\begin{itemize}
\item Jos kollektorin ja emitterin välinen jännite on liian suuri, tapahtuu läpilyönti.
\end{itemize}
}

\frame{
\frametitle{Hajakapasitanssit ja yksikkövahvistuksen rajataajuus}
\begin{itemize}
\item Transistorin sisäiset kapasitanssit saavat aikaan sen, että taajuuden kasvaessa vahvistuskerroin pienenee.
\item Valmistaja ilmoittaa usein yksikkövahvistuksen rajataajuuden $f_{\rm T}$ eli taajuuden, jolla $\beta$ on laskenut
arvoon 1.
\end{itemize}
}

\frame{
\frametitle{Early-jännite}
\begin{itemize}
\item Perinteisessä transistorimallissa virtavahvistuskerroin oletetaan vakioksi. Todellisuudessa kollektorivirta kasvaa,
kun kollektorin ja emitterin välinen jännite kasvaa.
\item Ominaiskäyrille piirretyt jatkeet leikkaavat pisteessä $-U_{\rm A}$, missä $U_{\rm A}$ on Early-jännite.
\item Tätä voidaan mallintaa kollektorin ja emitterin välille piirrettävällä vastuksella
\[
r_0=\frac{|U_{\rm A}|+|U_{\rm CE}|}{I_{\rm C}}\approx \frac{|U_{\rm A}|}{I_{\rm C}}
\]
\end{itemize}
}
