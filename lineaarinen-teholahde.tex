\frame{
\frametitle{Tasasuuntaajat ja regulaattorit}
\begin{itemize}
\item Käytännössä kaikki elektroniikkalaitteet vaativat syöttöjännitteeksi tasajännitteen.
\item Laitetta, joka muuttaa sähköverkosta saatavan vaihtosähkön elektroniikkalaitteelle sopivaksi
tasasähköksi, kutsutaan verkkolaitteeksi tai teholähteeksi. Arkikielessä käytetään myös termejä virtalähde, poweri,
adapteri tai muuntaja.
\item Teholähde voidaan toteuttaa kahdella päätavalla: lineaarisena teholähteenä tai hakkuriteholähteenä.
\end{itemize}
}

\frame{
\frametitle{Teholähteen ominaisuudet}
\begin{itemize}
\item Nimellisjännite
\item Maksimivirta/teho
\item Hurinajännite
\end{itemize}
}

\frame{
\frametitle{Lineaarinen teholähde}
Lineaarisessa teholähteessä
\begin{itemize}
\item verkkojännite alennetaan ensin verkkomuuntajalla. 
\item Seuraavaksi jännite tasasuunnataan (yleensä) diodisillalla.
\item Diodisillalta saatava sykkivä tasajännite suodatetaan tasaisemmaksi kondensaattorilla.
\item Jos tarvitaan erittäin stabiilia tasajännitettä, käytetään lisäksi regulaattoria, joka "leikkaa"\ jännitteen vaihtelut pois.
\end{itemize}
Verkkolaitteessa on tavallisesti myös sulake muuntajan ensiöpuolella.
}

\frame{
\frametitle{Muuntaja}
\begin{itemize}
\item Kaksi kelaa, joiden välillä keskinäisinduktanssia.
\item Toiseen kelaan syötetty muuttuva virta aikaansaa toiseen kelaan muuttuvan jännitteen.
\item Ensiö- ja toisiojännitteiden suhde eli muuntosuhde määräytyy suoraan käämien kierroslukujen perusteella.
\end{itemize}
}

\frame{
\frametitle{Tasasuuntaus}
\begin{itemize}
\item Muuntajalta saatava sinimuotoinen vaihtojännite tasasuunnataan sykkiväksi tasajännitteeksi.
\end{itemize}
}

\frame{
\frametitle{Suodatuskondensaattori}
\begin{itemize}
\item Sykkivä tasajännite ei kelpaa (harvinaisia poikkeuksia lukuunottamatta) sellaisenaan laitteen syöttöjännitteeksi.
\item Käyttämällä suodatuskondensaattoria voidaan sykkivää tasajännitettä tasoittaa vähemmän sykkiväksi.
\item Kondensaattori luovuttaa virtaa kuormalle, kun tasasuuntaajalta tuleva jännite laskee.
\item Suodatetun tasajännitteen vaihtelua kutsutaan hurinajännitteeksi (muita nimiä: ripple (engl.), rippeli, brummi).
\end{itemize}
}


\frame{
\frametitle{(Lineaarinen) regulaattori}
\begin{itemize}
\item Regulaattori leikkaa hurinajännitteen pois.
\item Esimerkiksi jos 5 voltin regulaattorille syötetään välillä 9--12 V vaihtelevaa tasajännitettä, regulaattori pitää lähtöjännitteen jatkuvasti viidessä voltissa.
\item Lineaarinen regulaattori koostuu säätöpiiristä ja tehotransistorista. Säätöpiiri vertaa lähtöjännitettä sisäiseen vertailujännitteeseen ja säätää transistorin johtavuutta niin, että lähtöjännite pysyy vakiona, vaikka kuorman virrankulutus ja tulojännite vaihtelisivat.
\item Tunnetuin regulaattorisarja on 78xx. Kaksi viimeistä numeroa kertovat jännitteen, esimerkiksi 7805 reguloi lähtöjännitteen viiteen volttiin.
\item Regulaattori vaatii tietyn minimijännitteen, jotta se toimii normaalisti. 78xx-sarjalla tämä on yleensä 2\ldots 2,5 volttia yli nimellisjännitteen. Lopullinen arvo kannattaa varmistaa valmistajan datalehdestä.
\item Regulaattoreita on saatavilla myös säädettävänä, esimerkiksi LM317.
\end{itemize}
}


\frame{
\frametitle{Esimerkki lineaarisesta teholähteestä}
\begin{center}
\begin{picture}(280,150)(0,-40)

\out{0,0}
\out{0,50}
\hln{0,0}{50}
\hln{0,50}{50}
\txt{0,25}{230\V\ 50\Hz}

\vl{50,0}{}
\vlr{75,-25}{}
\vlr{75,25}{}
\put(59,0){\line(0,1){50}}
\put(62,0){\line(0,1){50}}
\put(65,0){\line(0,1){50}}

\rd{75,-25}{D_2}
\rd{75,75}{D_1}

\vcj{125,25}
\vln{125,-25}{45}
\vln{125,30}{45}

\hln{75,25}{200}

\vc{150,25}{C_1}
\vc{235,25}{C_2}
\vz{275,25}{R_\mathrm{L}}

\put(175,60){\framebox(30,30){}}
\vln{190,25}{35}
\hgp{190,25}
\cn{190,25}
\stx{190,100}{Regulaattori}

\hln{125,75}{50}
\hln{205,75}{70}

\stx{169,80}{$\Uin$}
\stx{215,80}{$\Uout$}

\end{picture}
\end{center}
Koska muuntajassa on keskiulosotto, voidaan kokoaaltotasasuuntaus toteuttaa kahdella diodilla.
}

\frame{
\frametitle{Suodatuskondensaattorin mitoitus}
\begin{itemize}
\item Suodatuskondensaattori $C_1$ on valittava sellaiseksi, että regulaattorin tulojännite ei laske alle sallitun minimin.
\item Mitä suurempi kondensaattori, sitä pienempi on hurinajännite.
\item Jos kuorman ottama virta oletetaan vakioksi ja oletetaan, että purkautuminen kestää 10 millisekuntia, niin kondensaattorin vähimmäiskoko on
\[
C=\frac{I\cdot 10\, {\rm ms}}{\Delta U},
\]
missä $I$ on kuorman ja regulaattorin ottama virta ja $\Delta U$ on regulaattorille tulevan maksimijännitteen ja regulaattorin vaatiman minimijännitteen erotus.
\end{itemize}
}

\frame{
\frametitle{Kondensaattorin mitoittaminen: tarkempi tapa}
Kondensaattorin ei tarvitse antaa virtaa koko 10 millisekunnin jakson ajan. Voidaan helposti osoittaa, että seuraava nouseva jakso saavuttaa purkautuvan kondensaattorin jännitteen vaihekulman $\phi$ kuluttua, jolloin kondensaattorin minimikoko voidaan laskea:
\begin{itemize}
\item $C=\frac{I\Delta t}{\Delta U}$
\item $\Delta t=\frac{\phi}{360^\circ}\frac{1}{f_{\rm verkko}}$
\item $\phi=90^\circ+\arcsin \frac{U_{\rm min}+U_{\rm diodi(t)}}{U_{\rm max}+U_{\rm diodi(t)}}$.
\end{itemize}
Kaavassa $U_{\rm min}$ ja $U_{\rm max}$ ovat regulaattorille tuleva minimi- ja maksimijännite.
}

