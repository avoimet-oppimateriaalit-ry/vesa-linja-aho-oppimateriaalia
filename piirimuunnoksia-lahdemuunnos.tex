\frame {
  \frametitle{Piirimuunnokset}
\begin{enumerate}
\item Piirimuunnoksella tarkoitetaan toimenpidettä, jonka avulla piiri tai
piirin osa muunnetaan esitystavaltaan erilaiseksi mutta ulospäin samalla
 tavalla käyttäytyväksi piiriksi.
 \item Jo kurssilla käsitellyt jännitelähteiden sarjaankytkentä, vastusten rinnankytkentä
 sekä vastusten sarjaankytkentä ovat piirimuunnoksia.
 \item Tällä tunnilla käsitellään virtalähteiden rinnankytkentä sekä jännitelähde-virtalähdemuunnos.
\end{enumerate}
}


\frame {
\frametitle{Esimerkki piirimuunnoksesta}
Kaksi (tai useampi) vastusta muunnetaan yhdeksi, samalla tavalla käyttäytyväksi vastukseksi.
\begin{exampleblock}{Sarjaankytkentä}
\begin{center}
\begin{picture}(100,20)(50,-15)
\hz{0,0}{R_1}
\hz{50,0}{R_2}
%\ri{-10,0}{I}
%\ri{110,0}{I}
%\hln{-50,0}{50}
%\hln{100,0}{50}
\txt{125,0}{\Longleftrightarrow}
\hz{150,0}{R=R_1+R_2}
\end{picture}
\end{center}
\end{exampleblock}
\begin{exampleblock}{Rinnankytkentä}
\begin{center}
\begin{picture}(100,75)(0,-20)
\hz{0,0}{R_1}
\hz{0,50}{R_2}
\hln{-25,25}{25}
\hln{50,25}{25}
\vln{0,0}{50}
\vln{50,0}{50}
%\ru{0,10}{U}
%\ru{0,60}{U}
\txt{110,25}{\Longleftrightarrow}
\hz{130,25}{\vspace{-1cm}R=\frac{1}{\frac{1}{R_1}+\frac{1}{R_2}}}
\end{picture}
\end{center}
\end{exampleblock}
Tai sama kätevämmin konduktansseilla $G=G_1+G_2$.
}

\frame {
\frametitle{Virtalähteiden rinnankytkentä}
Kaksi (tai useampi) virtalähdettä muunnetaan yhdeksi, samalla tavalla käyttäytyväksi virtalähteeksi.
\begin{exampleblock}{Virtalähteet rinnan}
\begin{center}
\begin{picture}(270,50)(0,0)
\vj{0,0}{J_1}
\vj{50,0}{J_2}
\vdj{100,0}{J_3}
\hln{0,0}{150}
\hln{0,50}{150}
\out{150,0}
\out{150,50}

\txt{170,25}{\Longleftrightarrow}
\vj{200,0}{}
\txt{250,25}{J=J_1+J_2-J_3}
\hln{200,0}{50}
\hln{200,50}{50}
\out{250,0}
\out{250,50}
\end{picture}
\end{center}
\end{exampleblock}
Kuten jännitelähteiden rinnankytkentä, myös virtalähteiden sarjaankytkentä on määrittelemätön (arkikielellä: kielletty) asia piiriteoriassa, aivan
kuten nollalla jakaminen matematiikassa. Johtimessa ei voi samaan aikaan olla kahta erisuuruista virtaa!

}

\frame {
\frametitle{Jännitelähde-virtalähdemuunnos}
{\bf Jännitelähteen ja vastuksen sarjaankytkentä} käyttäytyy kuten {\bf virtalähteen ja vastuksen
rinnankytkentä}.
\begin{exampleblock}{Lähdemuunnos}
\begin{center}
\begin{picture}(150,55)(0,0)
\vst{0,0}{E}
\hz{0,50}{R}
\hln{0,0}{50}
\out{50,0}
\out{50,50}

\txt{66,25}{\Longleftrightarrow}

\vj{100,0}{J}
\vz{150,0}{R}
\hln{100,0}{75}
\hln{100,50}{75}
\out{175,0}
\out{175,50}
\color{red}
\txt{200,25}{E=RJ}
\end{picture}
\end{center}
\end{exampleblock}
}

\frame{
\frametitle{Tärkeää muistettavaa}
\begin{itemize}
\item Huomaa, että ideaalista jännite- tai virtalähdettä ei voi muuntaa yllä olevalla tavalla. Jännitelähteellä
on oltava sarja- ja virtalähteellä rinnakkaisresistanssa.
\item Vastuksen arvo pysyy samana, jännite- ja virtalähteen arvo saadaan kaavasta $E=RJ$, joka
perustuu Ohmin lakiin.
\item Lähdemuunnos ei ole vain piiriteoreettinen kuriositeetti. Lähdemuunnos sopivassa paikassa
säästää monen rivin kaavanpyörittelyltä, esimerkiksi transistorivahvistimien analyysissä.
\end{itemize}
}

\frame {
\frametitle{Muunnoksen perustelu}

\begin{exampleblock}{Lähdemuunnos}
\begin{center}
\begin{picture}(250,55)(0,0)
\vst{0,0}{E}
\hz{0,50}{R}
\hln{0,0}{50}
%\out{50,0}
%\out{50,50}
\hstp{50,0}{}
\ri{55,50}{I}
\du{50,0}{U}

%\txt{66,25}{\Longleftrightarrow}

\vj{150,0}{\frac{E}{R}}
\vz{183,0}{R}
\hln{150,0}{50}
\hln{150,50}{50}
\hstp{200,0}{}
\ri{205,50}{I}
\du{200,0}{U}
\color{red}
%\txt{200,25}{E=RJ}
\end{picture}
\end{center}
\end{exampleblock}

Vasen kuva
\[
I=\frac{E-U}{R}\qquad U=E-RI
\]
Oikea kuva:
\[
I=\frac{E}{R}-\frac{U}{R}=\frac{E-U}{R}\qquad U=(\frac{E}{R}-I)R=E-RI
\]
Molemmat piirit käyttäytyvät samalla tavalla.
}

\frame{
\frametitle{Esimerkki}
Ratkaise $U$.
\begin{center}
\begin{picture}(150,50)(0,0)
\vst{0,0}{E_1}
\hz{0,50}{R_1}
\vz{50,0}{R_3}
\hz{50,50}{R_2}
\vst{100,0}{E}
\hln{0,0}{100}
\du{57,0}{U}
\end{picture}
\end{center}

Muunnetaan piiri
\begin{center}
\begin{picture}(150,50)(0,0)
\vj{0,0}{J_1}
\vz{50,0}{R_1}
\vz{100,0}{R_3\hspace{-0.2cm}}
\vz{75,0}{R_2\hspace{-0.2cm}}
%\hz{50,0}{R_4}
\vj{150,0}{J_2}
\hln{0,0}{150}
%\hln{100,0}{50}
\hln{0,50}{150}
\hln{100,50}{50}
%\du{57,0}{U_1}
%\ri{57,50}{I}
\end{picture}
\end{center}
Ja ei muuta kuin vastaus pöytään:
\[
U=\frac{J_1+J_2}{G_1+G_2+G_3}
\]
}
\frame{
\frametitle{Erittäin tärkeä huomio}
\begin{itemize}
\item Vaikka vastuksen arvo pysyy samana muunnoksessa, vastus ei ole sama vastus! Esimerkiksi edellisessä
esimerkissä muuntamattoman vastuksen virta ei ole sama kuin muunnetun vastuksen virta!
\end{itemize}

}

\frame{
\begin{block}{Esimerkki}
Ratkaise virta $I$ muuntamalla virtalähteet jännitelähteiksi. $J_1=10\A$, $J_2=1\A$, 
$R_1=100\ohm$, $R_2=200\ohm$ ja $R_3=300\ohm$.
\end{block}

\begin{center}
\begin{picture}(150,50)(0,0)
\vj{0,0}{J_1}
\vz{50,0}{R_1}
\vz{100,0}{R_3}
\hz{50,50}{R_2}
%\hz{50,0}{R_4}
\vj{150,0}{J_2}
\hln{0,0}{150}
\hln{100,0}{50}
\hln{0,50}{50}
\hln{100,50}{50}
%\du{57,0}{U_1}
\ri{57,50}{I}
\end{picture}
\end{center}

Tämä on helppo ja nopea lasku; jos huomaat kirjoittavasi toista sivullista yhtälöitä, olet tehnyt
jotain väärin.
}

%LUENTO6

\frame{
\begin{block}{Ratkaisu}
Ratkaise virta $I$ muuntamalla virtalähteet jännitelähteiksi. $J_1=10\A$, $J_2=1\A$, 
$R_1=100\ohm$, $R_2=200\ohm$ ja $R_3=300\ohm$.
\end{block}

\begin{center}
\begin{picture}(150,50)(0,0)
\vj{0,0}{J_1}
\vz{50,0}{R_1}
\vz{100,0}{R_3}
\hz{50,50}{R_2}
%\hz{50,0}{R_4}
\vj{150,0}{J_2}
\hln{0,0}{150}
\hln{100,0}{50}
\hln{0,50}{50}
\hln{100,50}{50}
%\du{57,0}{U_1}
\ri{57,50}{I}
\end{picture}
\end{center}


\begin{center}
\begin{picture}(150,50)(0,0)
\vst{0,0}{R_1J_1}
\hz{0,50}{R_1}
\hz{100,50}{R_3}
\hz{50,50}{R_2}
%\hz{50,0}{R_4}
\vst{150,0}{R_3J_2}
\hln{0,0}{150}
%\hln{100,0}{50}
%\hln{0,50}{50}
%\hln{100,50}{50}
%\du{57,0}{U_1}
\ri{57,50}{I}
\end{picture}
\end{center}
\[
I=\frac{R_1J_1-R_3J_2}{R_1+R_2+R_3}=\frac{1000\V-300\V}{600\ohm}=\frac{7}{6}\A
\approx 1,17 \A.
\]
}
