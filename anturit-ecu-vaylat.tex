% HUOM! ANTURIKALVOISTA SUURI OSA LAINATTU MOOTTORINOHJAUKSEN K2010 LUENTO6 JA
% VÄYLÄKALVOT LUENTO8.



\frame{
\frametitle{Anturin määritelmä}
\begin{itemize}
\item Anturi on laite, joka muuttaa jonkin suureen (esimerkiksi kemiallisen tai fysikaalisen)  käsiteltävään muotoon
(jännitteeksi, virraksi, digitaaliseksi signaaliksi\ldots).
%\item
\end{itemize}
}


\frame{
\frametitle{Moottorinohjausjärjestelmän anturit}
\begin{description}
\item[1950-luku] $\lambda$-happianturi 
\item[1960] sähkömekaaninen paineanturi, pietsosähköinen nakutusanturi
\item[1970] hall-anturi, turvatyynyn venymäliuska-anturi, silikonipaineanturi
\item[1980] kuumalanka- ja ohutkalvoilmamassamittarit, integroitu paineanturi
\item[1990] mems-kiihtyvyysanturit, pietsosähköinen kiertonopeusanturi, mikromekaaninen
ilmamassa-anturi, mikromekaaninen kiertonopeusanturi

\end{description}
}



\frame{
\frametitle{Älykkäät anturit}
\begin{itemize}
\item Anturilta tulevan signaalin käsitteleminen tapahtuu itse anturiyksikössä.
\item Älykäs anturi voidaan kalibroida automaattisesti, käyttämällä useita antureita yhtäaikaa.
\item Suorittamalla signaalinkäsittely paikan päällä anturissa, moottorinohjausyksiköltä
vaaditaan vähemmän laskentatehoa.
\item Anturiliitäntä voidaan standardoida.
\end{itemize}
}

\frame{
\frametitle{Merkitys autotekniikassa}
\begin{itemize}
\item Sähkö- ja elektroniikkajärjestelmän osuus ajoneuvon arvosta on noin 26 \%.
\item Maailmassa valmistettavista antureista noin joka toinen sijoitetaan ajoneuvoon.
\end{itemize}
}

\frame{
\frametitle{Luotettavuus}
\begin{itemize}
\item Vikataajuus ilmoitetaan todennäköisyytenä, yksikkönä yleensä ppm/aikayksikkö.
\item Eli monenko miljoonasosan todennäköisyydellä anturi hajoaa aikayksikön aikana.
\item Ajoneuvoanturin vikataajuudeksi halutaan yleensä alle 10 ppm/10 vuotta.
\item Vrt. matkapuhelin (koko laite) 5000 ppm\footnote{Lähde: Autojen anturit (2009). Suhtaudun skeptisesti tuohon lukuun: ettäkö
vain viisi tuhannesta matkapuhelimesta hajoaa 10 vuoden aikana, huh?}.
\end{itemize}
}

\frame{
\frametitle{Kustannukset}
\begin{itemize}
\item Uudet ajoneuvot sisältävät helposti $>>150$ anturia...
\item Anturien oltava halpoja, tyypillisesti 1-30 euroa.
\item Luotettavuus- ja käyttöolosuhdevaatimukset kovia
\item Tärinä, iskut, lämpötila, kosteus, kemikaalit, sähköiset häiriöt\ldots
\item Toisissa sovelluksissa vaaditaan enemmän luotettavuutta kuin toisissa:
ohjaus, jarrutus, turvalaitteet, moottori/voimansiirto, alusta/renkaat, mukavuus,
diagnoosi, informaatio, varkaudenesto.
\end{itemize}
}

\frame{
\frametitle{Mittausperiaatteet}
\begin{itemize}
\item Asema-anturi
\item Nopeus- ja pyörintänopeusanturi
\item Kiihtyvyysanturi
\item Paineanturi
\item Voima- ja momenttianturi
\item Virtausmittari
\item Kaasu- ja pitoisuusanturi
\item Lämpötila-anturi
\item Optiset anturit
\end{itemize}
}

\frame{
\frametitle{Anturityypit}
\begin{itemize}
\item Moottorin pyörintänopeusanturit
\item Hall-vaiheanturit
\item Vaihteiston ohjauksen nopeusanturit
\item Pyörän nopeusanturit
\item Mikromekaaniset kiertonopeusanturit
\item Pietsosähköinen äänirauta-kiertonopeusanturi
\item Mikromekaaniset paineanturit
\item Korkeapaineanturit
\item Lämpötila-anturit
\item Kaasupoljinanturit
\item Ohjauskulman anturit
\item Vaihteiston ohjauksen asema-anturit
\item Akselianturit
\item Kuumakalvoilmamassamittarit
\item Pietsosähköinen nakutusanturi
\end{itemize}
}

\frame{
\frametitle{Lisää antureita}
\begin{itemize}
\item SMM-kiihtyvyysanturi
\item Tilavuusmikromekaaniset piikiihtyvyysanturit
\item Pietsosähköiset kiihtyvyysanturit
\item iBolt-voima-anturi
\item Momenttianturi
\item Ultraäänianturi
\item Sade- ja valoanturi
\item Lika-anturi
\item Kaksitasoinen $\lambda$-anturi
\item Laajakaistainen $\lambda$-anturi
\item Ilmastoinnin hiilidioksidianturi
\end{itemize}
}

\frame{
\frametitle{Potentiometri asema-anturina}
\begin{itemize}
\item Potentiometri = säätövastus.
\item Halpa, yksinkertainen, helppo varmentaa.
\item Kuluminen, iskut ja kiihdytykset vaikuttavat mittaustulokseen.
\item Käytetään esimerkiksi kaasupolkimen asennon, polttoainesäiliön pinnankorkeuden sekä
mittalevyn (KE- ja L-Jetronic) asennon tunnistamiseen.
\end{itemize}
}

\frame{
\frametitle{Lämpötila-anturit}
\begin{itemize}
\item Perustuvat tavallisesti NTC-vastukseen (PTC-vastusta käytetään harvoin).
\item Moottorin, imuilman, öljyn ja polttoaineen lämpötila (Diesel).
\end{itemize}
}

\frame{
\frametitle{Moottorin pyörintänopeus}
\begin{itemize}
\item Induktiivinen anturi
\item Hall-anturi
\item AMR-anturi
\end{itemize}
}

\frame{
\frametitle{Vaiheanturit (kampikulma)}
\begin{itemize}
\item Nokka-akselianturi tarvitaan kertomaan, mitkä sylintereistä ovat puristus- ja mitkä poistotahdissa.
\end{itemize}
}

\frame{
\frametitle{Ilmamäärä}
\begin{itemize}
\item Voidaan mitata joko massaa, virtausta tai painetta.
\item Kuumakalvomittari mittaa suoraan ilmamassan. 
\end{itemize}
}

\frame{
\frametitle{Nakutus}
\begin{itemize}
\item Pietsosähköinen anturi havaitsee nakutuksen.
\item Pietsokiteen puristaminen/taivuttaminen aikaansaa jännitteen (sama 
tekniikka on käytössä mm. savukkeensytyttimissä ja grilleissä).
\end{itemize}
}

\frame{
\frametitle{Paine}
\begin{itemize}
\item Imusarjan paine
\item Öljynpaine
\item Polttoaineen paine
\end{itemize}
}

\frame{
\frametitle{Jäännöshappipitoisuus ($\lambda$)}
\begin{itemize}
\item Kaksitasoinen ja laajakaistainen anturi.
\item Laajakaista-anturi mittaa tarkasti myös silloin, kun $\lambda$ ei ole noin 1.

\end{itemize}
}

% ECU Gasoline-kirjasta, sivut 262--267
\frame{
\frametitle{Moottorinohjausyksikkö (ECU)}
\begin{itemize}
\item ECU = Engine Control Unit = moottorinohjausyksikkö
\item Huom! Toinen merkitys lyhenteelle: ECU voi tarkoittaa myös
elektronista ohjainyksikköä (Electronic Control Unit) yleensä. Esimerkiksi
ilmatyynyn ohjainyksikkö (ACU) ja vaihteiston ohjausyksikkö (TCU) ovat
ECUja sanan jälkimmäisessä merkityksessä.
\item Moottorinohjausyksikkö lukee tietoa antureilta ja säätää (pääasiassa)
sytytyksen ajoitusta ja polttoaineensyöttöä ennalta ohjelmoitujen karttojen mukaan.
\item Nykyiset moottorinohjausyksiköt ovat älykkäitä ja ottavat huomioon
moottorin parametrien muuttumisen.
\end{itemize}

}

\frame{
\frametitle{Moottorinohjausyksikön vaatimukset}
Moottorinohjausyksikkö joutuu kovien ympäristörasitusten armoille.
\begin{itemize}
\item Korkeat ($> 100 ^\circ {\rm C}$) ja matalat ($-40 ^\circ {\rm C}$) lämpötilat.
\item Nopeat lämpötilan muutokset.
\item Altistuminen kemikaaleille (polttoneste, öljy).
\item Tärinä ja iskut.
\item Käyttöjännitteen vaihtelu.
\item EMC (Electromagnetic compatibility).
\end{itemize}
}

\frame{
\frametitle{Moottorinohjausyksikön toteutus}
\begin{itemize}
\item Piirilevyllä käytetään pintaliitoskomponentteja (kuten nykyaikaisessa elektroniikassa ylipäätään).
\item Kokonaisuus on sijoitettu metalli- tai muovikoteloon, jonka kyljessä on riviliittimet.
\end{itemize}
}

\frame{
\frametitle{Signaalinkäsittely - tulosignaalit}
ECU voi käsitellä kolmenlaisia tulosignaaleja.
\begin{itemize}
\item Analoginen tulosignaali. Jännitesignaali, joka voi määritellyllä välillä saada mitä tahansa arvoja ja 
voi muuttua jatkuva-aikaisesti.
\item Digitaalinen tulosignaali. Signaalilla on kaksi sallittua arvoa, 1 ja 0. Monet anturit lähettävät nykyään suoraan
digitaalista signaalia. Digitaalisignaalilla on tarkkaan määritellyt rajat nousu- ja laskuajoille sekä minimi- ja maksimijännitteille.
\item Pulssimuotoiset signaalit. Analogisen ja digitaalisen "välimaastossa". Signaali ei ole yhtä tarkkaan määriteltyjen rajojen
sisällä kuin digitaalisignaali. Esimerkiksi kierroslukuanturilta
tuleva signaali voi olla tällainen (lasketaan pulssien määrää per aikayksikkö).
\end{itemize}
}

\frame{
\frametitle{Signaalinkäsittely - signaalien esiprosessointi}
\begin{itemize}
\item Analoginen signaali muunnetaan digitaaliseksi (A/D-muunnos) käsittelemistä varten. Tyypillinen
A/D-muuntimen erottelukyky on 5 mV. Tällöin esimerkiksi nollan ja viiden voltin välisellä signaalilla on noin
1000 erilaista digitaalista arvoa.
\item Digitaalinen signaali ei vaadi mitään esikäsittelyä, vaan ohjausyksikkö voi käsitellä sitä suoraan.
\item Pulssimuotoisesta signaalista on karsittava häiriöt ja ylilyönnit pois. Sen jälkeen sitä voidaan käsitellä
digitaalisignaalina.
\end{itemize}
Signaaleille tehdään aluksi yksinkertainen tasonrajoitus ja alipäästösuodatus.
}

\frame{
\frametitle{Signaalinkäsittely}
\begin{itemize}
\item Moottorin ohjaaminen ei teoriassa poikkea mistään muusta tietokoneella toteutetusta säädöstä.
\item Mikrokontrolleri valvoo tulosignaaleja ja muuttaa lähtösignaaleja ennalta määrätyn ohjelman mukaisesti.
\item Tulosignaalit muodostavat moniulotteisia karttoja, joista lähtösignaalit lasketaan.
\end{itemize}
}

\frame{
\frametitle{Muisti}
\begin{itemize}
\item Mikrokontrollerin ohjelma on tallennettu ROM-muistille (Read Only Memory).
\item ROM-muisteja on erilaisia: Flash-EPROM voidaan ohjelmoida uudelleen elektronisesti,
EPROM voidaan tyhjentää UV-valolla ja sitten ohjelmoida uudestaan.
\item EPROM on yleensä oma erillinen piirinsä, Flash-muisti voidaan integroida samalle
sirulle mikrokontrollerin kanssa.
\item Mikrokontrollerin oma muisti on usein rajoitettu, siksi voidaan tarvita ulkoista muistipiiriä.
\item Moni valmistaja käyttää samaa ohjainyksikköä eri automalleissa; tuotatolinjan lopussa
vain ajetaan oikea ohjelma sisään (EoL = End-of-Line -ohjelmointi).
\end{itemize}
}

\frame{
\frametitle{Muisti}
\begin{itemize}
\item Kesken ajon muuttuvat tiedot tallennetaan kontrollerin RAM-muistiin (Random Access Memory). RAM-muistista
tiedot häviävät, jos ohjausyksiköltä katkaistaan virta (esimerkiksi akku irroitetaan).
\item Tärkeät, mutta auton käytön aikana muuttuvat tiedot (ajonestolaitteen salausavaimet, tärkeät moottorin kulumiseen liittyvät
tiedot) tallennetaan EEPROM-muistiin.
\item EEPROM-muisti eroaa Flash-muistista siten, että jokainen muistipaikka voidaan tyhjentää ja uudelleenkirjoittaa erikseen.
Flash-EPROMiin pitää tyhjentää \& kirjoittaa kerralla koko sisältö.
\end{itemize}
}

\frame{
\frametitle{ASIC-piirit}
\begin{itemize}
\item Aina ei pärjätä suoraan "kaupan hyllyltä"\ saatavilla mikrokontrollereilla. Esimerkiksi joku nopeutta
vaativa laskutoimitus voidaan hoitaa ASIC-piirillä (application specific integrated circuit).
\end{itemize}
}

\frame{
\frametitle{"Vahtikoira"}
\begin{itemize}
\item Mikrokontrollerin ohjelma voi mennä jumiin, kuten mikä tahansa tietokone (muttei yhtä todennäköisesti, onneksi).
\item Vikatilanteisiin varaudutaan ns. vahtikoirapiiriä käyttämällä.
\item Prosessi yksinkertaistettuna: kontrolleri ja vahtikoira kysyvät vuorotellen toisiltaan "oletko kunnossa?"
\item Jos vastausta ei tule määrätyn ajan kuluessa, käynnistetään elpymistoimenpiteet (esimerkiksi mikrokontrollerin uudelleenkäynnistys).
\end{itemize}
}



\frame{
\frametitle{Pääteaste}
\begin{itemize}
\item Mikrokontrolleripiiri itsessään jaksaa ohjata vain muutaman milliampeerin kuormia.
\item Ohjausyksikössä on pääteaste, joka vahvistaa mikrokontrollerilta saadut signaalit toimilaitteille sopiviksi.
\item Ohjauslähdöt on suunniteltu niin, että ne kestävät mahdollisen oikosulun maahan tai akun plusnapaan.
\item Lähdön oikosulkutilanteet tunnistetaan ja niistä annetaan virheilmoitus.
\end{itemize}
}

\frame{
\frametitle{Pääteaste}
Lähtösignaalit voidaan jakaa kahteen luokkaan.
\begin{itemize}
\item Kytkinsignaalit. Näillä on kaksi tilaa: päällä ja pois.
\item PWM-signaalit. Pulssinleveysmodulaatiolla (PWM) voidaan säätää monenlaisia toimilaitteita.
\end{itemize}
}


\frame{
\frametitle{Tiedonsiirto elektronisten järjestelmien sisällä ja välillä}
\begin{itemize}
\item CAN (Controlled area network, 1991)
\item LIN (Local interconnect network, 2001)
\item MOST (Media Oriented Systems Transport, 2001)
\item Bluetooth (1990-luvun loppu)
\item FlexRay (2006)
\end{itemize}
}

\frame{
\frametitle{Ajoneuvoväylätekniikan vaatimukset}
\begin{itemize}
\item Tiedonsiirtonopeus
\item Häiriönsietokyky
\item Reaaliaikaisuus
\item Verkkoon kytkettyjen laitteiden määrä
\end{itemize}
}

\frame{
\frametitle{Väylien luokittelu}
\begin{itemize}
\item A-luokka: hidas (alle 10 kbit/s). Ei-kriittiset anturit ja toimilaitteet. Esimerkki: LIN.
\item B-luokka: keskinopea (alle 125 kbit/s). Mukavuuslaitteet. Esimerkki: Low Speed CAN (CAN-B).
\item C-luokka: nopea (alle 1 Mbit/s). Moottorin- ja vaihteistonohjaus. Esimerkki: High Speed CAN (CAN-C).
\item C+-luokka: erittäin nopea (alle 10 Mbit/s). Moottorin- ja vaihteistonohjaus. Esimerkki: FlexRay.
\item D-luokka: erittäin nopea (yli 10 Mbit/s). Telematiikka ja multimedia.
\end{itemize}
}

\frame{
\frametitle{CAN-väylä}
\begin{itemize}
\item 1991 (500E-Mersu)
\item Mahdollistaa reaaliaikaisen ohjauksen. Esimerkki: automaattivaihteisto pyytää moottoria alentamaan momenttia.
\item 5 voltin logiikka, 120 $\ \Omega$ terminointi.
\item Voidaan käyttää yksi- tai kaksijohtimisena.
\item Väylälle mahdollista kytkeä ainakin 30 laitetta. 
\end{itemize}
}

\frame{
\frametitle{LIN-väylä}
\begin{itemize}
\item Yksijohtiminen, halpa. Vaihtoehto Low Speed CANille.
\item Yhteen väylään voidaan kytkeä enintään 16 laitetta.
\end{itemize}
}

\frame{
\frametitle{MOST}
\begin{itemize}
\item Nopea. Suunniteltu multimedialaitteile (DVD, TV, CD, navigointi).
\item Johdin- tai valokuitukäyttö.
\item Valokuidun etuja: ei häiriöitä (sisään eikä ulos), kevyt, voi vetää "mistä tahansa".
\end{itemize}
}

\frame{
\frametitle{Bluetooth}
\begin{itemize}
\item Käytetään matkapuhelimen ym. liittämiseksi auton äänentoistojärjestelmään.
\end{itemize}
}

\frame{
\frametitle{FlexRay}
\begin{itemize}
\item Suunniteltiin erityisesti turvalaitteita ajatellen.
\item Reaaliaikainen, nopea ja erittäin vikasietoinen tiedonsiirto.
\end{itemize}
}