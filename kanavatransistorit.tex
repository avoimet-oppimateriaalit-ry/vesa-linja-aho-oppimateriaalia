

\frame{
\frametitle{Kanavatransistorit}
\begin{itemize}
\item Tyypit MOSFET ja JFET
\item Kalvoilla käsitellään N-kanava JFET ja N-kanavainen avaustyyppinen MOSFET
\end{itemize}
\begin{center}
\begin{picture}(150,50)(0,0)

\njfet{0,0}{\rm JFET}
\benmos{100,0}{\rm MOSFET}
\vln{150,-3}{8}
\end{picture}
\end{center}
}

\frame{
\frametitle{n-kanava-JFET}
\begin{itemize}
\item Hilalla estosuuntainen diodiliitos -- ei kulje virtaa.
\item Hilan ja lähteen välisellä jännitteellä säädellään nieluvirtaa
\item Jos $U_{\rm GS}=0\V$, niin $I_{\rm D}=I_{\rm DSS}$, missä $I_{\rm DSS}$ on komponenttikohtainen vakio
\item Jos $U_{\rm GS}=U_{\rm P}$, niin $I_{\rm D}=0$. $U_{\rm P}$ on komponenttikohtainen vakio, ja se on negatiivinen,
esimerkiksi $-3$ volttia.
\item Näiden ääripäiden välillä $I_{\rm D}=I_{\rm DSS}(1-\frac{U_{\rm GS}}{U_{\rm P}})^2$ kun $\ U_{\rm DS}\ge U_{\rm GS}-U_{P}$
\end{itemize}
\begin{picture}(50,50)(-100,-20)

\njfet{0,0}{\rm JFET}
\txt{-10,0}{\rm G}
\txt{55,25}{\rm D}
\txt{55,-15}{\rm S}

\end{picture}
}

\frame{
\frametitle{MOSFET}
\begin{itemize}
\item Kuten JFET: $I_{\rm D}=0$ ja $U_{\rm GS}$ säätelee nieluvirtaa
\item Hilajännitteen oltava $>U_{\rm T}$, jotta virta kulkisi
\item $I_{\rm D}=K(U_{\rm GS}-U_{T})^2, \mbox{kun}\ U_{\rm GS} \ge U_{T}\ \mbox{ja}\ U_{\rm DS}\ge U_{\rm GS}-U_{T}$
\item $U_{\rm T}$ ja $K$ ovat komponenttikohtaisia vakioita
\item Jos hilajännite hyvin suuri (luokkaa $>10 \V$), kanava aukeaa täysin ja näyttää pieneltä resistanssilta
$R_{\rm DS on}$
\end{itemize}

\begin{picture}(50,50)(-100,-20)

%\njfet{0,0}{\rm JFET}
\txt{-10,0}{\rm G}
\txt{55,25}{\rm D}
\txt{55,-15}{\rm S}

\benmos{0,0}{\rm MOSFET}
\vln{50,-3}{8}

\end{picture}
}

