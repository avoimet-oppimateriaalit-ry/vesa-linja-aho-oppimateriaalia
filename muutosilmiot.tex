% TODO Kerro mikä on aikavakio.
\frame{
\frametitle{Muutosilmiöt}
\begin{itemize}
\item Tasajännitteen ja sinimuotoisen vaihtojännitteen lisäksi tärkeitä tarkasteltavia ovat muutosilmiöt.
\item Muutosilmiö tapahtuu esimerkiksi laitteita päälle kytkettäessä.
\item Esimerkiksi jos kondensaattorin ja vastuksen sarjaankytkentä kytketään jännitelähteeseen, kondensaattori varautuu.
\item Tällöin jännite muuttuu ajan funktiona. Miten se muuttuu? Edellyttää differentiaaliyhtälön ratkaisemista!
\item Monimutkaisemmat differentiaaliyhtälöt kannattaa ratkaista Laplace-muunnoksella.
\item Piirit, joissa on yksi kondensaattori tai kela, on helppo ratkaista {\rm yritteen} avulla.
\item Monimutkaisemmankin piirin voi ratkaista yritteen avulla, mutta yritteen keksiminen voi olla vaikeaa.
\end{itemize}
}

\frame{
\frametitle{Kondensaattori}
Kondensaattori on komponentti, jonka jännitteelle ja virralle pätee yhtälö:
\[
i=C\Du
\]


\begin{center}
\begin{picture}(50,50)(0,0)
\vc{25,0}{C}
\di{25,40}{i}
\du{40,0}{u}
\end{picture}
\end{center}
Aivan kuten aiemmin on opittu, että vastukselle pätee yhtälö $u=Ri$.

}

\frame{
\frametitle{Kondensaattori}
Yhtälö
\[
i=C\Du
\]
tarkoittaa, että mitä suurempi virta kondensaattorin läpi kulkee, sitä nopeammin sen jännite muuttuu.
Tai sama toisinpäin: mitä nopeammin kondensaattorin jännite muuttuu, sitä suurempi virta sen läpi
kulkee.

\begin{center}
\begin{picture}(50,50)(0,0)
\vc{25,0}{C}
\di{25,40}{i}
\du{40,0}{u}
\end{picture}
\end{center}
Symboli $C$ tarkoittaa kondensaattorin kapasitanssia, jonka yksikkö on faradi (F).
}

\frame{
\frametitle{Kondensaattori}
Integroimalla yhtälö
\[
i=C\Du
\]
puolittain saadaan:
\[
u=\frac{1}{C}\int i{\rm d}t + {\rm integrointivakio}
\]
tai määrättynä integraalina (valitsemalla alkuhetkeksi $t=0$ ja loppuhetkeksi joku
ajanhetki $t$)
\[
u=\frac{1}{C}\int_0^t i {\rm d}t + u(0).
\]
Termi $u(0)$ tarkoittaa kondensaattorin jännitettä ajanhetkellä nolla, ja sitä merkitään
usein myös $U_{\rm C0}$ tai $U_{0}$:
\[
u=\frac{1}{C}\int_0^t i {\rm d}t + U_{0}.
\]
}

\frame{
\frametitle{Yksinkertainen esimerkki}
Ladataan virtalähteellä kondensaattoria. Lukuarvot ovat:
\[
J=6\A  \qquad C=2\,{\rm F}\qquad U_0=0\V
\]
\begin{center}
\begin{picture}(100,50)(0,0)
\vc{100,0}{C}
\vj{0,0}{J}
\hln{0,0}{100}
\hln{0,50}{100}
\du{115,0}{u}
\ri{50,50}{i}

\end{picture}
\end{center}
\[
u=\frac{1}{C}\int_0^t i {\rm d}t + U_0=\frac{1}{2}\int_0^t 6 {\rm d}t+0=\frac{1}{2}\int_0^t 6 {\rm d}t=\frac{1}{2} 6t=3t
\]
Kondensaattorin jännite on alussa 0 volttia, sekunnin kuluttua 3 volttia, kahden sekunnin kuluttua 6 volttia\ldots

}

\frame{
\frametitle{Kondensaattori ja differentiaaliyhtälö}
Muutetaan hieman piiriä. Kytkin suljetaan ajanhetkellä $t=0$:
\[
E=12\V  \qquad C=2\,{\rm F}\qquad R=3\ohm \qquad U_0=5\V
\]
\begin{center}
\begin{picture}(100,50)(0,0)
\vc{100,0}{C}
\vst{0,0}{E}
\hso{0,50}{t=0}
\hln{0,0}{100}
%\hln{0,50}{50}
\hz{50,50}{R}
\du{115,0}{u}
\ri{50,50}{i}

\end{picture}
\end{center}
Kirchhoffin lakien ja kondensaattorin yhtälön mukaan
\[
i=\frac{E-u}{R}\ \mbox{ja}\ i=C\Du \Longrightarrow \frac{E-u}{R}=C\Du \Longrightarrow RC\Du + u = E
\]
Ratkeaa yritteellä
\[
u=B+Ae^{-\frac{t}{\tau}}.
\]

}

\frame{
\frametitle{Differentiaaliyhtälön ratkaiseminen}
Lasketaan yritteen derivaatta
\[
u=B+Ae^{-\frac{t}{\tau}} \Longrightarrow \Du=-\frac{A}{\tau}e^{-\frac{t}{\tau}}
\]
Ja sijoitetaan yrite derivaattoineen yhtälöön:
\[
RC\Du + u = E \Longrightarrow -\frac{RCA}{\tau}e^{-\frac{t}{\tau}}+B+Ae^{-\frac{t}{\tau}}=E
\]
Jotta yhtälö olisi tosi kaikilla $t$:n arvoilla, tulee päteä $\tau=RC$ ja $B=E$. Tällöin:
\[
-Ae^{-\frac{t}{\tau}}+Ae^{-\frac{t}{\tau}}=0
\]
Mistä saadaan $A$? Kondensaattorin alkujännitteestä. Ajanhetkellä $t=0$ tulee kaavan antaa
jännitteeksi 5 volttia:
\[
u=B+Ae^{-\frac{t}{\tau}} \Longrightarrow u=E+Ae^{-\frac{t}{RC}} \Longrightarrow 5=12+Ae^{-\frac{0}{2\cdot 3}}
\Rightarrow A=-7.
\]
Lopullinen vastaus jännitteelle:
\[
u=12-7e^{-\frac{t}{6}}.
\]

}

\frame{
\frametitle{Kela}
Tavallaan "kondensaattorin vastakohta"\  {--} yhtälöissä on jännitteet ja virrat vaihtaneet
paikkaa verrattuna kondensaattorin yhtälöihin:
\[
u=L\Di
\]


\begin{center}
\begin{picture}(50,50)(0,0)
\vl{25,0}{L}
\di{25,43}{i}
\du{40,0}{u}
\end{picture}
\end{center}
Sama integraalimuodossa
\[
i=\frac{1}{L}\int_0^t u {\rm d}t + I_{0}.
\]
}



% TODO Lisää konkan purkuesimerkki 85 uf ja 325 volttia ja 10 kohm


\frame{
\frametitle{Esimerkki}
Ratkaise kondensaattorin jännite $u$ ajan funktiona. Kytkin suljetaan ajanhetkellä $t=0$.
\[
R_1=R_2=1\ohm\quad C=1\,{\rm F}\quad E=10\V\quad U_0=0\V
\]
\begin{center}
\begin{picture}(100,50)(0,0)
\vc{100,0}{C}
\vst{-50,0}{E}
\hso{-50,50}{t=0}
\vz{50,0}{R_2}
\hln{-50,0}{150}
\hln{50,50}{50}
\hz{0,50}{R_1}
\du{115,0}{u}
\ri{75,50}{i}

\end{picture}
\end{center}
{\tiny Lopputulos: $u=5-5e^{-2\cdot t}=5(1-e^{-2\cdot t})\ (\rm volttia)$}

}


\frame{
\frametitle{Esimerkki}

Ratkaise kondensaattorin jännite $u$ ajan funktiona. Kytkin suljetaan ajanhetkellä $t=0$.
\[
R_1=R_2=1\ohm\quad C=1\,{\rm F}\quad E=10\V\quad U_0=0\V
\]
\begin{center}
\begin{picture}(100,50)(0,0)
\vc{100,0}{C}
\vst{-50,0}{E}
\hso{-50,50}{t=0}
\vz{50,0}{R_2}
\hln{-50,0}{150}
\hln{50,50}{50}
\hz{0,50}{R_1}
\du{115,0}{u}
\ri{75,50}{i}

\end{picture}
\end{center}

\begin{eqnarray*}
i=C\Du \qquad i&=&\frac{E-u}{R_1}-\frac{u}{R_2}\qquad \Rightarrow\\
\frac{R_1R_2C}{R_1+R_2}\Du+u&=&\frac{R_2}{R_1+R_2}E
\end{eqnarray*}

}

\frame{
\[
\frac{R_1R_2C}{R_1+R_2}\Du+u=\frac{R_2}{R_1+R_2}E
\]
Sijoitetaan yhtälöön tunnilta tuttu yrite derivaattoineen:
\[
u=B+Ae^{-\frac{t}{\tau}} \Rightarrow \Du=-\frac{A}{\tau}e^{-\frac{t}{\tau}}
\]
jolloin saadaan
\[
\frac{R_1R_2C}{R_1+R_2}(-\frac{A}{\tau}e^{-\frac{t}{\tau}})+B+Ae^{-\frac{t}{\tau}}=\frac{R_2}{R_1+R_2}E
\]
Jotta vakiotermit olisivat samat, täytyy päteä: $B=\frac{R_2}{R_1+R_2}E$. Eksponenttitermissä
täytyy olla $\tau=\frac{R_1R_2C}{R_1+R_2}$. Nyt yhtälö on ratkaistu:
\[
-Ae^{-\frac{t}{\tau}}+Ae^{-\frac{t}{\tau}}=0\qquad u=\frac{R_2}{R_1+R_2}E+Ae^{-\frac{t}{\frac{R_1R_2C}{R_1+R_2}}} 
\]
}

\frame{
Sijoitetaan vastaukseen lukuarvot:
\[
u=5+Ae^{-2t} 
\]
Vakio $A$ ratkeaa alkuehdosta. Ajanhetkellä $t=0$ kondensaattorin jännitteen tulee olla
nolla:
\[
0=5+Ae^{-2\cdot 0}\quad \Rightarrow \quad A=-5
\]
Lopullinen vastaus on siis:
\[
u=5-5e^{-2\cdot t}=5(1-e^{-2\cdot t})\ (\rm volttia).
\]

}


