\frame{
\frametitle{Konduktanssi}
\begin{itemize}
\item Resistanssilla tarkoitetaan kappaleen kykyä vastustaa sähkövirran kulkua.
\item Resistanssin käänteislukua kutsutaan konduktanssiksi. Konduktanssin
tunnus on $G$ ja yksikkö Siemens (S).
\item Konduktanssi kertoo kappaleen kyvystä johtaa sähköä.
\item Esimerkiksi jos $R=10 \ohm$ niin $G=0,1\Siemens$.
\end{itemize}

\begin{center}
$G=\frac{1}{R} \qquad U=RI \Leftrightarrow GU=I$ 

\begin{picture}(50,50)(0,0)
\hz{0,0}{G=\frac{1}{R}}
\ru{0,10}{U}
\ri{5,0}{I}
\hln{50,0}{15}
\hln{-15,0}{15}
\end{picture}
\end{center}
}


\frame {
\frametitle{Sähköteho}
\begin{itemize}
\item Teho tarkoittaa tehtyä työtä aikayksikköä kohti.
\item Tehon tunnus on $P$ ja yksikkö watti (W).
\item Elementin kuluttama teho on $P=UI$\begin{picture}(50,10)(-30,-5) \hz{0,0}{} \ri{8,0}{I} \ru{0,10}{U} \end{picture}
\item Jos kaava antaa positiivisen tehon, elementti kuluttaa tehoa. Jos kaava antaa negatiivisen tehon,
elementti luovuttaa tehoa.
\end{itemize}
}
\frame{
\frametitle{Sähköteho}
\begin{block}{Energia ei häviä piirissä}
Piirielementtien kuluttama teho = piirielementtien luovuttama teho.
\end{block}

\begin{center}
\begin{picture}(200,50)(0,0)
\vst{0,0}{E}
\vz{100,0}{R}
\hln{0,0}{100}
\hln{0,50}{100}
\ui{0,45}{I}
\di{100,42}{I}
\txt{200,50}{I=\frac{U}{R}}
\txt{200,25}{P_R=UI=U\frac{U}{R}=\frac{U^2}{R}}
\txt{200,0}{P_E=U\cdot(-I)=U\frac{-U}{R}=-\frac{U^2}{R}}
\end{picture}
\end{center}
Kuvassa vastus kuluttaa yhtä paljon tehoa kuin jännitelähde luovuttaa.
}



\frame{
\frametitle{Napa ja portti}
\begin{itemize}
\item Piirissä olevaa johdon liitäntäkohtaa nimitetään {\bf navaksi} tai nastaksi.
\item Kaksi napaa muodostavat {\bf portin} eli napaparin.
\item Helpoin esimerkki: auton akku, jolla sisäistä resistanssia.
\end{itemize}
\begin{center}
\begin{picture}(100,50)(0,0)
\vst{0,0}{E}
\hz{0,50}{R_{\rm S}}
\hln{0,0}{50}
\out{50,0}
\out{50,50}
\end{picture}
\end{center}

}

\frame{
\frametitle{Solmu}
\begin{itemize}
\item {\bf Solmulla} tarkoitetaan virtapiirin aluetta, jonka sisällä on sama potentiaali.
\item Palikkamenetelmä: laske kynä johonkin kohtaan johdinta. Ala värittää johdinta,
ja aina kun tulee vastaan komponentti, käänny takaisin. Väritetty alue on yksi solmu. 
\item Montako solmua on kuvan piirissä?
\end{itemize}
\begin{center}
\begin{picture}(150,50)(0,0)
\vst{0,0}{E}
\hz{0,50}{R_1}
\hz{50,50}{R_3}
\hz{100,50}{R_5}
\vz{50,0}{R_2}
\vz{100,0}{R_4}
\vz{150,0}{R_6}
\hln{0,0}{150}
\ri{8,50}{I}
\end{picture}
%$R_1=R_2=R_3=R_4=R_5=R_6=1\ohm\qquad E=9\V$
\end{center}
}

\frame{
\frametitle{Maa}
\begin{itemize}
\item Yksi solmuista voidaan {\bf nimetä} maasolmuksi.
\item Maasolmu-merkinnän käyttö säästää piirtämisvaivaa.
\item Auton akun miinusnapa on kytketty auton runkoon; näin muodostuu
suuri maasolmu.
\item Sanonta "tämän solmun jännite on (esim.) 12 volttia" tarkoittaa, että sen
solmun ja maan välinen jännite on (esim.) 12 volttia.
\end{itemize}
\begin{center}
\begin{picture}(150,50)(0,0)
\vst{0,0}{E}
\hz{0,50}{R_1}
\hz{50,50}{R_3}
\hz{100,50}{R_5}
\vz{50,0}{R_2}
\vz{100,0}{R_4}
\vz{150,0}{R_6}
\hln{0,0}{150}
\ri{8,50}{I}
\hgp{0,0}
\cn{0,0}
\end{picture}
\end{center}
}


\frame{
\frametitle{Maa}
\begin{itemize}
\item Maasolmu voidaan kytkeä laitteen runkoon tai olla kytkemättä
(symboli ei siis tarkoita, että laite on "maadoitettu").
\item Edellisen kalvon piiri voidaan piirtää myös näin:
\end{itemize}
\begin{center}
\begin{picture}(150,50)(0,0)
\vst{0,0}{E}
\hz{0,50}{R_1}
\hz{50,50}{R_3}
\hz{100,50}{R_5}
\vz{50,0}{R_2}
\vz{100,0}{R_4}
\vz{150,0}{R_6}
%\hln{0,0}{150}
\ri{8,50}{I}
\hgp{0,0}
\hgp{50,0}
\hgp{100,0}
\hgp{150,0}

%\cn{0,0}
\end{picture}
\end{center}
}
